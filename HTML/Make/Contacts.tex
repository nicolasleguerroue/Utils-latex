%#############################################################
%Add element in credits
\addcontentsline{toc}{chapter}{Préambule}
\chapter*{Préambule}

\section*{Informations}

\begin{items}{darkBlue}{\faFile}
    \item Document réalisé en \LaTeX{} par Nicolas Le Guerroué pour le Club de Robotique et d’Électronique Programmable de Ploemeur (\href{www.crepp.org}{CREPP}) 
\end{items}

\begin{items}{darkBlue}{\faHistory}
    \item Version du \today
\end{items}

\begin{items}{darkBlue}{\faFont}
    \item Taille de police: 11pt (\getCurrentFont)
\end{items}

\begin{items}{darkBlue}{\faComment}
    \item N'hésitez pas à faire des retours sur le document, cela permettra de l'améliorer
\end{items}
\begin{items}{darkBlue}{\faEnvelope}
    \item \underline{\href{mailto:nicolasleguerroue@gmail.com}{nicolasleguerroue@gmail.com}}
\end{items}

\begin{items}{darkBlue}{\faGithub}
    \item \href{https://github.com/CREPP-PLOEMEUR}{https://github.com/CREPP-PLOEMEUR\footnote{Click-droit et \bold{Copier l'adresse du lien}}}
\end{items} 

\begin{items}{darkBlue}{\faCertificate}
    \item Permission vous est donnée de copier, distribuer et/ou modifier ce document sous quelque forme et de 
    quelque manière que ce soit.
\end{items}

\begin{items}{red}{\faPrint}
    \item \bold{Dans la mesure du possible, évitez d'imprimer ce document si ce n'est pas nécessaire. Il est optimisé pour une visualisation sur un ordinateur et contient beaucoup d'images.}
\end{items}

    % \begin{items}{darkBlue}{\Triangle}
    %     \item Ce document centralise les supports utilisés et produits par le Club de Robotique et d’Électronique 
    %     Programmable de Ploemeur dans le cadre des ateliers \bold{Programmation de microcontrôleurs}.\\Un glossaire présente les acronymes et abbreviations utilisés.
    % \end{items}


    