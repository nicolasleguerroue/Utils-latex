
\chapter{Objet dynamique}

\section{Présentation}

Dans le cadre de l'enseignement Supervision de L'ENIB, nous avons développé dans le module \lib{ENIBSupervision}, un type \ib{DynamicObject}. Ce nouveau type hérite d'un Qwidget de la bibliothèque PyQt.\\

Il permet de représenter un objet de supervision (verin, moteur,...) et de l'afficher. Nous pourrons alors très simplement dans Qtdesigner afficher ce type d'objet.\\ 

\img{\rootImages/presentation_1.png}{exemple d'utilisation du type}{0.3}

Pour l'exemple nous avons crée un objet Moteur 1 que nous avons initialisé à l'état sens 1. 
Le widget affiche en partant du haut:
\begin{enumerate}
    \item Le nom de l'objet (choisi lors de l'initialisation) .
    \item Un bouton informations qui lorsque l'on place le curseur dessus affiche le nom de l'objet, son type, son état, la description de l'objet ainsi que s'il est en défaut.
    \item Une image représentant l'état actuel de l'objet.
\end{enumerate}

\img{\rootImages/presentation_2.png}{Objet en défaut}{0.3}

Lorsque qu'un défaut a été transmis à l'objet ou qu'il en a détecté un par lui même, le bouton informations devient rouge. Il indique à l'utilisateur qu'un défaut est actuellement présent sur l'objet. Si nous appuyons sur le bouton, une fenêtre s'ouvre indiquant les problèmes actuels. \\

Dans cet exemple nous avons simulé un vérin ayant ses capteurs S1 et S2 à l'état haut au même moment. Le programme a bien détecté que c'était un état impossible et à crée une erreur. Nous avons également transmis au vérin l'information qu'une erreur "test problème" était présente. 
 
 
 
 

 
\section{Utilisation simple du type}

Nous allons créer un programme simple permettant de représenter un objet dynamique. Pour cette expérimentation nous présenterons l'utilisation d'un objet vérin mais le fonctionnement est similaire pour tous les autres objets.\\

Nous allons commencer par un créer une fenêtre MainWindow sous QtDesigner (fonctionne pour tout type de fenêtre mais ici on l'utilise simplement pour un exemple simple).

\img{\rootImages/programme_1.png}{Choix de la fenêtre}{0.28}

On ajoute un layout à notre fenêtre pour que notre widget puisse prendre toute la place qui lui est disponible et puisse être redimensionné. On glisse l'icône sur la fenêtre et on le dimensionne pour prendre tout l'espace de la fenêtre.

\img{\rootImages/programme_2.png}{Emplacement des layouts}{0.26}

On ajoute ensuite un widget dans notre layout (on le glisse). On peut le renommer pour être plus explicite (fenêtre à droite).

\img{\rootImages/programme_3.png}{Nom du futur objet}{0.25}

On sélectionne notre widget, puis click-Droit + promote to. 

\img{\rootImages/programme_4.png}{Emplacement de promotion}{0.85}

Enfin, on sélectionne la classe \ib{DynamicObject}

\textcolor{red}{\bold{Si cette classe n'apparaît pas, il faut revenir au tutoriel  PyQt5 et ses outils, au chapitre 9 qui traite de l'importation des widgets personnalisés (localisé dans la classe \ib{DynamicObject})}} \\

On peut maintenant fermer QtDesigner et enregistrer notre MainWindow dans le dossier \dir{UI\_formFiles}. On exécute donc ensuite notre script \file{scriptExport.py}.\\

On peut maintenant ouvrir notre  fichier \file{UI\_MainWindow.py} present dans le dossier \dir{UI\_class}. On importe notre classe \ib{dynamicObject}.

\begin{Python}
from ENIBSupervision.UI_class.UI_DynamicObject import DynamicObject
\end{Python}


Si le fichier \lib{UI\_mainWindow.py} était vide jusqu'à présent on peut l'initialiser simplement comme ceci:

\begin{Python}
    super().__init__(parent)
    self.__ui = Ui_MainWindow()
    self.__ui.setupUi(self)
\end{Python}


Il faut maintenant initialiser l'objet. On utilise la fonction \italic{initObject()} interne à notre widget. En premier paramètre on fournit le type de l'objet (vérin, moteur, ...), puis le nom de l'objet et enfin la description de l'objet.

\begin{Python}
    self.__ui.verin_1.initObject("verin", "vérin 1", "Vérin pour déplacer la palette")
\end{Python}

On peut maintenant initialiser l'état de notre objet (on utilisera la même fonction pour actualiser plus tard l'objet). 

La fonction à utiliser diffère  des objets mais reprend la même structure. Pour un objet à n capteurs on utilisera la fonction \italic{Update} n**2 \italic{States} avec n paramètres.

Dans notre exemple la fonction à utiliser est \italic{Update4States()} avec en paramètres s1 et s2. 

\begin{Python}
    self.__ui.verin_1.update4States(0,0)
\end{Python}

A cette étape de l'expérimentation, vous devriez obtenir une fenêtre similaire en exécutant le programme.\\

\img{\rootImages/programme_10.png}{Vérin en position neutre}{0.6}

Notre objet est bien initialisé. Le nom apparaît au dessus, en passant le curseur sur l'objet le texte est conforme à l'objet et l'image affichée correspond bien à l'état transmis plus tôt (2 capteurs à 0).\\

Nous pouvons dorénavant actualiser notre objet dès qu'un capteur change de valeur. Pour ceci on utilisera la même fonction que précédemment.


Pour actualiser l'objet de manière synchrone à la table d'animation, vous pouvez vous inspirer du programme exemple.\\

La dernière fonctionnalité permettra de signaler un problème sur l'objet ( permettant d'afficher un bouton rouge avertissant l'utilisateur). Pour ceci on utilisera la fonction \italic{setProb()} avec en paramètre le problème à signaler.

\begin{Python}
    self.__ui.verin_1.setProb("test")
\end{Python}

On exécutant le programme, vous devriez constater qu'un bouton rouge est apparu. En cliquant dessus une fenêtre devrait s'ouvrir avec notre problème "test" bien signalé.

\img{\rootImages/programme_12.png}{Vérin en défaut}{0.8}

Lorsque le problème sera résolu, on pourra utiliser la fonction \italic{resetProb} avec en paramètre le problème pour le réinitialiser.
\begin{Python}
    self.__ui.verin_1.resetProb("test")
\end{Python}


\section{Création d'un nouvel objet}

Dans cette section nous aborderons la manière de créer un nouveau "type" d'objet qui pourra être représenté grâce à ce type \ib{dynamicObject}. Nous allons illustrer cela par la création de l'objet moteur.\\


On ouvre dans un éditeur de code le fichier \file{UI\_DynamicObject.py} present dans le dossier \dir{ENIBSupervision/UI\_class}.


Il faut d'abord ajouter dans la liste objectType, un string nommant le type de l'objet. Dans notre cas on ajoutera "moteur". 
\begin{Python}
        objectType = ["verin","moteur"]
\end{Python}


Ensuite, il nous faut créer une liste que nous nommerons dans notre exemple motorStates:\\

Son premier membre est un entier indiquant le nombre d'états que peut prendre l'objet. Dans notre cas un moteur à 2 capteurs, il peut donc prendre $ 2^2 $ états différents.
Son deuxième membre est une liste décrivant les états pris par l'objet :

\begin{enumerate}
    \item Les états sont du type str.
    \item On classe les états suivant l'ordre binaire naturel: (ckm1,ckm2) -> (0,0) ; (0,1) ; (1,0) ; (1,1).
    \item Les états sont écrits de la manière suivante : \\
    \bold{l'objet} ... (ex: ...est à l'arrêt ; ...tourne dans le sens 1).
\end{enumerate}

Dans notre exemple, on a donc : 

\begin{enumerate}
    \item élément 0: (0,0) - "est à l'arrêt"
    \item élément 1: (0,1) - "tourne dans le sens 2"
    \item élément 2: (1,0) - "tourne dans le sens 1"
    \item élément 3: (1,1) - "est en défaut"
\end{enumerate}

Il nous reste plus qu'à ajouter notre liste à la liste states (dans le même ordre que dans la liste objectType).

Nous obtenons donc:

\begin{Python}
    motorStates = [4, ["est à l'arrêt", "tourne dans le sens 2", "tourne dans le sens 1", "est en défaut"] ]
\end{Python}


Nous allons maintenant créer une liste permettant de signaler des états en défaut ou problématiques (ex: un verin qui aurait ses deux capteurs s1 et s2 à l'état haut).\\

Notre liste attend en paramètre:
\begin{enumerate}
    \item une liste contenant les numéros des états problématiques. Dans notre exemple le seul problème est lorsque les deux capteurs ckm1 et ckm2 sont à l'état 1. 
    \item Une seconde liste contenant des strings décrivant l'état signalée dans la liste précédente. Ici on aura simplement "ckm1 et ckm2 sont tous les 2 à l'état 1".
\end{enumerate}

A l'instar de states, on ajoute notre liste motorStateDefault (celle que l'on vient de créer) à la liste stateDefault.

\begin{Python}
    motorStateDefault = [ [3], ["ckm1 et ckm2 sont tous les deux à l'état 1"]]
    stateDefault = [verinStateDefault, motorStateDefault]
\end{Python}


La dernière étape consiste à créer des images pour représenter les différents états pris par l'objet. Les dimensions des images n'ont peu d'importance. Il est cependant recommandé de garder pour un même objet, des dimensions fixes. \\

Les images seront ensuite enregistrées au format png dans le dossier \dir{ENIBSupervision/UI\_class/images}  nommées comme ceci: \bold{monObjet\_numEtat}.\\

dans notre cas \file{moteur\_0.png} ; \file{moteur\_1.png} ; \file{moteur\_2.png} ; \file{moteur\_3.png}

\img{\rootImages/creation_4.png}{Images possibles d'un moteur}{0.6}


\newpage
\section{Utilisation des voyants}

Il est possible de placer des voyants qui changerons d'état en fonction des variables de l'automate. \\
La définition des voyants se situe dans le fichier \file{UI\_IndicLight.py}.\\
Le  nom de la classe promue est \ib{IndicLight}



Il convient tout d'abord de placer un widget de type "widget" sur la fenêtre. \\
Nous appellerons notre widget 'voyant\_capteur' \\

\img{\rootImages/dynamicObject_voyant.png}{Nom du voyant}{0.5}



Ensuite, il faut promouvoir le widget en \ib{IndicLight}.
Pour cela, on fait un click-droit sur l'arborescence des widgets en sélectionnant notre widget. \\
Dans "promote to" (promouvoir en) , il faut déclarer le fichier \file{UI\_IndicLight.py} de la même façon que le fichier \file{UI\_DynamicObject.py}

\img{\rootImages/dynamicObject_locate.png}{Importation de l'objet \ib{IndicLight}}{0.5}


L'emplacement du fichier \file{UI\_IndicLight.py} est le suivant : \\

\file{ENIBSupervision/UI\_class/UI\_IndicLight.py} \\


Une fois le widget promu, nous allons pouvoir enregistrer QtDesigner, fermer la fenêtre et actualiser l'interface avec le script \file{scriptExport.py}.\\ \\

La gestion des voyants se fera dans le fichier \lib{UI\_MainWindow.py} \\


Par défaut, lors de l'initialisation d'un voyant, ce dernier comporte deux états par défaut : 


\begin{enumerate}
    \item Un état '0' de couleur rouge sans texte
    \item Un état '1' de couleur verte sans texte
\end{enumerate}

Nous souhaitons maintenant modifier les propriétés de base de notre voyant

\subsubsection{Couleur du voyant}

\begin{Python}
self.__ui.voyant_capteur.changeState(0, [0,255,255]) 
\end{Python}
Cette instruction permet de redéfinir la couleur de l'état 0 du voyant 'voyant\_capteur' en cyan. \\
Le second argument de la fonction \italic{changeState} est un tableau comportant les valeurs de Rouge, Vert et Bleu du voyant (valeurs comprises entre 0 et 255) \\

\subsubsection{Texte du voyant}

Étant donné que le voyant hérite des labels (\ib{QLabel}), vous pouvez définir un texte pour le voyant de la manière suivante.

\begin{Python}
self.__ui.voyant_capteur.setText('capteur voyant')
\end{Python}


\subsubsection{Forme du voyant}

Le voyant peut être de forme rectangulaire (par défaut) ou arrondie avec l'instruction suivante :

\begin{Python}
self.__ui.voyant_capteur.setRoundCorner(True) #voyant arrondi
\end{Python}

\subsubsection{Couleur et taille du texte}

Deux fonctions sont à votre disposition pour changer la taille de la police (en pt) et sa couleur(RGB) : \italic{setFontSize} et \italic{setFontColor}
\begin{Python}
self.__ui.voyant_capteur.setFontSize(14)
self.__ui.voyant_capteur.setFontColor([150,150,150]) #gris
\end{Python}

Avec toutes ces instructions placées dans le constructeur de \\ \lib{UI\_MainWindow.py}, on obtient le voyant suivant.

\img{\rootImages/voyant.png}{Le voyant}{0.5}

\subsubsection{Un nouvel état}
Il est également possible d'ajouter un nouvel état.
\begin{Python}
self.__ui.voyant_capteur.setStateCount(2, [0,0,255]) 
\end{Python}
Cette instruction ajoute un troisième état appelé '2' de couleur bleue.

\subsubsection{Afficher un état}

Une fois que tous les états de notre voyant sont définis, il nous reste à changer l'état en fonction des conditions.
Pour définir l'état courant, on utilise l'instruction \italic{setState}
\begin{Python}
self.__ui.voyant_capteur.setState(1) 
\end{Python}
Le voyant est donc actualisé à l'état 1


