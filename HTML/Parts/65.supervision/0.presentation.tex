\chapter{Introduction}
\section{Objectifs}


Ce projet vise à mettre en place une interface de supervision en langage \emph{Python}.\newline
Ainsi, une interface en Python sera réalisée pour un contrôle de l’API (Automate Programmable Industriel) à distance. \newline
Nous avons opté pour la réalisation de notre propre logiciel de supervision pour plusieurs raisons : \newline

\begin{enumerate}
\item Contrôle fin de la communication API-Ordinateur
\item Indépendance aux logiciels propriétaires (PC-Vue….) \newline
\end{enumerate} 

Cependant, nous devons être conscients que la place des éléments graphiques est plus complexe.

\section{Supports pédagogiques}

Ce projet vise à donner suite à l'expérimentation de la création d'une interface Python pour la supervision. \\
Le logiciel précédent, permettant de contrôler un vérin à distance, était basé sur la bibliothèque Python TKinter. \\
Le protocole de communication restera inchangé, seul le logiciel pour l'interface sera différent.

\chapter{Pré-requis}
\section{Logiciels}

Afin de bien débuter, il convient que quelques logiciels soient préalablement installés.
Il s'agit de : \newline


\begin{enumerate}
\item \textbf{Python} \newline
La version minimal requise est la 3.7 sous peine de mauvaises installations et compatibilités entre les logiciels. \newline
Le logiciel est disponible à l'adresse \url{https://www.python.org/downloads} \newline
\item \textbf{Bibliothèque Python Pip} \newline 
Pip est une bibliothèque permettant d'installer d'autres bibliothèques Python. Toutes les bibliothèques nécessaires au projet de supervision sont disponibles via Pip. \newline
La bibliothèque est disponible à l'adresse \url{https://pip.pypa.io/en/stable} \newline

Il est également possible d'installer pip via Anaconda (ouvrir un terminal ou console) :
\begin{lstlisting}
conda install pip
\end{lstlisting}

\item \textbf{Logiciel Unity Pro} \newline 
Unity Pro sera utilisé pour générer le grafcet de supervision. \newline 
\end{enumerate}

\chapter{Logiciels et outils}


L'interface graphique sera réalisée avec le logiciel QtDesigner, en complément de la bibliothèque Python PyQt5.
Le protocole de communication se basera sur la bibliothèque Python PyModBus.
L'API sera programmé avec le logiciel Unity-Pro

\section{PyQt5}

\subsection{Présentation}

Qt est une bibliothèque originellement développée pour le C++. Elle a pour vocation d'aider à développer des interfaces graphiques utilisateurs (GUI en anglais) et propose de nombreux outils pour manipuler des fichiers, des bases de données etc...\newline

Elle est distribuée sous deux versions : \newline
- l'une est commerciale et nécessite de payer un abonnement régulier\newline
- l'autre est distribuée sous licence LGPL (c'est cette version qui a été utilisée dans notre projet)
Les deux versions proposent les mêmes fonctionnalitées à version identique.\\

PyQt5 n'est autre qu'un portage de la bibliothèque Qt C++ dans sa version 5 sur Python.\newline
A noter que PyQt5 dispose lui aussi d'une version commerciale mais que nous avons utilisé la version distribuée sous licence GPL.

\subsection{Pourquoi ?}

Nous avons choisi Qt pour de nombreuses raisons :
\begin{enumerate}
    \item \colors{red}{\bold{Multi-plateforme}}\newline PyQt5 a l'énorme avantage d'être portable sur de très nombreuses plateformes. Un seul programme peut être utilisé sur mobile comme sur Windows ou Linux.\newline
    Par exemple, dans notre cas, nous avons programmé de manière indifférente sur Windows ou Linux pendant toute la durée du projet.
    
    \item \colors{red}{\bold{Polyvalence}}\newline Comme dit un peu plus haut, PyQt5 propose de nombreux outils permettant la gestion de base de données, de fichiers xml, d'interfaces graphiques, de programmes multi-thread et bien d'autres choses encore. Cette polyvalence permet d'éviter d'utiliser plusieurs bibliothèque différente. Dans ce cas, une seule bibliothèque permet de faire énormément de choses.
    
    \item \colors{red}{\bold{Licence}}\newline Nous l'avons aussi expliqué un peu plus haut, PyQt5 est disponible sous licence GPL.V3 ce qui permet de l'utiliser gratuitement (sous les conditions imposées par le licence GPLv3)
    
    \item \colors{red}{\bold{Expérience}}\newline Plusieurs membres du groupe ont déjà utilisés Qt à plusieurs reprises. Ainsi, même si nous utilisions la version Python, le temps d'accoutumance à cette version particulière était plus court que si nous avions dû apprendre à utiliser TKinter.
\end{enumerate}

\section{QtDesigner}

\subsection{Présentation}
QtDesigner est un logiciel pour créer des interfaces graphiques.\newline
L'interaction entre les différents éléments graphiques appelés "widgets" sera faite en langage Python. 
Un widget est un élément grapĥique comportant des propriétés telles qu'une couleur, une taille, etc. \newline \newline


QtDesigner permet donc de gagner en efficacité en terme de conception graphique. \\
Lorsque les éléments graphiques de base sont devenus insuffisants, nous avons développés nous même les élements graphiques (Vérin, moteur, etc)



\section{PyModBus}
\subsection{Présentation}
\lib{PyModBus} est une bibliothèque Python pour communiquer entre des périphériques avec le protocole ModBus analogue à celui de l'API.

\subsection{Améliorations et documentation}

Cependant, pour des questions de robustesse et de facilité, nous utiliserons un module (ENIBSupervision) que nous avons développé pour simplifier les communications entre l'API et l'ordinateur. \newline
Ce module utilise la bibliothèque \lib{PyModBus} en interne.
 Ainsi, toutes les fonctions d'écriture/lecture des données sont protégées des mauvaises manipulations de programmation. \newline
Un tutoriel de prise en main du module \lib{ENIBSupervision} est disponible en annexe.

\subsection{Installation}

Il est possible d'installer la bibliothèque \lib{Pymodbus} de deux manières : \newline

\subsubsection{Installation par pip}
La commande pour installer Pymodbus est la suivante :
\begin{lstlisting}
pip install pymodbus
\end{lstlisting}

Pour une installation sur les machines de l'ENIB, il est nécessaire de préciser le proxy pour l'installation. \newline 

La commande suivante est à saisir dans un terminal :
\begin{lstlisting}
pip install --proxy "http://proxy.enib" pymodbus
\end{lstlisting}

\subsubsection{Installation par Anaconda}
 Il est également possible d'installer Pymodbus avec Anaconda : 
\begin{lstlisting}
conda install -c auto pymodbus
\end{lstlisting}

La bibliothèque officielle ainsi que l'ensemble de sa documentation est disponible à l'adresse \url{https://pymodbus.readthedocs.io/en/latest/} \newline 

\section{Pyuic5}

\subsection{Présentation}

Une fois l'interface graphique est réalisée à l'aide de QtDesigner, il faut utiliser \bold{pyuic5} pour convertir le fichier \file{.ui} en fichier python exploitable.\newline

La commande pour transformer le fichier UI est la suivante (via un terminal/console) :
\begin{lstlisting}
pyuic5.py -x fichier_qtdesigner_entree.ui -o fichier_python_sortie.py
\end{lstlisting}

Vous n'aurez jamais besoin d'écrire à la main dans le fichier python généré par pyuic5. En effet, \textbf{ce fichier doit être re-généré à chaque fois que vous modifiez l'interface depuis QtDesigner}, si vous oubliez,  vous ne constaterez aucun changement.

\subsection{Améliorations}

Étant donné que le fait de régénérer le fichier .ui à chaque changement peut être pénible, nous avons rédigé un script python qui transforme tout les fichier QtDesigner présent dans un dossier en fichier python. \newline
Cette section sera abordée plus en détail par la suite.

\subsection{Installation}

La commande pour installer Pyuic5 est la suivante : \newline

\begin{lstlisting}
pip install pyqt5-tools
\end{lstlisting}
Il est possible d'installer la bibliothèque avec Anaconda : \newline
\begin{lstlisting}
conda install -c anaconda pyqt
\end{lstlisting}
