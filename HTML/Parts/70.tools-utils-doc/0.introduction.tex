\part{Mise en place du répertoire}

\chapter{Introduction}

\section{Présentation}
Ce document a pour but de présenter les fonctionnalités de la bibliothèque Utils, qui n'est qu'un 
regroupement de bibliothèques pour simplifier l'utilisation de Latex. \\

Cette documentation, dans un contexte plus général, présente la mise en place et l'utilisation d'un répertoire de travail Latex. \\
Voici les bibliothèques disponibles: 

\begin{items}{darkBlue}{\faviconLeaf}
\item Badges
\item Bibliography
\item Colors
\item Debug
\item Electronic
\item Figures
\item Fonts
\item Footnote
\item Glossaries
\item Graphics
\item Header
\item Images
\item Index
\item Items
\item Labels
\item Layout
\item Links
\item Lipsum
\item Maths
\item MessageBox
\item Nomenclature
\item Objects3D
\item Parts
\item Pdf
\item Programming
\item Quotes
\item TableOfContent
\item Tables
\item Theorems
\item Titles
\item Tree
\end{items}



