\chapter{Bibliothèque Tree}

La bibliothèque \lib{Tree} permet de générer des arborescences de dossiers et fichiers.

\section{Création d'une arborescence simple}

\dirtree{%
.1 Root1. 
.2 SubRoot1.
.3 SubSubRoot1.
.3 SubSubRoot2.
.3 SubSubRoot3.
.2 SubRoot2.
.2 SubRoot3.
} 

\loc{Body}
\begin{Latex}{Code pour la création d'un arbre simple}
\dirtree{%
.1 Root1. 
.2 SubRoot1.
.3 SubSubRoot1.
.3 SubSubRoot2.
.3 SubSubRoot3.
.2 SubRoot2.
.2 SubRoot3.
}
\end{Latex}


Le \bold{.Numéro} représente la profondeur de l'arborescence, c'est à dire le niveau. \\
Chaque fin de ligne doit impérativement se terminer par un point et la ligne \bold{\\dirtree\{} doit se terminer par un symbole '\%'.


\section{Création  d'une arborescence plus évoluée}

Avec l'utilisation de la bibliothèque \lib{Labels} \footnote{Se référer à la section \link{\titleName{Labels}}}, il est possible de mettre des couleurs très facilement.\\

\dirtree{%
.1 \dir{Root1}. 
.2 \dir{SubRoot1}.
.3 \file{SubSubRoot1}.
.3 \file{SubSubRoot2}.
.3 \file{SubSubRoot3}.
.2 \dir{SubRoot2}.
.2 \dir{SubRoot3}.
}


\loc{Body}
\begin{Latex}{Code pour la création d'une arborescence plus évoluée}
\dirtree{%
.1 \dir{Root1}. 
.2 \dir{SubRoot1}.
.3 \file{SubSubRoot1}.
.3 \file{SubSubRoot2}.
.3 \file{SubSubRoot3}.
.2 \dir{SubRoot2}.
.2 \dir{SubRoot3}.
}
\end{Latex}


\section{Création  d'une arborescence dans une figure}

Avec la bibliothèque \lib{Figures}, on peut intégrer une arborescence au sein d'une figure non centrée.\\
Actuellement, on ne peut pas centrer l'arbre sous peine de dégrader le rendu.


\createNoCenteredFigure{Exemple d'arborescence}{
    \dirtree{%
    .1 \dir{Root1}. 
    .2 \dir{SubRoot1}.
    .3 \file{SubSubRoot1}.
    .3 \file{SubSubRoot2}.
    .3 \file{SubSubRoot3}.
    .2 \dir{SubRoot2}.
    .2 \dir{SubRoot3}.
    }
}

\loc{Body}
\begin{Latex}{Code pour la création d'une arborescence dans une figure}
    \createNoCenteredFigure{Exemple d'arborescence}{
        \dirtree{%
        .1 \dir{Root1}. 
        .2 \dir{SubRoot1}.
        .3 \file{SubSubRoot1}.
        .3 \file{SubSubRoot2}.
        .3 \file{SubSubRoot3}.
        .2 \dir{SubRoot2}.
        .2 \dir{SubRoot3}.
        }
    }
\end{Latex}

%############################################################
%###### Package 'Trees' 
%###### This package contains some tools to generate trees
%###### Author  : Nicolas LE GUERROUE
%###### Contact : nicolasleguerroue@gmail.com
%############################################################

\RequirePackage{dirtree} %Tree
\typeout{>>> Utils: Package 'Tree' loaded !}
%###############################################################

\renewcommand*\DTstylecomment{\rmfamily\color{black}}
\renewcommand*\DTstyle{\ttfamily\textcolor{black}}





