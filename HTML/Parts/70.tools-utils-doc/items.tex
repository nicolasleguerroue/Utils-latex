\chapter{Bibliothèque Items} \label{Items}

La bibliothèque \lib{Items} permet de gérer les listes à puce.

\section{Création d'un liste}

\begin{items}{orange}{\Triangle}
    \item A
    \item B
    \item C
\end{items}

\loc{Body}
\begin{Latex}{Code pour la création d'une liste}
\begin{items}{orange}{\Triangle}
    \item A
    \item B
    \item C
\end{items}
\end{Latex}


\subsection{Options}

\begin{items}{orange}{\Triangle}
    \item Triangle
    \item Circle
    \item Bullet
    \item \faGear
    \item \faFileText
    \item ...
\end{items}

\begin{Latex}{Options disponibles}
\Triangle
\Bullet 
\Circle
\faGear
\faFileText
\end{Latex}

\messageBox{\faviconInfo}{green}{green}{Il est possible de personnaliser les puces en utilsant la bibliothèque \lib{Fonts} (Section \italic{\titleName{Fonts}})
Par exemple, on peut créer une liste avec le symbole \faFileText~et obtenir le résultat suivant :
\begin{items}{orange}{\faFileText}
    \item Settings
    \item Files
    \item Directory
\end{items}

}{white}

\begin{Latex}{Personnalisation des puces}
    \begin{items}{orange}{\faFileText}
        \item Settings
        \item Files
        \item Directory
    \end{items}
\end{Latex}


%############################################################
%###### Package 'name' 
%###### This package contains ...
%###### Author  : Nicolas LE GUERROUE
%###### Contact : nicolasleguerroue@gmail.com
%############################################################
\RequirePackage{enumitem}
\RequirePackage{pifont}
\typeout{>>> Utils: Package 'Items' loaded !}
%################################################################

%Use default list  newenvironment{itemize} #
%Use default numeroted list newenvironment{enumerate} #


\newcommand{\setBulletList}                  %Set default display by bullet in list #
{\renewcommand{\labelitemi}{$\bullet$}}

\newcommand{\Triangle}           %Display triangle in item list #
{$\blacktriangleright$}
\newcommand{\Circle}                          %Display circle in item list #
{$\circ$} 
\newcommand{\Bullet}                         %Display bullet in item list #
{$\bullet$}

\newenvironment{items}[2] %Create list (items) #color labelForm
{      
        \begin{itemize}[font=\color{#1}, label=#2]  
    }
    { 
        \end{itemize}
}

\newenvironment{List}[2] %Create list (List)#color labelForm
{      
        \begin{itemize}[font=\color{#1}, label=#2]  
    }
    { 
        \end{itemize}
}
