\chapter {Bibliothèque Theorems}

La bibliothèque \lib{Theorems} gère les théorèmes.\\

Les Théorèmes sous Latex sont des sections d'informations précises (remarque, question, réponse, propriété\ldots) mais dont la mise en page est transparente tout comme la numérotation.\\


\section{Utilisation avec les environnements}

Cette méthode n'est pas la plus conseillée, elle peut provoquer des erreurs d'affichage lors de la compilation Latex vers HTML.
Cependant, quand le paragraphe est conséquent, cette méthode est à privilégier par rapport à l'utilisation avec des commandes classiques.


Par exemple, on souhaite créer une question : 

\subsection{Création d'une question}

\begin{question}
    Quelle heure est-il ?
\end{question}

\loc{Body}
\begin{Latex}{Code pour la création d'une question}
\begin{question}
    Quelle heure est-il ?
\end{question}
\end{Latex}

L'ajout d'une deuxième question se fait en ajoutant le même code.

\begin{question}
    Quelle est ma deuxième question ?
\end{question}


\subsection{Création d'une réponse}

\begin{reponse}
    il est 18 h.
\end{reponse}

\loc{Body}
\begin{Latex}{Code pour la création d'une reponse}
\begin{reponse}
    il est 18 h.
\end{reponse}
\end{Latex}

\subsection{Création d'une propriété}

\begin{propriete}
    Un produit scalaire est commutatif.
\end{propriete}

\loc{Body}
\begin{Latex}{Code pour la création d'une propriete}
\begin{propriete}
    Un produit scalaire est commutatif.
\end{propriete}
\end{Latex}

\subsection{Création d'une proposition}

\begin{proposition}
    Les chats sont des mammifères.
\end{proposition}

\loc{Body}
\begin{Latex}{Code pour la création d'une proposition}
\begin{proposition}
    Les chats sont des mammifères.
\end{proposition}
\end{Latex}

\subsection{Création d'une remarque}

\begin{remarque}
remarque sur Latex
\end{remarque}

\loc{Body}
\begin{Latex}{Code pour la création d'une remarque}
\begin{remarque}
remarque sur Latex
\end{remarque}
\end{Latex}

\subsection{Création d'un exemple}

\begin{exemple}
    Ceci est un exemple d'exemple
\end{exemple}

\loc{Body}
\begin{Latex}{Code pour la création d'une exemple}
\begin{exemple}
    Ceci est un exemple d'exemple
\end{exemple}
\end{Latex}

\subsection{Création d'une définition}

\begin{definition}
    Une phrase est un ensemble de mots.
\end{definition}

\loc{Body}
\begin{Latex}{Code pour la création d'une definition}
\begin{definition}
    Une phrase est un ensemble de mots.
\end{definition}
\end{Latex}


\subsection{Création d'une solution}

\begin{solution}
   La solution est triviale
\end{solution}

\loc{Body}
\begin{Latex}{Code pour la création d'une solution}
\begin{solution}
    La solution est triviale
\end{solution}
\end{Latex}



%############################################################
%###### Package 'Theorems' 
%###### This package contains some tools to set theorems
%###### Author  : Nicolas LE GUERROUE
%###### Contact : nicolasleguerroue@gmail.com
%############################################################
\RequirePackage{amssymb}     %math
\RequirePackage{amsthm}     %math
\typeout{>>> Utils: Package 'Theorems' loaded !}
%################################################################
%###############################################################
\newtheorem{question}{Question} %new question #content
\newtheorem{reponse}{$>>>$} %new answer #content
\newtheorem{propriete}{Propriété} %new property #content
\newtheorem{proposition}{Proposition} %new preposition #content
\newtheorem{remarque}{Remarque} %new remarque #content
\newtheorem{exemple}{Exemple} %new example #content
\newtheorem{definition}{Definition} %new definition #content
\newtheorem{solution}{Solution} %new definition #content

\newcommand{\Question}[1]{  %Add question #question
    \begin{question} #1 \end{question}
}

\newcommand{\Reponse}[1]{  %Add answer #question
    \begin{reponse} #1 \end{reponse}
}