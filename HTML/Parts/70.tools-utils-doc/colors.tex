\chapter{Bibliothèque Colors} \label{colors}

La bibliothèque \lib{Colors} gère les couleurs dans le document.

\section{Afficher du contenu en couleur}

\colors{blue}{Ceci est du texte en bleu}

\colors{red}{Ceci est du texte en rouge}

\colors{green}{Ceci est du texte en vert}\\


\loc{Body}
\begin{Latex}{Code pour l'affichage en couleur}
\colors{blue}{Ceci est du texte en bleu}
\colors{red}{Ceci est du texte en rouge}
\colors{green}{Ceci est du texte en vert}
\end{Latex}


\section{Liste des couleurs disponibles}


\begin{items}{white}{\Bullet}
    \item \colors{red}{red}
    \item \colors{green}{green}
    \item \colors{blue}{blue}
    \item \colors{orange}{orange}
    \item \colors{yellow}{yellow}
    \item \colors{gray}{gray}
    \item \colors{brown}{brown}
    \item \colors{cyan}{cyan}
    \item \colors{black}{black}
    \item \colors{purple}{purple}
    \item \colors{magenta}{magenta}


    \item \colors{rose}{rose}
    \item \colors{darkBlue}{darkBlue}
    \item \colors{darkBrown}{darkBrown}
    \item \colors{darkRed}{darkRed}
    \item \colors{darkOrange}{darkOrange}
    \item \colors{darkGray}{darkGray}

\end{items}


\section{Création de couleurs personnalisées}

Les nouvelles couleurs doivent être placées dans le fichier \file{Make/Colors.tex} avec la syntaxe suivante : 
\loc{Make/Colors.tex}

\begin{Latex}{Une couleur personnalisée}
\definecolor{colorName}{RGB}{valueRED, valueGREEN, valueBLUE}
\end{Latex}

Où 
\begin{items}{blue}{\Bullet}
    \item valueRED représente la couleur \colors{red}{Rouge} sur un niveau compris entre 0 et 255
    \item valueGREEN représente la couleur \colors{green}{Verte} sur un niveau compris entre 0 et 255
    \item valueBLUE représente la couleur  \colors{blue}{Bleue} sur un niveau compris entre 0 et 255
\end{items}


%############################################################
%###### Package 'Color' 
%###### This package contains some colors
%###### Author  : Nicolas LE GUERROUE
%###### Contact : nicolasleguerroue@gmail.com
%############################################################
\RequirePackage{color}              %Add colors
%############################################################
\typeout{>>> Utils: Package 'Colors' loaded !}
%################################################################

\IfFileExists{Make/Colors.tex}
{
    %Blue
\definecolor{darkBlue}{HTML}{034f84}
\definecolor{lightBlue}{HTML}{92a8d1}
\definecolor{navy}{HTML}{12305f}
\definecolor{sky}{HTML}{00aacf}

%Green
\definecolor{green}{HTML}{007c2e}
\definecolor{lemon}{HTML}{d5f4e6}

\definecolor{rose}{RGB}{100,100,100}
\definecolor{darkOrange}{RGB}{100,100,100}
\definecolor{darkRed}{RGB}{100,100,100}
\definecolor{darkBrown}{RGB}{100,100,100}

\definecolor{darkOrange}{RGB}{100,100,100}



%Gray
\definecolor{lightGray}{HTML}{878f99}
\definecolor{darkGray}{HTML}{222831}



}
{
    \raiseWarning{Colors file 'Make/Colors.tex' no loaded}
}

%Default command to set content color newcommand{color}[2]  #color content
\newcommand{\colors}[2]{%Display content in color #color content
{
    \color{#1}{#2}}
}
