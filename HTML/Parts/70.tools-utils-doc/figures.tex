\chapter{Bibliothèque Figures}

La bibliothèque \lib{Figures} permet de créer des figures et d'afficher la liste des figures dans le document

\section{Création d'une figure centrée}

\createFigure{Figure en exemple}{Ceci est un contenu de figure}

\loc{Body}
\begin{Latex}{Code pour la création de figure}
\createFigure{Figure en exemple}{Ceci est un contenu de figure}
\end{Latex}



\section{Création d'une figure non centrée}

\createFigure{Figure en exemple}{Ceci est un contenu de figure non centrée}

\loc{Body}
\begin{Latex}{Code pour la création de figure non centrée}
\createNoCenteredFigure{Figure en exemple}{Ceci est un contenu de figure}
\end{Latex}

\section{Affichage de la liste des figures}

\loc{main.tex}
\begin{Latex}{Code pour l'affichage de la liste des figures}
\displayListOfFigures{Liste des figures}
\end{Latex}



\section{Références des figures}

Il est possible de faire référence à une figure si celle-ci est explicité avec la commande \bold{createFigureref}.\\

% \createFigureRef{Figure en exemple}{Ceci est un contenu de figure}{ref}
% Ceci est une référence à la \figureName{ref}.\\

\loc{Body}
\begin{Latex}{Code pour la référence d'une figure}
  Ceci est une référence à la figure \figureName{Fonts}.
\end{Latex}

%############################################################
%###### Package 'Figures' 
%###### This package contains some tools to display Figures (listoffigures)
%###### Author  : Nicolas LE GUERROUE
%###### Contact : nicolasleguerroue@gmail.com
%############################################################
\typeout{>>> Utils: Package 'Figures' loaded !}
\RequirePackage{graphics}
\RequirePackage{graphicx}
\RequirePackage{nameref}
\RequirePackage{caption}
\RequirePackage{subcaption}

\newcommand{\displayListOfFigures}[1]{%Display the listOfFigures #tocName
    \renewcommand{\listfigurename}{#1}
    \addcontentsline{toc}{chapter}{#1}
    \vspace{-4cm}
    \listoffigures
}

\newcommand{\createFigure}[2]  %Create new figure #name content
{
    \begin{figure}[h]
        \centering
    \raiseMessage{Creating new figure ['#1']}
        {#2}
        \caption{#1}
    \end{figure}
}

\newcommand{\titleFigure}{default}
\newcommand{\setTitleFigure}[1]{
    \renewcommand{\titleFigure}{#1}
}
\newenvironment{Figure}[2]  %create Figure #title ref
{
    
    \begin{figure}[h]
        \label{AZE}
        \begin{center}
            \setTitleFigure{#1}
            \raiseMessage{Creating new figure ['\titleFigure']}
}
{
        \caption{\titleFigure}
        \end{center} 
    \end{figure}  
}


\newcommand{\createFigureRef}[3]  %Create new figure with ref #name #content #reference
{
    \begin{figure}[h]
        \label{#3}
        \centering
    \raiseMessage{Creating new figure ['#1']}
        {#2}
        \caption{#1}
    \end{figure}
}

\newcommand{\createNoCenteredFigure}[2]  %Create new figure without centering #name #content
{
    \begin{figure}[h]
    \raiseMessage{Creating new figure ['#1']}
        {#2}
        \caption{#1}
    \end{figure}
}


\newcommand{\figureRef}[1] %Display number of fugure with '(Figure X.X)' format #figureReference
{(Figure \ref{#1})} 
\newcommand{\figureName}[1]   %Display number of fugure with 'figure X.X' format #figureReference
{figure \ref{#1}} 