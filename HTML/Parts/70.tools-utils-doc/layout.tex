\chapter{Bibliothèque Layout}

La bibliothèque \lib{Layout} gère la page de garde et la mise en forme du text (gras, italique...).

La mise en page est séparée en 4 parties : frontmatter (début du document, numéroation romaine), mainmatter (avant le premier chapitre), appendix (annexes) et backmatter avant les tables et bibliographies

\section{Mise en gras}

\bold{Texte en gras}
\sn

\loc{Body}
\begin{Latex}{Code pour mettre le texte en gras}
\bold{Texte en gras}
\end{Latex}

\section{Mise en italique}

\italic{Texte en italique}
\sn
\loc{Body}
\begin{Latex}{Code pour mettre le texte en italique}
\italic{Texte en gras}
\end{Latex}

\section{Mise en gras et italique}

\ib{Texte en gras et italique}
\sn

\loc{Body}
\begin{Latex}{Code pour mettre le texte en gras et italique}
\ib{Texte en gras et italique}
\end{Latex}


\section{Ajout d'un espace vertical}

Lorem ipsum dolor sit amet, consectetuer adipiscing elit. Ut purus elit, vestibulum
ut, placerat ac, adipiscing vitae, felis. Curabitur dictum gravida mauris. Nam arcu li- \sn

bero, nonummy eget, consectetuer id, vulputate a, magna. Donec vehicula augue eu
neque. Pellentesque habitant morbi tristique senectus et netus et malesuada fames ac
turpis egestas. Mauris ut leo. Cras viverra metus rhoncus sem. Nulla et lectus vestibu-

\loc{Body}
\begin{Latex}{Code pour ajouter un espace vertical}
\sn
\end{Latex}

%############################################################
%###### Package 'Layout' 
%###### This package contains some tools to set page layout or text
%###### Author  : Nicolas LE GUERROUE
%###### Contact : nicolasleguerroue@gmail.com
%############################################################
\RequirePackage{lmodern}

\RequirePackage{xcolor}             %define new colors
\RequirePackage{xparse}
\RequirePackage{amssymb}     %math
\RequirePackage{amsthm}     %math
\typeout{>>> Utils: Package 'Layout' loaded !}

%################################################################
\begin{comment}
@begin
@command \bold
@des 
Cette commande permet de mettre le texte en gras
@sed
@input Texte à mettre en gras
@begin_example 
\bold{texte en gras}
@end_example
@end
\end{comment}

\renewcommand{\bold}[1]{%Set text in bold #content
    \textbf{#1}
}

%################################################################
\begin{comment}
@begin
@command \italic
@des 
Cette commande permet de mettre le texte en italique
@sed
@input Texte à mettre en italique
@begin_example 
\italic{texte en italique}
@end_example
@end
\end{comment}

\newcommand{\italic}[1]{%Set text in italic #content
    \textit{#1}
}

%################################################################
\begin{comment}
@begin
@command \ib
@des 
Cette commande permet de mettre le texte en italique et en gras
@sed
@input Texte à mettre en italique et gras
@begin_example 
\ib{texte en italique et gras}
@end_example
@end
\end{comment}

\newcommand{\ib}[1]{%Set text in italic and bold #content
    \textit{\textbf{#1}}
}

%################################################################
\begin{comment}
@begin
@command \bi
@des 
Cette commande permet de mettre le texte en italique et en gras
@sed
@input Texte à mettre en italique et gras
@begin_example 
\bi{texte en italique et gras}
@end_example
@end
\end{comment}

\newcommand{\bi}[1]{%Set text in italic and bold #content
    \textit{\textbf{#1}}
}


\begin{comment}
@begin
@command \n
@des 
Cette commande permet de faire un saut de ligne
@sed
@begin_example 
Text \n Next text
@end_example
@end
\end{comment}

\newcommand{\n}{\\}%Add newline #

%################################################################
\begin{comment}
@begin
@command \sn
@des 
Cette commande permet de faire un espace vertical
@sed
@begin_example 
Text \sn Next text
@end_example
@end
\end{comment}

\newcommand{\sn}{\vskip 0.5cm} %Set small vertical space #

%################################################################

%#######################################
