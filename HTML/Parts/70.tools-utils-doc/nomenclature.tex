\chapter{Bibliothèque Nomenclature}
\label{addNomenclature}

La Bibliothèque \lib{Nomenclature} permet de générer des nomenclatures.

\section{Ajout d'un élément dans la nomenclature}

Chaque élément est de la forme :\\

\loc{Make/Nomenclature.tex}
\begin{Latex}{Format pour la nomenclature}
\nomenclature[category]{$symbole$}{définition}

%Par exemple
\nomenclature[E]{$r$}{Rapport cyclique d'un signal périodique}
\end{Latex}
et doit être rajouté dans le fichier \dir{Make}/\file{Nomenclature.tex}\\


Dans la commande, \bold{category} indique le type de grandeur. Il y a actuellement 6 types de grandeurs définies dans le fichier \dir{Utils}/\file{Nomenclature.sty}\footnote{Cependant, rien ne vous empêche d'en ajouter en modifiant la bibliothèque}: 

\begin{items}{blue}{\Bullet}
\item P pour les \bold{Constantes physiques}
\item O pour les \bold{Autres symboles}
\item N pour les \bold{Nombres spéciaux}
\item A pour les \bold{Amplificateurs Opérationnels}
\item M pour les \bold{Mécanique}
\item E pour les \bold{Électronique}
\end{items}  


\subsection{Ajout des unités}

Pour ajouter une unité, il suffit d'invoquer la commande \lib{addUnit} à la fin du dernier argument de la commande \lib{nomenclature}.

\subsection{Ajout d'une unité}

\loc{Make/Nomenclature.tex}
\begin{Latex}{Ajout d'une unité}
\nomenclature[A]{$\varepsilon$}{Tension différentielle $(\varepsilon = E_+ - E_-)$\addUnit{V}}
\end{Latex}
    

\subsection{Exemple de nomenclature}

La nomenclature est dans le fichier \dir{Make}/\file{Nomenclature.tex}.\\

\loc{Make/Nomenclature.tex}
\begin{Latex}{Exemple de nomenclature}

\nomenclature[E]{$r$}{Rapport cyclique d'un signal périodique}
\nomenclature[A]{$A_d$}{Coefficient d'amplification, gain différentiel }
\nomenclature[A]{$\varepsilon$}{Tension différentielle $(\varepsilon = E_+ - E_-)$\addUnit{V}}
\nomenclature[A]{$E_+$}{Tension entrée non inverseuse \addUnit{V}}
\nomenclature[A]{$E_-$}{Tension entrée inverseuse \addUnit{V}}
\nomenclature[E]{$\eta$}{Rendement d'un mécanisme \addUnit{\%}}
\nomenclature[E]{$\varphi$}{Déphasage entre deux signaux \addUnit{rad}}
  
\end{Latex}



\section{Affichage de la nomenclature}

Pour afficher la nomenclature, il suffit de faire la commande : \\

\loc{main.tex}
\begin{Latex}{Code pour l'affichage de la nomenclature}
\displayNomenclature{Nomenclature}{Make/Nomenclature.tex}
\end{Latex}

\img{\rootImages/nomenclature.png}{Rendu d'une nomenclature}{0.5}


%############################################################
%###### Package 'Nomenclature' 
%###### This package contains some tools to set nomenclature
%###### Author  : Nicolas LE GUERROUE
%###### Contact : nicolasleguerroue@gmail.com
%############################################################
\RequirePackage{nomencl}  %nomenclature
%################################################################
\typeout{>>> Utils: Package 'Nomenclature' loaded !}

\makenomenclature

%##### Convention [Nomenclature]
  \renewcommand{\nomgroup}[1]{%Set nomenclature #nomenclatureName
  \item[\bfseries
  \ifthenelse{\equal{#1}{P}}{Constantes physiques}{%
  \ifthenelse{\equal{#1}{O}}{Autres symboles}{%
  \ifthenelse{\equal{#1}{N}}{Nombres spéciaux}{ 
  \ifthenelse{\equal{#1}{A}}{Amplificateurs Opérationnels}{ 
  \ifthenelse{\equal{#1}{M}}{Mécanique}{ 
  \ifthenelse{\equal{#1}{E}}{Électronique}{}}}}}}%
  ]}


  % Add unit on convention [nomenclature]
%----------------------------------------------
\newcommand{\addUnit}[1]{%Add unit in nomenclature #unit
\renewcommand{\nomentryend}{\hspace*{\fill}#1}}
%----------------------------------------------


\newcommand{\displayNomenclature}[2]{ %Display the nomenclature #tocName #filename
    \input{#2}
    \printnomenclature
    \addcontentsline{toc}{chapter}{\nomname}
}
