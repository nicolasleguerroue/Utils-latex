\chapter{Bibliothèque Graphic}

La bibliothèque \lib{Graphic} permet de générer des graphiques depuis diverses sources de données.

\section{Affichage d'un graphique 2D avec insertion des données depuis un fichier txt (csv)}

\begin{graphic}{0.8}{0.6}{0}{2.1}{-1.1}{1.1}{t(ms)}{vs}{Oscilloscope}
\addPointsFromCSV{red}{comma}{src_examples/input_1.txt}
\addLegend{voie A}
\end{graphic}

\loc{Body}
\begin{Latex}{Code pour l'affichage d'un graphique 2D avec insertion des données depuis un fichier txt (csv)}
\begin{graphic}{0.8}{0.6}{0}{2.1}{-1.1}{1.1}{t(ms)}{vs}{Oscilloscope}
  \addPointsFromCSV{red}{comma}{src_examples/input_1.txt}
  \addLegend{voie A}
\end{graphic}
\end{Latex}



\section{Affichage d'un graphique 2D avec insertion des données depuis une liste de points}

\begin{graphic}{0.8}{0.4}{0}{40}{-1}{6}{t(s)}{Tension (V)}{Signal numérique}
\addPoints{blue}{(0,0)(10,0)(10,5)(15,5)(15,0)(20,0)(20,5)(25,5)(25,0)(30,0)(30,5)(35,5)(35,0)(100,0)}
\addLegend{Tension (V)}
\end{graphic}

\loc{Body}
\begin{Latex}{Code pour l'affichage d'un graphique 2D avec insertion des données depuis une liste de points}
\begin{graphic}{0.8}{0.4}{0}{40}{-1}{6}{t(s)}{Tension (V)}{Signal numérique}
  \addPoints{blue}{(0,0)(10,0)(10,5)(15,5)(15,0)(20,0)(20,5)(25,5)(25,0)(30,0)(30,5)(35,5)(35,0)(100,0)}
  \addLegend{Tension (V)}
\end{graphic}
\end{Latex}


\section{Affichage d'un graphique 2D avec insertion des données depuis une équation}

La variable est $x$ est les fonction trigonométriques sont en degrés.\\

\setSamples{1000}
\setMarker{markerNone}
\setFillColor{darkBlue}
\begin{graphicFigure}{0.4}{0.4}{0}{3}{-1}{5}{t(s)}{Tension V}{Signal analogique}

  
  \addTrace{green}{-10}{10}{sin(360*x)}
  \addTrace{blue}{-10}{10}{2*sin(720*x)}
  \addLegend{sin(360*x)}
\end{graphicFigure}
\setDefaultFillColor
%\enableGrid

\loc{Body}
\begin{Latex}{Code pour l'affichage d'un graphique 2D avec insertion des données depuis une équation}
  \setSamples{1000}
  \setMarker{markerNone}
  \begin{graphicFigure}{0.4}{0.4}{0}{3}{-1}{5}{t(s)}{Tension V}{Signal analogique}
    \addTrace{green}{-10}{10}{sin(x)}
  \addLegend{Legende}
  \end{graphicFigure}
\end{Latex}

\newpage
\section{Affichage de deux graphiques}

  \begin{figure}[h!]  
    \centering 
      \begin{subfigure}[b]{0.4\linewidth}
        \begin{graphic}{0.8}{1}{0}{1.2}{-1}{5}{t(s)}{Tension V}{h}
          \addTrace{green}{-10}{10}{sin(x)/x}
          \addLegend{sin(t)}
          \end{graphic}%NO END  LINE HERE
        \caption{Origine} 
      \end{subfigure}
    \begin{subfigure}[b]{0.4\linewidth}
      \begin{graphic}{0.8}{1}{0}{4}{-0.3}{0.3}{t(s)}{Tension V}{g}
        \addPointsFromCSV{blue}{comma}{src_examples/jack01.txt}
        \addLegend{g(t)}
        \end{graphic}%NO END  LINE HERE
    \caption{Bruit}
    \end{subfigure}
    \caption{Les tensions de service}
    \end{figure}  


  
    \loc{Body}
    \begin{Latex}{Code pour l'affichage d'un graphique 2D avec insertion des données depuis plusieurs sources}

      \begin{figure}[h!]  
        \centering 
          \begin{subfigure}[b]{0.4\linewidth}
            \begin{graphic}{0.8}{1}{0}{1.2}{-1}{5}{t(s)}{Tension V}{h}
              \addTrace{green}{-10}{10}{sin(x)/x}
              \addLegend{sin(t)}
              \end{graphic}%NO END  LINE HERE
            \caption{Origine} 
          \end{subfigure}
        \begin{subfigure}[b]{0.4\linewidth}
          \begin{graphic}{0.8}{1}{0}{4}{-0.3}{0.3}{t(s)}{Tension V}{g}
            \addPointsFromCSV{blue}{comma}{src_examples/jack01.txt}
            \addLegend{g(t)}
            \end{graphic}%NO END  LINE HERE
        \caption{Bruit}
        \end{subfigure}
        \caption{Les tensions de service}
        \end{figure}  
    \end{Latex}

  

\section{Paramètres des graphiques}

\subsection{Affichage de la grille}
Il est possible d'afficher ou non la grille avec les directives suivantes : 

\loc{Body}
\begin{Latex}{Affichage de la grille}
\enableGrid
\disableGrid
\end{Latex}


\subsection{Résolution des tracés}
Il est possible de modifier le nombre de points affichés pour une courbe :\\

\loc{Body}
\begin{Latex}{Définition du nombre d'échantillons}
\setSamples{1000}
\end{Latex}

\subsection{Modification des marqueurs}

Il est possible de changer les marqueurs des points avec la directive suivante : 


\loc{Body}
\begin{Latex}{Changement des marqueurs}
\setMarker{\markerCircle}
\end{Latex}

Les marqueurs disponibles sont les suivants : \\

\loc{Body}
\begin{Latex}{Liste des marqueurs}
\markerNone
\markerCircle
\markerSquare
\markerTriangle
\markerDiamond
\markerStart
\end{Latex}

\subsection{Modification de la couleur de remplissage}

Il est possible de mettre une couleur de remplissage des graphiques :

\loc{Body}
\begin{Latex}{Modifications de la couleur de remplissage}
\setFillColor{red}
\end{Latex}

On peut restaurer la couleur par défaut avec la directive : \\

\loc{Body}
\begin{Latex}{Restauration de la couleur d'origine}
\setDefaultFillColor
\end{Latex}

\subsection{Modification de l'épaisseur des lignes}

Il est possible de changer l'épaisseur des lignes (en mm) avec la commande suivante :

\loc{Body}
\begin{Latex}{Modifications de l'épaisseur des lignes}
\setLineWidth{0.2}
\end{Latex}

%############################################################
%###### Package 'Graphics' 
%###### This package contains some tools to create graphics 2D or 3D
%###### Author  : Nicolas LE GUERROUE
%###### Contact : nicolasleguerroue@gmail.com
%############################################################
%\ProvidesPackage{Utils}[2013/01/13 Utils Package]
%############################################################
\RequirePackage{csvsimple} 
\RequirePackage{tikz}  
\RequirePackage{pgfplots}  
\RequirePackage{pgf}  
\RequirePackage{version}            %use commented code

\RequirePackage{graphics} 
\RequirePackage{graphicx} 
\RequirePackage{caption}
\RequirePackage{subcaption} 
\RequirePackage{version}      
\typeout{>>> Utils: Package 'Graphics' loaded !}

\pgfplotsset{compat=1.7}
%###### Checking if babel is loaded
\makeatletter
\@ifpackageloaded{babel}
{% if the package was loaded
\newcommand{\setGraphic}{%Internal command to display graphics #
    \shorthandoff{:;!?}} 
\frenchbsetup{StandardLists=true} %to include if using \RequirePackage[french]{babel} -> rounded list
}
{%else:
\newcommand{\setGraphic}%Internal command to display graphics __internal__2 #
{} 
}


\makeatother
%############################################################
%### WARNING : USE \shorthandoff{:;!?} before \begin{tikzpicture} 
%### environment
%############################################################
%grid
\newcommand{\grid}{grid=both}%Value of grid #
\newcommand{\enableGrid}{    %enable grid on graphic #
    \renewcommand{\grid}{grid=both}
}
\newcommand{\disableGrid}{    %disable grid on graphic #
    \renewcommand{\grid}{}
}

%Samples
\newcommand{\samples}{400}  %Value of samples #
\newcommand{\setSamples}[1]{        %Set samples count
    \renewcommand{\samples}{#1}
}

%Color
\newcommand{\defaultFillColor}{none}  
\newcommand{\fillColor}{\defaultFillColor}  
\newcommand{\setDefaultFillColor}{        %Set default fill color
    \renewcommand{\fillColor}{\defaultFillColor}
}
\newcommand{\setFillColor}[1]{        %Set fill color #color
    \renewcommand{\fillColor}{#1}
}

%LineWidth
\newcommand{\defaultLineWidth}{0.1}  
\newcommand{\lineWidth}{\defaultLineWidth}  
\newcommand{\setDefaultLineWidth}{        %Set default fill color
    \renewcommand{\lineWidth}{\defaultLineWidth}
}
\newcommand{\setLineWidth}[1]{        %Set fill color #color
    \renewcommand{\lineWidth}{#1}
}

%marker
\newcommand{\markerNone}{otimes}  %Any marker #
\newcommand{\markerCircle}{otimes}  %Circle marker #
\newcommand{\markerSquare}{square}  %Square marker #
\newcommand{\markerTriangle}{triangle}  %Triangle marker #
\newcommand{\markerDiamond}{diamond}  %Diamond marker #
\newcommand{\markerStart}{star}  %Star marker #
\newcommand{\marker}{otimes} 
\newcommand{\setMarker}[1]{        %Set graphic marker #marker
    \renewcommand{\marker}{#1}
}


\newenvironment{graphicFigure}[9]  %Display graphe inside figure #widht height minAbs maxAbs minOrd maxOrd xLegend yLegend title
{
    \raiseMessage{Creating new graphic figure [title='#9']}
    \begin{center}
        \makeatletter
        \def\@captype{figure}
        \makeatother

        \newcommand{\TitleGraphic}%Internal command to display title graphic (Figure)# 
    {#9} 
        \begin{tikzpicture}
        \setGraphic %command to display with frenchb babel
    \begin{axis}[width=#1\linewidth,height=#2\linewidth,xmin=#3,xmax=#4,  ymin=#5, ymax=#6, scale only axis,xlabel=#7,ylabel=#8,\grid, legend cell align={left}] %grid=both
    }
    { 
    \end{axis}
        \end{tikzpicture}
        \captionof{figure}{\TitleGraphic}
    \end{center}
}


\newenvironment{graphic}[9]  %Display graphe without figure #widht height minAbs maxAbs minOrd maxOrd xLegend yLegend title
{
    \raiseMessage{Creating new graphic [title='#9']}
    \newcommand{\TitleGraphic}%Internal command to display title graphic (No figure) # 
    {#9} 
        \begin{tikzpicture}
        \setGraphic %command to display with frenchb babel
    \begin{axis}[width=#1\linewidth,height=#2\linewidth,xmin=#3,xmax=#4,  ymin=#5, ymax=#6, scale only axis,xlabel=#7,ylabel=#8, title=#9, \grid , legend cell align={left}] %grid=both
    }
    { 
    \end{axis}
        \end{tikzpicture}

}
    
%############################################################
\newcommand{\addPoints}[2]{ %Add points in graph #color coords
    \addplot+[thick,mark=\marker, color=#1, fill=\fillColor, line width=\lineWidth mm] coordinates{#2};
}

%############################################################
\newcommand{\addTrace}[4]{ %Display equation #color xBegin xEnd equation
    \raiseMessage{Creating new trace [equ='#4']}
    \addplot [#1, domain=#2:#3, mark=\marker, samples=\samples, fill=\fillColor, line width=\lineWidth mm] {#4};
}

%############################################################
\newcommand{\addPointsFromCSV}[3]{%Add points on graphe from CSV #color delimitor filename
    \IfFileExists{#3}{
    \addplot+[thick, mark=none, color=#1, fill=\fillColor, line width=\lineWidth mm] table[mark=\marker ,col sep=#2] {#3};
    \raiseMessage{File '#3' loaded !}
    }
    {\raiseError{[import failed]'#3' \stop}
    }
}

%############################################################
\newcommand{\addLegend}[1]{ %Add legend on graphe #legend
    \legend{#1}
}

%############################################################
\newenvironment{graphic3D}[1]  %Display graphe without figure (3D) #title
{
    \raiseMessage{Creating new 3D graphic [title='#1']}
    \newcommand{\TitleGraphic}%Internal command to display title graphic # 
    {#1} 
        \setGraphic %command to display with frenchb babel
\begin{tikzpicture}
    \begin{axis}[
      title=#1,
      hide axis,
      colormap/cool]
    }
    { 
    \end{axis}
        \end{tikzpicture}

}
