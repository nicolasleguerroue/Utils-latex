\chapter{Bibliothèque Glossaries}
\label{addDef}
La bibliothèque \lib{Glossaries} permet de manipuler un glossaire


\section{Ajout d'une définition au glossaire}

\loc{Make/Glossaries.tex}
\begin{Latex}{Ajout d'une définition au glossaire}
\newglossaryentry{Definiton}{name=Identifiant,description={Description}}
\end{Latex}

\section{Référence au glossaire}

Pour faire référence au glossaire (pour qu'il soit affiché au glossaire), il faut saisir la commande suivante : \\
\loc{Body}
\begin{Latex}{Référence d'un terme au glossaire}
\glossary{Identifiant}
\end{Latex}

\messageBox{\faviconInfo}{green}{green}{Toute référence au glossaire provoquera un affichage de la définition en note de bas de page}{white}

\section{Affichage du glossaire}

Pour afficher le glossaire située dans le dossier \dir{Make}/\file{Glossaries.tex}, il faut saisir la commande suivante : \\

\loc{main.tex}
\begin{Latex}{Code pour l'affichage du glossaire}
\displayGlossaries{Glossaire}
\end{Latex}

\subsection{Rendu du glossaire}
\img{\rootImages/glossaire.png}{Un rendu du glosssaire}{0.6}

%############################################################
%###### Package 'Glossaries' 
%###### This package contains some tools to set glossaries
%###### Author  : Nicolas LE GUERROUE
%###### Contact : nicolasleguerroue@gmail.com
%############################################################
\RequirePackage[xindy, acronym, nomain, toc,nonumberlist]{glossaries}
%################################################################
\typeout{>>> Utils: Package 'Glossaries' loaded !}

% Default command for glossary (renewcommand) {gls}[1] #content

\renewcommand{\glossary}[1]{%display glossarie #filename
\addtocounter{glossaryCounter}{1} 
\gls{#1}
}

\newcounter{glossaryCounter}  %Create new counter
\setcounter{glossaryCounter}{0}

\newcommand{\displayGlossaries}[1]{%Display the glossaries #tocName
    \ifnum \value{glossaryCounter}>0
    {
        \renewcommand {\acronymname}{#1}
        \printnoidxglossaries
        \printglossaries
        \typeout{>>> Utils: Glossary loaded !}
        \addcontentsline{toc}{chapter}{#1}
    } 
    \else 
    {
        \typeout{>>> Utils: Glossary no loaded !}
    }
    \fi
}

\newcommand{\addGlossaryInput}[3]{  %Add glossary input #name description meaning
    \newglossaryentry{#1}
    {
        name=#1,
        description={\italic{#2} \\#3}
    }
}

