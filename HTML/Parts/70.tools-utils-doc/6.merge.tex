\chapter{Fusion de projets}

Le choix d'un dossier par partie (Parts/XXX) permet de fusionner très facilement des projets.\\

Pour fusionner deux projets, il suffit de copier-coller le contenu du dossier \dir{Images} et \dir{Parts} du projet A 
dans le dossier de projet qui contiendra la fusion (projet B). Lors de la compilation, \lib{make} va gérer la 
fusion automatiquement.


\chapter{Ajout de bibliothèques personnelles}


Après avoir extrait les 2 dossiers, lancer dans chaque dossier la commande
\begin{Bash}{Création du fichier .sty}
pdflatex *.ins
\end{Bash}

Puis remonter dans le dossier contenant les 2 dossier puis
\begin{Bash}{Création du fichier .sty}
sudo cp -r acrotex /usr/share/texmf/tex/latex
sudo cp -r eq-save /usr/share/texmf/tex/latex
cd /usr/share/texmf/tex/latex
sudo texhash
\end{Bash}
    
