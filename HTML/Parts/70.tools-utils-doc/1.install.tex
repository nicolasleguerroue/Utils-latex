\chapter{Installation}

\section{Installation des outils}


La bibliothèque \lib{Utils-latex} regroupe les outils pour installer et compiler un répertoire de travail Latex.
\messageBox{\faviconInfo}{green}{green}{L'ensemble des outils présentés est optimisé pour une utilisation avec le logiciel Visual Studio Code}{black}

Il faut donc saisir la commande suivante dans le dossier où l'on souhaite créer le répertoire de travail : 

\begin{Bash}{Clone de la bibliothèque et démarrage de l'installation}
git clone https://github.com/nicolasleguerroue/Utils-latex.git && Utils-latex/tools/init
\end{Bash}

On vous demande ensuite si vous souhaitez mettre à jour le système.
\img{\rootImages/acceptInstall.png}{Installation des paquets}{0.7}
Répondre \bold{y} est fortement conseillé. A ce moment là, l'ensemble des logiciels et paquets nécessaires vont être installés.\\
Voici les paquets installés : \\
\begin{items}{blue}{\faviconLeaf}
\item texlive-full
\item git
\item texlive-lang-european
\item okular
\item gnuplot
\end{items}

Enfin, on vous demande si vous souhaitez installer les extensions VScode, encore une fois, l'acceptation est vivement recommandée.
\img{\rootImages/installExtensions.png}{Installation des extensions VSCode}{0.7}

Le logiciel VSCode se lance une fois que l'installation est terminée.


\section{Configuration des tâches}

Lors de l'ouverture du répertoire de compilation, un script de vérification va se lancer automatiquement.
Pour ce faire, il va falloir activer le lancement des tâches.\\ 
Il faut faire le raccourci clavier \shortcut{CTRL+SHIFT+P} et saisir \bold{Manage Automatic tasks in Folder} 
\img{\rootImages/manage.png}{Lancement des tâches au démarrage}{0.5}

Puis valider l'action avec \bold{Manage Automatic tasks in Folder}

\img{\rootImages/allowManager.png}{Activation des tâches au démarrage}{0.5}

Ce script exécuté au démarrage permet de vérifier l'intégrité des fichiers (paramètres, outils) avant de lancer une compilation.