\chapter{Architecture}

\section{Organisation du projet}

\createNoCenteredFigure{Arborescence du projet}{
\dirtree{%
.1 \dir{Projet}. 
.2 \dir{Images}.
.3 \dir{Intro}.
.3 \dir{Content}.
.3 \dir{OtherDirectory}.
.2 \dir{Make}.
.3 \file{Alias.tex}.
.3 \file{Bibliography.bib}.
.3 \file{Colors.tex}.
.3 \file{Contacts.tex}.
.3 \file{Glossaries.tex}.
.3 \file{Index.tex}.
.3 \file{Nomenclature.tex}.
.3 \file{Rules.tex}.
.3 \file{Versions.tex}.
.2 \dir{Output}.
.2 \dir{Parts}.
.3 \dir{0.Intro}.
.3 \dir{1.Content}.
.3 \dir{2.OtherDirectory}.
.2 \dir{Settings}.
.2 \sym{Utils}.
.2 \dir{Utils-latex}.
.3 \dir{bash}.
.3 \dir{HTML}.
.3 \dir{tools}.
.3 \dir{Utils}.
.3 \dir{.vscode}.
.3 \file{make}.
.2 \sym{.vscode}.
.2 \file{main.tex}.
.2 \sym{make}.
}
}
 
Chaque projet est constitué de 8 dossiers et de 2 fichiers situés à la racine du projet.\\
Les fichiers ou dossiers violets (SYM) sont des liens symboliques dont la source est située dans le dossier \dir{Utils-latex}.


    \begin{items}{gray}{\faFolder}
    \item Le dossier \dir{Images} contient l'ensemble des images du projet.
    Chaque image doit faire partie de la même partie que son document source associé.\\

    La gestion des emplacements des images est indiqué dans la section \link{\nameref{handleImages}}\\

    \item Le dossier \dir{Make} contient les fichiers annexes du projet: 
    \begin{items}{white}{}
        \item Le fichier \file{Alias.tex} regroupe les nouvelles commandes utilisées pour les abréviations.

        \item Le fichier \file{Bibliography.bib} recense la bibliographie du projet. \\Pour plus d'informations, consulter la section \link{\nameref{biblio}}.

        \item Le fichier \file{Colors.tex} est un fichier pour centraliser les nouvelles couleurs définies.
        Pour ajouter une couleur, veuillez vous référer à la section \link{\nameref{colors}}

        \item Le fichier \file{Contacts.tex} est une page pour contacter l'auteur et contient les informations sur les 
        droits et les licences du projet.

        \item Le fichier \file{Glossaries.tex} contient le glossaire.\\ Pour ajouter une définition, 
        veuillez vous référer à la section \link{\nameref{addDef}}
        
        \item Le fichier \file{Index.tex} contient l'index. \\Pour plus d'informations, consulter la section \link{\nameref{index}}.

        \item Le fichier \file{Nomenclature.tex} contient la structure de la nomenclature\footnote{Les unités et 
        grandeurs physiques par exemple}.\\
        Pour ajouter un élément, consulter la section \link{\nameref{addNomenclature}}

        \item Le fichier \file{Rules.tex} contient les conventions pour le projet. Il peut contenir les types de commandes, 
        les conventions de nommage du projet \ldots

        \item Le fichier \file{Versions.tex} contient les différentes versions du projet. \\ Pour plus d'information, 
        consulter la partie \link{\nameref{addVersion}}

    \end{items}
    \item Le dossier \dir{Output} contient les fichiers de compilation générés de manière automatique. 
    \bold{Vous n'aurez pas à modifier les fichiers à cette emplacement.}

    \item Le dossier \dir{Parts} contient les différentes parties du projet. Il est possible de scinder son projet 
    en grandes parties (Introduction, Chapitre1, Chapitre2, Conclusion), chaque dossier contenu dans le 
    dossier \bold{Parts} représente ces parties.\\

    Dans chacun de ces dossiers, vous pouvez créer autant de fichiers Latex que vous voulez, il seront compilés dans 
    l'ordre croissant si vous mettre un numéro au début du nom de fichier.\\

    Pour nommer les dossiers dans le dossier \dir{Parts}, \colors{red}{il faut impérativement commencer le nom avec un 
    numéro suivi d'un point.} Par exemple \dir{0.Intro} puis \dir{1.Content}.\\
    \colors{red}{Les espaces sont interdits dans les noms des dossiers mais les symboles ~\_~ et - sont acceptés.}\\

    Pour chaque dossier crée dans le dossier \dir{Parts}, il faudra créer un dossier avec le même nom (sans le numéro et le point) dans le dossier 
    \dir{Images}, sous peine de voir une volée d'erreurs lors de la compilation.\\

    \item Le dossier \dir{Settings} qui contient les fichiers de configuration du projet.\\
    Pour configurer le projet, reportez-vous à la section \link{\nameref{setSettings}}

    \begin{items}{white}{}

        \item Le fichier \file{ChapterAlias.tex} enregistre le nom des chapitres ('Section' au lieu de 'Chapitre' par exemple).
        \item Le fichier \file{DocumentClass.tex} sauvegarde la classe du document.
        \item Le fichier \file{FHeaderLine.tex} s'occupe de l'épaisseur des lignes entre l'en-tête et le pied de page.
        \item Le fichier \file{Fonts.tex} sélectionne la police du document.
        \item Le fichier \file{Header.tex} affiche la première page
        \item Le fichier \file{Hyperref.tex} enregistre les métadonnées du fichier.
        \item Le fichier \file{Includes.tex} est un fichier généré automatiquement pour importer tous les paramètres cités.
        \item Le fichier \file{ListMargin.tex} règle l'indentation des listes.
        \item Le fichier \file{LOF.tex} autorise l'affiche de la liste des figures
        \item Le fichier \file{Margins.tex} s'occupe des marges du document.
        \item Le fichier \file{NomenclatureName.tex} enregistre le nom de la nomenclature.
        \item Le fichier \file{Theme.tex} enregistre le thème des pages.
        \item Le fichier \file{TocDepth.tex} enregistre la profondeur du sommaire.
        \item Le fichier \file{TocSize.tex} enregistre l'agencement du sommaire.
        \item Le fichier \file{SectionColor.tex} enregistre la couleur des sections.
        \item Le fichier \file{SubSectionColor.tex} enregistre la couleur des sous-sections.
        \item Le fichier \file{TitlePrefix.tex} enregistre le préfixe des sections et sous-sections.

    \end{items}
    \item Le dossier \sym{Utils} est un dossier symbolique qui contient les bibliothèques Latex du projet. Le fichier \file{Utils.sty} est généré 
    dynamiquement, toute écriture manuelle sera écrasée à la prochaine compilation.

    \item Le dossier \dir{Utils-latex} contient toutes les bibliothèques du projet ainsi que les outils de compilation. 
    Il est configuré en tant que sous-module Git afin de séparer le contenu des bibliothèques et le contenu Latex à compiler.
    Ce dossier contient :
    \begin{items}{gray}{\faFolder}

        \item Le dossier \dir{bash} qui regroupe plusieurs bibliothèques : 

        \begin{items}{blue}{}
            \item La bibliothèque \lib{Charts.sh} qui permet de générer des graphiques
            \item La bibliothèque \lib{Colors.sh} qui gère les couleurs dans le terminal
            \item La bibliothèque \lib{Directories.sh} qui permet de gérer les fichiers et dossiers
            \item La bibliothèque \lib{Networks.sh} qui surveille l'état de la connexion Internet
            \item La bibliothèque \lib{System.sh} qui permet de faire des fonctions System
            \item La bibliothèque \lib{Time.sh} qui gère les dates
        \end{items}

        \item Le dossier \dir{HTML} qui regroupe plusieurs bibliothèques utilisées pour convertir le code source Latex en fichier HTML : 

        \begin{items}{blue}{}
            \item La bibliothèque \lib{Badge.py} qui gère les badges
            \item La bibliothèque \lib{Color.py} qui gère les couleurs
            \item La bibliothèque \lib{Command.py} qui gère les types de commandes disponibles
            \item La bibliothèque \lib{Glossary.py} qui gère le glossaire
            \item La bibliothèque \lib{Image.py} qui gère les images et leur intégration au sein du fichier HTML\footnote{Compatibilité Wordpress}
            \item La bibliothèque \lib{Label.py} qui gère labels
            \item Le fichier \lib{generateHTML.py} qui génère les pages HTML
            \item Le fichier \lib{startWebServer.py} qui démarre un serveur Web pour afficher la page Web de rendu
            \item Le dossier \dir{HTML\_render} contient les pages Web générées
        \end{items}
        \item Le dossier \dir{tools} regroupe plusieurs fichiers : 

        \begin{items}{blue}{}
            \item Le fichier \lib{checking} qui permet de vérifier l'intégrité du dossier de projet lors du lancement de VScode
            \item Le fichier \lib{init} qui initialise le dossier du projet
            \item Le fichier \lib{install} qui installe les paquets et utilitaires pour la compilation
            \item Le fichier \lib{compileAnyFiles.py} désactive la compilation pour l'ensemble des dossiers contenus 
            dans le dossier \dir{Parts}. \footnote{Concrètement, cela revient à ajouter un point devant le nom du 
            dossier pour que le compilateur ignore le dossier.}
            \item Le fichier \file{compileAllFile.py} autorise la compilation des dossiers contenus dans le dossier \footnote{On retire les points devant les noms.}
            \item Le fichier \file{generateSnippets.py} créer un fichier de snippets VScode
            afin de gérer l'autocomplétion VSCode. \bold{Ce fichier doit obligatoirement être appelé depuis le 
            fichier \lib{make}}
        \end{items}


    \end{items}

    \item Le dossier \sym{.vscode} contient tous les fichiers de configuration du répertoire VScode : 
    \begin{items}{cyan}{}

        \item Le fichier \file{settings.json} contient les tâches exécutables VScode via un bouton \footnote{Cf. \link{\nameref{allTasks}}}. Ces tâches permettent de lancer la compilation et les utilitaires du répertoire.
        \item Le fichier \file{task.json} est le fichier qui recense les tâches exécutables sous VSCode.
    \end{items}


    Et voici les deux fichiers situés à la racine: 
    \begin{items}{black}{\Triangle}

        \item Le fichier \file{main.tex} est le fichier principal du projet. C'est dans ce fichier qu'on définit notamment :

        \begin{items}{darkBlue}{\Triangle}
            \item L'import de la classe du document
            \item Le titre de la page
            \item Le choix de la disposition du document
            \item Le choix de la présence des fichiers dans le dossier \dir{Make}, c'est à dire les fichiers 
            personnalisables tels que la bibliographie, l'index, le glossaire\ldots
        \end{items} 
        \item Le fichier \sym{make} est le fichier de compilation. \bold{Vous n'aurez pas besoin de modifier ce 
        fichier pour une utilisation classique}
    \end{items}
\end{items}


