\chapter{Bibliothèque Images}

La bibliothèque \lib{Images} permet d'ajouter des images.

\section{Définition de l'espace de nommage}
\label{handleImages}

Lors de l'ajout d'une image, il n'est pas nécessaire de connaître le nom du dossier courant.\\

Pour les commandes \lib{img}, \lib{imgr} et \lib{imgf}, l'argument \bold{Référence} doit être de la forme \bold{\\rootImages+NomImage.extension}.

La macro \lib{\\rootImages} prendra lors de la compilation la valeur du dossier courant.

Par exemple, pour ajouter une image appelée \file{monImage.png} provenant du dossier \\
\dir{Images/Part1}, il suffit d'écrire : 

\begin{Latex}{Code pour l'espace de nommage}
\img{\rootImages/monImages.png}{Ma légende}{0.5}
\end{Latex}

Le troisième argument représente le rapport de taille de l'image.


\section{Ajout d'une image non-flottante}

\img{\rootImages/tux.png}{Légende de l'image}{0.5}

\loc{Body}
\begin{Latex}{Code pour l'ajout d'une image non-flottante}
\img{\rootImages/tux.png}{Légende de l'image}{0.5}
\end{Latex}

\section{Ajout d'une image non-flottante avec une rotation}

\imgr{\rootImages/tux.png}{Légende de l'image}{0.5}{45}

Le dernier argument de la commande représente l'angle de rotation en °.

\loc{Body}
\begin{Latex}{Code pour l'ajout d'une image non-flottante avec une rotation}
\imgr{\rootImages/tux.png}{Légende de l'image}{0.5}{45}
\end{Latex}


\section{Ajout d'une image référencée}

\subsection{Afficher l'image référencée}

Il est possible de faire référence à une image avec la commande suivante : 

\loc{Body}
\begin{Latex}{Code pour référencer une image}
\imgn{\rootImages/tux.png}{Une image référenceée}{0.5}{tux}
\imgn{\rootImages/tux.png}{Une autre image référencée}{0.5}{tux2}
\end{Latex}

\imgn{\rootImages/tux.png}{Une image référenceée}{0.5}{tux}
\imgn{\rootImages/tux.png}{Une autre image référencée}{0.5}{tux2}

\subsection{Faire référence dans le document}

Il existe 2 façons de référencer une figure : 

\begin{items}{blue}{\Triangle}
    \item Via une référence explicite
    \begin{Latex}{Référence par le numéro}
La \figureName{tux} représente Tux.
\end{Latex}
    La \figureName{tux} représente Tux.
    \item Via une référence entre parenthèse
\begin{Latex}{Référence entre parenthèse}
L'image de Tux apparaît sur le document \figureRef{tux2}
\end{Latex}
    L'image de Tux apparaît sur le document \figureRef{tux2}
  \end{items}

\section{Ajout d'une image flottante}

Lorem ipsum dolor sit amet, consectetuer adipiscing elit. Ut purus elit, vestibulum
ut, placerat ac, adipiscing vitae, felis. Curabitur dictum gravida mauris. Nam arcu li-
bero, nonummy eget, consectetuer id, vulputate a, magna. Donec vehicula augue eu
neque. Pellentesque habitant morbi tristique senectus et netus et malesuada fames ac
turpis egestas. Mauris ut leo. Cras viverra metus rhoncus sem. Nulla et lectus vestibu-

\imgf{\rootImages/tux.png}{Légende}{0.5}{0.5}{r}{10}

lum urna fringilla ultrices. Phasellus eu tellus sit amet tortor gravida placerat. Integer
sapien est, iaculis in, pretium quis, viverra ac, nunc. Praesent eget sem vel leo ultrices
bibendum. Aenean faucibus. Morbi dolor nulla, malesuada eu, pulvinar at, mollis ac,
nulla. Curabitur auctor semper nulla. Donec varius orci eget risus. Duis nibh mi, congue
Lorem ipsum dolor sit amet, consectetuer adipiscing elit. Ut purus elit, vestibulum
ut, placerat ac, adipiscing vitae, felis. Curabitur dictum gravida mauris. Nam arcu li-
bero, nonummy eget, consectetuer id, vulputate a, magna. Donec vehicula augue eu
neque. Pellentesque habitant morbi tristique senectus et netus et malesuada fames ac
turpis egestas. Mauris ut leo. Cras viverra metus rhoncus sem. Nulla et lectus vestibu-


\loc{Body}
\begin{Latex}{Code pour l'ajout d'une image lottante}
\imgf{\rootImages/tux.png}{Légende}{0.5}{0.5}{r}{10}
\end{Latex}


\section{Ajout d'une image dans les parties}

Il est possible d'ajouter une image dans les pages présentant les parties du document (Commande \bold{part}).



%############################################################
%###### Package 'Images' 
%###### This package contains some tools to display images
%###### Author  : Nicolas LE GUERROUE
%###### Contact : nicolasleguerroue@gmail.com
%############################################################
\RequirePackage{lmodern}
\RequirePackage{graphicx}           %Images
\RequirePackage{caption}            %légende
\RequirePackage{float}              %image floating
\RequirePackage{wrapfig}
\RequirePackage{subcaption}         %Subcaption

%################################################################


\newcounter{imgCounter}  

    
\newcommand{\img}[3]{%Add simple image #filename legend ratio
    \IfFileExists{#1}{\begin{figure}[H]\centering\ \includegraphics[scale=#3]{#1}\caption{#2}\end{figure} \addtocounter{imgCounter}{1} \raiseMessage{Image '#1' [size=#3,id \arabic{imgCounter}] loaded !} 
    }{\raiseWarning{Image '#1' no loaded}}
}

\newcommand{\imgn}[4]{%Add simple image with reference name #filename legend ratio reference
    \IfFileExists{#1}{\begin{figure}[H]\centering\
    \includegraphics[scale=#3]{#1}\caption{#2}\label{#4}\end{figure} \addtocounter{imgCounter}{1} \raiseMessage{Image '#1' [size=#3,id \arabic{imgCounter}] loaded !} 
    }{\raiseWarning{Image '#1' no loaded}}
}


\newcommand{\imgHeader}[3]{%Add simple image #filename legend ratio
    \IfFileExists{#1}{\includegraphics[scale=#3]{#1} \addtocounter{imgCounter}{1} %\raiseMessage{Image '#1' [size=#3,id \arabic{imgCounter}] loaded !}  
    }{\raiseWarning{Image '#1' no loaded}}
}

\newcommand{\imgr}[4]{%Add simple image with rotation #filename legend ratio angle
    \IfFileExists{#1}{\begin{figure}[H]\centering\ \includegraphics[scale=#3,angle=#4]{#1}\caption{#2}\end{figure} \addtocounter{imgCounter}{1} \raiseMessage{Image '#1' [size=#3,id \arabic{imgCounter},angle=#4] loaded !} }{\raiseWarning{Image '#1' no loaded}}
}


\newcommand{\imgf}[6]{%Add simple floating image #filename legend ratioImage ratioPage side numberOfLines
    \IfFileExists{#1}{
    \begin{wrapfigure}[#6]{#5}{#4\textwidth}\centering\ \includegraphics[scale=#3]{#1}\caption{#2}\end{wrapfigure} \addtocounter{imgCounter}{1} \raiseMessage{Image '#1' [size=#3,id \arabic{imgCounter},side=#5] loaded !} }{\raiseWarning{Image '#1' no loaded}}
}

  
    



    %################################################################