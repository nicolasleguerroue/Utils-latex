\chapter{Bibliothèque Electronic}

La bibliothèque \lib{Electronic} permet de générer des chronogrammes et des schémas électriques



\section{Création de chronogrammes fixes}

\begin{numeric}{Exemple 1}
    D1 &  20{C}   \\
    D2 &  [green] 1H1L1L1L1H1L1L1H1L1H1L1H1L1H1L1H  \\
    D7 &  [black] 1H1L1L1L1H1L1L1H1L1H1L1H1L1H1L1H  \\
    D8 & 8D5U7U5D \\
    D9 & LLL 2{0.1H 0.1L} 0.6H HH \\
    D10 & ZZ G ZZ G XX G X \\
    D11 & [d] 4{5D{Text}} 0.2D \\
    D12 & [L][timing/slope=1.0] HL HL HL HL HL \\
  \end{numeric}


\loc{Body}
\begin{Latex}{Code pour la création de chronogrammes fixes [exemple 1]}
  \begin{numeric}{exemple 1 - chronogramme fixe}
    D1 &  20{C}   \\
    D2 &  [green] 1H1L1L1L1H1L1L1H1L1H1L1H1L1H1L1H  \\
    D7 &  [black] 1H1L1L1L1H1L1L1H1L1H1L1H1L1H1L1H  \\
    D8 & 8D5U7U5D \\
    D9 & LLL 2{0.1H 0.1L} 0.6H HH \\
    D10 & ZZ G ZZ G XX G X \\
    D11 & [d] 4{5D{Text}} 0.2D \\
    D12 & [L][timing/slope=1.0] HL HL HL HL HL \\
  \end{numeric}
\end{Latex}


  \begin{numeric}{Exemple 2 - Chronogramme du compteur 4 bits}
    INPUT &  CC [blue]16{CC} CCC   \\
    D0 &  HL 8{LHHL} LHL   \\
    D1 &  H  4{LLLLHHHH} LLLL \\
    D2 &  H 2{LLLLLLLLHHHHHHHH} LLLL   \\
    D3 &  H{LLLLLLLLLLLLLLLLHHHHHHHHHHHHHHHH} LLLL  \\
    END &  LL [green]14{LL} LHHLLL  \\
    VALUE & L 2D{0} 2D{1} 2D{2} 2D{3} 2D{4} 2D{5} 2D{6} 2D{7} 2D{8} 2D{9} 2D{10} 2D{11} 2D{12} 2D{13} 2D{14} 2D{15} 2D{0} 2D{1}  \\
  \end{numeric}%

\loc{Body}
\begin{Latex}{Code pour la création de chronogrammes fixes [exemple 2]}
  \begin{numeric}{Exemple 2 - Chronogramme du compteur 4 bits}
    INPUT &  CC [blue]16{CC} CCC   \\
    D0 &  HL 8{LHHL} LHL   \\
    D1 &  H  4{LLLLHHHH} LLLL \\
    D2 &  H 2{LLLLLLLLHHHHHHHH} LLLL   \\
    D3 &  H{LLLLLLLLLLLLLLLLHHHHHHHHHHHHHHHH} LLLL  \\
    END &  LL [green]14{LL} LHHLLL  \\
    VALUE & L 2D{0} 2D{1} 2D{2} 2D{3} 2D{4} 2D{5} 2D{6} 2D{7} 2D{8} 2D{9} 2D{10} 2D{11} 2D{12} 2D{13} 2D{14} 2D{15} 2D{0} 2D{1}  \\
  \end{numeric}%
\end{Latex}


\section{Création de chronogrammes flottants}

Notre signal d'horloge (\texttiming{[blue]CCCCCC}) provient d'un oscillateur à quartz.\\
Notre signal d'horloge (\texttiming[timing/draw grid]{LHLHLHLHLHLHLHL}) provient d'un oscillateur à quartz. 

\loc{Body}
\begin{Latex}{Code pour la création de chronogrammes flottants}
  Notre signal d'horloge (\texttiming{[blue]CCCCCC}) provient d'un oscillateur à quartz.\\
  Notre signal d'horloge (\texttiming[timing/draw grid]{LHLHLHLHLHLHLHL}) provient d'un oscillateur à quartz. 
\end{Latex}



\section{Création de schémas électriques}

  
  \begin{schema} {Exemple de schéma électrique}
  
    \addPower{6,5}{power1}{$+5V$}
    \addGround{4,0}{gnd1}{}
  
    \setDeviceBackgroundColor{white}
    \setRotate{0}
    \addLogicGate{0,0}{mynor}{nor}{}{A}{B}{G1}
  
    \setDeviceBackgroundColor{green}
    \addLogicGate{0,2}{mynand}{nand}{}{C}{D}{G2}
    \addLogicGate{2,1}{myor}{or}{}{}{}{G3}
    \resetColors
            
    \addTransistor{6,1}{npnA}{nmos}{B}{C}{E}
    \addTransistor{6,3}{pnpA}{pmos}{b}{e}{c}
  
    \resetColors
    \addTransistor{10,2}{npnR}{nmos}{b}{e}{c}
  
    \addWire{mynor.out}{myor.in 2}{\orthogonalWireA}
    \addWire{mynand.out}{myor.in 1}{\orthogonalWireA}

    \addWire{mynand.out}{pnpA.B}{\orthogonalWireA}
    \addWire{pnpA.C}{npnA.C}{\orthogonalWireA}
  
    \addWire{pnpA.E}{power1}{\orthogonalWireA}
  
    \addWire{npnA.E}{gnd1}{\orthogonalWireA}
  
    \addNode{$(pnpA.C)+(1,0)$}{node1}{}
    \addWire{pnpA.C}{node1}{\orthogonalWireA}
  
    \setDeviceBackgroundColor{red}
    \addLed{myor.out}{\Right}{npnA.B}{\orthogonalWireA}{L1}
    \addResistor{node1}{\Right}{npnR.B}{\orthogonalWireA}
            
  \end{schema}
  
  \loc{Body}
  \begin{Latex}{Code pour la création de schémas électriques}
    \begin{schema} {Exemple de schéma électrique}
  
      \addPower{6,5}{power1}{$+5V$}
      \addGround{4,0}{gnd1}{}
    
      \setDeviceBackgroundColor{white}
      \setRotate{0}
      \addLogicGate{0,0}{mynor}{nor}{}{A}{B}{G1}
    
      \setDeviceBackgroundColor{green}
      \addLogicGate{0,2}{mynand}{nand}{}{C}{D}{G2}
      \addLogicGate{2,1}{myor}{or}{}{}{}{G3}
      \resetColors
              
      \addTransistor{6,1}{npnA}{nmos}{B}{C}{E}
      \addTransistor{6,3}{pnpA}{pmos}{b}{e}{c}
    
      \resetColors
      \addTransistor{10,2}{npnR}{nmos}{b}{e}{c}
    
      \addWire{mynor.out}{myor.in 2}{\orthogonalWireA}
      \addWire{mynand.out}{myor.in 1}{\orthogonalWireA}
  
      \addWire{mynand.out}{pnpA.B}{\orthogonalWireA}
      \addWire{pnpA.C}{npnA.C}{\orthogonalWireA}
    
      \addWire{pnpA.E}{power1}{\orthogonalWireA}
    
      \addWire{npnA.E}{gnd1}{\orthogonalWireA}
    
      \addNode{$(pnpA.C)+(1,0)$}{node1}{}
      \addWire{pnpA.C}{node1}{\orthogonalWireA}
    
      \setDeviceBackgroundColor{red}
      \addLed{myor.out}{\Right}{npnA.B}{\orthogonalWireA}{L1}
      \addResistor{node1}{\Right}{npnR.B}{\orthogonalWireA}
              
    \end{schema}
  \end{Latex}
  

\section{Création de modèles de Thévenin et Norton}


\subsection{Modèle de Thévenin}

\thevenin{E}{Z}[A][B]

\loc{Body}
\begin{Latex}{Un modèle de Thévenin}
  \thevenin{E}{Z}[A][B]
\end{Latex}

\subsection{Modèle de Norton}

\norton{I}{Z}[C][D][-3]

\loc{Body}
\begin{Latex}{Un modèle de Norton}
  \norton{I}{Z}[A][B][-3]
\end{Latex}

\section{Création de moteur à courant continu}

\begin{tikzpicture}
  \mcc{(0,0)}
\end{tikzpicture}

%############################################################
%###### Package 'Electronic' 
%###### This package contains some tools to generate electronic circuits
%###### Author  : Nicolas LE GUERROUE
%###### Contact : nicolasleguerroue@gmail.com
%############################################################

\RequirePackage{tikz-timing}

\RequirePackage{graphics} 
\RequirePackage{graphicx}
\RequirePackage{pgf}
\RequirePackage{tikz}
\RequirePackage{circuitikz}
\usetikzlibrary{babel}  

\RequirePackage{ifthen}   
\typeout{>>> Utils: Package 'Electronics' loaded !}

%############################ Settings ##############################
\tikzset{
    timing/table/axis/.style={->,>=latex},
    timing/table/axis ticks/.style={},   
}

%Direction of some device such as resistor, led...
%1.5 is the minimum length of device according my runs
\newcommand{\Up}{0,1.5}     %Componant Direction to up #
\newcommand{\Down}{0,-1.5} %Componant Direction to down #
\newcommand{\Right}{1.5,0} %Componant Direction to right #
\newcommand{\Left}{-1.5,0} %Componant Direction to left #


%###### Length of components
\newcommand{\bipolesLength}[1]{#1cm}%Default size of components #size
%Length update
\newcommand{\setBipolesLength}[1]{%Set default size of components #size
    \renewcommand{\bipolesLength}{#1}
    \ctikzset{bipoles/length=\bipolesLength cm}
}


%############ Mirrors and inverting
\newcommand{\Mirror}{}%Mirror rotate of component #
\newcommand{\Invert}{}%Invert rotate of component #

\newcommand{\setMirror}{%Set mirror #
    \renewcommand{\Mirror}{,mirror}
}

\newcommand{\setNoMirror}{%Set no mirror #
    \renewcommand{\Mirror}{}
}

\newcommand{\setInvert}{ %Set invert #
    \renewcommand{\Invert}{,Invert}
}

\newcommand{\setNoInvert}{ %Set no invert #
    \renewcommand{\Invert}{}
}


%############## Rotate ###########
\newcommand{\rotate}{0} %default rotate value #

%Update
\newcommand{\setRotate}[1]{ %Set rotate value #angle
    \renewcommand{\rotate}{#1}
}%End \setRotate


%####################### Colors
\newcommand{\deviceBorderColor}{black} %Default colors of border color #
\newcommand{\deviceBackgroundColor}{white}%default colors of background color #

\newcommand{\setDeviceBorderColor}[1]{ %Set color of device #color
    \renewcommand{\deviceBorderColor}{#1}  
    \renewcommand{\deviceBackgroundColor}{white} 
}

\newcommand{\setDeviceBackgroundColor}[1]{ %Set background color of device #color
    \renewcommand{\deviceBorderColor}{black} 
    \renewcommand{\deviceBackgroundColor}{#1}
}

\newcommand{\resetColors}{ %Reset colors #
    \renewcommand{\deviceBorderColor}{black} 
    \renewcommand{\deviceBackgroundColor}{white} 
}

%####################################################################
%############## draw device #########################################



%\ifthenelse{\equal{#1}{0}}{A.}{no A.}
%Init

\begin{comment}
    @begin
    @command \addLogicGate
    @des 
    Cette commande permet de dessiner une porte logique à double entrée. Pour dessinder une porte inverseuse, utiliser la commande \addNotGate
    @sed
    @input Coordonnées de la porte en (x,y) sans parenthèse
    @input Référence de la porte pour s'accrocher aux entrées et sorties
    @input Type de la porte [nand, nor, or, and, or, xor]
    @input Label de sortie (laisser vide si absence de label souhaité)
    @input Label de l'entrée 1 (laisser vide si absence de label souhaité)
    @input Label de l'entrée 2 (laisser vide si absence de label souhaité)
    @input Nom de la porte [NOR1, AND1...]
    @begin_example 
    \addLogicGate{5,5}{logicgate}{nand}{S}{A}{B}{L1}
    @end_example
    @end
    \end{comment}

\newcommand{\addLogicGate}[7] {%Add logic gate #coord reference type outputLabel inputLabel1 inputLabel2 name
    \raiseMessage{Adding logic gate device [type=#3]}
    \ifthenelse{\equal{\deviceBorderColor}  {black}}
    {\draw (#1)         node (#2) [rotate=\rotate,xshift=0cm,fill=\deviceBackgroundColor,#3 port] {#7}}%if equal to black
    {\draw (#1)         node (#2) [rotate=\rotate,xshift=0cm,color=\deviceBorderColor,#3 port] {#7}}

    (#2.out)  node      [anchor=south west, yshift=-0.3cm] {#4}
    (#2.in 1) node (A1)     [anchor=east,xshift=0cm,yshift=+0.3cm]       {#5}
    (#2.in 2) node (B1)     [anchor=east,xshift=0cm,yshift=+0.3cm]       {#6};
}


\newenvironment{schema}[1] %Create newt schema #name
{
    \begin{center}
        \makeatletter
        \def\@captype{figure}
        \makeatother
        \newcommand{\TitleSchema}%Print title schema #
        {#1}
        %\shorthandoff{:;!?} %Compulsory if frenchb package is used (from babel)
        \raiseMessage{Creating new schema ['#1']}
        \begin{tikzpicture}
            %\setGraphic %command to display with frenchb babel
    }
    { 
        \end{tikzpicture}
   % \caption{\TitleSchema}
    \end{center}
}


\newenvironment{numeric}[1]%create new chronogram #name
{
\begin{center}
    \makeatletter
    \def\@captype{figure}
    \makeatother
    \newcommand{\TitleNumeric}%use var to print title #
    {#1}
    \raiseMessage{Creating new chronogram ['#1']}
\begin{tikztimingtable}
}
{
\end{tikztimingtable}%
\caption{\TitleNumeric}
\end{center}
}


\newcommand{\addTransistor}[6] {%add transistor #coord name type B C E

    \raiseMessage{Adding transistor device [type=#3]}
    \ifthenelse{\equal{\deviceBorderColor}  {black}}
    {\draw (#1)         node (#2) [xshift=0cm,fill=\deviceBackgroundColor,#3] {}}%if equal to black
    {\draw (#1)         node (#2) [xshift=0cm,color=\deviceBorderColor,#3] {}}

    (#2.B)  node      [anchor=south west, xshift=0cm, yshift=0cm] {#4} 
    (#2.C) node (A1)     [anchor=north,xshift=0.3cm,yshift=+0.1cm]       {#5}
    (#2.E) node (B1)     [anchor=south,xshift=0.3cm,yshift=0.1cm]       {#6};
}

\newcommand{\addWire}[3] {%Add wire #node1 node2 direction
    \draw (#1) #3 (#2);
}

\newcommand{\orthogonalWireA}{-|}%Set wire vertical direction 1#
\newcommand{\orthogonalWireB}{|-}%Set wire vetical direction 2#
\newcommand{\directWire}{--}%Set wire hrizontal direction #


\newcommand{\addNode}[3] {%Add node #coord label value
    \node (#2) at (#1) {#3};
}


\newcommand{\addPoint}[3] { %Add point #coord color width
    \filldraw [#2] (#1) circle (#3pt);
}


\newcommand{\addPower}[3] {  %Add power supply #coord name value
    \raiseMessage{Adding power device [name=#2, value=#3]}
    \draw (#1) node (#2) [vcc] {#3};
}

\newcommand{\addGround}[3] { %Add ground #coord name value
    \draw (#1) node (#2) [ground] {#3};
}


\newcommand{\addResistor}[4] {%Create resistor #beginCoord orientation endCoord
    \raiseMessage{Adding resistor device}
    \draw (#1) to[R,l=$R$] +(#2) #4 (#3);
}


\newcommand{\addLed}[5] {  %Create led #beginCoord orientation endCoord type name
    \raiseMessage{Adding LED device [name=#5]}
    \ifthenelse{\equal{\deviceBorderColor}  {black}}
    {\draw (#1) to[leD,l_=#5,fill=\deviceBackgroundColor] +(#2) #4 (#3);}
    {\draw (#1) to[leD,l_=#5,color=\deviceBorderColor] +(#2) #4 (#3);}
}


\newcounter{datasheetCounter}  
   
\newcommand{\addDatasheet}[1]{%Add simple image #filename legend ratio
    \raiseMessage{Adding datasheet [name=#1]}
    \addtocounter{datasheetCounter}{1}
}
