\chapter{Bibliothèque Electronic}

La bibliothèque \lib{Electronic} permet de générer des chronogrammes et des schémas électriques



\section{Création de chronogrammes fixes}

\begin{numeric}{Exemple 1}
    D1 &  20{C}   \\
    D2 &  [green] 1H1L1L1L1H1L1L1H1L1H1L1H1L1H1L1H  \\
    D7 &  [black] 1H1L1L1L1H1L1L1H1L1H1L1H1L1H1L1H  \\
    D8 & 8D5U7U5D \\
    D9 & LLL 2{0.1H 0.1L} 0.6H HH \\
    D10 & ZZ G ZZ G XX G X \\
    D11 & [d] 4{5D{Text}} 0.2D \\
    D12 & [L][timing/slope=1.0] HL HL HL HL HL \\
  \end{numeric}


\loc{Body}
\begin{Latex}{Code pour la création de chronogrammes fixes [exemple 1]}
  \begin{numeric}{exemple 1 - chronogramme fixe}
    D1 &  20{C}   \\
    D2 &  [green] 1H1L1L1L1H1L1L1H1L1H1L1H1L1H1L1H  \\
    D7 &  [black] 1H1L1L1L1H1L1L1H1L1H1L1H1L1H1L1H  \\
    D8 & 8D5U7U5D \\
    D9 & LLL 2{0.1H 0.1L} 0.6H HH \\
    D10 & ZZ G ZZ G XX G X \\
    D11 & [d] 4{5D{Text}} 0.2D \\
    D12 & [L][timing/slope=1.0] HL HL HL HL HL \\
  \end{numeric}
\end{Latex}


  \begin{numeric}{Exemple 2 - Chronogramme du compteur 4 bits}
    INPUT &  CC [blue]16{CC} CCC   \\
    D0 &  HL 8{LHHL} LHL   \\
    D1 &  H  4{LLLLHHHH} LLLL \\
    D2 &  H 2{LLLLLLLLHHHHHHHH} LLLL   \\
    D3 &  H{LLLLLLLLLLLLLLLLHHHHHHHHHHHHHHHH} LLLL  \\
    END &  LL [green]14{LL} LHHLLL  \\
    VALUE & L 2D{0} 2D{1} 2D{2} 2D{3} 2D{4} 2D{5} 2D{6} 2D{7} 2D{8} 2D{9} 2D{10} 2D{11} 2D{12} 2D{13} 2D{14} 2D{15} 2D{0} 2D{1}  \\
  \end{numeric}%

\loc{Body}
\begin{Latex}{Code pour la création de chronogrammes fixes [exemple 2]}
  \begin{numeric}{Exemple 2 - Chronogramme du compteur 4 bits}
    INPUT &  CC [blue]16{CC} CCC   \\
    D0 &  HL 8{LHHL} LHL   \\
    D1 &  H  4{LLLLHHHH} LLLL \\
    D2 &  H 2{LLLLLLLLHHHHHHHH} LLLL   \\
    D3 &  H{LLLLLLLLLLLLLLLLHHHHHHHHHHHHHHHH} LLLL  \\
    END &  LL [green]14{LL} LHHLLL  \\
    VALUE & L 2D{0} 2D{1} 2D{2} 2D{3} 2D{4} 2D{5} 2D{6} 2D{7} 2D{8} 2D{9} 2D{10} 2D{11} 2D{12} 2D{13} 2D{14} 2D{15} 2D{0} 2D{1}  \\
  \end{numeric}%
\end{Latex}


\section{Création de chronogrammes flottants}

Notre signal d'horloge (\texttiming{[blue]CCCCCC}) provient d'un oscillateur à quartz.\\
Notre signal d'horloge (\texttiming[timing/draw grid]{LHLHLHLHLHLHLHL}) provient d'un oscillateur à quartz. 

\loc{Body}
\begin{Latex}{Code pour la création de chronogrammes flottants}
  Notre signal d'horloge (\texttiming{[blue]CCCCCC}) provient d'un oscillateur à quartz.\\
  Notre signal d'horloge (\texttiming[timing/draw grid]{LHLHLHLHLHLHLHL}) provient d'un oscillateur à quartz. 
\end{Latex}



\section{Création de schémas électriques}

  
  \begin{schema} {Exemple de schéma électrique}
  
    \addPower{6,5}{power1}{$+5V$}
    \addGround{4,0}{gnd1}{}
  
    \setDeviceBackgroundColor{white}
    \setRotate{0}
    \addLogicGate{0,0}{mynor}{nor}{}{A}{B}{G1}
  
    \setDeviceBackgroundColor{green}
    \addLogicGate{0,2}{mynand}{nand}{}{C}{D}{G2}
    \addLogicGate{2,1}{myor}{or}{}{}{}{G3}
    \resetColors
            
    \addTransistor{6,1}{npnA}{nmos}{B}{C}{E}
    \addTransistor{6,3}{pnpA}{pmos}{b}{e}{c}
  
    \resetColors
    \addTransistor{10,2}{npnR}{nmos}{b}{e}{c}
  
    \addWire{mynor.out}{myor.in 2}{\orthogonalWireA}
    \addWire{mynand.out}{myor.in 1}{\orthogonalWireA}

    \addWire{mynand.out}{pnpA.B}{\orthogonalWireA}
    \addWire{pnpA.C}{npnA.C}{\orthogonalWireA}
  
    \addWire{pnpA.E}{power1}{\orthogonalWireA}
  
    \addWire{npnA.E}{gnd1}{\orthogonalWireA}
  
    \addNode{$(pnpA.C)+(1,0)$}{node1}{}
    \addWire{pnpA.C}{node1}{\orthogonalWireA}
  
    \setDeviceBackgroundColor{red}
    \addLed{myor.out}{\Right}{npnA.B}{\orthogonalWireA}{L1}
    \addResistor{node1}{\Right}{npnR.B}{\orthogonalWireA}
            
  \end{schema}
  
  \loc{Body}
  \begin{Latex}{Code pour la création de schémas électriques}
    \begin{schema} {Exemple de schéma électrique}
  
      \addPower{6,5}{power1}{$+5V$}
      \addGround{4,0}{gnd1}{}
    
      \setDeviceBackgroundColor{white}
      \setRotate{0}
      \addLogicGate{0,0}{mynor}{nor}{}{A}{B}{G1}
    
      \setDeviceBackgroundColor{green}
      \addLogicGate{0,2}{mynand}{nand}{}{C}{D}{G2}
      \addLogicGate{2,1}{myor}{or}{}{}{}{G3}
      \resetColors
              
      \addTransistor{6,1}{npnA}{nmos}{B}{C}{E}
      \addTransistor{6,3}{pnpA}{pmos}{b}{e}{c}
    
      \resetColors
      \addTransistor{10,2}{npnR}{nmos}{b}{e}{c}
    
      \addWire{mynor.out}{myor.in 2}{\orthogonalWireA}
      \addWire{mynand.out}{myor.in 1}{\orthogonalWireA}
  
      \addWire{mynand.out}{pnpA.B}{\orthogonalWireA}
      \addWire{pnpA.C}{npnA.C}{\orthogonalWireA}
    
      \addWire{pnpA.E}{power1}{\orthogonalWireA}
    
      \addWire{npnA.E}{gnd1}{\orthogonalWireA}
    
      \addNode{$(pnpA.C)+(1,0)$}{node1}{}
      \addWire{pnpA.C}{node1}{\orthogonalWireA}
    
      \setDeviceBackgroundColor{red}
      \addLed{myor.out}{\Right}{npnA.B}{\orthogonalWireA}{L1}
      \addResistor{node1}{\Right}{npnR.B}{\orthogonalWireA}
              
    \end{schema}
  \end{Latex}
  

\section{Création de modèles de Thévenin et Norton}


\subsection{Modèle de Thévenin}

\thevenin{E}{Z}[A][B]

\loc{Body}
\begin{Latex}{Un modèle de Thévenin}
  \thevenin{E}{Z}[A][B]
\end{Latex}

\subsection{Modèle de Norton}

\norton{I}{Z}[C][D][-3]

\loc{Body}
\begin{Latex}{Un modèle de Norton}
  \norton{I}{Z}[A][B][-3]
\end{Latex}

\section{Création de moteur à courant continu}

\begin{tikzpicture}
  \mcc{(0,0)}
\end{tikzpicture}

%############################################################
%###### Package 'Electronic' 
%###### This package contains some tools to generate electronic circuits
%###### Author  : Nicolas LE GUERROUE
%###### Contact : nicolasleguerroue@gmail.com
%############################################################
\RequirePackage{tikz-timing} %Timing diagram tikz #
\RequirePackage{graphics} %Images #
\RequirePackage{graphicx} %Images #
\RequirePackage{pgf} %Circuit #
\RequirePackage{tikz}   %schema #
\RequirePackage{circuitikz} %Electronic #
\usetikzlibrary{babel}  %langage #
\RequirePackage{ifthen} %Conditions #
\RequirePackage{xargs} %Optionnal args #
%############################################################
\typeout{>>> Utils: Package 'Electronics' loaded !}
%############################################################
\newboolean{boolRaccourcisElec}
\setboolean{boolRaccourcisElec}{true}



%############################ Settings ##############################
\tikzset{
    timing/table/axis/.style={->,>=latex},
    timing/table/axis ticks/.style={},   
}

%Direction of some device such as resistor, led...
%1.5 is the minimum length of device according my runs
\newcommand{\Up}{0,1.5}     %Componant Direction to up #
\newcommand{\Down}{0,-1.5} %Componant Direction to down #
\newcommand{\Right}{1.5,0} %Componant Direction to right #
\newcommand{\Left}{-1.5,0} %Componant Direction to left #


%###### Length of components
\newcommand{\bipolesLength}[1]{#1cm}%Default size of components #size
%Length update
\newcommand{\setBipolesLength}[1]{%Set default size of components #size
    \renewcommand{\bipolesLength}{#1}
    \ctikzset{bipoles/length=\bipolesLength cm}
}


%############ Mirrors and inverting
\newcommand{\Mirror}{}%Mirror rotate of component #
\newcommand{\Invert}{}%Invert rotate of component #

\newcommand{\setMirror}{%Set mirror #
    \renewcommand{\Mirror}{,mirror}
}

\newcommand{\setNoMirror}{%Set no mirror #
    \renewcommand{\Mirror}{}
}

\newcommand{\setInvert}{ %Set invert #
    \renewcommand{\Invert}{,Invert}
}

\newcommand{\setNoInvert}{ %Set no invert #
    \renewcommand{\Invert}{}
}


%############## Rotate ###########
\newcommand{\rotate}{0} %default rotate value #

%Update
\newcommand{\setRotate}[1]{ %Set rotate value #angle
    \renewcommand{\rotate}{#1}
}%End \setRotate


%####################### Colors
\newcommand{\deviceBorderColor}{black} %Default colors of border color #
\newcommand{\deviceBackgroundColor}{white}%default colors of background color #

\newcommand{\setDeviceBorderColor}[1]{ %Set color of device #color
    \renewcommand{\deviceBorderColor}{#1}  
    \renewcommand{\deviceBackgroundColor}{white} 
}

\newcommand{\setDeviceBackgroundColor}[1] %Set background color of device #color
{ 
    \renewcommand{\deviceBorderColor}{black} 
    \renewcommand{\deviceBackgroundColor}{#1}
}

\newcommand{\resetColors}{ %Reset colors #
    \renewcommand{\deviceBorderColor}{black} 
    \renewcommand{\deviceBackgroundColor}{white} 
}

%####################################################################
%############## draw device #########################################

\newcommand{\addLogicGate}[7] {%Add logic gate #coord reference type outputLabel inputLabel1 inputLabel2 name
    \raiseMessage{Adding logic gate device [type=#3]}
    \ifthenelse{\equal{\deviceBorderColor}  {black}}
    {\draw (#1)         node (#2) [rotate=\rotate,xshift=0cm,fill=\deviceBackgroundColor,#3 port] {#7}}%if equal to black
    {\draw (#1)         node (#2) [rotate=\rotate,xshift=0cm,color=\deviceBorderColor,#3 port] {#7}}

    (#2.out)  node      [anchor=south west, yshift=-0.3cm] {#4}
    (#2.in 1) node (A1)     [anchor=east,xshift=0cm,yshift=+0.3cm]       {#5}
    (#2.in 2) node (B1)     [anchor=east,xshift=0cm,yshift=+0.3cm]       {#6};
}


\newenvironment{schema}[1] %Create newt schema #name
{
    \begin{center}
        \makeatletter
        \def\@captype{figure}
        \makeatother
        \newcommand{\TitleSchema}%Print title schema #
        {#1}
        %\shorthandoff{:;!?} %Compulsory if frenchb package is used (from babel)
        \raiseMessage{Creating new schema ['#1']}
        \begin{tikzpicture}
            %\setGraphic %command to display with frenchb babel
    }
    { 
        \end{tikzpicture}
   % \caption{\TitleSchema}
    \end{center}
}


\newenvironment{numeric}[1]%create new chronogram #name
{
\begin{center}
    \makeatletter
    \def\@captype{figure}
    \makeatother
    \newcommand{\TitleNumeric}%use var to print title #
    {#1}
    \raiseMessage{Creating new chronogram ['#1']}
\begin{tikztimingtable}
}
{
\end{tikztimingtable}%
\caption{\TitleNumeric}
\end{center}
}


\newcommand{\addTransistor}[6] {%add transistor #coord name type B C E

    \raiseMessage{Adding transistor device [type=#3]}
    \ifthenelse{\equal{\deviceBorderColor}  {black}}
    {\draw (#1)         node (#2) [xshift=0cm,fill=\deviceBackgroundColor,#3] {}}%if equal to black
    {\draw (#1)         node (#2) [xshift=0cm,color=\deviceBorderColor,#3] {}}

    (#2.B)  node      [anchor=south west, xshift=0cm, yshift=0cm] {#4} 
    (#2.C) node (A1)     [anchor=north,xshift=0.3cm,yshift=+0.1cm]       {#5}
    (#2.E) node (B1)     [anchor=south,xshift=0.3cm,yshift=0.1cm]       {#6};
}

\newcommand{\addWire}[3] {%Add wire #node1 node2 direction
    \draw (#1) #3 (#2);
}

\newcommand{\orthogonalWireA}{-|}%Set wire vertical direction 1#
\newcommand{\orthogonalWireB}{|-}%Set wire vetical direction 2#
\newcommand{\directWire}{--}%Set wire hrizontal direction #


\newcommand{\addNode}[3] {%Add node #coord label value
    \node (#2) at (#1) {#3};
}


\newcommand{\addPoint}[3] { %Add point #coord color width
    \filldraw [#2] (#1) circle (#3pt);
}


\newcommand{\addPower}[3] {  %Add power supply #coord name value
    \raiseMessage{Adding power device [name=#2, value=#3]}
    \draw (#1) node (#2) [vcc] {#3};
}

\newcommand{\addGround}[3] { %Add ground #coord name value
    \draw (#1) node (#2) [ground] {#3};
}


\newcommand{\addResistor}[4] {%Create resistor #beginCoord orientation endCoord
    \raiseMessage{Adding resistor device}
    \draw (#1) to[R,l=$R$] +(#2) #4 (#3);
}


\newcommand{\addLed}[5] {  %Create led #beginCoord orientation endCoord type name
    \raiseMessage{Adding LED device [name=#5]}
    \ifthenelse{\equal{\deviceBorderColor}  {black}}
    {\draw (#1) to[leD,l_=#5,fill=\deviceBackgroundColor] +(#2) #4 (#3);}
    {\draw (#1) to[leD,l_=#5,color=\deviceBorderColor] +(#2) #4 (#3);}
}



\newcommandx*{\thevenin}[4][3=$A$,4=$B$]		{\begin{circuitikz} %Add Thevenin schematic #VoltageName, InternalImpedanceName, HighVoltageNamge, LowVoltageName
    \draw (0,0)%
to[short,o-]	(-1.5,0)
to[V,v=$#1$]	(-1.5,1.5)
to[R,l=$#2$]	(-1.5,3)
to[short,-o]	(0,3);
\node[anchor=north] at (0,3) {#3};
\node[anchor=south] at (0,0) {#4};
;\end{circuitikz}
}

\newcommandx*{\norton}[5][3=$A$,4=$B$,5=-2.5]		{\begin{circuitikz} %Add Norton schematic #VoltageName, InternalImpedanceName, HighVoltageNamge, LowVoltageName
    \draw (0,0)%
to[short,o-]	(#5,0)%
to[I,i=$#1$]	(#5,3)%
to[short,-o]	(0,3);%
\draw (-1,0) to[R,l=$#2$,*-*] (-1,3);
\node[anchor=north] at (0,3) {#3};%
\node[anchor=south] at (0,0) {#4};%
    ;\end{circuitikz}%
}

%Changement des interrupteurs dans circuitikz
\makeatletter
% create the shape
\pgfcircdeclarebipole{}{\ctikzvalof{bipoles/interr/height 2}}{spst}{\ctikzvalof{bipoles/interr/height}}{\ctikzvalof{bipoles/interr/width}}{

\pgfsetlinewidth{\pgfkeysvalueof{/tikz/circuitikz/bipoles/thickness}\pgfstartlinewidth}

\pgfpathmoveto{\pgfpoint{\pgf@circ@res@left}{0pt}}
\pgfpathlineto{\pgfpoint{.6\pgf@circ@res@right}{\pgf@circ@res@up}}
\pgfusepath{draw}   
}

% make the shape accessible with nice syntax
\def\pgf@circ@spst@path#1{\pgf@circ@bipole@path{spst}{#1}}
\tikzset{switch/.style = {\circuitikzbasekey, /tikz/to path=\pgf@circ@spst@path, l=#1}}
\tikzset{spst/.style = {switch = #1}}
\makeatother








% \ctikzset{voltage/european label distance=1.8}



\newcommandx*{\mcc}[3][1=0,3=]{\begin{scope}[shift={#2}]\begin{scope}[rotate=#1]
    \draw (0,0) -- (0.5,0);
    \draw (0.5,-0.25) rectangle (0.75,0.25);
    \draw (2,0) -- (2.5,0);
    \draw (1.75,-0.25) rectangle (2,0.25);
    \draw (1.25,0) circle (0.5);
    \node[anchor=center] at (1.25,0) {\footnotesize Mcc};
    \ifthenelse{\equal{#3}{}}
        {}
        {\draw[-45] (2.25,0.65) -- (0.25,0.65);
        \node at (1.25,1) {#3};
        }
;\end{scope};\end{scope};
}

\newcommandx*{\switchOpen}[4][1=0,3=,4=]{\begin{scope}[shift={#2}]\begin{scope}[rotate=#1]
    \draw (0,0) -- (0.25,0)-- (0.7,0.25);
    \draw (0.75,0) -- (1,0);
    \fill[black] (0.25,0) circle (0.05);
    \fill (0.75,0) circle (0.05);
    \ifthenelse{\equal{#3}{}}
        {}
        {\node[anchor=south] at (0.5,0.5) {#3};}
    \ifthenelse{\equal{#4}{}}
        {}
        {\draw[-triangle 45] (0.8,-0.25) -- (0.2,-0.25);
        \node[anchor=north] at (0.5,-0.25) {#4};}
;\end{scope};\end{scope};
}

\newcommandx*{\switchClosed}[4][1=0,3=,4=]{\begin{scope}[shift={#2}]\begin{scope}[rotate=#1]
    \draw (0,0) -- (1,0);
    \fill[black] (0.25,0) circle (0.05);
    \fill (0.75,0) circle (0.05);
    \ifthenelse{\equal{#3}{}}
        {}
        {\node[anchor=south] at (0.5,0.25) {#3};}
    \ifthenelse{\equal{#4}{}}
        {}
        {\draw[-triangle 45] (0.8,-0.25) -- (0.2,-0.25);
        \node[anchor=north] at (0.5,-0.25) {#4};}
;\end{scope};\end{scope};
}


\newcommand{\resistorColor}[1] %Display label color according value #
{ 
    \ifthenelse{\equal{#1}{0}}
    {\badge{white}{black}{0}}   %if equal to 0
    {\ifthenelse{\equal{#1}{1}}{\badge{white}{brown}{1}} %if equal to 1
        {\ifthenelse{\equal{#1}{2}}{\badge{white}{red}{2}} %if equal to 2
            {\ifthenelse{\equal{#1}{3}}{\badge{black}{orange}{3}} %if equal to 3
                {\ifthenelse{\equal{#1}{4}}{\badge{black}{yellow}{4}} %if equal to 4
                    {{\ifthenelse{\equal{#1}{5}}{\badge{white}{green}{5}} %if equal to 5
                        {\ifthenelse{\equal{#1}{6}}{\badge{white}{cyan}{6}} %if equal to 6
                            {\ifthenelse{\equal{#1}{7}}{\badge{white}{blue}{7}} %if equal to 7
                                {\ifthenelse{\equal{#1}{8}}{\badge{black}{gray}{8}} %if equal to 8
                                    {\ifthenelse{\equal{#1}{9}}{\badge{black}{white}{9}} %if equal to 9 
{#1}}}}}}}}}}}}
