\chapter{Bibliothèque Tables}


La bibliothèque \lib{Table} gère les tableaux.

\section{Création d'un tableau}
  

\begin{tableFigure}{|c|c|c|c|}{Réponse sur le sens du courant en fonction des tensions $U_A$ et $U_B$}
    \rowcolor{darkBlue}  
    \hline
    \color{white}{$U_A$ (V)} & \color{white}{$U_B$ (V)} & \color{white}{Sens du courant} & \color{white}{$U_A-U_B$}\\
    \hline
    10 & 5 & \colors{blue}{De A vers B} & \cellcolor{green} 5\\
    \hline
    5 & 10 & \colors{blue}{de B vers A} & -5\\
    \hline
    5 & 5 & \colors{blue}{Aucun courant ne circule} & 0\\
    \hline
  \end{tableFigure}

  
  \loc{Body}
  \begin{Latex}{Code d'exemple}
    \begin{tableFigure}{|c|c|c|c|}{Réponse sur le sens du courant en fonction des tensions $U_A$ et $U_B$}
      \rowcolor{darkBlue}  
      \hline
      \color{white}{$U_A$ (V)} & \color{white}{$U_B$ (V)} & \color{white}{Sens du courant} & \color{white}{$U_A-U_B$}\\
      \hline
      10 & 5 & \colors{blue}{De A vers B} & \cellcolor{green} 5\\
      \hline
      5 & 10 & \colors{blue}{de B vers A} & -5\\
      \hline
      5 & 5 & \colors{blue}{Aucun courant ne circule} & 0\\
      \hline
    \end{tableFigure}
\end{Latex}

\section{Personnalisation des tableaux}

\subsection{Couleur d'arrière-plan d'une ligne}


La commande suivante permet de changer l'arrière-plan d'une ligne du tableau :

\begin{Latex}{Couleur d'arrière-plan des tableaux}
\cellcolor{red} 5
\end{Latex}

\subsection{Couleur d'arrière-plan d'une cellule}


La commande suivante permet de changer l'arrière-plan d'une cellule du tableau :

\begin{Latex}{Couleur d'arrière-plan des tableaux}
\rowcolor{darkBlue} 
\end{Latex}


\section{Affichage de la liste des tables}

\loc{main.tex}

\begin{Latex}{Code pour l'affichage de la liste des tables}
\displayListOfTables{Liste des tables}
\end{Latex}

L'argument passé à la macro est le nom visible sur la page ainsi que dans la table des matière.



%############################################################
%###### Package 'Tables' 
%###### This package contains some tools to display Tables (listoftables)
%###### Author  : Nicolas LE GUERROUE
%###### Contact : nicolasleguerroue@gmail.com
%############################################################
\typeout{>>> Utils: Package 'Tables' loaded !}

\newcommand{\displayListOfTables}[1]{%Display the listOfTables #tocName
\renewcommand{\listtablename}{#1}
\addcontentsline{toc}{chapter}{#1}
%\chapter*{#1}
\vspace{-4cm}
\listoftables
}


