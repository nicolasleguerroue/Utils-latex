\chapter{Bibliothèque Header}

La bibliothèque \lib{Header} permet de gérer la mise en page globale du document, en particulier : 

\begin{items}{blue}{\Triangle}
  \item Le type de page de garde : 
  \begin{items}{green}{\Triangle}
    \item Page de garde avec une image
    \item Page de garde sans image
  \end{items}
  \item Créer des parties avec des images (commande \lib{part} améliorée)
  \item Définit l'en-tête et le pied de page
\end{items}
\section{Mise en forme de la page de garde avec une image}

\loc{Header}
\begin{Latex}{Code pour la mise en forme de la page de garde avec une image}
\setHeaderImage{Emplacement_image}{0.8}{Titre}{sous-titre}{Auteurs}{\today \\ \pageref{LastPage} pages}
\end{Latex}


\section{Mise en forme de la page de garde sans image}

\loc{Header}
\begin{Latex}{Code pour la mise en forme de la page de garde sans image}
  \setHeader{Titre}{Auteur 1 \\ Auteur 2}{Date}
\end{Latex}

\section{Mise en forme de la page des parties}

\subsection{Ajout d'une image}
\loc{Body}
\begin{Latex}{Code pour la mise en forme de la page des parties}
  \partImg{Partie}{Images/file.png}{0.2}
\end{Latex}


\subsection{Ajout d'un texte sous l'image}
La commande \lib{addPartText} doit impérativement être appelée avant la commande \lib{partImg}

\loc{Body}

\begin{Latex}{Code pour la mise en forme du texte sous la partie}
\addPartText{Texte sous l'image}
\partImg{Partie}{Images/file.png}{0.2}
\end{Latex}


\section{Ajout d'un trait entre l'en-tête et le corps de la page}

\loc{Settings/FHeaderLine.tex}
\begin{Latex}{Code pour l'ajout d'un trait entre l'en-tête et le corps de la page}
  \setHeaderLine{0.2}
\end{Latex}

Ce code est généré automatiquement lors de l'ouverture du projet. Pour le modifier, il faut configurer les paramètres du projet en allant à la section \link{\nameref{setSettings}}.

\section{Ajout d'un trait entre le corps de la page et le bas de page}

\loc{Settings/FHeaderLine.tex}
\begin{Latex}{Code pour l'ajout d'un trait entre le corps de la page et le bas de page}
  \setFooterLine{0.2}
\end{Latex}

Ce code est généré automatiquement lors de l'ouverture du projet. Pour le modifier, il faut configurer les paramètres du projet en allant à la section \link{\nameref{setSettings}}.



\section{Définition de la présentation globale des pages}

\loc{Header}
\begin{Latex}{Code pour la définition de la présentation globale des page}
  \addPresentation
  {Titre} {Centre} {\currentChapter}
  {Gauche} {} {\currentPage}
\end{Latex}

\section{Redéfinition des titres des chapitres}

Par défaut, l'utilisation du mot clé \lib{chapter} force à utiliser le mot-clé \bold{Chapitre X}. Pour utiliser un autre nom, utiliser la commande suivante : 

\loc{Settings/ChapterAlias.tex}
\begin{Latex}{Code pour la redéfinition des titres des chapitres}
  \setAliasChapter{Section}
\end{Latex}
Ce code est généré automatiquement lors de l'ouverture du projet. Pour le modifier, il faut configurer les paramètres du projet en allant à la section \link{\nameref{setSettings}}.



\section{Mettre le document en pleine page}
\loc{Header}
\begin{Latex}{Code pour mettre le document en pleine page}
\setFullPage
\end{Latex}

\section{Récupérer le chapitre courant}

Le chapitre courant sera en majuscule.
\loc{Body}
\begin{Latex}{Code pour récuperer le chapitre courant}
\currentChapter
\end{Latex}


%############################################################
%###### Package 'Layout' 
%###### This package contains some tools to set page layout or text
%###### Author  : Nicolas LE GUERROUE
%###### Contact : nicolasleguerroue@gmail.com
%############################################################
\RequirePackage{lmodern}
\RequirePackage{graphicx}           %Images
\RequirePackage{caption}            %legend
\RequirePackage{textcomp}           %special characters
\RequirePackage{fancyhdr}           %headers and footers
\RequirePackage{lastpage}           %page counter
\RequirePackage{float}              %image floating
\RequirePackage{wrapfig}            %Float figures
\RequirePackage{subcaption}         %Subcaption
\RequirePackage{geometry}
%############################################################
\typeout{>>> Utils: Package 'Header' loaded !}
%############################################################

\newcommand{\setHeader}[3]{  %Set minimal page header #title author date 
  \title{#1}
  \author{#2}
  \date{#3}
  \maketitle
}

\newcommand{\partImg}[3]{  %Set minimal part page with image #subtitle imgSource ratio
    \part[#1]{#1 \\ \vspace*{2cm} \makebox{\centering \includegraphics[width=#3\textwidth]{#2}}}
}

\newcommand{\setHeaderImage}[6]{ %Set header with image #imagePath Title subtitle authors info
\begin{titlepage}
  \begin{sffamily}
  \begin{center}
    \includegraphics[scale=#2]{#1} \sn \sn
    \hfill
    \vspace{3cm}
%\HRule \\[0.4cm]
\begin{center}
    {\Huge \textbf{#3}} \sn
    \textbf{#4}\sn \sn
\end{center}
\sn \sn
 #5 \sn
   \vfill
   {\large #6}
  \end{center}
  \end{sffamily}
\end{titlepage}
}

\newcommand{\addPresentation}[6]{ %Set document presentation (header and footer) #rightHeaderContent centerHeaderContent leftHeaderContent rightFooterContent centerFooterContent leftFooterContent
\fancypagestyle{classic}{
    \rhead{#3}  
    \lhead{#1}
    \chead{#2}
    \rfoot{#6}  
    \cfoot{#5}
    \lfoot{#4}
}

\@ifclassloaded{report}{
\makeatletter
\renewcommand\chapter{
  \if@openright\cleardoublepage\else\clearpage\fi
                      \thispagestyle{classic} %Thème 'classic'
                      \global\@topnum\z@
                      \@afterindentfalse
                      \secdef\@chapter\@schapter}
\makeatother
}%End renew chapter

\@ifclassloaded{book}{
\makeatletter
\renewcommand\chapter{
  \if@openright\cleardoublepage\else\clearpage\fi
                      \thispagestyle{classic} %Thème 'classic'
                      \global\@topnum\z@
                      \@afterindentfalse
                      \secdef\@chapter\@schapter}
\makeatother
}%End renew chapter

\pagestyle{classic}
}


\newcommand{\setRightHeader}[1] %Set right header content #content
{
  \rhead{#1}
} 
\newcommand{\setCenterHeader}[1] %Set center header content #content
{
  \chead{#1}
} 
\newcommand{\setLeftHeader}[1] %Set left header content
{
  \lhead{#1}
} 

\newcommand{\setRightFooter}[1] %Set right footer content #content
{
  \rfoot{#1}
} 
\newcommand{\setCenterFooter}[1] %Set center footer content #content
{
  \cfoot{#1}
} 
\newcommand{\setLeftFooter}[1] %Set left footer content #content
{
  \lfoot{#1}
} 



\newcommand{\setHeaderLine}[1] %Enable header line (disable with arg to 0) #width
{  
  \renewcommand{\headrulewidth}{#1pt} 
}
\newcommand{\setFooterLine}[1] %Enable footer line (disable with arg to 0) #width
{ 
  \renewcommand{\footrulewidth}{#1pt} 
}
  
\newcommand{\currentChapter} %Get current chapter name #
{
  \leftmark
} 

\@ifclassloaded{report}{

\newcommand{\setAliasChapter}[1] %create alias chapter #name
{ 
  \makeatletter
  \renewcommand{\@chapapp}
  {
    #1
  }  
  \makeatother
}
}%End if
{
  \newcommand{\setAliasChapter}[1]{
}
}

\@ifclassloaded{book}{
\newcommand{\currentChapter}{\leftmark}
\newcommand{\setAliasChapter}[1]{
\makeatletter
\renewcommand{\@chapapp}{
  #1
  }  
\makeatother
}
}%End if


\newcommand{\currentPage} %Get current page number on total #
{ 
  \thepage{}
} 

