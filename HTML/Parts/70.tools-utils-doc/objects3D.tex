\chapter{Bibliothèque Object3D}

La bibliothèque \lib{Object3D} gère les figures 3D et les graphiques 3D.

\section{Affichage d'un graphique 3D avec insertion des données depuis une équation}

\plot{Titre 3D}{sin(x)*cos(y)}


\loc{Body}
\begin{Latex}{Code pour l'affichage graphique 3D avec insertion des données depuis une équation}
\plot{Titre 3D}{sin(x)*cos(y)}
\end{Latex}

\section{Affichage de sphères en 3D}

\ball{red}{2}
\ball{green}{3}
\ball{blue}{4}
\ball{darkBlue}{5}

\loc{Body}
\begin{Latex}{Code pour l'affichage de sphères en 3D}
\ball{red}{2}
\ball{green}{3}
\ball{blue}{4}
\end{Latex}

Il est possible d'utilise ces balles pour décorer les listes : \\

\begin{items}{darkBlue}{\ball{blue}{1.5}}
    \item A
    \item B
    \end{items}

\loc{Body}
\begin{Latex}{Code pour décorer les listes avec les balles}
\begin{items}{darkBlue}{\ball{blue}{1.5}}
\item A
\item B
\end{items}
\end{Latex}


%############################################################
%###### Package 'Object3D' 
%###### This package contains tools to display 3D Objects
%###### Author  : Nicolas LE GUERROUE
%###### Contact : nicolasleguerroue@gmail.com
%############################################################
\RequirePackage{tikz}  %Schematic
\RequirePackage{pgfplots}  %Plot
\RequirePackage{pgf}  %Plot
\RequirePackage{graphics} %include figures
\RequirePackage{graphicx} %include figures
\RequirePackage{caption}    %caption
\RequirePackage{subcaption} %Subcaption
%################################################################
\typeout{>>> Utils: Package 'Objects3D' loaded !}
%################################################################
\pgfplotsset{compat=1.7}
%################################################################
\newcommand{\ball}[2] %Create a 3D ball #color size
{  
    \tikz\path[shading=ball,ball color=#1] circle (#2mm);
}
%############################################################

\newcommand{\plot}[2] %Create a 3D plot #title equation
{
    \raiseMessage{Creating new plot [title='#1']}
    \begin{tikzpicture}
    \setGraphic %command to display with frenchb babel
     \begin{axis}[title={#1}, xlabel=x, ylabel=y]
     \addplot3[surf,domain=0:360,samples=50]
     {#2};
     \end{axis}
     \end{tikzpicture}
     }
        