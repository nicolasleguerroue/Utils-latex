\chapter{Bibliothèque Titles}

La bibliothèque \lib{Titles} gère les titres.\\

\subsection{Titre de chapitre}
\loc{Body}
\begin{Latex}{Code pour l'ajout d'un titre}
\chapter{Titre}
\end{Latex}

\section{Titre de section}
\loc{Body}
\begin{Latex}{Code pour l'ajout d'une section}
\section{Titre de section}
\end{Latex}

\section{Couleur de section}

Il est possible de changer la couleur des sections avec la commande suivante : \\

\loc{Settings/SectionColor.tex}
\begin{Latex}{Code pour la modification des couleurs des section}
\setColorSection{blue}
\end{Latex}


\subsection{Titre de sous-section}

\loc{Body}
\begin{Latex}{Code pour l'ajout d'une sous-section}
\subsection{Titre de sous-section}
\end{Latex}

\section{Couleur de sous-section}

Il est possible de changer la couleur des sous-sections avec la commande suivante : \\

\loc{Settings/SubSectionColor.tex}
\begin{Latex}{Code pour la modification des couleurs des section}
\setColorSubSection{blue}
\end{Latex}


\subsubsection{Titre de sous-sous-section}
\loc{Body}
\begin{Latex}{Code pour l'ajout d'une sous-sous-section}
\subsection{Titre de sous-sous-section}
\end{Latex}


\section{Références des titres}

Il est possible de faire référence au nom du chapitre ci celui-ci est explicité avec la mention \bold{label}.\\

Ceci est une référence à la section appelée \titleName{Fonts}.\\

\loc{Body}
\begin{Latex}{Code pour la référence à un titre}
Ceci est une référence à la section appelée \titleName{Fonts}.
\end{Latex}

Avec la bibliothèque \lib{Labels}, vous pouvez mettre un label pour rendre plus visible le lien : 

Un lien la bibliothèque \link{\titleName{Fonts}}.

\loc{Body}
\begin{Latex}{Code pour la référence à un titre}
Un lien vers la bibliothèque \link{\titleName{Fonts}}.
\end{Latex}

\section{Paramétrage de la numérotation des titres}

\loc{Settings/TitlePrefix.tex}
\begin{Latex}{Code pour la numérotation des sections}
\enableSectionPrefix
%\disableSectionPrefix
\end{Latex}

%############################################################
%###### Package 'Titles' 
%###### This package contains some tools to set title color
%###### Author  : Nicolas LE GUERROUE
%###### Contact : nicolasleguerroue@gmail.com
%############################################################
\RequirePackage[explicit]{titlesec} %Colors of titles
%################################################################
\typeout{>>> Utils: Package 'Titles' loaded !}
% \@ifclassloaded{report}{
%     \typeout{>>> Utils: CHAPTER!}
%     \titleformat{\chapter}[display] {\fontsize{17pt}{12pt}\selectfont \bfseries}{\textcolor{blue} {\chaptertitlename\ \thechapter: #1}}{20pt}{\Huge}
% }


%Default part newcommand{part} #partName
%Default chapter newcommand{chapter} #chapterName
%Default section newcommand{section} #sectionName
%Default subsection newcommand{subsection} #subSectionName
%Default subsubsection newcommand{subsubsection} #subSubSectionName

\newcommand{\prefixSection} %Content of prefix section # 
{\thesection}  

\newcommand{\enableSectionPrefix}{ %Enable section prefix #
    \renewcommand{\prefixSection}{\thesection}  
}
\newcommand{\disableSectionPrefix}{ %Disable section prefix #
    \renewcommand{\prefixSection}{}  
}



\newcommand{\prefixSubSection} %Content of prefix subsection # 
{\thesubsection}  

\newcommand{\enableSubSectionPrefix}{ %Enable subsection prefix
    \renewcommand{\prefixSubSection}{\thesubsection}  
}
\newcommand{\disableSubSectionPrefix}{ %Disable subsection prefix
    \renewcommand{\prefixSubSection}{}  
}

\newcommand{\sectionColor} 
{darkBlue}  

\newcommand{\setSectionColor}[1]{ %Set section color #color
    \renewcommand{\sectionColor}{#1}  
}

\newcommand{\subSectionColor} 
{orange}  
\newcommand{\setSubSectionColor}[1]{    %Set subsection color #color
    \renewcommand{\subSectionColor}{#1}  
}

\titleformat{\section}[display] {\fontsize{17pt}{12pt}\selectfont \bfseries}{\textcolor{\sectionColor} {\prefixSection~#1}}{20pt}{\Huge}
\titleformat{\subsection}[display] {\fontsize{15pt}{12pt}\selectfont \bfseries}{\textcolor{\subSectionColor} {\prefixSubSection~#1}}{20pt}{\large}

\titlespacing*{\section}{0pt}{20pt}{-30pt}
\titlespacing*{\subsection}{0pt}{20pt}{-20pt}


\newcommand{\hsp}{\hspace{20pt}}
\titleformat{\chapter}[hang]{\Huge\bfseries}{\thechapter\hsp\textcolor{red}{|}\hsp}{0pt}{\Huge\bfseries}

\newcommand{\titleRef}[1] %Get number of title #titleRef
{
    \ref{#1}
}

\newcommand{\titleName}[1] %Get content of title #titleRef
{
    \nameref{#1}
}



