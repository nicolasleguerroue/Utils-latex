\chapter{Compilation}

La compilation du projet se fait grâce au fichier \lib{make} situé à la racine du projet.\\
L'ensemble des outils de compilation sont disponibles de deux façons :

\begin{items}{blue}{\Bullet}
    \item Via un terminal
    \item Via le logiciel VSCode
\end{items}

L'avantage du logiciel VScode est qu'il fournit des raccourcis clavier et une interface graphique plus évoluée (boutons)


\section{Compilation avec un terminal}


\subsection{Compilation classique}

Une compilation classique a pour objectif de générer le fichier PDF de rendu, appelé \file{main.pdf} et situé à la 
racine du projet.\\

La commande est la suivante :

\begin{Bash}{Compilation complète du projet}
./make --full
\end{Bash}

Lors de la compilation, plusieurs fichiers sont générés à la racine, dont : 

\begin{items}{darkBlue}{\Triangle}
    \item Le fichier \file{.render\_report.tex} (fichier caché) qui contient la première partie des fichiers journaux de compilation
    \item Le fichier \file{.render\_report\_logs.tex} qui contient la seconde partie des fichiers journaux de compilation\footnote{Les messages de compilation générés par la bibliothèque Utils sont situés dans ce fichier.}
    \item Une image \italic{Part.png} qui affiche le nombre de ligne pour chaque fichier compilé contenu dans le dossier \bold{Parts}
    \img{\rootImages/Part.png}{Nombre de ligne pour les parties}{0.7}
    \item Une image \italic{Utils.png} qui affiche le nombre de ligne pour chaque fichier contenu dans le dossier \bold{Utils}
    \img{\rootImages/Utils.png}{Nombre de ligne pour les bibliothèques}{0.7}
    \item Un fichier \file{standlone.tex} qui contient l'ensemble du code latex généré. Ne pas supprimer ce fichier, il est utilisé pour la conversion du code Latex en fichier source HTML.
\end{items}

Lors de la compilation, différents messages s'affichent : 

\img{\rootImages/messages.png}{Message d'ajout d'élements de la bibliothèque Utils}{0.5}
\img{\rootImages/warnings.png}{Message d'avertissements}{0.5}

\subsection{Ajout de version}\label{addVersion}

Il est possible d'ajouter une version de projet en invoquant la commande suivante : 

\begin{Bash}{Mise à jour Git}
./make --version
\end{Bash}

La date de la mise à jour vous sera demandée ainsi que le contenu de la mise à jour.

\img{\rootImages/addVersion.png}{Ajout d'une version}{0.5}


\label{setLayout}
\subsection{Mise à jour de l'autocomplétion}

Il est possible de mettre à jour l'autocomplétion sous VScode pour les bibliothèques Utils.
En invoquant le paramètre \bold{--snippet}, il est possible de générer le fichier VScode qui va ajouter l'autocomplétion.\\
Ce fichier est appelée \file{output.snippet-code} et se situe dans le dossier \dir{.vscode}

\begin{Bash}{Mise à jour de l'autocomplétion}
./make --snippet
\end{Bash}

La commande se termine en affichant le nombre de commandes documentées.

%\img{\rootImages/snippetsCount.png}{Fin de la commande de génération de l'auto-complétion}{0.3}






\section{Compilation avec VSCode}

L'ensemble des commandes présentées précédemment sont disponibles sous VSCode.\\

\subsection{Ouverture d'un projet}

Il faut ouvrir un terminal à la racine du projet et lancer la commande : 

\begin{Bash}{Ouverture de l'arborescence avec VScode}
code .
\end{Bash}

\messageBox{\faviconInfo}{green}{green}{Lors de chaque ouverture du répertoire de projet VSCode, un script de vérification des fichiers va se lancer. Les fichiers manquants seront regénérés à leur valeur par défaut.}{white}

Le logiciel se lance avec une interface similaire :

\img{\rootImages/vscode_opened.png}{Le logiciel VSCode ouvert}{0.45}

On observe sur la gauche l'arborescence du projet. En cliquant sur les dossiers, ces derniers se déroulent et affichent leur contenu.

\img{\rootImages/vscode_part.png}{L'arborescence du projet}{0.5}

    
\subsection{Lancement de la compilation}

Il y deux méthodes pour lancer les outils de compilation sous VSCode :

\begin{items}{blue}{\Bullet}
    \item Via les raccourcis clavier (Tâches)
    \item Via les boutons sur l'interface graphique
\end{items}

\subsection{Les raccourcis clavier}

\label{allTasks}
L'ensemble des commandes et outils sont disponibles en saisissant le raccourci \shortcut{CTRT+SHIFT+B} \img{\rootImages/example.png}{Visualisation des commandes}{0.35}


Les commandes sont classées par catégorie, à savoir : 

\begin{items}{blue}{\Triangle}

    \item \bold{Compilation}

    Les commandes pour compiler le projet. 
    \item \bold{Settings}

    Les commandes pour définir les paramètres du projet.
    \item \bold{Display}

    Les commandes pour afficher ou non des éléments (liste des figures...)
    \item \bold{Part}

    Les commandes pour gérer les parties
\end{items}

\subsection{Les commandes de compilation}

Les différentes commandes sont de 3 types :

\begin{items}{green}{\faviconLeaf}
\item Une commande qui lance un script en arrière-plan, sans argument, repérée avec la mention \badge{white}{darkOrange}{script}
\item Une liste avec valeurs déterminées, repérées avec la mention \badge{white}{cyan}{élement}.\\La valeur par défaut est de la couleur bleue foncée (\badge{white}{blue}{Defaut})
\item Zone de saisie utilisateur, repérée avec la mention \badge{white}{purple}{Entrée utilisateur}
\end{items}

\begin{items}{blue}{\Triangle}

    \item Compilation
    \begin{items}{cyan}{\Triangle}
        \item \bold{Full}

        Lance une compilation complète\\ \badge{white}{darkOrange}{script}
        \item \bold{Display HTML render}

        Lance la page d'affichage du rendu HTML du code Latex. Avant de faire cette commande, il faut faire la commande \bold{Compilation > Convert Latex to HTML}\\ \badge{white}{darkOrange}{script}
        \item  \bold{Enable all files}

        Autorise la compilation complète du répertoire \dir{Parts}\\ \badge{white}{darkOrange}{script}
        \item  \bold{Disable all files}

        Empêche la compilation complète du répertoire \dir{Parts}\\ \badge{white}{darkOrange}{script}
        \item  \bold{Display PDF}

        Ouvre le PDF de sortie (main.pdf) avec le logiciel \lib{okular}\\ \badge{white}{darkOrange}{script}
        \item  \bold{Display errors}

        Affiche le fichier de sortie de compilation\\ \badge{white}{darkOrange}{script}
        \item  \bold{Convert Latex to HTML}

        Lance les scripts python pour convertir le Latex en HTML\\ \badge{white}{darkOrange}{script}
        \item  \bold{Light (without bibliography, index and nomenclature)}

        Lance une compilation rapide sans bibliographie, index et nomenclature. Cette compilation ne compile pas l'intégralité du document, la table des matière sera donc absente mais cela a le mérite de compiler rapidement.\\ \badge{white}{darkOrange}{script}
    \end{items}
\end{items}

Pour trier les commandes disponibles, il faut saisir le type de commande (Compilation, Settings...) et les commandes seront triées.

\img{\rootImages/categories.png}{Trie des commandes}{0.5}

\subsection{Les commandes de configuration du projet}\label{setSettings}

Il est possible d'éditer de nombreux paramètres de manière graphique sous VScode.
Les configurations se font via les commandes de type \bold{Settings}

    \begin{items}{black}{\faGear}
        
        \item \bold{Depth of TOC}

        Définit la profondeur du sommaire.

        \badge{white}{cyan}{-1} \badge{white}{cyan}{0} \badge{white}{blue}{1} \badge{white}{cyan}{2}

        La profondeur du sommaire correspond aux types de titre affichés.

        \begin{tabular}{|c|c|c|c|c|}\hline
         Profondeur & Parties & Chapitres & Sections & Sous-sections\\ \hline
        -1 & \colors{green}{\faviconCheck} & \colors{red}{\faviconClose} & \colors{red}{\faviconClose} & \colors{red}{\faviconClose}\\ \hline
        0 & \colors{green}{\faviconCheck} & \colors{green}{\faviconCheck} & \colors{red}{\faviconClose} & \colors{red}{\faviconClose}\\ \hline
        1 & \colors{green}{\faviconCheck} & \colors{green}{\faviconCheck} & \colors{green}{\faviconCheck} & \colors{red}{\faviconClose}\\ \hline
        2 & \colors{green}{\faviconCheck} & \colors{green}{\faviconCheck} & \colors{green}{\faviconCheck} & \colors{green}{\faviconCheck}\\ \hline
        \end{tabular}
        \item \bold{Update of font}

        Définit la police du document

        \badge{white}{blue}{calibri} \badge{white}{cyan}{default}

        \item  \bold{Update of theme}

        Définit le thème du document

        \badge{white}{blue}{Lenny} \badge{white}{cyan}{Glenn} \badge{white}{cyan}{Sonny}
        \badge{white}{cyan}{Conny} \badge{white}{cyan}{Rejne} \badge{white}{cyan}{Bjarne}

        \item  \bold{Update of margins}

        Définit les marges en centimètre du document. Il est possible de définir les marges verticales et horizontales.

        Marges horizontales : \badge{white}{blue}{2} \\
        Marges verticales : \badge{white}{blue}{2}

        \item  \bold{Update of metadata}

        Ouvre le fichier de définition des métadonnées

        \begin{tabular}{cc}
            Titre du fichier PDF & \badge{white}{blue}{main} \\
            Auteur du fichier & \badge{white}{blue}{username}\footnote{Pour ajouter des auteurs, il faut séparer avec une virgule} \\
            Sujet du document & \badge{white}{blue}{main}\\
        \end{tabular}

        \item  \bold{Update of line widht}

        Définit l'épaisseur des traits pour les en-tête et pied de page.

        \badge{white}{purple}{Valeur à saisir} \badge{white}{blue}{0.2}
        \item  \bold{Update of document class}

        Définit la classe du document

        \badge{white}{blue}{utils\_report} \badge{white}{cyan}{report} \badge{white}{cyan}{utils\_article} \badge{white}{cyan}{article}
        \badge{white}{cyan}{beamer} \badge{white}{cyan}{utils\_book} \badge{white}{cyan}{book}
        \item  \bold{Update of nomenclature}

        Définit le nom de la nomenclature.

        \badge{white}{purple}{Valeur à saisir} \badge{white}{blue}{Nomenclature}

        \item  \bold{Update of section color}

        Définit la couleur des sections

        \item  \bold{Update of subsection color}

        Définit la couleur des sous-sections

    \end{items}

\subsection{Les boutons}

Au lieu d'utiliser les raccourcis avec \shortcut{CTRT+SHIFT+B}, il est possible d’exécuter les outils avec les boutons en bas de VSCode : 

\img{\rootImages/buttons.png}{Visualisation des boutons}{0.4}



\section{Autocomplétion des commandes}

Lors du démarrage de l'espace de travail, un script va générer le fichier pour l'autocomplétion des commandes.\\
Chaque commande commence par le mot clé "Utils" puis un point et le nom de la bibliothèque associée. \\

Par exemple, pour obtenir les commandes de la bibliothèque \lib{Images}, en écrivant "Images" dans un document Latex vous obtiendrez le résultat suivant : 

\imgn{\rootImages/Utils_lib.png}{Autocomplétion des commandes}{0.4}{autocompletion}

Vous pouvez parcourir les commandes disponibles puis saisir la touche \shortcut{Enter}.\\
La commande sera écrite à la place du curseur.





