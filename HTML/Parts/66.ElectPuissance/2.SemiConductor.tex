%\part{Electronique de puissance}
\chapter{Les composants semi‐conducteurs de puissance}

Les semi-conducteurs abordés içi seront utilisé en mode commutation (pas d'amplification).\\

Il sont classés en 3 catégories selon le mode de commutation d’un état à un autre : 


\begin{items}{blue}{\Triangle}
    \item Non commandable (la diode)
    \item Semi-commandable (le thyristor « classique », et le triac)
    \item Entièrement commandable (le thyristor blocable par la gâchette, et le transistor de
    puissance).
\end{items}

\section{Les diodes de puissance}

\subsection{Présentation}

La diode de puissance est le semi-conducteur qui présente une structure la plus simple
de toutes. \\
Elle est constituée de deux couches semi-conductrices dopées p-n
respectivement, qui forment ainsi une jonction pn. \\
Voici quelques applications des diodes de puissance.

\begin{items}{blue}{\Triangle}
    \item Les montages redresseurs
    \item Les alimentations à découpage
    \item Les diodes de roues libre (assurer la continuité de courant dans une charge inductive)
\end{items}

\subsection{Broches}

La diode de puissance présente deux électrodes dites anode et cathode. 

\begin{items}{blue}{\Triangle}
  \item La tension inverse
  \item Le courant moyen
  \item La vitesse de commutation
\end{items}

\img{\rootImages/diode.png}{Schéma d'une diode}{0.5}

\subsection{Les critères de conduction}

La diode de puissance entre en conduction dès que \bold{la tension vD entre l’anode et la
cathode est positive}. En mode conduction, la chute de tension directe vd entre l’anode
et la cathode est de l’ordre de 1,5 à 3 V. \\
Dans les études suivantes, la chute de tension aux bornes des diodes sera considérée comme nulle.


Elle se met en mode blocage à \bold{l’extinction de son courant iD}. Dans ce mode, la diode de puissance présente une tension inverse de
plusieurs kilovolts.

\img{\rootImages/carac.png}{Caractéristique d'une diode}{0.5}

\section{Le thyristor de puissance}

Du point de vue structure, le thyristor de puissance est constitué de quatre couches semi-
conductrices, respectivement dopées p-n-p-n, pour former trois jonctions.\\

\subsection{Broches}
Du point de vue utilisation, ce composant présente trois électrodes : \bold{l’anode, la cathode, et la
gâchette}

\begin{items}{blue}{\Triangle}
    \item Les convertisseurs de courant
    \item Les hacheurs, 
    \item les gradateurs
    \item Les onduleurs pour des applications grande puissance
\end{items}

À noter que les thyristors conçus pour des applications à 60 Hz (50 Hz), tels les
convertisseurs de coutant et les gradateurs, ont un pouvoir en tension et en courant
beaucoup plus grand que ceux conçus pour applications à fréquence élevée. Le pertes par commutations deviennent non négligeables.

\subsection{Les critères de conduction}

Deux conditions sont nécessaires pour qu’un thyristor entre en mode conduction : 


\begin{items}{blue}{\Triangle}
  \item Une tension vT positive entre l’anode et la cathode
  \item une impulsion de courant (ou un train
  d’impulsions), suffisante en amplitude et en durée, appliquée à la gâchette. La tension
  et le courant de gâchette sont de l’ordre de quelques volts et une centaine de
  milliampères respectivement. \\
\end{items}

Une fois à l’état passant, le thyristor conduit tant que son courant iT est supérieur au
courant de maintien im. La chute de tension directe vd entre l’anode et la cathode qui en
résulte est de l’ordre de 1,5 à 3 V.


Comme dans le cas de la diode de puissance, le thyristor de puissance se met en mode
blocage à l’extinction de son courant iT (Il est possible d'injecter un courant négatif pour couper le thyristor -> commutation forcée). Dans ce mode, le thyristor de puissance
présente une tension inverse de plusieurs kilovolts.

GTO : Gate Turn Off


\subsection{Commutation naturelle}



\section{Les transistors bipolaires}

Une des moyens pour créer notre circuit de puissance est le transistor bipolaire. \index{bipolaire}. Ce composant possède trois broches : 

\begin{items}{blue}{\Triangle}

  \item Le collecteur (C)
  \item la base (B)
  \item l'émetteur (E)

\end{items}

\img{\rootImages/bjt.png}{La représentation du transistor bipolaire}{0.1}

\subsection{Conventions}

Afin de simplifier les calculs par la suite, posons les normes suivantes : 

\begin{items}{blue}{\Triangle}

  \item Le courant entrant dans le Collecteur est appelé $I_{C}$
  \item Le courant entrant dans la Base est appelé $I_{B}$
  \item Le courant sortant de l'émetteur est appelé $I_{E}$

  \item La tension entre la Base et l’Émetteur est appelée $V_{be}$
  \item La tension entre le Collecteur et l’Émetteur est appelée $V_{ce}$
\end{items}

\img{\rootImages/courants.png}{Conventions du transistor bipolaire}{0.6}


Les flèches au sein du transistor indiquent le sens de déplacement du courant sur les broches.

\subsubsection{Les familles de transistors bipolaires}

Les transistors bipolaires sont classés en deux catégories : 

\begin{items}{blue}{\Triangle}

  \item Les transistors NPN\footnote{Le nom de ces familles provient du type de jonction utilisé en interne. Pour plus de renseignements, consulter les diodes et semi-conducteurs}
  \item Les transistors PNP

\end{items}

Le principe de fonctionnement est similaire entre ces deux familles, seul le branchement et le niveau de commande diffère.\\ Dans ce document, nous utiliserons essentiellement des transistors NPN car ces derniers utilisent des grandeurs positives.

\img{\rootImages/pnppnp.jpeg}{Transistors NPN et PNP}{0.6}


\subsection{Les paramètres de sélection du transistor}

Notre transistor doit dans un premier temps répondre à deux contraintes : 

\begin{items}{blue}{\Triangle}

  \item La tension admissible sur $V_{ce}$\footnote{Cette tension est indiquée dans les documentations techniques} doit être supérieure à la tension d'alimentation de notre circuit.\\
  Concrètement, si notre circuit est alimenté en 48V mais que le transistor ne supporte pas plus de 30V, il va être détruit.
  \item Le transistor doit supporter un courant plus élevé que le courant maximal transitant dans notre circuit.
  Pour contrôler un moteur consommant 1 Ampère, je dois donc choisir un transistor pouvant contrôler au moins 2 Ampère.
\end{items}

Pour la suite de la présentation, on supposera que notre transistor a été dimensionné pour répondre à ces deux contraintes.


\subsection{Le principe}

Ce type de transistor fonctionne comme une vanne pour une canalisation. Il est possible de réguler le débit de la canalisation avec la vanne.\\

Le transistor bipolaire permet de contrôler un courant important avec un faible courant.\\

\img{\rootImages/barrage.png}{Le rôle du transistor}{0.3}.

Ici, notre transistor joue le rôle de la vanne et permet de bloquer le courant (électrons) ou bien de les laisser passer. \\


Le courant de l'élément à contrôler (moteur, résistance de puissance) transite entre le collecteur et l'émetteur et le courant de commande passe par la base, comme l'illustre la figure suivante.\\

\img{\rootImages/courant_main.png}{Courant de commande et de puissance}{0.6}

La relation fondamentale reliant le courant de puissance et de commande est la suivante : 

$$ \boxed{ I_{C} = \beta \cdot I_{B} }$$

Le paramètre $\beta$, appelé \bold{gain du transistor}\footnote{le gain \index{gain (transistor)}est sans dimension (unité) et est appelé $ h_fe$ dans les documentations} est une caractéristique interne de notre transistor, c'est à dire qu'il dépend du type de transistor que nous choisissons.\\
Les courants $I_{C}, I_{B},I_{E}$ sont exprimés dans la même unité (Ampère, milliampères..) pour une formule homogène.\\

Les transistors de puissance possède des gains de l'ordre de la dizaine alors que les transistors de signal (faibles courants) ont un gain pouvant facilement atteindre 200 ou 300.

\messageBox{Remarque}{cyan}{white}{Plus notre $\beta$ est faible, plus il va falloir injecter un courant important dans notre base}{black}

\Question{Et que devient notre broche Émetteur" ?}

\begin{reponse}
  
  Notre émetteur est relié à la masse du circuit et permet de le fermer pour que les électrons puissent circuler.\\

  Le courant circulant dans l'émetteur est simplement la somme des courants entrant dans le transistor. \\
  d'où : $$ \boxed{ I_{E} = I_{B} + I_{C} }$$

  \end{reponse}



\subsection{Exemple}

On souhaite commander l'arrêt et la marche d'un moteur consommant au maximum 0.5A et alimenté avec une tension de 9V. \\
Nous choisissons un transistor permettant de commuter 1A (sécurité) avec $\beta=30$ \\

\Question{Quel doit-être le courant injecté dans la base ?}


\begin{reponse}
  On applique la formule précédent et on obtient : 

  $$  I_{B} = \frac{I_{C}}{\beta} = \frac{0.5}{30} = 16 mA $$

  \end{reponse}

 

\subsection{Mise en pratique}

\subsubsection{Branchements}

    Maintenant que nous connaissons les tensions et courants nécessaires à notre transistor et à notre moteur, nous allons le commander avec une carte Arduino. \\

    Tout d'abord, il convient de placer le moteur entre notre alimentation et le collecteur.\\
    

\messageBox{Remarque}{cyan}{white}{Toutes les charges à contrôler avec ce type de transistor se placent entre l'alimentation et le collecteur.}{black}

    Enfin, il ne nous reste plus qu'à relier une sortie numérique de l'Arduino vers notre base par l'intermédiaire d'une résistance.

    \bold{La résistance va servir à imposer le courant dans la base de notre transistor}.

    Nous obtenons donc le schéma suivant.

    \img{\rootImages/schema_pnp.png}{Branchement du transistor bipolaire}{0.4}

    \subsection{Dimensionnement de la résistance}

    On souhaite obtenir un courant de $16 mA$ dans notre base et on sait que l'Arduino délivre du $5V$ en sortie.\\

    Nous somme donc tentés de dire que $R_b = \frac{U_{arduino}}{ I_{B}} = \frac{5}{0.016} = 312 \Omega$ \footnote{On part de la loi de Ohm qui dit que $U=R.I$}\\


    Hélas, il y a peu de chance que votre moteur tourne dans les conditions optimales.\\
    Il convient d'avoir à l'esprit que notre $\beta$ trouvé dans la documentation n'est que théorique et qu'il peut être en réalité inférieur.

    \messageBox{Remarque}{cyan}{white}{Une des conventions non officielles admet que pour de la commutation en Tout ou Rien, on divise la valeur théorique de notre $\beta$ par 2. \\Nous allons donc prendre donc un $\beta$ valant 15.}{black}
    
    On refait donc les calculs.

    $$  I_{B} = \frac{I_{C}}{\beta} = \frac{0.5}{15} = 32 mA $$

    Une dernière chose : les transistors bipolaires entraînent une chute de tension entre la base et l'émetteur ($V_{be}$).\\
    Cette chute de tension dépend de la technologie des transistors bipolaires : 

    \begin{items}{blue}{\Triangle}

      \item $0.7V$ pour les transistors au silicium
      \item $0.3V$ pour les transistors au germanium
    \end{items}
    Dans l'extrême majorité des cas, on utilisera des transistors au silicium. La tension disponible aux bornes de la résistance est donc de $4.3V$ ($5-0.7$)

    D'où : 

    $$ \boxed{ R_{b} = \frac{U_{arduino}-V_{be}}{I_b} = \frac{4.3}{0.032} = 134 \Omega} $$


Il faut dissiper la chaleur du transistor


Protection contre les régimes transitoirs : Varistor 



\section{Les transistors MOSFET}

     Nous avons vu l'utilisation des transistors bipolaires. \\
     Ces derniers sont assez contraignants à mettre en oeuvre car ils sont commandés en courant.

     Nous allons utiliser cette fois-ci la technologie des \glossary{MOSFET} \footnote{MOSFET : Metal Oxide Semiconductor Field Effect Transistor = Transistor à effet de champ à structure métal-oxyde-semi-conducteur} car ces derniers ont l'avantage d'être contrôlés en \bold{tension}.

     Ce composant possède trois broches : 
     
     \begin{items}{blue}{\Triangle}
     
       \item Le drain (D)
       \item la porte (G)\footnote{G pour Gate}
       \item la source (S)
     
     \end{items}
     
     \img{\rootImages/mosfet.png}{La représentation du transistor MOSFET}{0.1}
     
     \subsection{Conventions}
     
     Afin de simplifier les calculs par la suite, posons également les normes suivantes : 
     
     \begin{items}{blue}{\Triangle}
     
       \item Le courant entrant dans le Drain est appelé $I_{D}$
       \item Le courant entrant dans la Porte est appelé $I_{G}$
       \item Le courant sortant de la Source est appelé $I_{S}$

       \item La tension entre la Porte et la Source est appelée $V_{GS}$
       \item La tension entre le Drain et la Source est appelée $V_{DS}$
     \end{items}
     
     \subsubsection{Les familles de transistors MOSFET}

     Les transistors MOSFET sont classés en deux catégories : 
     
     \begin{items}{blue}{\Triangle}
     
       \item Les transistors MOSFET à canal N \footnote{Le nom de ces familles provient du type de jonction utilisé en interne. Pour plus de renseignement, consulter les diodes et semi-conducteurs}
       \item Les transistors MOSFET à canal P
     
     \end{items}
     
     Le principe de fonctionnement est similaire entre ces deux familles, seul le branchement et le niveau de commande diffère.\\
     Dans ce document, nous utiliserons essentiellement des transistors MOSFET à canal N car ces derniers utilisent des grandeurs 
     positives.
     
     \img{\rootImages/mos.png}{Transistors à canal N et P}{0.6}

     \section{Les paramètres de sélection du transistor}
     
     Les paramètres de sélection de nos transistors MOSFET sont identiques aux transistors bipolaires, c'est à dire :
     
     \begin{items}{blue}{\Triangle}
     
       \item La tension admissible sur $V_{DS}$ du transistor
       \item Le courant admissible entre le Drain et la Source.
     \end{items}
     
     Pour la suite de la présentation, on supposera que notre transistor a été dimensionné pour répondre à ces deux contraintes.


     \subsection{Le principe}

     Ce type de transistor fonctionne comme les transistors bipolaires mais est commandé en tension et non en courant.

     Par analogie, le drain joue le rôle du collecteur, la source celui de l'émetteur et la porte celui de la base.\\
     Le courant de l'élément à contrôler (moteur, résistance de puissance) transite entre le drain et la source et la tension de commande est aux bornes de la porte.\\
     
     Les transistors MOSFET deviennent passant\footnote{Le transistor laisse passer tout le courant autorisé.} lorsque la tension sur la porte dépasse une tension de déclenchement appelée $V_{GS_{th}}$.
     Cette valeur est généralement comprise entre $2$ et $4$ Volts.\\

     \bold{Lorsque cette tension} $V_{GS_{th}}$ est atteinte, notre transistor peut être remplacé d'un point de vue électrique entre le drain et la source par une résistance de très faible valeur, appelée $R_{DS_{on}}$

    \section{Comparaison avec les transistors bipolaires}

    Par nature, la porte du MOSFET est vue comme un condensateur. Le transistor ne consomme pas de courant, excepté pendant les commutations.\\
    Ainsi, le courant est nul dans la porte pour maintenir le moteur en marche alors que pour un bipolaire, il faut maintenir un courant dans la base.\\

    Les MOSFET sont donc plus économes en énergie que les bipolaires.\\
    De plus, ils peuvent généralement supporter des courants plus importants que les bipolaires.\\

    En revanche, en hautes fréquences, les MOSFET sont moins réactifs du fait de leur capacité en entrée.
     \subsection{Mise en pratique}

     Nous souhaitons faire tourner le même moteur que celui utilisé avec notre transistor bipolaire.\\
     Nous allons le commander avec une carte Arduino.

     \subsubsection{Branchements}

     Tout d'abord, il convient de placer le moteur entre notre alimentation et le drain.\\
     
 
    \messageBox{Remarque}{cyan}{white}{Toutes les charges à contrôler avec ce type de transistor se placent entre l'alimentation et le drain.}{black}
 
     Enfin, il ne nous reste plus qu'à relier une sortie numérique de l'Arduino vers notre porte \bold{sans} résistance. Nous obtenons donc le schéma suivant.
 
     \img{\rootImages/schema_mosfet.png}{Branchement du transistor MOSFET}{0.45}



\section{Les redresseurs de puissance}

Transformer une tension alternative en tension continue.


% in, ia, iG, ifd et ifi sont respectivement le courant nominal en mode conduction, le courant
% d’accrochage, le courant de gâchette, le courant de fuite en mode blocage en
% © M. Ouhrouche, 2021polarisation directe, et le courant de fuite en mode blocage en polarisation inverse.
% vpdmax et vpimax sont les tensions maximales en mode blocage en polarisation directe et
% inverse respectivement. Le courant d’accrochage ia est le niveau minimal requis à
% atteindre pour le curant iT afin que l’amorçage ai lieu. Alors que l’application de vpdmax
% aux bornes du thyristor provoquerait un amorçage intempestif même en l’absence d’une
% impulsion de courant de gâchette.
% La figure 2.6 ci-dessous donne un exemple de formes d’ondes du courant et de la
% tension aux bornes d’un thyristor de puissance. La figure montre que le thyristor bloque
% aussi une tension positive. On dit alors que le thyristor est un interrupteur bidirectionnel
% en tension.


% \subsection{Amorçage du thyristor de puissance}

% Il existe divers circuits électroniques pour amorcer un thyristor de puissance. Ces
% circuits diffèrent entre autres les uns des autres par le moyen mis en œuvre pour assurer
% l’isolation galvanique entre l’électronique basse puissance et l’électronique de
% puissance. \\

% Le circuit dans la figure 2.7 ci-après donne un exemple d’un circuit de principe
% d’amorçage utilisant un transformateur d’impulsion TI. La diode au primaire de TI est
% une diode de roue libre pour assurer la continuité de courant au blocage du thyristor.
% Alors que la diode au secondaire de TI redresse le courant de celui-ci. Ces deux diodes
% sont de type rapide.
% © M. Ouhrouche, 2021VC C
% n:1
% TI
% Figure 2.7 Circuit d’amorçage avec transformateur d’isolation
% Le rapport de transformation n doit être judicieusement choisi, puisqu’il joue un rôle
% important dans le conditionnement de l’impulsion de la tension appliquée à la gâchette
% du thyristor.
% 2.3.2 Commutation de courant
% L’extinction du courant dans un thyristor menant au blocage de celui-ci peut se faire
% naturellement ou d’une façon forcée. On parle alors à propos de la commutation
% naturelle et de la commutation forcée selon le cas.
% La commutation naturelle a lieu dans les convertisseurs à entrée alternative, tels que les
% convertisseurs de courant et les gradateurs. Pour illustrer le principe de la commutation
% naturelle, considérons le circuit dans le figure 2.8 ci-après. Initialement, le thyristor T1
% étant passant, et le thyristor T2 en mode blocage. Dans ces conditions, le courant iT1
% dans le thyristor T1 est égal au courant i0. Le thyristor T2 peut être amorcé dès que la
% tension alternative v2 devient supérieure v1. À l’amorçage de T2, son courant iT2 croît
% linéairement de zéro à i0, alors que iT1 décroît de i0 à zéro conformément à la loi des
% courants de Kirchhoff.
% © M. Ouhrouche, 2021i0
% iT1
% iT2
% T2
% T1
% Xs
% v1
% Xs
% v2
% Figure 2.8 Commutation naturelle
% La commutation forcée a lieu dans les convertisseurs à entrée continu, comme les
% hacheurs de courant par exemple. Le blocage d’un thyristor nécessite l’utilisation d’un
% circuit dit d’aide à la commutation, formé à la base d’un circuit oscillant LC.
% Il existe une multitude de circuits d’aide à la commutation, et la figure 2.9 ci-dessous
% en donne un exemple. Ce circuit que nous avions construit dans le cadre d’un projet
% d’ingénierie durant nos études supérieures sera analysé en détail au chapitre 6.
% D
% iTa
% L2
% Tp
% L1
% C
% 
%  
%  
% Ta
% Ti
% Figure 2.9 Circuit d’aide à la commutation
% Tp, Ti, et Ta, désignent respectivement le thyristor principal (à bloquer), le thyristor
% d’inversion, et le thyristor auxiliaire. L’amorçage de Ti permet l’inversion de la polarité
% de la tension aux bornes du condensateur. Le courant du condensateur circule à travers
% © M. Ouhrouche, 2021C, L1, Tp et Ti lors de la première moitié de la période d’oscillation L’amorçage de Ta
% par la suite permet au courant du condensateur de circuler lors de la deuxième moitié
% de la période d’oscillation à travers le thyristor principal Tp en y entrant par sa cathode
% afin d’annuler le courant de celui-ci. La diode D offre un chemin pour le courant du
% condensateur après le blocage de Tp.
% 2.4
% LE THYRISTOR BLOCABLE PAR LA GÂCHETTE
% L’utilisation d’un circuit d’aide à la commutation pour bloquer un thyristor classique
% en résulte des montages encombrants. Le thyristor blocable par la gâchette, connu sous
% le vocable anglais Gate-turn-off (FTO), diffère du thyristor classique à l’effet qu’il est
% entièrement commandable par la gâchette; c’est-à-dire qu’une impulsion de courant de
% gâchette le met en mode passant, et une impulsion négative de courant de gâchette le
% met en mode blocage. Il offre ainsi un avantage par rapport au thyristor classique en
% réduisant l’encombrement. La figure 2.10 donne le symbole d’un thyristor GTO.
% Anode
% iT
% 
% vT
% 
% iG
% 
% Gâchette
% vG
% 
% Cathode
% Figure 2.10 Le thyristor blocable par la gâchette
% Les circuits d’amorçage du thyristor GTO sont similaires à ceux utilisés pour
% l’amorçage du thyristor classique. Quant au circuit de « blocage », il doit pouvoir
% générer une brève impulsion négative de courant (quelques microsecondes seulement),
% mais dont l’amplitude se compare au courant nominal. La figure 2.11 donne un
% exemple de circuit pour générer des impulsions négatives de courant de gâchette pour
% le blocage d’un GTO. Nous avertissons le lecteur qu’il ne s’agit là que d’un circuit de
% principe élaboré à partir des notions de base de circuits électriques et électroniques. Le
% condensateur était initialement lors de la mise en conduction du GTO, et la polarité de
% © M. Ouhrouche, 2021sa tension doit être telle qu’indiquée dans le schéma. La mise en conduction du photo-
% thyristor entraîne une décharge rapide du condensateur à travers la résistance et la
% gâchette du GTO.
% 
% iG  0
% Figure 2.11 Circuit de principe de blocage d’un GTO
% Alors que l’utilisation du thyristor classique s’impose dans des applications de grandes
% puissances à 60 Hz (50 Hz), le thyristor GTO est utilisé dans des applications à fort
% courant et vitesse de commutation élevée, telles les entraînements à courant alternatif,
% et les entraînements à courant continu par hacheur pour des applications en traction
% électriques notamment