
\chapter{Introduction à la conversion statique d'énergie électronique}

\section{Application des moteurs à courant continu}

Les moteurs à courant continu sont adaptés pour les systèmes de levage (grue). Le couple est trè élevé à l'arrêt.
La vitesse de rotation d'une machine dépend de la fréquence.
Il est pertinent de démarrer un moteur en douceur pour éviter les appels de courant.
\section{Les types de transistors}

\begin{items}{blue}{\Triangle}
    \item Les transistors BJT (Base Junction Transistor) sont des transistors bipolaires contrôlés en courant
    \item Les transistors \glossary{MOSFET} sont des transistors commandés en tension
    \item Les transistors IGBT sont des transistors combinant les avantages des  IGBT et des MOSFET
\end{items}

\section{Protection des circuits}


On protège les circuits avec des radiateurs, des fusibles, des condensateurs, etc\\
Pour s'opposer : 
\begin{items}{blue}{\Triangle}
    \item aux fortes variations de tension on utilise des condensateurs
    \item aux fortes variations de courant on utilise des inductances
\end{items}

\section{Qualité d'un signal}

\begin{items}{blue}{\Triangle}
    \item La distorsion harmonique individuelle (DHI) nous renseigne sur l’importance en amplitude de chaque harmonique individuellement par rapport au fondamental

    $$ DHI = \frac{X_{neff}}{X1_{eff}}$$
    \item La distorsion harmonique totale (DHT) nous renseigne sur l’importance en amplitude de chaque harmonique individuellement par rapport au fondamental

    $$ DHT = \frac{\sqrt{ \sum X_{neff}^2}}{X_{eff}}$$
    \item Le facteur de distorsion (FD) nous renseigne sur l’importance du fondamental par rapport au signal ondulé dans son intégralité
\end{items}

\subsection{Les signaux alternatifs}

Les signaux alternatifs ne contiennent que des harmoniques impairs. L’harmonique de
rang n=1, c’est-à-dire V1, est appelé le fondamental, et sa fréquence, dite fréquence
fondamentale, est celle du réseau (60 Hz ou 50 Hz).


$$ x(t) = 120 sin(wt)+40sin(3wt)-25sin(5wt) $$



$$ DHT = $$