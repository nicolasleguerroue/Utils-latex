% \addPartText{Gravure de circuits imprimés}

% \part{Gravure}

\chapter{Gravure de circuits imprimés}

\section{Liste du matériel}

\begin{items}{blue}{\Bullet}
	\item Perchlorure de fer
	\item Carbonate de Sodium
	\item Acétone
	\item Plaque de cuivre 
	\item Perceuse à colonne
	\item Forêts (8-12.5 mm)
	\item Une insoleuse
	\item Un ordinateur
	\item Une imprimante laser
	\item Du papier calque ou transparent
\end{items}

\subsection{Dessiner son circuit imprimé}

Avant de se lancer dans la fabrication de son circuit imprimé, il convient de le concevoir sur un logiciel de CAO electronique.
Il existe de nombreux logiciels tels que:

\begin{items}{blue}{\Bullet}
	\item Kicad
	\item Eagle
	\item EasyEDA
	\item Altium 
	\item Target 3001
	\item Proteus-ISIS
	\item CircuitMaker
\end{items}

\subsection{Impression en noir inversé}

Une fois votre circuit réalisé sur l'ordinateur, il vous suffit de l'imprimer
Imprimer votre circuit à l'imprimante laser pour avoir un tracé noir, dense et sans manques, à haute résolution, sur un transparent (calque).\\
Pour plus de précision, on utilisera ce tracé sur la face inférieure (à cause de la légère épaisseur du transparent).\\

Marquer les 4 coins pour le repérage de position de typon si on fait un circuit double face. (Attention, les imprimantes à jets d'encre ne font pas toujours des détails précis, les gouttes d'encre ont tendance à diffuser).\\

Le film est un négatif donc inverser l'image avec Photoshop.\\

Imprimer en noir tout ce qui doit rester en cuivre (les pistes, les pastilles de soudure, le plan de masse).

\subsection{Découpe du cuivre}
Choisir du cuivre entre simple face au double face selon le circuit souhaité.
Couper votre plaque de cuivre.

\subsection{Découpe du film}
Découper un morceau de film bleu photosensible à la bonne taille, qui dépasse de 1 cm (un cutter et une bonne surface de coupe sont recommandés, sinon aux ciseaux). N'ouvrir l'emballage du film sensible que le temps nécessaire (ne pas laisser au soleil).

Le film photosensible est pris en sandwich entre deux protections : soft, en face interne du rouleau (mat), et hard, sur le dessus (brillant)

Décoller et retirer le film de protection soft du film photosensible (astuce, le tirer avec un bout de scotch depuis un coin) et appliquer cette face soft, coté collant, sur le cuivre. Eviter les bulles et les vagues, bien à plat. Le cuivre doit être bien propre, sans trace de doigts (nettoyer à l'acétone et gratter au préalable, le cuivre doit être dégraissé).
Replier et coller le film sur la seconde face du cuivre si on fait un PCB double face.

\subsection{Chauffer}
Passer la plaque plastifiée dans la plastifieuse à chaud (2 fois) pour une bonne adhésion. Ou au fer à repasser réglage léger.

\subsection{Couper les bords}
Couper le surplus, pas trop près des bords de la plaque.
Elle est alors sensibilisée, prête à l'emploi.

\subsection{Exposition}
Exposer la plaque à la lumière.
Le circuit PCB à réaliser est imprimé en négatif avec une imprimante. Poser le calque sur sa plaque, soigneusement aligné. Une plaque de verre (sans poussière) permet de le bloquer dessus si besoin.
Exposer à la lumière du jour ou un tube fluo 15 w pendant 15 minutes. Le papier qui est vert clair avant exposition passe au bleu sombre.
Faire le second côté éventuel (avec des repères de calage).

30 minutes avec une lampe à économie d'énergie, probablement 15 minutes si exposé au soleil. Le circuit apparaît imprimé sur la plaque.

\subsection{Révélation}

Après exposition, retirer la seconde couche protectrice (hard) et le masque. Puis développer chimiquement le cuivre au carbonate de sodium (0.85 %), on peut s'aider d'une brosse à dents pour les pistes fines. ("lessive de soude" à 30% mettre 3 cuillères / litre) 5 minutes à 15-35 °C, le produit se conserve.

\subsection{Gravure}
Puis préparer une solution de gravure à 2-5 \% d'hydroxyde de sodium (soude : utiliser avec des gants, attention aux vapeurs).
Plonger la plaque dedans et attendre que ça devienne transparent, retirer la plaque (avec une pince), rincer à l'eau et peler le papier.

9bis - Autre méthode de gravure sans perchlorure
\url{http://www.bidouille.org/elec/gravure}
\url{http://wiki.jelectronique.com/realisation_de_circuits_imprimes}

Mélanger 3 volumes d'eau chaude, 2 volumes d'acide chlorhydrique 23\% et 2 d'eau oxygénée 110 volumes (ou dosage 1 + 0.6 + 0.3) (+ gants et protections, ne pas verser l'eau dans l'acide), pour éviter d'utiliser du perchlorure très salissant. prendre des récipients en verre. Durée 30 à 40 secondes.

\subsection{Perçages}
Puis percer les trous (forêt au carbure de tungstène avec perceuse à colonne sur support vertical).

\subsection{Finitions}
Etamer, vernir, placer les composants, souder et tester le circuit.