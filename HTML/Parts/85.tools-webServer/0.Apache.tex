\chapter{Installation d'un serveur Apache}

\section{Présentation} \label{apache}

Apache est un logiciel qui va faire l’interface entre le serveur et les requêtes émises par le client. Ainsi, toutes les demandes d’affichage des pages Web passent par le logiciel Apache.
Ce dernier va écouter le port HTTP qui est le n°80 par défaut. \\

\img{\rootImages/schema.png}{Le rôle du serveur Apache}{0.6}

Les requêtes sécurisées passent, quant à elles par le port n°443. \\

\section{Installation}
Pour installer Apache, rien de plus simple :
\begin{Bash}{Installation de Apache}
sudo apt-get install -y apache2
\end{Bash}


\section{Une petite explication sur les ports}

Pour recevoir et transmettre des données à d’autres ordinateurs, un ordinateur a besoin de ports. Cependant, les ports physiques tel que le port USB, VGA, RS232 et ETHERNET sont peu nombreux. Ainsi, des ports virtuels ont été créés. \\

Chaque port virtuel est codé sur 16 bits et permet de faire communiquer un service ou un logiciel.\\
Par exemple, le service SSH communique sur le port 22. 
Il y a donc potentiellement 65536 ports disponibles. \\
Certains numéros de port sont réservés à certains services \\
Les ports de 0 à 1023 sont déjà réservés. Il conviendra de prendre un numéro de port non pris par un service dans les configurations à venir. \\

Le logiciel Apache gère l’emplacement du répertoire du site Web. \\
\bold {Par défaut, le code source du futur site est localisé dans /var/www/html}


\section{Masquer le serveur}

Tout d’abord, il convient de limiter le nombre d’information que le serveur va afficher en cas de problème. Moins l’utilisateur en sait, meilleure sera la sécurité globale.\\

Nous allons donc désactiver l’affichage du type de serveur et le système d’exploitation le faisant fonctionner.\\

\img{\rootImages/error.png}{L'affichage du système d'exploitation}{0.8}
Ainsi, nous voyons clairement le type de serveur en saisissant une adresse inexistante basée sur le serveur, avec les informations concernant le port (80) et le système d’exploitation Linux. \\

Editons le fichier \bold{/etc/apache2/conf-enabled/security.conf}

\begin{Bash}{Ouverture du fichier de configuration}
sudo nano /etc/apache2/conf-enabled/security.conf 
\end{Bash}

Si la commande \bold{nano} n'est pas reconnu, soit vous utiliser un éditeur de texte que vous avez l'habitude, soit vous installer \bold{nano} de la manière suivante : 

\begin{Bash}{Installation de nano}
sudo apt-get install -y nano
\end{Bash}


Ensuite, nous allons remplacer la ligne :
\begin{Bash}{Ancienne directive}
ServerTokens Os 
\end{Bash}
par
\begin{Bash}{Nouvelle directive}
ServerTokens Prod 
\end{Bash}

De plus, on va remplacer 

\begin{Bash}{Ancienne directive}
ServerSignature On
\end{Bash}
par
\begin{Bash}{Nouvelle directive}
ServerSignature Off
\end{Bash}



Enfin, on actualise le serveur avec

\begin{Bash}{Rédémarrage du serveur Apache}
sudo service apache2 restart
\end{Bash}

\section{Changer le port d'écoute d'Apache}

On édite le fichier /etc/apache2/ports.conf

\begin{Bash}{Edition du fichier de configuration des ports}
sudo nano /etc/apache2/ports.conf
\end{Bash}
Puis changer la ligne Listen si vous souhaiteez changer de port.\\
Si vous voulez ajouter un port d'écoute, ajouter une ligne Listen XXX


\section{Ajout d'un serveur virtuel}

/etc/apache2/sites-available/default
Exemple
\begin{Bash}{Fichier VirtualHost}


<VirtualHost *:80>
        # The ServerName directive sets the request scheme, hostname and port that
        # the server uses to identify itself. This is used when creating
        # redirection URLs. In the context of virtual hosts, the ServerName
        # specifies what hostname must appear in the request's Host: header to
        # match this virtual host. For the default virtual host (this file) this
        # value is not decisive as it is used as a last resort host regardless.
        # However, you must set it for any further virtual host explicitly.
        #ServerName www.example.com

        ServerAdmin webmaster@localhost
        DocumentRoot /var/www/html

        # Available loglevels: trace8, ..., trace1, debug, info, notice, warn,
        # error, crit, alert, emerg.
        # It is also possible to configure the loglevel for particular
        # modules, e.g.
        #LogLevel info ssl:warn

        ErrorLog ${APACHE_LOG_DIR}/error.log
        CustomLog ${APACHE_LOG_DIR}/access.log combined

        # For most configuration files from conf-available/, which are
        # enabled or disabled at a global level, it is possible to
        # include a line for only one particular virtual host. For example the
        # following line enables the CGI configuration for this host only
        # after it has been globally disabled with "a2disconf".
        #Include conf-available/serve-cgi-bin.conf
</VirtualHost>

# vim: syntax=apache ts=4 sw=4 sts=4 sr noet

\end{Bash}


\section{Protection d’un répertoire spécifique}

\paragraph{
Dans certains cas, on souhaite interdire, ou du moins restreindre l’accès d’un dossier au utilisateurs.
Une des solutions consiste à configurer un fichier htaccess. \newline
}
Un fichier htaccess est un fichier de configuration Apache permettant de restreindre des répertoires.
Ce dernier est de la forme suivante : \newline 


\begin{Bash}{Ancienne directive}
AuthUserFile /var/www/html/Admin/.htpasswd
AuthName "Veuillez saisir votre mot de passe pour ouvrir le dossier d'administration"
AuthType Basic
Require valid-user
\end{Bash}

Pour activer la lecture de ce fichier par le serveur Apache, il faut déclarer que la lecture est activée.

Dans le fichier 
\begin{Bash}{Ouverture du fichier}
sudo nano /etc/apache2/apache2.conf
\end{Bash}

Mettre tous les 
\begin{Bash}{\protect }
AllowOverride None
\end{Bash}
à 
\begin{Bash}{\protect }
AllowOverride All
\end{Bash}


\section{Htaccess}

\begin{Bash}{Fichier Htaccess}
AuthUserFile /var/www/html/root	//.htpasswd
AuthName "Accès protégé par un mot de passe"
AuthType Basic
Require valid-user
\end{Bash}

\section{Htpasswd}
\begin{Bash}{Fichier Htpasswd}
nico:\$apr1$KzVtay1X$NnfyAYUyQ0SN/PnxKu0aI1

\end{Bash}

\section{Définir les erreurs possibles}

Cette section ne traite plus de la sécurité mais traite du confort d’utilisation du site en cas d’erreurs.

Il est possible de personnaliser le message d’erreur en fonction de son type.

\begin{Bash}{Fichier htaccess}
ErrorDocument 404 /Warning/notfound.html
ErrorDocument 401 /Warning/unhautorized.html
\end{Bash}

\section{Afficher les erreurs Apache}
\begin{Bash}{Fichier log Apache}
cat /var/log/apache2/error.log 
\end{Bash}
