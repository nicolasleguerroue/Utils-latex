
\chapter{Introduction}

\section{Contexte de la Slovénie}
La Slovénie est un pays très marqué par ses paysages alpins. En effet de nombreux massifs y sont présents, on peut citer le massif du Pohorje ou bien les Alpes kamiques. Le point culminant de la Slovénie, le Triglav (2864m) figure même sur le drapeau du pays ou bien sur les pièces nationales de 50 centimes.\n

De ce fait, le ski alpin est le sport le plus populaire en Slovénie. De nombreux sportifs slovènes sont présents au niveau international. Nous pouvons par exemple citer Andrej Jerman et Tina Maze sportifs considérés comme faisant partie des meilleurs skieurs au monde.\n

Passée la saison, la neige laisse place aux grandes étendues d'herbe verdoyantes invitant les touristes à envahir le paysage pour pratiquer diverses activités telles que la randonnée, le vélo ou encore des promenades à cheval dans les zones rurales.\n

La Slovénie reste un pays très préservé de l'activité humaine, presque 60 \% de la superficie du pays est boisé. Elle possède un grand nombre d'espaces protégés (355 sites classés dans le réseau Natura 2000). Le pays se revendique comme une destination d'eco-tourisme idéal, il fait  par ailleurs partie des dix meilleures destinations écologiques selon Green Destination. \n






\section{Objectifs de l'entreprise}

Suite à l'arrivée du Coronavirus il y a déjà 5 ans, la Slovénie, pour qui le tourisme est un domaine important a rencontrée de nombreuses difficultés. \n 

Au vu de la nécessité de relancer l'économie locale et d'offrir à nouveau la possibilité de sortir et de profiter des différentes activités offertes par les paysages slovènes, notre société a pour but de proposer des solutions pour permettre ce retour à la normale dans les meilleures conditions malgré l'épidémie qui sévit.\n

Notre société proposera un ensemble d'applications mobiles pour les stations de skis: \n
\begin{enumerate}
    \item carte de densité des clients sur la station. 
    \item redirection des utilisateurs vers des zones moins fréquentées.
    \item gestion du matériel des stations (matériel disponible , matériel à nettoyer, ...)
\end{enumerate}
  
L'objectif étant de permettre aux touristes de profiter aux mieux des activités proposées sur le site. Tout en croisant le moins de personnes possible et ainsi limiter les risques de propagation du virus. \n   
  
Notre entreprise s'engage à préserver la biodiversité locale ainsi qu'à sensibiliser tous les utilisateurs de nos services aux problèmes environnementaux. \n

Nous nous engageons également à promouvoir au maximum les sociétés locales, et si ce n'est pas possible, des sociétés européennes. 







\chapter{Organisation de l'entreprise}



\section{Emplacement de l'entreprise}

Notre entreprise proposant des services personnalisés pour chaque station de ski (ou autre), son emplacement sera donc fortement influencé par la proximité avec ces stations.\n 

\img{\rootImages//carte.png}{corona-map mai 2020}{0.25}

Comme nous le montre cette infographie, l'ensemble des stations de ski se trouve au Nord-Ouest du pays (où se trouvent les principaux pics montagneux du pays). Nous souhaitons donc nous installer dans le périmètre du cercle tracé. \\




\section{Fonctionnement de l'entreprise}

Notre société "Coronavacances" n'est composée que de cinq associés travaillant en collaboration dans leur domaine respectif. Notre société se définit comme start-up. 
Toute la gestion, création et mise à jour des applications ainsi que la gestion humaine et financière de l'entreprise sont réalisées par nos cinq associés.\\

Nous travaillons en collaboration avec de nombreuses entreprises locales et partenaires pour la production matérielle.
Nous travaillons évidemment en collaboration avec le client lors de la réalisation d'une application personnalisée.\\
 
Chaque associé est responsable d'un département spécifique, choisi suivant les goûts et aptitudes de chacun. Les responsabilités sont les suivantes : \\


\img{\rootImages/membres.png}{Répartition des rôles}{0.18}

\textbf{Alexis MAQUIGNON (Département Commercial)} \italic{\n "Je suis attiré par le contact avec le client et la construction d'un projet complet avec celui-ci. Je suis là pour accompagner chaque projet individuellement et conseiller du mieux possible."} \n

\textbf{Baptiste LÉO (Département Production \& Logistique)} \italic{\n "J'adore me rendre sur chaque site et imaginer le futur de celui-ci. Je me charge de l'élaboration et de l'installation de la logistique et je suis particulièrement proche du terrain et de ses réalités."} \n

\textbf{Nicolas GAUTIER (Département Communication \& design)} \italic{\n "Passionné par le design et les arts graphiques, je me charge de la conception graphique des applications ainsi que de la communication."} \n

\textbf{Mathieu CHARLES (Département Informatique)} \italic{\n "J'aime l'informatique et les possibilités que cette technologie offre. Fervent défenseur du libre sur internet, je tiens à ce que nos solutions développées suivent cette éthique. Je gère le développement des solutions logicielles et l'assistance informatique de l'entreprise."} \n

\textbf{Nicolas LE GUERROUÉ (Département Électronique):} \italic{"Je suis passionné d'électronique depuis mon plus jeune âge. Je m'occupe de la conception électronique des modules de communication sur sites en collaboration avec mon collègue Mathieu C."}\n

\section{Organisation interne}

Si nous avons décidé de nous répartir les tâches en fonction de nos prédispositions personnelles, nous avons aussi choisi de mettre en place une hiérarchie horizontale au sein de notre entreprise.\\
En effet, nous avons la conviction qu'une entreprise dans laquelle chaque membre possède la même force est un bon moyen d'amener à des discussions et de tirer le niveau vers le haut pour proposer le meilleur service possible. \\
C'est pour cela que nous persisterons à conserver l'équité des salaires et des responsabilités, si notre entreprise venait à s'agrandir.\\
Ceci toujours dans le but de conserver notre éthique et de promouvoir des relations saines entre les associés de l'entreprise.


\chapter{Services}

\section{Convictions}

Parce que nous souhaitons instaurer une relation de confiance avec les utilisateurs de nos services, nous publierons l'intégralité de nos logiciels sous une licence libre.\n
Nous sommes en effet persuadés que cette solution apportera de nombreux avantages tels que :
\begin{enumerate}
    \item La possibilité pour n'importe qui le souhaitant de consulter le code source des logiciels proposés afin de vérifier que ce que nous faisons ne va pas à l'encontre de ses convictions
    \item La possibilité que des personnes extérieures à l'entreprise contribuent activement à l'amélioration du système. Ceci pourrait apporter par exemple des détections et des corrections de failles de sécurité beaucoup plus rapides qu'à la normale, des résolutions de bogues simplifiés pour nos équipes et bien d'autres.
    \item La possibilité pour les utilisateurs de personnaliser relativement facilement notre solution pour l'adapter exactement à ses besoins si jamais nous n'en avons pas la possibilité.
    \item La possibilité que n'importe quel domaine puisse en profiter, y compris si celui-ci ne possède pas les moyens de payer les services de suivi que nous proposons. Nous souhaitons en effet rendre notre solution accessible au plus grand nombre afin de promouvoir un tourisme plus responsable et plus sûr. \footnote{La section Service "Sponat" aborde une solution de développement d'entreprise.}
\end{enumerate}

Ainsi, même si nous proposons des produits "logiciels", ce que nous vendons sont des services. L'accompagnement des entreprises dans la mise en place de nos solutions logicielles.
Un service de suivi et d'aide à l'utilisation de nos solutions logicielles.



Notre entreprise met à la disposition des usagers différents services. Ces services sont les suivants :

\section{Service "LOGAL"}

\subsection{Principe}

Le service "LOGAL" \footnote{"LOcalisation par GALiléo"} utilise les nouvelles technologies pour ré-orienter les flux de touristes vers les zones les moins convoitées actuellement de la station. \newline


Notre société a mis au point une application tout à fait révolutionnaire. Cette application utilisera le système Galiléo de ses utilisateurs afin de lui transmettre des données en temps réel sur le flux des touristes, et permettra à l’utilisateur en question de se diriger vers une zone peu fréquentée et contribuer à une sécurité accrue.\n
L'utilisateur pourra donc récupérer de nombreuses informations en temps réel. Ces informations concerneront aussi bien les amateurs que les grands sportifs.
En effet, en renseignant leur capacité sportive et leur âge\footnote{A condition que l'utilisateur l'accepte}, cela leur permettra de se diriger vers les pistes adéquates tout en tenant compte de la fréquentation

\img{\rootImages//application.png}{prototype application}{0.18}



\newpage



\subsection{Système de localisation}

Nous avons opté pour le système satellite de géo-localisation européen appelé Galiléo. \newline 
Ce système, mis en place en 2020 est stable et performant (précision de base de 4 m) et évite la dépendance du réseau américain GPS.

\subsection{Critères requis}
L’efficacité d’une telle application réside dans le nombre d’utilisateurs. Il faut donc un nombre supérieur à 65\% de touristes qui adhèrent à ce projet de manière à ce que celui-ci soit profitable.\n

\subsection{Charte éthique}
Nous tenons très rigoureusement à vous informer des normes étiques de l’application. En effet ce type de projet suscite généralement de nombreux contestataires. Notre société conçoit que des inquiétudes demeurent chez certains utilisateurs. Il est donc nécessaire pour nous d’afficher une totale transparence vis-à-vis de vous et de vos données collectées.\n
Concrètement, l’unique donnée récoltée par l’application est votre position Galiléo. Dans le seul but de vous proposer un séjour plus sûr, votre position est récoltée seulement si vous nous l’autorisez et seulement dans les horaires d’activité du site. \newline 
Une fois votre séjour terminé, il vous suffira de désinstaller l’application.\n

\section{Service "SPONAT"} 

\subsection{Principe}

Le service "SPONAT" \footnote{"SPOrts NATure} est un service qui permettra à l'usager de profiter pleinement du milieu montagneux. \n
En cas d'obstacle sur une piste par exemple, l'utilisateur sera aussi tôt averti par l'intermédiaire de la fonction "talk-to-speech".
Les endroits remarquables seront également répertoriés par l'intermédiaire des autres utilisateurs qui souhaitent faire découvrir un nouvel endroit, par exemple un joli panorama. \n

Parce que le sport n'a pas vocation à être toujours fatiguant, l'application détecte votre temps d'activité physique et vos performances et détecte automatiquement une baisse des performances. Une baisse importante signifie généralement une fatigue. \n \n
De ce fait, il convient de se reposer ou bien de changer d'activité, pour en faire une moins physique. \n
C'est là que notre application intervient de nouveau et propose un panel d'activités adaptées en valorisant au mieux les activités locales.

\img{\rootImages//application2.png}{prototype application}{0.18}

\section{Extension du réseau}

Ce service proposé permettra à de jeunes entreprises ayant les mêmes intérêts pour la nature et la santé de se développer avec des bases saines.
En effet, chaque entreprise de service (culturel, culinaire...) peut, si elle le souhaite, faire partie du réseau et promouvoir ses valeurs aux utilisateurs. \n
Pour y parvenir, il suffira simplement d'en faire la demande à notre entreprise qui déterminera si l'entreprise en question partage la même politique que le réseau d'entreprise.