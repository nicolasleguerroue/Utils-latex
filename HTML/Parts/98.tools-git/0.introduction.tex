
\chapter{Utilisation des ressources du CREPP}

\section{Présentation}

Ce document a pour objectif d'expliquer le fonctionnement de Git pour les ateliers du CREPP.
Un répertoire Git a été créé pour centraliser les supports des ateliers ainsi que les codes et projets produits depuis la création du CREPP en 2012.\\
Il est donc en perpétuelle amélioration.

\section{Localisation}
Le répertoire est disponible à l'adresse suivante : \link{\url{https://github.com/CREPP-PLOEMEUR}} ou bien en passant sur
 le site du CREPP\footnote{\link{\url{crepp.org}}}, dans la section \bold{Ressources GIT}

\img{\rootImages/localisation.png}{Accès aux ressources GIT}{0.5}

\section{Menu principal}


Une fois le lien cliqué, vous tombez directement sur cette interface 

\img{\rootImages/git_home.png}{La page principale du répertoire Git}{0.35}

Les répertoires facilement accessibles sont :

\begin{items}{blue}{\Bullet}
    \item Le répertoire \bold{Supports\_PDF} regroupe les supports PDf utilisés pour les Ateliers Arduino et Microcontrôleurs.
    \item Le répertoire \bold{Codes\_Arduino} est une compilation des codes Arduino utilisés lors des ateliers.
    \item Le répertoire \bold{Codes\_ESP12} est une compilation des codes ESP12 des ateliers.
\end{items}

Sous ces trois répertoires, vous avez accès à l'ensemble des répertoires du CREPP. Pour afficher tous les répertoires, vous pouvez cliquer 
sur \bold{View all repositories} en bas de la page.

\img{\rootImages/view.png}{Afficher l'ensemble des répertoires}{0.45}

Il est possible de classer les répertoires en cliquant sur \bold{Sort} et de les classer en fonction de :

\begin{items}{blue}{\Bullet}
    \item Les dates de modifications
    \item Les noms
\end{items}

\img{\rootImages/sort.png}{Trier les répertoires}{0.45}

Actuellement, voici les répertoires : 

\begin{items}{orange}{\faBookmark} 
\item codes\_Arduino
\item Codes\_ESP12
\item Codes\_Pico
\item Projets\_Ateliers\_Jeunes
\item Projet\_Capteur\_pollution\_atmospherique
\item Projet\_Cocci-Bot
\item Projet\_Crepp-Rap
\item Projet\_Fauteuil\_roulant
\item Projet\_MySensors
\item Projet\_Pot\_Qui\_pense
\item Projet\_Raspi-Bot
\item Projet\_SimpleCDBot
\item Projet\_Tracker\_Solaire
\item Projet\_Ventilateur
\item Supports\_PDF 
\end{items}


\section{Exploration d'un répertoire}

\subsection{L'arborescence}

En cliquant sur un répertoire, l'arborescence de ce dernier apparaît avec le premier rang des dossiers et les fichiers sur le même niveau.\\
En cliquant sur les noms des dossiers, on peut parcourir l'arborescence du répertoire complet.\\

\img{\rootImages/main.png}{L'arborescence du répertoire}{0.30}

Une description du répertoire est disponible avec un fichier README.md. (ici le répertoire Codes\_Arduino)

\img{\rootImages/readme.png}{Une description du répertoire}{0.35}

Les principaux langages utilisés dans le répertoires sont indiqués à droite de la section \bold{README}.

\subsection{Récupération d'un fichier}

\subsubsection{Un fichier contenant du code}
Pour récupérer le contenu  d'un fichier particulier, il faut parcourir l'arborescence pour le trouver.\\
Une fois le fichier localisé, il faut cliquer sur le fichier afin de voir son contenu : 


\img{\rootImages/content.png}{Un contenu de fichier}{0.35}

Enfin, il ne reste plus qu'à cliquer sur \bold{Copy raw contents}
pour copier tout le texte du fichier dans le presse-papier.

\img{\rootImages/raw.png}{Copier le contenu d'un fichier}{0.6}


\subsubsection{Un fichier PDF}

Pour récupérer le fichier PDF, après l'avoir localisé, il suffit de cliquer sur le bouton \bold{Download}.

\img{\rootImages/download_pdf.png}{Téléchargement d'un fichier PDF}{0.35}

\messageBox{Remarque}{orange}{white}{Cette méthode est contraignante quand nous sommes ammenés à manipuler plusieurs fichiers au sein d'un même répertoire. \\La méthode suivante va vous expliquer comment télécharger directement tout un répertoire pour travailler par la suite en local.}{black}


\section{Téléchargement d'un répertoire}

Pour télécharger un répertoire dans son intégralité, il faut tout d'abord se placer à la racine de celui-ci en cliquant sur le 
nom du répertoire (Codes\_Arduino):

\img{\rootImages/name.png}{Déplacement à la racine du répertoire}{0.35}

Ensuite, il faut cliquer sur le bouton vert \shortcut{Code} pour dérouler un petit menu puis cliquer sur \bold{Download ZIP}

\img{\rootImages/zip.png}{Téléchargement du répertoire}{0.5}

Le répertoire va se télécharger au format ZIP dans vos téléchargements avec le suffixe \bold{-master}.\\ 
Par exemple, le répertoire \bold{Codes\_Arduino} sera téléchargé sous le nom \bold{Codes\_Arduino-master.zip}.\\

Il ne vous reste plus qu'à extraire le fichier pour explorer le répertoire.

