\chapter{Conclusion}

J'ai dans un premier temps analysé le système actuel de la gestion de l'énergie des balises. \\
J'ai ensuite mesuré le rendement du système de recharge des supercondensateurs en reproduisant l'environnement de charge et de décharge et j'ai essayé de trouver un composant qui ferait la même tâche plus efficacement. J'ai prouvé que les composants du marché ne permettent pas d'avoir de meilleurs rendements dans le cadre de notre utilisation.\\

Sur la partie finale de mon stage, j'analyserai les performances des batteries à très faible température  avec et sans le système des supercondensateurs.\\


J’ai trouvé que la taille de l’entreprise était très propice au développement personnel ainsi qu’au développement des connaissances et j’ai eu l’occasion d’apprendre beaucoup de choses et de consolider mes acquis en électronique.\\
J'ai également pu améliorer mes connaissances dans les domaines des alimentations, des batteries et des supercondensateurs. \\

Ce stage m'a donc permis de me conforter dans mon orientation professionnelle, à savoir le domaine de l'électronique avec une partie d'informatique pour contrôler un système électronique.\\


Le développement de balises de détresse permet de sauver de nombreuses vies que ce soit en aéronautique, dans le domaine militaire ou maritime, ce qui ajoute un côté plus humain dans le travail réalisé.\\




