\chapter{Présentation de l'entreprise}

\section{Le groupe Orolia}

\subsection{Les domaines d'activité}

Le groupe Orolia regroupe 3 grands domaines d'activités : 

\begin{items}{blue}{\Circle}
  \item Les horloges atomiques
  \item Les balises de détresse 
  \item Les simulateurs \glossary{GNSS}  (Global Navigation Satellite System)
\end{items}

Orolia fait partie des leaders mondiaux dans les domaines des balises de détresse aéronautiques et militaires.\\
Le groupe est numéro un mondial pour les horloges de précision (horloges atomiques), et numéro deux pour les simulateurs.\\

Le groupe compte aujourd’hui 450 employés.

\subsection{Localisation}

L'entreprise est présente, comme indiqué sur la \figureName{localisation} à Les Ulis (France),\\Rochester (USA), Montréal (Canada), Neufchâtel (Suisse) et à Guidel (France), lieu de ce stage.

\imgn{\rootImages/localisation.png}{Localisation des filiales}{0.45}{localisation}


\section{Le site de Guidel}

A l'origine, la principale activité de l’entreprise de Guidel était la protection de sites sensibles comme les centrales nucléaires ou les bases militaires par le biais de systèmes électriques et électroniques.\\
A partir de 1986, le site se spécialise dans les balises de détresse et c'est en 2012 que l'entreprise prend le nom de Orolia SAS.


Le site de Guidel s'occupe essentiellement des balises de détresse qui peuvent être sous différentes gammes :

\begin{items}{blue}{\Circle}
    \item Balises \glossary{PLB} (Personal Locator Beacon) militaires
    \item Balises \glossary{ELT} (Emergency Locator Transmitter) pour l’aéronautique \figureRef{balises}
\end{items}


\imgn{\rootImages/balises.png}{Un ensemble de balises ELT }{0.10}{balises}


La \figureName{organigrame} montre la répartition dans les différents services des 90 employés d'Orolia à Guidel.
\imgn{\rootImages/diagramme.png}{Organigramme du site de Guidel}{0.4}{organigrame}


La conception et l'étude des balises se fait dans le département Recherche \& Développement. 
\newline
\section{Le département Recherche \& Développement}

Le service R\&D compte 25 employés, alternants ou stagiaires, ce qui représente de loin le plus grand service du site de Guidel.\\

Au sein du service R\&D, le responsable de l’équipe, Stéphane JINCHELEAU, gère l’organisation du travail en parallèle des trois chefs de projet qui contrôlent la répartition et l’avancement des tâches. \\
Ce sont eux aussi qui communiquent avec les clients des projets (Airbus et Boeing, principalement).\\ 
Le service est scindé en 3 salles pour répartir en fonction des "métiers", à savoir la partie Mécanique, Hardware, Software et Système.\\
Chaque stagiaire possède un maître de stage qui lui sert d’interlocuteur privilégié pour les questions techniques, même si tous les équipiers peuvent être questionnés pour des sujets précis.\\ 

Lors de mon stage, j'avais un maître de stage orienté Software, Quentin L'HELGUEN mais je pouvais aussi compter sur Benjamin ADAM pour la partie Hardware du stage.\\
Le service R\&D a tout au long du stage veillé à ce que j'ai le matériel adapté (oscilloscope, composants électroniques...) et j'avais une grande liberté pour commander du matériel si besoin.\\
Tout au long de mon passage à Orolia, j'ai donc constaté que l'entreprise souhaitait sincèrement que je progresse dans mes compétences.
