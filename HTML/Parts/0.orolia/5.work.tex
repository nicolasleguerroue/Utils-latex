
\chapter{Présentation du travail réalisé}

\section{Objectifs du stage}

Le premier objectif de ce stage était d'analyser dans un premier temps la gestion de l'énergie sur les balises. \\ 
Il a donc fallu caractériser l'ensemble des modules permettant la conversion et la répartition de l'énergie à travers les balises.\\

Un second objectif était d'étudier le système des supercondensateurs et d'analyser les supercondensateurs , partie qui sera abordée par la suite.\\

L'objectif final était d'améliorer l'autonomie des balises grâce aux études et expériences précédentes.\\

L'ensemble des résultats et analyses ont été rédigées dans un cahier de conception demandé par l'entreprise.\\
Pour m'aider dans la compréhension du système actuel, j'ai souvent utilisé le logiciel LTSpice ou Tina afin de simuler le comportement du système.







\section{Analyse de la batterie}

La base de l'alimentation est la batterie, il convient donc de la caractériser.

Comme vu précédemment, les performances de la batterie diminuent lorsque la température diminue. Cependant, le courant tiré par la balise entre en compte. \\

Le cas le plus critique pour l'utilisation de la batterie est lors d'une émission d'un burst en 406 MHz. La partie Puissance de la balise nécessite un courant de 2A sous une tension de 7.6V afin d'émettre avec suffisamment de puissance le signal aux satellites.\\

Des mesures réalisées par la société ont montré qu'on ne pouvaient pas tirer plus de 1.5A sur un ensemble BAT400 ou BAT800 si l'on souhaitait effectuer correctement les émissions.
Ce courant (en ampères) sera par la suite appelé \bold{$I_{max}$}

Sous ces basses températures, analysons le phénomène de Résistance Série Équivalente (ESR) qui pénalise la batterie.

\subsection{Analyse de la résistance interne}

A faible température, la capacité de la batterie diminue et son \glossary{ESR} augmente de manière significative. \\

J'ai essayé de mesurer la valeur de l'ESR de la batterie. Pour cela, je suis parti du principe que l'ESR engendre une chute de tension parasite lors du passage d'un courant. \\
J'ai utilisé une charge dynamique afin d'imposer le courant de 1 ampère en sortie de la cellule placée dans une étuve à -40°C, comme indiqué sur la \figureName{esrDynamicLoad}.\\ Les valeurs sont ensuite relevées avec un oscilloscope.

\imgn{\rootImages/box.png}{Montage pour mesurer l'ESR à -40°C}{0.7}{esrDynamicLoad}

Pour un même courant demandé (1A), la chute de tension due à l'ESR à -40°C augmente considérablement. La \figureName{esr} montre la chute de tension due à l'ESR pour un courant de 1A sur une cellule de 3V.

\imgn{\rootImages/esr_pile.png}{Mise en évidence de l'ESR à -40°C}{0.7}{esr}

L'ESR d'une cellule vaut donc :

$$ R_{ESR} = \frac{\Delta V}{I} = \frac{0.843}{1} = 0.843 \Omega$$

Avec $I$ le courant débité par la celulle en A, $\Delta $ la chute de tension mesurée


La \figureName{esrHot} en annexe met en évidence le lien entre la température et l'ESR. J'ai soumis la pile à plusieurs salves de 1A afin de réchauffer le coeur de la pile et on constate que l'ESR diminue progressivement.\\

Sur un ensemble BAT400 ou BAT800 de 12V, lorsque un courant de 1A est tiré, une chute de tension de 3.4V apparaît et la tension apparente n'est plus que de 8.6V.\\


Le système actuel cherche donc à limiter le courant drainé par la batterie et pour cela, un ensemble de supercondensateurs (configuration 2s2p) fait office de réservoir d'énergie intermédiaire lors des appels importants de courant de la balise.\\

Les condensateurs sont intégrés de la façon suivante au sein de la chaîne d'énergie.


\imgn{\rootImages/hybrid2.png}{Configuration des supercondensateurs}{0.55}{config}

La partie verte correspond au mode Hybrid sur la \figureName{arch} et s'explique car la balise doit émettre dans les 5 secondes. Or, les condensateurs sont déchargés et il n'est pas possible de les charger en 5 secondes. La partie Puissance tire l'énergie directement sur la batterie lors des 6 premiers bursts.


Les supercondensateurs peuvent délivrer de forts courants à basse température. Ainsi, excepté lors des 6 premiers Bursts en 406 MHz, la partie Puissance tire le courant des supercondensateurs qui sont rechargés entre les bursts avec un courant moyen constant afin de réduire l'impact de l'ESR et d'améliorer les performances de la batterie.\\


D'un point de vue capacité de batterie, il est plus efficace de tirer un faible courant sur la pile qu'un courant plus élevé. 

Au lieu d'avoir la courbe de courant bleue (figure suivante) pour le courant sur la batterie, on va chercher à obtenir la courbe rouge. L'énergie débitée par la pile avec ces deux courbes est la même , seule le profil du courant évolue.


\begin{graphic}{0.8}{0.5}{0}{32}{-0.1}{1.6}{t(s)}{$I ~~(A)$}{Courbes de courant entre deux bursts}
\addPoints{blue}{(0,0.033)(1,0.033)(1,1.38)(2,1.38)(2,0.033)(31,0.033)(31,1.38)(32,1.38)(32,0.033)}
\addPoints{red}{(0,0.08)(32,0.08)}
\addPoints{green}{(0,1.5)(32,1.5)}
\addLegend{Courant pulsé (A), Courant lissé (A), $I_{max}$ Batterie ($A$)}
\end{graphic}





\section{Analyse d'un convertisseur}

\subsection{Besoins techniques du convertisseur}


Pour la partie Puissance, il faut une tension de 7.6V (2A) avec un pack batterie de 12V (ou batterie de 6V pour les balises PLB). 

Or, nous avons vu précédemment qu'on ne peut pas tirer 2A avec la batterie à -40°C.\\
L'utilisation d'un régulateur de tension linéaire n'est pas acceptable car si un courant de 2A est tiré en sortie du régulateur, cette même valeur de courant sera tirée sur la batterie.\\

En revanche, il est possible de réduire le courant tiré sur la batterie en utilisant un convertisseur DC-DC.\\ Ce dernier va convertir la tension d'entrée en une tension plus faible. A puissance d'entrée et de sortie du module égale, une tension plus grande sur l'entrée par rapport à la tension de sortie va imposer un courant plus faible en entrée (par rapport à la sortie).\\
La figure suivante indique les différents courants et tensions pour un convertisseur (rendement de 90\%) en séquence de burst.\\


\imgn{\rootImages/converter.png}{Principe d'un convertisseur}{0.55}{converter}


Les convertisseurs, lorsqu'ils sont utilisés dans leur plage de courant nominal possèdent de très bons rendements contrairement aux régulateurs linéaires. Cependant, ces régulateurs sont utilisés dans les balises lorsque une tension stable est recherchée.\\


J'ai étudié principalement 3 topologies de convertisseurs DC-DC , dont deux qui sont utilisés sur les balises :

\begin{items}{blue}{\Triangle}
    \item Convertisseur \glossary{Buck} : Ce module engendre en sortie une tension plus petite que la tension en entrée.
    \item Convertisseur \glossary{Boost} : Ce module engendre en sortie une tension plus élevée que la tension en entrée.
    \item Convertisseur \glossary{Buck-Boost} non-inverseur : Ce module regroupe les deux topologies précédentes pour produire une tension plus faible ou plus élevée que la tension d'entrée.
\end{items}


\subsection{Principe}
Les convertisseurs Buck-Boost (ou Boost) se basent sur les propriétés des bobines à stocker de l'énergie. En commutant une bobine, il est possible de faire varier la tension de sortie.\\
Ces convertisseurs utilisent généralement la même configuration pour leur commande, à savoir un pont diviseur de tension pour fixer la tension de sortie, une résistance pour fixer la fréquence du module et une broche pour activer ou non le module.

\subsection{Mesure des rendements}

L'un des objectifs du stage est de mesurer les rendements des modules Buck-Boost utilisés. En effet, sur les différentes balises les modules Buck-Boost varient et leur efficacité également. 

J'ai pu simuler et mesurer les rendements de 3 modules différents. Cependant, seul une des caractérisation sera présentée. Une première étape a été de simuler un convertisseur puis les mesures ont été réalisées.

\subsubsection{Mesures par simulation}

J'ai simulé le convertisseur LTC3119 sous LTSpice afin d'avoir une première estimation des valeurs de rendement en condition nominale des burst, c'est à dire 7.6V et 2A en charge.

Connaissant la tension de sortie et d'entrée du module, le calcul du rendement se fait donc ainsi : 

$$ \eta = \frac{Vs \cdot Is}{Ve \cdot Ie} *100 ~~[\%]$$
Avec $V_s$ et $V_s$ en $V$, $I_s$ et $I_s$ en $A$ \\

J'ai réalisé le schéma suivant \figureRef{ltc3119} sous LTSpice et avec l'aide de la commande MEAS (Ces instructions sont disponibles en annexes \ref{meas}), j'ai pu mesurer le courant en entrée du module ainsi qu'en sortie, une fois le régime permanent établi.


\imgn{\rootImages/LTC3119.png}{Le schéma de simulation d'un LTC3119}{0.3}{ltc3119}


Les rendements des modules LTC3112, LTC3124 et LTC3119 sont disponibles en annexes \figureRef{ltc3112Efficiency}


\subsubsection{Mesures réelles}
Pour mesurer réellement les rendements de ces modules, j'ai soit : 

\begin{items}{blue}{\Triangle}
    \item Démonté de vieilles balises pour souder directement les câbles de l'alimentation et simulé la batterie avec une alimentation externe \figureRef{rendement_mesures}.
    \imgn{\rootImages/mesures.jpg}{Mesure de rendements}{0.07}{rendement_mesures}
    \item Utilisé des modules de démonstration fourni par les fabricants (Analog Device).
\end{items}


J'ai réalisé une première série de mesures mais l'écart des rendements entre la simulation et la pratique était très élevé (plus de 20\%), j'ai donc remis en cause le montage et constaté une chute de tension élevée aux bornes des câbles d'alimentation.\\
J'ai augmenté la section des fils des câbles d'alimentation afin de me rapprocher des résultats des simulations\\

Les résultats des rendements mesurés sont disponibles en annexes \figureRef{ltc3112Efficiency}.\\


Les différences de rendement dans les graphiques montrant les résultats s'expliquent car de nombreux phénomènes n'ont pas été simulés sous LTSpice comme la résistance des câbles.\\De plus, il ne s'agit que d'une simulation, on arrive rapidement aux limites du monde théorique.
\\Les rendements mesurés sont inférieurs aux rendements simulés ce qui est cohérent.\\
Plus la tension en entrée est faible plus le courant en entrée sera important pour une puissance égale.\\

Les trois modules sont néanmoins capables de fournir suffisamment d'énergie à la partie Puissance lors de l'émission des burst, même si la tension en entrée diminue.\\
A chaque fois que les 2A étaient consommés par la charge, le courant tiré sur l'alimentation était inférieur au $I_{max}$ de la batterie.\\ L'utilisation de convertisseurs pour la partie Puissance est donc justifiée.



\newpage
\section{Analyse des supercondensateurs}

Nous avons vu que les supercondensateurs permettaient de délivrer un courant élevé pendant un bref instant afin de s'affranchir de la température. Cependant, il convient de caractériser ce système afin de connaître ses performances.

\subsection{Caractérisation des supercondensateurs}

J'ai cherché dans un premier temps à déterminer la capacité (F) et la Résistance Série Équivalente ($\Omega$).\\ Cette première donnée est relativement imprécise de la part des fabricants ($\pm$~20\%).

Le supercondensateur mesuré est un BCAP0005-P270 avec une tension nominale de 2.7V et une capacité nominale de 5F.

\subsubsection{Mesure de la capacité}

Pour mesurer la capacité, j'ai simplement fait un pont RC et j'ai soumis le supercondensateur à une tension de 2.7V. 
La tension du supercondensateur a atteint 63\% de la valeur nominale ($1.7V$) en 1072s avec une résistance de 189.9 $\Omega$.
d'ou 
$$ C = \frac{t_{63\%}}{R} = \frac{1072}{189.9} = 5.64 F$$

\subsubsection{Détermination de la Résistance Série Équivalente}


Le principe utilisé pour mesurer l'ESR est le même que celui utilisé pour l'ESR de la batterie à -40°C, c'est à dire qu'on impose un courant de charge de 1 ou 2A et on regarde la chute de tension aux bornes du supercondensateur. J'ai mesuré l'ESR lorsque les supercondensateurs étaient chargés à 2.7V (tension nominale).\\
La valeur de l'ESR mesurée vaut en moyenne $55~ m\Omega$ ce qui est plus faible que la valeur de l'ESR de la batterie. Les supercondenseurs permettent de tirer un courant important même à faible température.\\

La courbe du relevé de l'expérience est disponible avec la \figureName{esrCapa}.

\subsection{Mesure du courant}

Toujours dans l'optique de ne pas dépasser $I_{max}$ sur la batterie, un capteur de courant est placé en amont des supercondensateurs pour connaître le courant.\\

La figure \figureRef{current} indique l'emplacement du capteur.

\imgn{\rootImages/sensor.png}{Emplacement du capteur de courant}{0.46}{current}

Une simulation du composant (\figureName{currentSensor} en annexe) a été réalisée avec le logiciel TINA afin de vérifier sa validité, notamment son temps de réponse.

La \figureName{currentSensorResponse} simule le temps de réponse en sortie du filtre passe-bas du capteur.\\

\imgn{\rootImages/transient.png}{Temps de réponse du filtre}{0.46}{currentSensorResponse}


J'ai ensuite réalisé une mesure réelle. J'ai fixé un courant de 1A et j'ai mesuré le temps que met le capteur à produire la tension image de ce courant.\\
La relation entre le courant et la tension de sortie est la suivante : 

$$ V_{s} = \frac{I}{1.25}~~ [V]$$


%# \begin{graphic}{0.8}{0.5}{0}{1}{-0.1}{1.5}{t(s)}{Vs}{Temps de mesure du capteur}
%# \addPoints{red}{(0,0)(0.1,0)(0.1,1)(0.8,1)(1,1)}
%# \addTrace{blue}{0.1}{1}{1.25*(1-exp{-(x-0.1)/(0.075)})}
%# \addLegend{Courant (A), Vs (V)}
%# \end{graphic}

%# A ev
J'ai constaté que le signal de sortie met 362 $ms$ à atteindre sa valeur nominale. Or, le temps entre deux acquisitions de courant est de 10 ms. Il a donc fallu réduire le temps de réponse du filtre à moins de 10 $ms$ en modifiant le filtre.

Une mesure de comparaison entre le capteur avec et sans filtre passe-bas est disponible en annexe (\figureName{annexe_filter}).


\subsection{Le système de recharge actuel}


%#Ajouter courbe standards -> 
Le système de recharge actuel des 4 supercondensateurs est un chargeur Buck contrôlé via un signal \glossary{PWM} (Pulse Width Modulation) en provenance du microcontrôleur de la balise.


Pour charger les supercondensateurs, différents algorithmes ont été mis en place par la société, certains sont plus efficaces que d'autres. Le rendu de la charge avec l'algorithme de la balise Tauceti est disponible en annexe \figureRef{annexe_tauceti}.\\

\subsubsection{Mesure des rendements}


\imgf{\rootImages/nucleo.jpg}{Une carte Nucleo F411}{0.3}{0.5}{right}{10}

L'un des objectifs de mon stage était de caractériser le système actuel de recharge pour connaître les rendements.\\
Le système actuel est fonctionnel mais peu de données de rendement étaient disponibles. Lors des mesures, j'ai réalisé un code en C (organisation du code sur la \figureName{code}) avec une carte de développement Nucleo F411 afin de reproduire la séquence de charge et décharge des supercondensateurs.\\
\vspace{1cm}
\imgn{\rootImages/diagramme.png}{Graphique des modules C de code}{0.6}{code}

Chaque module gère des fonctions propres à l'expérience (chargeur de supercondensateurs, module de décharge, gestions des interrupteurs, communication UART...) et comprend un fichier d'en-tête et de source code C.\\


L'ensemble du fonctionnement de la balise est régie par le module \bold{Beacon} qui est vu comme la principale machine à état.


\subsection{Mesures des rendements}

Ainsi, pour mesurer les rendements du chargeur de supercondensateurs, il suffit d'appuyer sur le bouton utilisateur de la Nucleo afin de lancer le cycle : 

\begin{items}{blue}{\Bullet}
    \item Décharge des supercondensateurs jusqu'à 4.3V. Cette tension est la tension des supercondensateurs en fin de burst.
    \item Calcul du courant nécessaire en fonction du temps de charge souhaité
    
    Ce courant est simple à calculer car étant donné qu'il est constant, l'évolution de la tension est linéaire et le courant est proportionnel au temps de charge. On sait que : 
        $$ Q = I\cdot t~[C] ~~~~~~~~ et ~~~~~~~~~Q = C \cdot \Delta  V$$ avec $Q$ la charge en Coulomb, $I$ le courant dans le condensateur en Ampère, $t$ le temps de charge en s et $\Delta V$ la différence de potentiel aux bornes du condensateur.
    D'ou :
    $$ <I> = \frac{C \cdot\Delta V}{t}~[A]$$

Ainsi, par exemple, pour charger un condensateur de 5F de 0V à 5.4V en 30 s, il faut un courant (mA) de 
$$ <I> = \frac{5 \cdot 5.4}{30} = 900 mA$$
    \item Activation de la \glossary{PWM} et de l'algorithme de charge
    \item Scrutation de la fin de la charge (tension à 5.4V) et acquisition des valeurs de courant
\end{items}

Afin d'automatiser le calcul du rendement, je me suis basé sur l'énergie accumulée dans les supercondensateurs pour calculer le rendement. En effet, l'énergie accumulée (en Joule) par un condensateur vaut : 

$$ E_C = \frac{C\cdot \Delta V^2}{2} [J] $$

Il s'agit donc de l'énergie stockée dans le condensateur sur une période $t_{charging} $ (s)

Ensuite, on relève le courant débité par l'alimentation (capteur de courant) tous les 10 ms et on stocke le résultat dans un tableau le temps de la durée de charge du condensateur.

$$ <I_e> = \frac{1}{N} \cdot \sum_{n=0}^{N}{array[n]}$$

D'où : 

$$ <P_{condensateur}> = V_{CC}~\cdot <I>~[W]$$

Comme $E = P.t~[W.s] $ et que $ 1Wh\cdot s = 1 J$, on en déduit que :

$$ E[J] =  V_{CC}~ \cdot (\frac{1}{N} \cdot \sum_{n=0}^{N}{array[n]}) \cdot t_{charging} $$


Avec N le nombre d'échantillons du courant, $t_{charging}$ le temps de charge du condensateur en s.\\


Étant donné que les supercondensateurs seront une seule fois déchargés au démarrage de la balise (cas critique), les essais des rendements ont été fait pour une tension des supercondensateurs comprise entre 4.3V et 5.4V (on ignore la charge initiale).\\

Il a été ensuite possible de lancer une séquence avec plusieurs bursts pour faire une moyenne des rendements.


La \figureName{arctarusCurrent} présente un cycle de recharge et décharge de supercondensateurs.\\
La courbe rouge représente l'image du courant dans les supercondensateurs et la courbe bleue la tension des supercondensateurs (V).
\imgn{\rootImages/simulation_burst_1.png}{Séquence de bursts avec l'algorithme Arctarus}{0.35}{arctarusCurrent}


La \figureName{resultBuck} est disponible en annexe et présente les résultats des rendements pour des temps de charges de 5, 10 et 30s.\\ Les rendements sont d'autant meilleurs que le courant en entrée est élevé (sous peine de ne pas dépasser le courant de
saturation de l'inductance).\\
Sur les premières 24 heures, la plupart des recharges se font en 30 secondes (cas nominal) et le rendement pour ces recharges va de 45\% (batterie à 12V) à 56\% (batterie à 8V).

%# Une fois que le système actuel de recharge des supercondensateurs a été caractérisé, la seconde étape était de vérifier si leur présence est indispensable.\\

%# \newpage
%# \section{Caractérisation du système sans supercondensateurs}

%# Un des objectifs de mon stage était de savoir si on gardait le système des supercondensateurs ou non.\\
%# En effet, on peut se demander s'il ne valait pas mieux de perdre un peu de capacité de batterie en tirant un courant plus élevé mais compenser l'énergie perdue en retirant le chargeur de supercondenseurs qui ne possède pas un bon rendement.

%# Pour répondre à la question, j'ai simulé et mesuré le comportement d'une balise en faisant abstraction des supercondensateurs,  c'est à dire en supprimant le circuit de recharge et en mettant la batterie en entrée des convertisseurs utilisés pour la partie Puissance.
 
%#  Ce principe de branchement direct est indiqué sur la \figureName{arch}, plus précisément avec le bloc Direct où l'on supprime le chargeur de supercondensateurs.


%# \subsection{Calculs préliminaires}

%# Afin d'estimer la capacité d'un pack BAT800, j'ai réalisé un tableur où j'ai rentré précisément le cycle des bursts avec les durée et les courants. De ce tableur en a découlé une capacité estimée pour 48 h de fonctionnement (en Ah).

%# \subsection{Expérience}

%#  La séquence des bursts a dû être reproduite avec un microcontrôleur, en accord avec les périodes des bursts (Section \ref{frequencies}).\\

%# J'ai donc placé la batterie à -40°C dans une étuve 2 heures avant de lancer la simulation afin que le coeur de la batterie soit à -40°C.\\

%# Pour simuler la balise, j'ai utilisé deux convertisseurs LTC3112 , l'un pour simuler les émissions en 406 MHz (2A) et le second pour la consommation de la balise hors bursts et en Homing (30
%# mA moyen sur la batterie).\\

%# Le schéma suivant (\figureName{battery}) m'a donc permis de réaliser l'expérience. 

%# Lors de mes premières expériences, j'avais utilisé un régulateur linéaire \glossary{LDO} (Low DropOut) pour simuler le courant hors-burst mais le modèle était incohérent vis à vis du courant tiré sur la batterie en dehors des burst à 406 MHz.

%# \imgn{\rootImages/cold.png}{Schéma pour la batterie sans supercondensateur}{0.5}{battery}


%# La figure suivante présente la montage. On aperçoit l'étuve en arrière-plan avec les câbles de la batterie qui sortent.


%# \begin{figure}[!h]
%# \begin{minipage}[b]{.45\linewidth}
%#     \imgn{\rootImages/montage.jpg}{Montage globale}{0.1}{montage}


%# \end{minipage}
%# \begin{minipage}[b]{.45\linewidth}
%#     \imgn{\rootImages/photo.jpg}{Montage des convertisseurs}{0.1}{montage}
%# \end{minipage}
%# \end{figure}


%# \subsection{Résultats}


%# Afin de vérifier si le système fonctionne sans supercondensateur, il a fallu vérifier que toutes les séquences de Burst ont bien été réalisées. 


%# Le système sans supercondensateur a tenu près de XX heures. Cela peut sembler élevé par rapport à la contrainte des 48 heures mais il faut prendre en compte la pré-décharge de la batterie.
%# %#Ref pour question à l'oral

%# Les résultats détaillés des essais de la batterie sans supercondensateurs sont disponibles en annexe.



\newpage
\section{Optimisation du chargeur de supercondensateurs}

J'ai cherché un composant pour améliorer le rendement du chargeur de supercondensateurs. 

La contrainte de ce composant était de pouvoir contrôler le courant pour ne pas dépasser le $I_{max}$ de la batterie.
Un asservissement en tension aurait entraîné des pics de courants non maîtrisés et inacceptables vis-à-vis du $I_{max}$.\\
De plus, il fallait que ce composant soit le plus autonome possible.
je me suis orienté dans mes recherches sur deux technologies disponibles sur le marché.

\begin{items}{blue}{\Circle}
    \item Des chargeurs de supercondensateurs linéaires
    \item Des convertisseurs DC-DC dont le courant est programmable soit par un bus \glossary{I2C} (Inter-Integrated Circuit) ou bien en fixant une résistance pour déterminer le courant de charge.
\end{items}

La première catégorie, ayant des rendements faibles (50\% environ, la \figureName{ltc4425} montre les rendements mesurés sur le LTC4425, un chargeur linéaire), je me suis porté sur les convertisseurs programmables, plus précisément sur le \bold{LT3120} qui utilise une résistance pour fixer le courant. \\

\imgn{\rootImages/ltc4425.png}{Rendements du LTC4425 (\%)}{0.5}{ltc4425}

Pour fixer le courant, je suis parti sur le choix technologique des potentiomètres numériques\footnote{Il était possible d'utiliser un Convertisseur-Analogique Numériques et même des transistors en commutation mais c'est la méthode la plus simple} qui ne sont ni plus ni moins que des circuits intégrés avec des résistances internes qui commutent par l'intermédiaire de transistors.

Ce composant est prévu pour fonctionner sous deux modes différents : 


\begin{items}{blue}{\Circle}
    \item Le mode PWM est prévu pour les courants en sortie assez élevés ($> 1A$)
    \item Le mode Burst pour les courants inférieurs
\end{items}


La résistance de contrôle du courant maximal se calcul de la manière suivante : 

  $$ R_{P} = \frac{795}{2\cdot I_{m}\cdot R_{S}}  $$
  
  avec $I_{m}$ en $A$ et $R_S$ en $\Omega$ (Il s'agit de la résistance \glossary{Shunt} utilisé par le composant pour connaître le courant en sortie).\\
  
 
Lors des essais en mode Burst, il m'a été impossible de contrôler un courant inférieur à 250 mA. J'ai donc contacté le fournisseur du composant afin d'expliquer mon expérience et les capacités du composant.

De plus, même si le courant en sortie était constant, celui en entrée ne l'était pas du tout, on avait un comportement étrange en fin de charge, comme le montre la figure suivante : 

\imgn{\rootImages/fluctuation.png}{Mesure d'une charge avec le LT3120}{0.3}{fluc}

La courbe blue représente la tension des supercondensateurs (V) et la courbe rouge le courant en entrée du module.\\

Ce composant ne répondant pas en totalité au cahier des charges (comportement étrange), je  n'ai pas pu valider l'intérêt d'utiliser ce composant qui prenait une grande place sur un circuit imprimé.

%#- Image comportemennt LT3120
