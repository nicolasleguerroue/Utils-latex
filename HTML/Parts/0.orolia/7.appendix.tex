
\chapter{Annexes}

\subsection{Verrou de l'alimentation Arctarus}


Lorsque une balise est est mode ON, il faut que son alimentation soit active. Pour cela, on utilise un verrou présenté ci-contre.


\imgn{\rootImages/latch_lt.png}{Verrouillage de l'alimentation Arctarus}{0.7}{annexes_verrou_arctarus}

L'ensemble du verrouillage est assuré par le transistor Q1.

\subsection{Filtre du capteur de courant}
Ces deux signaux sont l'image du courant à travers le capteur, c'est à dire la tension en sortie du filtre passe-bas.\\
Le signal bleu est le signal sans filtre et le rouge avec filtre.
\imgn{\rootImages/comp.png}{Efficacité du filtre passe-bas}{0.35}{annexe_filter}


%## \newpage
%## \section{Balise Kiwi}
%## \subsection{Verrou de l'alimentation Kiwi}
%## \imgn{\rootImages/kiwi_supply.png}{Verrouillage de l'alimentation Kiwi}{0.7}{annexe_verrou_kiwi}


\subsection{Allure de l'algorithme de charge de Tauceti}

La courbe bleue représente la tension des supercondensateurs (V) et la rouge leimage du courant (A) en entrée de la balise en fonction du temps (s). \\Contrairement à l'algorithme utilisé pour reproduire les charges et décharges des supercondensateurs, le courant est moins constant car l'asservissement est géré différemment.
\imgn{\rootImages/charge_tauceti.png}{Allure de l'algorithme Tauceti}{0.4}{annexe_tauceti}
%#erreur statique
%#Ajouter algo Arctarus


\subsection{Mesure de l'ESR pour les supercondensateurs}
La courbe bleue représente la tension des supercondensateurs (V) et la rouge le courant (A) dans les supercondensateurs. La tension de charge est de 5.4V car deux condensateurs ont été mis en série pour augmenter la tension maximale.

\imgn{\rootImages/ESR_2A.png}{Mesure de l'ESR des supercondensateurs avec une charge à courant constant de 2A}{0.3}{esrCapa}

\subsection{Rendement du Buck}


Ce tableau présente les rendements du chargeur de supercondensateurs en fonction de la tension d'alimentation (V) et du temps de charge (s).
Dans notre cas nominal (30 secondes), nous pouvons espérer un rendement compris entre $45.7\%$ et $57.9\%$.

\imgn{\rootImages/time.png}{Mesure des rendements pour le module Buck}{0.7}{resultBuck}


\subsection{Simulation des convertisseurs}
\label{meas}

Les instructions suivantes permettent de mesurer le rednement des convertisseurs sous LTSpice.
% \begin{Cpp}{Instructions LTSpice pour les mesures des rendements}
% .tran 0 6.4m 6m 100u startup
% .step param load list 5 4 3 2.5 2 1.5 1 0.5
% .meas POWER_INPUT AVG -V(INPUT)*I(Vin)
% .meas POWER_OUTPUT AVG V(OUTPUT)*I(I1)
% .meas EFFICIENCY PARAM POWER_OUTPUT/POWER_INPUT
% .meas OUTPUT_VOLTAGE AVG V(OUTPUT)
% .meas OUTPUT_CURRENT AVG I(I1)
% \end{Cpp}


\subsection{Rendements des convertisseurs}

Voici les rendements simulés et mesurés pour les 3 convertisseurs disponibles pour une utilisation nominale (2A, 7.6V)
A chaque fois, le rendement réel était inférieur au rendement théorique.

\imgn{\rootImages/LTC3112_eff.png}{Rendements des modules LTC3112}{0.7}{ltc3112Efficiency}

\imgn{\rootImages/LTC3124_eff.png}{Rendements des modules LTC3124}{0.7}{ltc3124Efficiency}


\imgn{\rootImages/LTC3119_b.png}{Rendements des modules LTC3119}{0.9}{ltc3119Efficiency}


\subsection{Mise en évidence entre la température et l'ESR}

La figure suivante indique la chute de tension (V) pour une cellule de 3V placée dans l'étuve à -40°C. 
En imposant un courant de décharge suffisamment élevé à la cellule, le coeur de cette dernière se réchauffe ce qui fait baisser son ESR.

\imgn{\rootImages/salve.png}{Lien entre la température et l'ESR}{0.45}{esrHot}



%# \subsection{Mesures de la batterie sans supercondensateur}

%# La courbe bleue représente la tension de la batterie (V), la courbe jaune la tension en sortie du LDO (V), la courbe orange la tension en sortie du convertisseur (V) fournissant le 7.56V et la courbe grise le courant dans la batterie (A).

%# Pour vérifier la faisabilité du système sans supercondensateur, il a fallu vérifier que l'ensemble des Bursts s'est correctement déroulé, c'est à dire en vérifiant que les convertisseur ont pu délivrer la puissance nécessaire pour les émissions.

%# \imgn{\rootImages/b1.png}{Évolution de la batterie pour les bursts de 5 secondes}{0.7}{burst_Fast}

%# \imgn{\rootImages/b2.png}{Évolution de la batterie pour les bursts de 30 secondes}{0.5}{burst_Medium}

\subsection{Simulation du capteur de courant}

La figure suivante présente la simulation du capteur de courant effectuée avec le logiciel TINA.
\imgn{\rootImages/current_sensor.png}{Simulation du capteur de courant}{0.8}{currentSensor}


\newpage
\imgr{\rootImages/planning.png}{Planning}{0.35}{90}