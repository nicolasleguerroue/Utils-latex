\section{Effacer la mémoire de l'ESP8266}

Dans un premier temps, nous allons effacer le contenu de la puce ESP8266. Ceci nous permettra de partir sur des bases saines. \\

\messageBox{Point-clé}{orange}{white}{Maintenant, vous pouvez brancher votre ESP8266 sur un des ports USB de votre ordinateur.}{black}

Il convient ensuite d'installer les outils adéquats.

\subsection{Installation de Pip3}

\lib{pip3} est un utilitaire Python qui va nous permettre d'installer le petit programme pour effacer l'ESP8266. \\
On l'installe de la manière suivante : 

\begin{Bash}{Installation de Pip3}
sudo apt-get -y install python3-pip
\end{Bash}


\subsection{Installation de Esptool}

L'utilitaire qui va se charger d'exécuter cette opération s'appelle \lib{esptool}.

Pour l'installer, on effectue

\begin{Bash}{Installation de Esptool}
pip3 install esptool
\end{Bash}

Voici le résultat de la commande sur le terminal : 

\img{\rootImages/esptool_download.png}{Résultat de l'installation de pip3}{0.28}



\subsection{Récupération du port USB}

L'ESP8266 étant raccordé, l'ordinateur lui a affecté un nom de port de type \italic{/dev/ttyUSBx} avec x représentant le numéro du périphérique USB. \\

Pour récupérer la valeur de ce numéro, nous allons lancer la commande suivante : 

\begin{Bash}{Récupération du numéro du port série}
ls /dev/ttyUSB*
\end{Bash}

\img{\rootImages/lsusb.png}{Résultat de la commande}{0.8} %@change

Dans le cas présent,le nom du port est \italic{/dev/ttyUSB0}

\subsection{Effacer la mémoire}

Lancer la commande suivante : 
\begin{Bash}{Effacer la mémoire de l'ESP8266}
esptool.py --port /dev/ttyUSB0 erase_flash
\end{Bash}

Évidemment, si vous avez un autre numéro de port avec la commande \bold{esptool.py flash\_id}, vous mettez votre numéro.

Voici le résultat de la commande sur le terminal : 

\img{\rootImages/esptool_erase.png}{Résultat de la commande pour effacer l'ESP8266}{0.6}


\section{Installer le firmware sur l'ESP8266}

Maintenant que l'ESP8266 est vide, il nous reste à installer son logiciel (firmware) fin qu'il puisse utiliser le Wifi selon deux modes : 

\begin{items}{blue}{\Triangle}
    \item Point d'accès : l'ESP8266 créer son propre réseau Wifi
    \item Connexion : l'ESP8266 peut se connecter à un réseau Wifi pour dialoguer
\end{items}

\subsection{Récupération du logiciel}

Le logiciel se présente sous fichier binaire (.bin) et est disponible à l'adresse suivante : \\

\url{http://micropython.org/download/esp8266/}\\

Je vous invite à télécharger la dernière version stable (latest)

\img{\rootImages/esplatest.png}{Récupération du logiciel pour l'ESP8266}{0.55}

\subsection{Installation du logiciel}

\bold{Tout d'abord, placez vous dans le même répertoire que votre fichier binaire installé précédemment et ouvrez un terminal.} \\

La commande \shortcut{ls} devrait confirmer votre contenu du répertoire.

\begin{Bash}{Vérification du répertoire}
ls
\end{Bash}

\img{\rootImages/ls.png}{Contenu du répertoire}{0.7}

Il ne vous reste plus qu'a saisir la commande pour installer le firmware. \\
\begin{Bash}{Installation du firmware}
esptool.py --port /dev/ttyUSB0 --baud 460800 write_flash --flash_size=detect 0 esp8266-20190125-v1.10.bin
\end{Bash}

Comme précédemment, si vous avez un nom de fichier différent, je vous laisse le soin de changer de nom afin de coïncider avec le vôtre.

\img{\rootImages/espinstall.png}{Résultat de la commande pour installer le firmware}{0.45}

Après le déploiement du firmware, le module redémarre et il est configuré en point d’accès WiFi avec pour nom \bold{MicroPython-6953}. \\ Les chiffres correspondent aux quatre derniers chiffres de l'adresse MAC du module. 


\section{Configurer le mot de passe WebRepl}

\subsection{Installation de WebRepl}

Le logiciel \lib{WebRepl} va nous permettre de se connecter à l'ESP8266 afin de saisir des commandes Python.

Le logiciel est disponible à l'adresse suivante : \\

\url{https://github.com/micropython/webrepl}\\

Cliquez ensuite sur \shortcut{Code} puis \shortcut{Download Zip}

\img{\rootImages/github.png}{Téléchargement de WebRepl sur Github}{0.45}

Veuillez commencer par extraire l'archive.
Celle-ci contient les fichiers suivants : 

\img{\rootImages/webrepl.png}{Contenu du dossier WebRepl}{0.45}

\subsection{Installation de screen}

\lib{screen} est un utilitaire qui va nous permettre de se connecter à l'ESP8266 via le câble USB car actuellement, il nous est impossible d'utiliser WebRepl.\\
On installe l'utilitaire avec la commande suivante : 
\begin{Bash}{Installation de screen}
sudo apt-get install -y screen
\end{Bash}

\subsection{Création du mot de passe}

\subsubsection{Utilisation de screen}

On peut accéder à l’interpréteur de commande Python REPL \footnote{Read Evaluate Print Loop} via le port série en tapant la commande suivante dans un terminal :

\begin{Bash}{Exécution de screen}
screen /dev/ttyUSB0 115200
\end{Bash}


\messageBox{Point-clé}{orange}{white}{\bold{Il faut appuyer sur la touche Entrée pour afficher l'invité de commande MicroPython.}}{black}

\img{\rootImages/console.png}{Console screen}{0.6}


Pour quitter Screen, il faut appuyer sur les touches \shortcut{CTRL + a} puis écrire \shortcut{:quit} \\

Ensuite, entrez la commande suivante via le terminal Screen :
\begin{Bash}{Commande pour créer un mot de passe}
>>> import webrepl_setup
\end{Bash}

Le système vous demande tout d'abord d’activer l’accès par le réseau Wifi dès le démarrage en saisissant \shortcut{E}.

\img{\rootImages/enable.png}{Activation de l'ESP8266 screen}{0.6}

Il vous invite ensuite à saisir le mot de passe pour l'accès à WebRepl. Ici le mot de passe choisi est \shortcut{crepp}
\img{\rootImages/password.png}{Activation de l'ESP8266 screen}{0.6}
Enfin, saisissez \shortcut{y} pour redémarrer l'ESP8266.\\
A ce moment là, les lignes suivantes apparaissent : 
\img{\rootImages/end.png}{Fin de la configuration}{0.6}

\bold{La configuration de l'ESP8266 est terminée}


\subsection{Connexion à WebRepl}



\subsubsection{Connexion au réseau de l'ESP8266}

Tout d'abord, veuillez chercher parmi les réseaux Wifi le réseau de l'ESP8266 appelé MicroPython-XXXX avec XXX représentant les 4 derniers chiffres de l'adresse MAC de l'ESP8266.

Lors de la connexion, un mot de passe est demandé, saisir \shortcut{micropythoN}

\subsection{Lancement de WebRepl}

 Veuillez lancer, à l'aide de votre navigateur Internet, le fichier \file{webrepl.html}  situé dans le dossier WebRepl précédemment installé. \\
 Pour cela, veuillez faire un \shortcut{Click droit + "Ouvrir avec le navigateur Web ..."}. \\
Vous tombez sur l’interface Web. Il suffit de cliquer sur \shortcut{Connect} et d’entrer le mot de passe que vous avez définie.
(en l'occurrence \bold{crepp}).


\section{Se connecter automatiquement à un réseau Wifi}

L’inconvénient avec la méthode présente est de jongler entre les 2 accès WiFi (Wifi classique ou réseau ESP8266).\\
Or, il est possible de configurer le module pour qu’il se connecte sur votre box WiFi en tant que client afin d'éviter les désagréments des connexions WiFi. \\
Pour cela il suffit de se connecter à l'ESP8266 via le port série et de taper les commandes suivantes : 
\begin{Bash}{Commandes de connexion}
import network
wlan = network.WLAN(network.STA_IF)
wlan.active(True)
wlan.connect('ssid', 'password')
\end{Bash}

Vous pouvez vérifier la nouvelle adresse IP fournie par votre box en tapant la commande :
\begin{Bash}{Vérification}
>>> wlan.ifconfig()
\end{Bash}

Par contre il est nécessaire de passer les commandes suivantes après la connexion à votre box :

\begin{Bash}{Initialisation}
>>> import webrepl
>>> webrepl.start()
\end{Bash}

