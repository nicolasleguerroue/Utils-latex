%\addPartText{Théorie sur les signaux et leur traitement}
\part{Les signaux}
\chapter{Les signaux}

\section{Introduction}

Un signal est une évolution d'une grandeur physique.\\
Il permet donc de caractériser cette grandeur qui peut ensuite être traitée par un circuit adapté.


\begin{itemize}
    \item Les signaux numériques
    \item Les signaux analogiques
\end{itemize} 

\subsection{Les grandeurs physiques}

Afin de mieux comprendre comment un signal peut se former, nous allons étudier les notions de tensions, 
de courant et de résistances électriques.
Pour cela, plongeons nous dans la matière, au niveau des électrons

\subsection{Les atomes}


\subsection{Les électrons}

Les électrons forment la couche extérieure de l'atome, element insécable de notre univers.

%image atome.

Lorsqu'un électron se déplace dans l'espace, ce dernier engendre un courant.\\

%\box{Le courant électrique est le nombre d'éléctrons par unité de temps}
%\boxed{Le~courant~électrique~est~le~nombre~d'éléctrons~par~unité~de~temps}\\


La charge électrique représente quant à elle le nombre d'électron à un endroit donné et à un temps précis.

\section{La tension électrique}

Nous avons vu que le courant est une nombre de charges par unité de temps. \\
Comment pouvons nous engendrer ce courant ? Tout simplement en soumettant une différence de potentiel, 
autrement dit une \bold{tension}

\section{Analogie}
\newcommand{\A}{$A~$~}
\newcommand{\B}{$B~$~}

Afin de mieux comprendre ce principe, faisons une analogie.

Prenons un rivière, un barrage et une montagne.\\

Soit deux points sur une route dont nous pouvons régler la hauteur en chaque point. On considère une goutte d'eau entre ces points. \\
Notre objectif est de déplacer notre goutte d'un point à l'autre.\\

>> Comment faire ?\\

En surélevant le point \A et en abaissant notre point \B, notre goutte va descendre du point le plus élevé au plus bas.\\

Le courant fonctionne de la même manière : \\

\bold{Il ira toujours du potentiel le plus haut au plus faible.}

Pour la petite histoire, historiquement, le sens du courant étaient du potentiel le plus élevé au plus faible.\\
Bien des années plus tard, les scientifiques se sont rendu compte que le courant allait en réalité du
potentiel le plus faible au plus élevé. Néanmoins, la convention a été gardée...


\subsection{Exemples}
\begin{question}
Soient \A et \B nos deux points du circuits. Ces deux points sont soumis à une tension $U_A$ et $U_B$\\
Quel est le sens de parcours du courant (de \A vers \B, de \B vers \A ou bien autre chose) ?

\begin{figure}[!h]
    \centering
\begin{tabular}{|c|c|c|}
    \hline
    $U_A$ (V) & $U_B$ (V) & Sens du courant ?\\
    \hline
    10 & 5 & \\
    \hline
    5 & 10 & \\
    \hline
    5 & 5 &\\
    \hline
\end{tabular}
\caption{Question sur le sens du courant en fonction des tensions $U_A$ et $U_B$}
\end{figure}
\end{question}

