\chapter{Introduction}

\section{Objectif}

L'objectif est d'automatiser le feutrage de la laine, étape fastidieuse à la main.

On part du principe que si, sur une surface en laine ( 80\% ), on pose une petite quantité de laine pure, on pourra avec une aiguille spéciale enfoncer des brins de laine dans le support et ceux-ci y resteront à la remontée de l’aiguille ( les brins de laine s’accrochent entre eux avec une facilité incroyable ).\\
 Pour une petite surface on utilise un support avec 6 à 9 aiguilles et tenir dans la main le dit support avec un aller et retour vertical devient fastidieux avec des risques de blessures.\\
\bold{C’est cette action qu’il faut automatiser. }


\section{Les versions}

Une premiere version comporte un simple balayage du plateau.\\
La seconde version devrait comporter un traitement d'image avec une caméra pour adapter le balayage.


\section{État des lieux}

Actuellement, des essais ont été réalisés avec la CNC qui était stocké au FabLab.\\

\section{Liste du matériel}


\begin{items}{blue}{\Triangle}
    \item Une ancienne CNC (modèle de Xavier Hinault)
    \item Support d'aiguilles
\end{items}


\section{Communication avec la CNC}

L'ensemble des communications ont lieu avec une liaison série (UART)


La liaison \glossary{UART} est une liaison série avec deux broches :

\begin{items}{green}{\Triangle}
   \item RX
   \item TX
\end{items}

\subsubsection{Protocole}

\begin{items}{blue}{\Circle}
   \item Un bit de start toujours à 0 pour synchroniser la communication
   \item Un champ de données de 8 bits
   \item Un bit de parité (dans notre cas aucune parité)
   \item Un bit de stop
\end{items}

\subsubsection{Vitesse de communication}

La liaison étant asynchrone, il faut que les périphériques communiquent à la même vitesse. Cette dernière est normalisée et représente le nombre de bit par seconde (baud\footnote{1 baud représente 1 symbole par seconde.})

En l’occurrence, la CNC communique à 115200 bauds.


\chapter{Rappels du G-Code}

Le G-Code est un langage machine utilisé par les machines numériques.
Chaque instruction G-Code se termine un retour à la ligne (valeur 13 dans la table ASCII). Par souci de lisibilité, il ne sera pas affiché.
\section{Le référentiel}
Soit on effectue les déplacements de manière relative (+xmm depuis la position actuelle) ou absolue (depuis l'origine avec les capteurs fin de course)
Pour définir le référentiel, on utilise soit la commande suivante pour le référentiel absolu:

\begin{Cpp}{Référentiel absolu}
G90
\end{Cpp}
et pour le référentiel relatif : 
\begin{Cpp}{Référentiel relatif}
G91
\end{Cpp}

Par la suite, on va utiliser le mode relatif.

\section{Les déplacements}

Pour se déplacer on va utiliser la commande suivante : 

\begin{Cpp}{La commande de déplacement}
G01 X[mm] Y[mm] Z[mm]
\end{Cpp}

Les arguments entre crochets correspondent aux déplacements voulus sur les axes concernés.\\
Ainsi, si on souhaite se déplacer de 20 mm vers le centre sur l'axe X, on va utiliser la commande : 

\begin{Cpp}{Les déplacements}
G01 X20
\end{Cpp}

Les axes non concernés par le déplacement ne sont pas présents dans la commande.

\section{Le retour à l'origine}

Le retour en position d'origine se fait via la commande suivante : 

\begin{Cpp}{Retour à l'origine}
G28 [X] [Y] [Z]
\end{Cpp}

Ainsi, pour effectuer un retour à l'origine sur l'axe X on fait la commande 

\begin{Cpp}{Retour à l'origine sur X}
G28 X
\end{Cpp}

\chapter{Algorithme de déplacement}

L'objectif est de ne pas faire de déplacement sur X ou Y lorsque le moteur Z est activé afin d'éviter de casser les aiguilles.


Au début du programme, le porte-aiguille est placé aux coordonnées (0,0) avec le porte-aiguille en position de repos (position haute).\\
Pendant toutes les durées des déplacements en X et Y, le porte-aiguille reste en position de repos.
\begin{items}{blue}{\Triangle}
\item Le porte-aiguille se déplace en Y de la moité de la largeur de la tête contenant les aiguilles (disons 8mm)
\item Le cycle Z correspond à plusieurs aller-retour du porte-aiguille et à leur retour en position de départ (capteur de fin de course activé)
\end{items}

Ce cycle se reproduit pour balayer l'ensemble du plateau.

\img{\rootImages/cycle.png}{Cycle de poinçonnage}{0.5}


\chapter{Programmation Python}

\section{L'interface graphique}

Le choix de l'interface graphique s'est porté sur les bibliothèques PyQt5.\\


\section{Installation des outils}


\begin{Bash}{Installation des outils}
sudo apt install python-pip
sudo apt-get install python-qt5
sudo apt-get install pyqt4-dev-tools 
pip install pyserial
sudo apt-get install qt4-designer
\end{Bash}


    
\section{Un premier programme}

Ce premier programme, sans interface graphique (juste dans un terminal) permet de vérifier le bon déplacement de la tête sur l'ensemble du plateau.

\img{\rootImages/algo.png}{Déplacement de la tête}{0.4}


Pour lancer le programme, il faut saisir la commande suivante : 
\begin{Bash}{Lancement du programme}
python GCode.py
\end{Bash}

Le programme demande ensuite les dimensions du plateau (X et Y) et il va générer au fur et à mesure le G-Code pour balayer le plateau.

\subsection{Code source}

\begin{Python}
#!/usr/bin/python
# -*- coding: utf-8 -*-

from math import floor
import serial
import time

ser = serial.Serial( '/dev/ttyACM0', 115200,serial.EIGHTBITS, serial.PARITY_NONE, serial.STOPBITS_ONE, timeout=0.1)
ser.flushInput()  #Vide le buffer

if(open):
    print("Communication fonctionnelle !")
else:
    print("Mauvaise communication")

decalagePas = 8 #moitie du porte outil en mm

yWidth = input("Quelle est la longueur du plateau ? (mm) : ")
xWidth = input("Quelle est la largeur du plateau ? (mm) : ")

yIter = floor(int(yWidth)/8)
xIter = floor(int(xWidth)/8)

nbBlocX = xIter/2
deplacementZ = 10 #mm
vitesse=4 #max 6

print("Nombre d'arret sur X : "+str(xIter))
print("Nombre d'arret sur Y : "+str(yIter))

def sendCommand(command):
    print("Commande : "+str(command))
    ser.write(command+'\n')
    result=""
    waitingOK = True
    while (waitingOK==True): # tant que au moins un caractère en réception
        char=ser.read() # on lit le caractère
        if char=='\n': # si saut de ligne, on sort du while
            result=""
        else: #tant que c'est pas le saut de ligne, on l'ajoute à la chaine 
            result=result+char	
            if(result=="<ok>"):
                print("Commande valide")
                waitingOK = False


#Init
print("Mode relatif")
sendCommand("G91") #Mode relatif

#Retour Home
print("Retour origine")
sendCommand("G28 X")
sendCommand("G28 Y")


#Generation du GCode
gcode =""
for xBloc in range(0, int(nbBlocX)):
    #Code bloc
    
    for xStep in range(0,int(yIter)):
        gcode += "G01 Y"+str(decalagePas)+" F4"+"\n"
        gcode += "G04 P0.1"+"\n"
    
    gcode += "G01 X"+str(decalagePas)+" F"+str(vitesse)+"\n"
    
    for xStep in range(0,int(yIter)):
        gcode += "G01 Y-"+str(decalagePas)+" F4"+"\n"
        gcode += "G04 P0.1"+"\n"
    
    gcode += "G01 X"+str(decalagePas)+" F"+str(vitesse)+"\n"
    

print(gcode)

allCommands = gcode.split("\n")
print("Nombre d'instructions a executer :" +str(len(allCommands)))

for line in allCommands:
    sendCommand(line)
        
 

\end{Python}



\section{Une première interface}

Voici la première version d'une interface graphique réalisée avec PyQt4.

Le code est disponible sur le Git du club à \\
l'adresse \link{\url{https://github.com/CREPP-PLOEMEUR/Feutrage-de-laine}}

\subsection{Utilisation de l'interface}

Tout d'abord, veuillez télécharger le dossier complet puis se rendre dans le dossier \dir{V4\_GUI\_PyQt4}

Ensuite, faire click-droit et \bold{Ouvrir un terminal ici}

Ensuite, saisir la commande suivante pour lancer l'interface : 

\begin{Bash}{Ouverture de l'interface}
python GUIMain.py
\end{Bash}

L'interface se lance. Elle se compose de 3 onglets : 

\begin{items}{blue}{\Triangle}
    \item Présentation
    \item Paramètres
    \item Lancement du cycle
\end{items}

Veuillez vous rendre à l'onglet \bold{Présentation}

\img{\rootImages/c1.png}{Onglet Présentation}{0.25}

Il détaille le déplacement du porte aiguilles sur le schéma.\\

Une barre de statut est disponible sur tous les onglets et indique l'état courant du cycle. Au démarrage de l'application, il indique \bold{Non connecte}.\\

il faut donc se connecter à la machine.\\
Pour cela, il faut sélectionner le port série disponible dans la liste (ttyACMX généralement) et cliquer sur \shortcut{Connect}.\\
En cas de bonne connexion, la barre de statut indique \bold{Connecte}.

\img{\rootImages/c2.png}{Connexion effectué}{0.5}

Il faut ensuite cliquer sur l'onglet \bold{Paramètres}

\img{\rootImages/c3.png}{Paramètres du cycle}{0.25}

Voici les différents paramètres du cycle
\begin{items}{blue}{\Triangle}
    \item La longueur du plateau en mm [0-300mm comme valeur possible]
    \item La largeur du plateau en mm [0-300mm comme valeur possible]
    \item Déplacement du porte-aiguilles : Cela correspond à la distance parcourue (en mm) par le porte-aiguilles entre deux séquences de poinçonnage
    \item Déplacement de l'axe Z : Cela correspond à l'image de la distance pour effectuer une rotation pour planter le porte-aiguilles et le faire remonter en position de repos. Il faudra jouer avec cette valeur pour que le porte-aiguilles fasse précisément un aller-retour jusqu'à la position de repos.
    \item Nombre d'actions de l'outil-aiguilles : Cela correspond au nombre d'aller-retour effectués par le porte aiguilles entre 2 déplacements (X ou Y) sur le plateau
    \item Vitesse sur l'axe X et Y : La vitesse de déplacement sur l'axe X et Y comprise entre 1 et 6
    \item Vitesse sur l'axe Z : La vitesse de déplacement sur l'axe Z comprise entre 1 et 10
\end{items}

Une fois tous les paramètres renseignés, nous allons pouvoir lancer un cycle. Pour cela, il faut se rendre dans l'onglet \bold{Lancement du cycle}

Il ne reste plus qu'à lancer le cycle en appuyant sur le bouton \shortcut{Lancer le cycle}.

\img{\rootImages/c4.png}{Lancement d'un cycle}{0.25}

Une fois le cycle lancé, il est possible de l’arrêter en cliquant sur \shortcut{Arreter le cycle}.

\img{\rootImages/c5.png}{Un cycle en cours}{0.25}

Lorsque le cycle est arrêté, si vous cliquez sur \shortcut{Lancer le cycle}, ce dernier va reprendre du début, c'est à dire en coordonnées (0,0).









