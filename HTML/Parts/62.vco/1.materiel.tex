
\chapter{Cahier des charges}

\subsection{Caractérisation du matériel}

Pour réaliser notre synthétiseur analogique, nous avons à disposition  :


\begin{items}{blue}{\Bullet}
    \item Un MCV4 (convertisseur MIDI / Tension) alimenté en 9V (tension continue)
    \item Un clavier générant une trame MIDI alimenté en 12V (tension continue)
    \item une plaque d’expérimentation LABDEC
    \item Des composants électroniques de base (résistances, condensateurs, AOP TL081...)
\end{items}

Dans un premier temps, nous allons caractériser la tension du MCV4 en fonction de la note jouée au clavier.\\
 Le MCV4 sera en mode Volt/Hz afin de simplifier le dimensionnement du système (relation linéaire entre les grandeurs d’entrée). \\

On cherche donc à déterminer la relation qui existe entre le signal d’entrée du MCV4 qui code la fréquence des notes jouées au piano et la 
tension analogique délivrée par le MCV4. Pour cela, on mesure la tension en sortie du MCV4 à l’aide d’un voltmètre. Pour chaque note jouée, on obtient :


\img{\rootImages/tableur.png}{Equation du VCO}{1}


Ainsi, nous obtenons la relation suivante : 


$$ Tension_{MCV4} =0.00217020673595 \cdot Frequence + 0.002907571050445 $$ 


Cette relation servira donc à vérifier la validité du système final en comparant la fréquence de sortie de notre 
synthétiseur avec la fréquence théorique donnée par cette relation.\\

\section{Contraintes}

Le montage général est imposé et sera sous cette forme:

\img{\rootImages/global.png}{Forme global du montage}{1}

Ce système devra générer un signal:

\begin{items}{blue}{\Bullet}
    \item Triangulaire
    \item Symétrique avec une composante moyenne nulle
    \item de valeur crête à crête 12V
    \item De fréquence f (f=1/T) qui correspond à la note jouée au piano
\end{items}

On va imposer certaines valeurs pour certains composants : 

\begin{items}{blue}{\Bullet}
\item une résistance R1 de 100K
\item un condensateur C de15nF
\item une alimentation symétrique +14 /-14V
\end{items}

Notre objectif sera donc de déterminer les valeurs de R3, R4 et R5 en sachant que R2 et R6 seront fixées de manière (presque) arbitraire.






