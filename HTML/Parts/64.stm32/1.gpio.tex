\chapter{Les GPIOs}

 \section{Question 1}

\subsection{1.a}

Les broches LEDs 

\begin{itemize}

  \item LED\_RED : \outputPin{OUTPUT} sur la broche \genericPin{PB4}
  \item LED\_BLUE : \outputPin{OUTPUT} sur la broche \genericPin{PA9}
  \item LED\_GREEN : \outputPin{OUTPUT} sur la broche \genericPin{PC7}

\end{itemize}

Les broche des interrupteurs 

\begin{itemize}

  \item SW\_UP : \inputPin{INPUT} sur la broche \genericPin{PA4}
  \item SW\_DOWN : \inputPin{INPUT} sur la broche \genericPin{PB0}
  \item SW\_LEFT : \inputPin{INPUT} sur la broche \genericPin{PC1}
  \item SW\_RIGHT : \inputPin{INPUT} sur la broche \genericPin{PC0}
  \item SW\_CENTER : \inputPin{INPUT} sur la broche \genericPin{PB5}

\end{itemize}

\subsection{1.b}

Étant donné que les LEDs sont configurées en mode \bold{anode commune}, 
il convient de mettre un niveau logique bas pour activer ces dernières. \\

\img{\rootImages/leds.png}{Schéma des LEDs}{0.6}

Un niveau logique haut ne créé pas de différence de potentiel aux bornes des LEDs, de ce fait, aucun courant ne circule.
  
\subsection{1.c}

Lors d'un appui sur un interrupteur, le niveau logique associé est un niveau haut (3.3V). \\
Ce niveau se justifie par le type de montage. En effet, nous distinguons un montage en mode \bold{Pull-down} 
avec la résistance à la masse, ce qui implique que lors d'un appui, le courant circule dans
la résistance et toute la tension est au borne de la résistance.

\img{\rootImages/buttons.png}{Schéma des interrupteurs}{0.4}

\section{Question 2}

\subsection{2.a)}
\subsubsection*{Instruction}
L'adresse du registre \reg{RCC\_AHB1ENR} est \adr{0x40023830}

\subsubsection{Instruction}
L'instruction est la suivante : 

\begin{Cpp}{Affectation de l'horloge au port C}
#define PORT_C 2

RCC->AHB1ENR |= 1u << PORT_C;	
\end{Cpp}
\subsubsection{Justification de l'instruction}

Le registre \reg{RCC\_AHB1ENR} est de la forme suivante (Avec GPIOAEN le bit de poids faible): \\
\bitred{RESERVED} \bitgreen{GPIODEN} \bitgreen{GPIOCEN} \bitgreen{GPIOBEN} \bitgreen{GPIOAEN} \\
\bitblue{X}\bitblue{X}\bitblue{X}\bitblue{X}\bitblue{X}\bitblue{X}    \quad\quad \quad\quad \bitgreen{X}   \quad\quad \quad\quad          \bitgreen{X}    \quad\quad \quad\quad  \bitgreen{X}    \quad\quad \quad\quad\bitgreen{X}\\



Pour synchroniser l'horloge au port voulue, il suffit de mettre à 1 le bit du port C.

Pour cela, on commence par placer un '1' devant le bon bit, dans notre cas sur \bold{GPIOCEN}. On utilise l'opérateur de décalage à gauche.\\



$1 << 2$ revient à écrire \bin{100} puis en faisant ou \bold{OU} logique, cela permet de passer le bit à 1 si ce dernier est à 0 et de ne rien changer si il est déja à 1.


Le fait de faire un \bold{OU} permet de ne pas affecter les autres bits du registre, ce qui est souhaité. 

\subsubsection{Exemple}
Prenons en considération les 4 premiers bits de notre registre \reg{RCC\_AHB1ENR} et observons le résultat avec notre opération.\\
Ici, le port B et A sont déja activés (LSB).\\

\begin{tabular}{rl|l}
  Opérateur & Données & Information \\
\hline
    & 0011 & (RCC\_AHB1ENR tronqué)\\
   | & 0100 & ($1 << pin$)\\
   \hline
   = & 0111 & (RCC\_AHB1ENR tronqué)\\
  
\end{tabular} \\

On constate que le port C est bien relié à l'horloge et que les autres ports n'ont pas été affectés.

\subsection{2.b)}

\begin{itemize}

\item \reg{GPIOC\_MODER} : Ce registre permet de définir le mode de la broche. Il existe 4 modes.

  \begin{itemize}
    \item 00 : entrée  (mode par défaut)
    \item 01 : sortie
    \item 10 : broches alternatives (SPI, I2C, UART...)
    \item 11 : analogique
  \end{itemize}

 \item \reg{GPIOC\_PUPDR} : Ce registre permet de définir les éventulles résistances de rappel. Il existe 4 modes.

  \begin{itemize}
    \item 00 : aucune résistance de rappel
    \item 01 : mode Pull-Up
    \item 10 : mode Pull-Down
    \item 11 : mode réservé
  \end{itemize}

  Dans notre cas, les broches \inputPin{PC0} et \inputPin{PC1} sont en entrée \bold{sans résistance de pull-up/pull-down}.

\end{itemize}

\subsubsection{Instructions}

  \begin{Cpp}{Configurations des broches en entrée}

  int pc0 = 0;                //Broche associée au port
  int pc1 = 1;                //Broche associée au port

  int upDown = 0b00;   //Aucune résistance de rappel
  int input = 0b00     //broche en entrée

  //Etat de la broche
  GPIOC->MODER=GPIOC->MODER & ~(0b11 << (pc0*2) ) | input << (pc0*2); 
  GPIOC->MODER=GPIOC->MODER & ~(0b11 << (pc1*2) ) | input << (pc1*2);

  //pull-up/pull-down ?
  GPIOC->PUPDR=GPIOC->PUPDR & ~(0b11 << (pc0*2) ) | upDown << (pc0*2);
  GPIOC->PUPDR=GPIOC->PUPDR & ~(0b11 << (pc1*2) ) | upDown << (pc1*2);

\end{Cpp}

\subsubsection{Justification de l'instruction}

Le registre \reg{GPIOC\_MODER} est de la forme suivante, de manière tronquée (Avec MODER0 les 2 bits de poids les plus faibles): \\
\bitgreen{MODER3} \bitgreen{MODER2} \bitgreen{MODER1} \bitgreen{MODER0}\\
\quad\quad\quad\bitgreen{X}\bitgreen{X} \quad\quad\quad\bitgreen{X}\bitgreen{X}\quad\quad\quad\bitgreen{X}\bitgreen{X}    \quad\quad\bitgreen{X}\bitgreen{X}\\


Où \bitgreen{XX} represente l'état de la broche.\\

On souhaite mettre nos deux broches en entrée. Pour cela, on va utiliser l'opérateur de décalage pour sélectionner le duo de bits voulus (en fonction du numéro de la broche). \\

Cependant, on souhaite écraser la paire de bits car si on utilise la technique précédente, si on veut passer du mode  \bold{11} au mode \bold{00}\footnote{Ce mode ne sera pas utilisé mais cela permet de rendre l'opération générique pour tous les modes}, le bit de droite ne sera pas affecté.\\

On va donc réinitialiser les deux bits en faisant un complément de la valeur \bold{0x3} avec un décalage de \bold{$2$ fois le numéro de la broche.} \\


Ce coefficient est justifié par le fait que l'état de chaque broche est défini sur 2 bits. \\

Ensuite, il ne nous reste plus qu'à faire un \bold{ET} avec le registre pour ne pas modifier les autres broches puis faire un \bold{OU} avec notre mode souhaité. \\
Il est impératif que le mode soit exprimé en valeur hexadécimal.

\subsubsection{Exemple}
Prenons en considération les 8 premiers bits de poids faibles (4 premières broches du port) de notre registre \reg{GPIOC\_MODER} et observons le résultat avec notre opération.\\
On souhaite mettre la broche 4 au mode \bold{00} \\

\begin{tabular}{rl|l}
  Opérateur & Données & Information \\
\hline
    & 10110011 & (GPIOC\_MODER)\\
  \& & 00111111 & not(0b11 $<< 2*$ pin) \\
  \hline
  = & 00110011 &  \\
  | & 00000000 & (\bold{0b00}$<<2*$ pin)  \\
  \hline
   = & 00110011 & (GPIOC\_MODER)  \\
\end{tabular} \\

Le registre \reg{GPIOC\_PUPDR} est modifié de la même façon que \\
le registre \reg{GPIOC\_MODER}.


\subsection{2.c)}

\begin{itemize}

\item \reg{GPIOC\_MODER} : Voir question précédente

 \item \reg{GPIOC\_OTYPER} : Ce registre permet de définir le type de configuration de sortie. Il existe 2 modes.

  \begin{itemize}
    \item 0 : sortie push pull
    \item 1 : sortie à drain ouvert
  \end{itemize}

\item \reg{GPIOC\_OSPEEDR} : Ce registre permet de définir la vitesse des broches de sortie. Il existe 4 modes.

  \begin{itemize}
    \item 00 : Vitesse faible
    \item 01 : Vitesse intermédiaire
    \item 10 : Vitesse elevée
    \item 11 : Vitesse maximale
  \end{itemize}



\end{itemize}


\subsubsection{Instructions}

  \begin{Cpp}{Configurations des paramètres des sorties}
  int pc7 = 0;   //Broche associée au port
  
  int output_type = 0x0;   //sortie push-pull
  int speed = 0b01;        //vitesse medium
  int upDown = 0b00;       //Aucune résistance de rappel

  //Type de sortie
  GPIOC->OTYPER = GPIOC->OTYPER & ~(0b1 << pc7 ) | output_type << pc7; 
  //Vitesse
  GPIOC->OSPEED=GPIOC->OSPEED & ~(0b11 << (pc7*2) ) | speed << (pc7*2);
  //Pullup-down ?
  GPIOC->PUPDR=GPIOC->PUPDR & ~(0b11 << (pc7*2) ) | upDown << (pc7*2);

\end{Cpp}

\subsubsection{Présentation}

Les registres \reg{GPIOC\_OTYPE} et \reg{GPIOC\_SPEEDR} sont modifiés de la même façon que le registre \reg{GPIOC\_MODER}  si ce n'est que le registre \\
\reg{GPIOC\_OTYPER} ne nécéssite pas de décaler de deux fois le numéro de la broche car chaque information sur le type de sortie est stockée dans un bit et non deux.


\section{Question 3}

\subsection{a)}

\subsection{Instructions BSRR}

\begin{Cpp}{Instruction BSRR}
  
  #define pin 7 //Broche de la led sur le port C
  void green_led(uint32_t state) {

    GPIOC->BSRR = (state)?  1u << pin  :  0b1 <<(16u)<<pin;
    
  }
  

\end{Cpp}

\subsection{Justification de l'instruction BSRR}

Ce registre utilise les 16 premiers bits de poids les plus faibles pour mettre la sortie à 1 et 
les 16 bits suivants pour mettre la sortie à 0. \\

Lorsque \bold{state} est à 0, on veut donc écrire dans le bit BR7 qui force la broche à 0. 
D'ou le décalage de la valeur 1 de 16 bits puis du décalage de la pin. \\
On peut écrire la valeur brute dans le registre sans faire de masque dans la mesure où 
les changements d'états se font dès qu'un \bold{1} est présent. Au tour d'horloge suivant, les bits sont réinitialisés à 0.

Lorsque \bold{state} est à 1, on veut donc écrire dans le bit BS7 qui force la broche à 1. 
D'ou le décalage de la pin uniquement. \\



\subsection{Instructions ODR}

\begin{Cpp}{Instruction ODR}
  
  #define pin 7 //Broche de la led sur le port C
  void green_led(uint32_t state) {

    GPIOC->ODR = (state) ? GPIOC->ODR | (0b1 <<pin) : GPIOC->ODR & ~(0b1 <<pin);
     
  }
  

\end{Cpp}

\subsection{Justification de l'instruction ODR}

Ce registre utilise les 16 premiers bits de poids les plus faibles pour mettre la sortie à 1 ou 0.

Lorsque \bold{state} est à 1, on veut donc écrire dans le bit ODR7 qui force la broche à 1. D'ou le décalage de la valeur 1 de la pin puis le \bold{OU} pour ne pas affecter les autres sorties. \\


Lorsque \bold{state} est à 0, on veut donc écrire dans le bit ODR7 qui force la broche à 0. 

Il suffit de réinitialiser le bit en le mettant à 0 avec un décalage du port d'un bit complémenté et en faisant un \bold{ET}
avec la valeur actuelle du registre.

\subsection{b)}

\begin{Cpp}{Instruction input}

uint32_t input() {

    uint32_t output;

    output = GPIOC->IDR & (0b1 << 0);  //Right
    output |= _GPIOC->IDR & (0b1 << 1); //Left
    
    return output;
}

\end{Cpp}

\subsection{Justification de l'instruction input}

Le registre \reg{GPIOC->IDR} est accessible en lecture seule.

Il suffit de faire un \bold{ET} avec la valeur 1 décalé de la valeur de la broche et de mettre le résulat dans une variable.

On effectue la même opération si ce n'est que l'on fait un \bold{OU} avec la variable \bold{output} afin de ne pas écraser le bit lu précédemment.


\subsection{c) Code final}


\begin{Cpp}{Code final}

  #define LED 7
  
  void green_led(uint32_t state){

    GPIOC->BSRR = (state)? 0b1 << LED : 0b1 << (16u) << LED;

  }//End green_led

  #define SW_RIGHT (1u)
  #define SW_LEFT (1u<<1)

  uint32_t input(){

    uint32_t output;
    output = _GPIOC->IDR & (0b1)); //Right
    output |= _GPIOC->IDR & (0b1 << 1); //Left
  
    return output;

  }//End input

  int main(){
  
    _RCC->AHB1ENR |= RCC_AHB1ENR_GPIOCEN; //Clock

    //Settings LED
    //Output
    GPIOC->MODER=GPIOC->MODER & ~(0b11 << (LED*2) ) | 0b01 << (LED*2); 
    //Type of output
    GPIOC->OTYPER = GPIOC->OTYPER & ~(0b1 << LED ) | 0b0 << LED; 
    //Medium speed
    GPIOC->OSPEED=GPIOC->OSPEED & ~(0b11 << (LED*2) ) | 0b01 << (LED*2);
    //no Pullup-down
    GPIOC->PUPDR=GPIOC->PUPDR & ~(0b11 << (LED*2) ) | 0b00 << (LED*2);


    //Settings buttons SW_LEFT (PC1) and SW_RIGHT (PC0)
    //Right button
    GPIOC->MODER=GPIOC->MODER & ~(0b11 << (0*2) ) | 0b00 << (0*2); 
    GPIOC->PUPDR=GPIOC->PUPDR & ~(0b11 << (0*2) ) | 0b00 << (0*2);
    //Left buttons
    GPIOC->MODER=GPIOC->MODER & ~(0b11 << (1*2) ) | 0b00 << (1*2); 
    GPIOC->PUPDR=GPIOC->PUPDR & ~(0b11 << (1*2) ) | 0b00 << (1*2);
  
    while (true){
  
      uint32_t button_value = input();
  
      if(button_value & (1u)){
  
        green_led(0);
  
      }//End if
  
      else if(button_value & (1u << 1) ){
      
        green_led(1u);

      }//End else if

    }//End while

  return 0;
  }//End main

\end{Cpp}

