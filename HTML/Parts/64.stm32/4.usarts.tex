\part{Les USARTS}
\chapter{Présentation}

L'objectif de cette section est de pouvoir communiquer par liaison série avec la maquette équipée d'un microcontrôleur STM32F411.\\
À cette fin, un terminal est utilisé sur le PC de développement pour gérer un des ports du PC. Il permettra de recevoir ou d'émettre des caractères depuis ou vers la maquette.\\

Sur la maquette, les signaux RxD et TxD de l'USART2 du microcontrôleur sont connectés au circuit STLINK, présent sur la carte NUCLEO, qui permet la connexion au PC de développement via une liaison USB. Cette liaison sera transparente pour l'utilisateur : seule la liaison série gérée par l'USART2 est étudiée. Aucun contrôle de flux ne sera pris en charge, seuls les signaux d'émission TxD et de réception RxD de caractères seront utilisés. \\

Les fonctions développées pour gérer une USART permettront de gérer les USART présentes dans
le microcontrôleur même si seule l'UART2 sera utilisée dans ce laboratoire. \\

Pour communiquer par liaison série avec la maquette, un terminal doit être ouvert sur le PC
de développement : Applications | Programmation | GtkTerm. \\
Lorsque la fenêtre de ce terminal est active, il est associé au clavier et à un port du PC. Avant de l'utiliser il est nécessaire de configurer le terminal : choix du port associé, format des trames utilisées, débit, ... S'assurer qu'un port série virtuel est disponible dans la liste des ports séries proposés par le terminal. Le sélectionner et le configurer pour avoir les mêmes spécifications que celles de l'USART.\\

\bold{Sélectionner le port ttyACM0}
\chapter{Fonctions de base}

\section{Émission et réception d'un caractère (coupleur géré par scrutation)}

\subsection{Définition de uart\_init}

Pour information, la présentation des registres et instructions utilisées ne tient pas compte des \bold{\#ifdef USE\_USARTX} pour éviter d'alourdir la présentation.\\
Cependant, le code finale de la fonction \func{uart\_init} comprendra les conditions.


\subsubsection{Activation de l'horloge}

\begin{Cpp}{Activation de l'horloge} 
//Activation de l'horloge sur USART1
_RCC->APB2ENR |= 1<<4;

//Activation de l'horloge sur USART2
_RCC->APB1ENR |= 1<<17;

//Activation de l'horloge sur USART6
_RCC->APB2ENR |= 1<<5;
\end{Cpp}

\img{\rootImages/clockUSART.png}{Registre APB1ENR}{0.6}

\img{\rootImages/usart16.png}{Registre APB2ENR}{0.6}


\subsubsection{Association des signaux}

\begin{Cpp}{Paramétrage du type de broche} 
#define USART_PIN_CONFIG (PIN_MODE_ALTFUNC | PIN_OPT_RESISTOR_NONE | PIN_OPT_AF7)
\end{Cpp}



On définit ici que ce sont des broches alternatives utilisée pour l'USART sans résistance de rappel.
\begin{Cpp}{\func{io\_configure}} 
//USART1
//Cette fonction ne pourra pas être appelée
io_configure(USART1_GPIO_PORT, USART1_GPIO_PINS, USART_PIN_CONFIG, NULL);

//USART2
io_configure(USART2_GPIO_PORT, USART2_GPIO_PINS, USART_PIN_CONFIG, NULL);
//USART_GPIO_PORT is PIN2 | PIN3

//USART6
//Cette fonction ne pourra pas être appelée
io_configure(USART6_GPIO_PORT, USART6_GPIO_PINS, USART_PIN_CONFIG, NULL);
\end{Cpp}


\subsubsection{Gestion du baudrate}

\begin{Cpp}{Paramétrage du baudrate}
//USART1
u->BRR = (sysclks.apb2_freq/baud);

//USART2
u->BRR = (sysclks.apb1_freq/baud);

//USART6
u->BRR = (sysclks.apb2_freq/baud);
\end{Cpp}

ici, chaque USART doit être paramétré avec la fréquence apb qui lui est propre.\\


\subsubsection{Activation de l'USART}

\begin{Cpp}{Activation de l'USART}
//Activation de l'USART
u->CR1 |=	(1 << 13);

//Activation de la transmission 
u->CR1 |=	( 1 << 3);
//Activation de la réception
u->CR1 |=	( 1 << 2);
\end{Cpp}

\img{\rootImages/13.png}{Registre CR, bit 13}{0.6}
\img{\rootImages/3.png}{Registre CR, bit 2}{0.6}
\img{\rootImages/2.png}{Registre CR, bit 3}{0.6}

\subsubsection{Configuration pour les interruptions}

Pour chaque USARt, nous allons préparer l'éventuelle utilisation de l'USART avec les interruptions sur RX.

\begin{Cpp}{Interruptions}
//USART1

usart1_cb = cb;     //On enregistre la fonction à appeler
irq_number = 37;    //Depuis la table des vecteurs
irq_priority = 44;  //Niveau de priorité

//USART2

usart2_cb = cb;
irq_number = 38;
irq_priority = 45;

//USART6

usart6_cb = cb;
irq_number = 71;
irq_priority = 78;

\end{Cpp}


Il ne reste plus qu'à autoriser les interruptions si \bold{cb} n'est pas null.

\begin{Cpp}{Autorisations d'interruptions}

if (cb) {
			
	u->CR1 |= USART_CR1_RXNEIE;

	NVIC_EnableIRQ(irq_number); //Active l'interruption
	NVIC_SetPriority(irq_number,irq_priority);
}//End if cb

\end{Cpp}

\subsection{Code uart\_init complet}

\begin{Cpp}{Code uart\_init complet}

int uart_init(USART_t *u, uint32_t baud, uint32_t mode, OnUartRx cb)
{
	IRQn_t	irq_number;
	uint32_t irq_priority;

	if (u == _USART1) {
#ifdef USE_USART1
		//Activation de l'horloge
		_RCC->APB2ENR |= 1<<4;
		io_configure(USART2_GPIO_PORT, USART2_GPIO_PINS, USART_PIN_CONFIG, NULL);
		u->BRR = (sysclks.apb2_freq/baud);
		//fonction d'interuption
		usart1_cb = cb;
		irq_number = 37;
		irq_priority = 44;
#else
	return -1;
#endif
	 } 
	 else if (u == _USART2) {
#ifdef USE_USART2
		//Activation de l'horloge
		_RCC->APB1ENR |= 1<<17;
		io_configure(USART2_GPIO_PORT, USART2_GPIO_PINS, USART_PIN_CONFIG, NULL);
		u->BRR = (sysclks.apb1_freq/baud);
		//fonction d'interuption
		usart2_cb = cb;
		irq_number = 38;
		irq_priority = 45;
#else
	return -1;
#endif

	} else if (u == _USART6) {
#ifdef USE_USART6
		//Activation de l'horloge
		// configure Tx/Rx pins
		_RCC->APB2ENR |= 1<<5;
		//io_configure(USART6_GPIO_PORT, USART2_GPIO_PINS, USART_PIN_CONFIG, NULL);
		u->BRR = (sysclks.apb2_freq/baud);
		//fonction d'interuption
		usart6_cb = cb;
		irq_number = 71;
		irq_priority = 78;
#else
	return -1;
#endif
	}
		//format des données 8/9bits
		u->CR1 |= ((mode & 0b1) << 12);
		//Bit de stop
		u->CR2 |= (((mode>>4) & 0b11) << 12); //move mode to left to get field
		//Parité
		u->CR1 |= (((mode>>8) & 0b111) << 8);
		//USART ENABLE
		u->CR1 |=	(1 << 13);
		//Transmitter Enable
		u->CR1 |=	( 1 << 3);
		//Receiver Enable
		u->CR1 |=	( 1 << 2);
 
	// Setup NVIC
	if (cb) {
			
		u->CR1 |= USART_CR1_RXNEIE;

		NVIC_EnableIRQ(irq_number); //Active l'interruption
		NVIC_SetPriority(irq_number,irq_priority);

	}
    return 1;
}

\end{Cpp}
	

\subsection{Définition de uart\_putc }
 
 \begin{Cpp}{Définition de la fonction \func{uart\_putc}}
 
void uart_putc(USART_t *u, char c)
{
	while (!(u->SR & (1<<6))) //on attend que le traitement du premier caractère
	{
		//Ne rien faire
	}
  	u->DR = c;
	while (!(u->SR & (1<<6))) //on attend que le traitement du premier caractère
	{
		//Ne rien faire
	}
}//End uart_putc
\end{Cpp}

Pour envoyer un caractère, il suffit de mettre un \bold{char} dans le registre \reg{DR}
\img{\rootImages/registre_DR.png}{Registre DR}{0.6}

\subsection{Définition de uart\_puts}

 \begin{Cpp}{Définition de la fonction \func{uart\_puts}}
void uart_puts(USART_t *u, char *s)
{
	while (!(u->SR & (1<<6))) //on attend que le traitement du premier caractère
	{
		//Ne rien faire
	}
    while (*s != 0)
    {
        u->DR = *s;  //On affecte au registre DR le contenu à l'adresse s
        s++;        //On applique l'arithmétique des pointeurs
		while (!(u->SR & (1<<6))) //on attend que le traitement du premier caractère
		{
			//Ne rien faire
		}
    }
}
\end{Cpp}

\subsection{Définition de uart\_getc}
\begin{Cpp}{Fonction getc()}
char uart_getc(USART_t *u) {

    //On attend que le bit 5 passe a 1 (donnée reçu)
	while(!(u->SR & (1<<5))){};
	return (char)u->DR;//la valeur est contenue dans le registre DR

}
\end{Cpp}

\img{\rootImages/registre_SR.png}{Registre SR}{0.6}
\img{\rootImages/registre_SR_Bit5.png}{Registre SR, bit 5}{0.6}

\section{Test de l'émission d'un caractère}

Pour vérifier l'émission d'un caractère sans interruption, nous allons tester le code du MAIN1 : 

\begin{Cpp}{Code MAIN1}
int main()
{
	uart_init(_USART2,115200,UART_8N1,NULL);//On initialise la communication à 115200 bauds, pas de bits de parité, 1 bit de stop.
	uart_puts(_USART2, "Entrez du texte ici : ");//On affiche une chaine de caractère sur le terminal GTkTerm

	uart_putc(_USART2, 'A');//On envoie le caractère 'A'
	uart_putc(_USART2, 'B');//On envoie le caractère 'B'
	uart_putc(_USART2, 'C');//On envoie le caractère 'C'
	
	uart_puts(_USART2,"\r\nC'est un message du STM32F411 :-)\r\n");

	uart_puts(_USART2, "Entrez du texte ici : ");

	while (1) {
		uart_putc(_USART2, uart_getc(_USART2));
	}
	//ici, on scrute tous les caractères entrant (du terminal vers la carte) et on les renvoie aussitôt sur le terminal, ce qui fait que cette boucle fait une recopie du clavier.

	return 1;
}
\end{Cpp}


\section{Réception d'un caractère par interruption}

Le code nécessaire à la configuration des interruptions par réception de caractère a été défini dans la fonction \func{uart\_init}.

\section{Test de la réception d'un caractère}


Pour vérifier l'émission d'un caractère avec interruption, nous allons tester le code du MAIN2 : 

\begin{Cpp}{Code MAIN2}
static void on_rx_cb(char c) //Fonction à appeler lors d'une réception d'un caractère.
{
	uart_putc(_USART2, c);//On renvoie le caractère du buffer vers le terminal.
}

int main()
{
	uart_init(_USART2, 115200, UART_8N1, on_rx_cb);
	//On définit que la réception d'un caractère provoque l'appel de la fonction on_rx_cb.

	uart_puts(_USART2, "Entrez du texte ici : "); //On affiche du texte

	while (1) ;

	return 1;
}
\end{Cpp}

A première vue, on peut penser que le code ne fait rien.\\
Or, à chaque appui sur une touche du clavier, la carte va recevoir une demande d'interruption par la broche \bold{RX} et va appeler la fonction\func{on\_rx\_cb.}. La recopie du clavier est donc transparente pour l'utilisateur.

\chapter{Gestion d'un terminal}

\section{Présentation}

Un terminal "VT100" prend en compte des séquences de caractères permettant de modifier certains de ses attributs.\\
On pourra, par exemple, effacer l'écran du terminal, positionner le curseur à un emplacement choisi, changer la couleur des caractères affichés, etc.\\
Les commandes VT100 commencent toutes par la séquence de caractères ESC + '[', soit la chaîne "\\x1b["

Il est possible de recevoir, selon la touche appuyée sur le clavier, plus d'un caractère hexadécimal.\\
Des caractères accentués ne faisant pas partie de la table ASCII (codage sur 7 bits) peuvent être obtenus par un codage sur plusieurs octets (UTF-8).\\
Dans ce cas, l'émission d'un caractère (lettre) accentué se traduit par plus d'un caractère hexadécimal à prendre en compte à la réception.\\
D'autres touches associées à des fonctions de contrôle, telles que "page up", flèches gauche/droite, etc, génèrent également des séquences de caractères hexadécimaux.\\
Dans tous les cas, pour interpréter correctement l'action sur le clavier, il est nécessaire de prendre en compte la séquence hexadécimale complète reçue.

\section{Tests}

\begin{Cpp}{Code MAIN3}
static void on_rx_cb(char c)
{
	char  s[34];				//Chaine de 34 caractère max
	num2str(s,c,16);			//Permet de prendre en compte la séquence hexadéciaml complète
	uart_puts(_USART2, " 0x");	//début de séquence
	uart_puts(_USART2, s);		//On envoie la séquence
}


int main()
{
	//Initialisation de la communication à 115200 bauds avec interruptions sur la broche RX.
	uart_init(_USART2,115200,UART_8N1,on_rx_cb);

	//On efface le terminal
	uart_puts(_USART2,"\x1B[2J\x1B[H");

	uart_puts(_USART2,"On affiche un message ici");

	// positionnement du curseur ligne 20, col 5
	uart_puts(_USART2,"\x1B[20;5H");

	// on écrit en couleur
	uart_puts(_USART2,"\x1B[31mA partir de maintenant, entrez du texte :\x1B[0m");

	while (1) ;

	return 1;
}
\end{Cpp}
\chapter{Fonction printf}

\section{Prise en compte d'un nombre variable d'arguments}

\begin{items}{black}{\Bullet}

\item void va\_start(va\_list ap, param) : la macro va\_start fait pointer ap
sur le premier argument variable fourni à la fonction. param est le nom du
dernier argument nommé

\item type va\_arg(va\_list ap, type) : la macro va\_arg renvoie le premier
argument variable et fait pointer ap sur l'argument suivant. type est le type de
l'argument qui va être lu. La macro va\_arg génère une expression de ce même
type
\item void va\_end(va\_list ap) : la macro va\_end remet tout à état initial avant
le retour à la fonction appelante.
\end{items}

\section{Codage de la fonction printf allégée}


Tout d'abord, on va créer des variables locales à la fonction.

\begin{Cpp}{Variables locales}
	char* tmp_string;		//Chaîne de caractère demandée
	int tmp_s_int;			//Entier signé
	unsigned int tmp_u_int; //Entier non signé
\end{Cpp}

\subsection{Les caractères}

\begin{Cpp}{Gestion des caractères}
	case 'c':
		ch = va_arg(ap, char);  	//On récupère le paramètre correspondant à "\%c" (caractère)
	    uart_putc(u,ch); 			//Envoie du caractère
		break;
\end{Cpp}


\subsection{Les chaînes de caractères}

\begin{Cpp}{Gestion des chaînes de caractères}
    case 's':
		tmp_string = va_arg(ap,char*);
		uart_puts(u,tmp_string); //On récupère le pointeur associé à la chaîne de caractère
		break;
\end{Cpp}

\subsection{Les entiers non-signés}

\begin{Cpp}{Gestion des entiers non-signés}
	case 'u':
		tmp_u_int = va_arg(ap,unsigned int);	//On récupère le paramètre corespondant à "\%u" (unsigned int)
		num2str(s,(unsigned int)tmp_u_int,10);	//on convertit le nombre en chaine de caractère base 10
					uart_puts(u,s);							//On envoie la chaine
					break;
\end{Cpp}

\subsection{Les entiers signés}

\begin{Cpp}{Gestion des entiers signés}
    case 'd':
	    tmp_s_int = va_arg(ap,int);
		if(tmp_s_int < 0) { //Si valeur négative
			uart_putc(u,'-'); 
			tmp_s_int = tmp_s_int *(-1);//Remet en valeur positive
		}//End if
		num2str(s,(unsigned int)tmp_s_int,10);	//on convertit le nombre en chaîne de caractère base 10
		uart_puts(u,s);						//On envoie la chaîne
		break;
\end{Cpp}

\subsection{Les caractères hexadécimaux}

\begin{Cpp}{Gestion des caractères hexadécimaux}
	case 'x':
	    tmp_u_int = va_arg(ap,unsigned int);	//On récupère le paramètre corespondant à "\%x" (hexa)
		num2str(s,(unsigned int)tmp_u_int,16);	//on convertit le nombre en chaine de caractère base 16
		uart_puts(u,s);							//On envoie la chaine
		break;
\end{Cpp}

\section{Code complet de la fonction uart\_printf}


\begin{Cpp}{Définition de uart\_printf}

	__gnuc_va_list ap;
	char          *p;
	char           ch;
	unsigned long  ul;
	char           s[34];

	char* tmp_string;		//Chaine de caractère demandée
	int tmp_s_int;			//Entier signé
	unsigned int tmp_u_int; //Entier non signé
	
	va_start(ap, fmt);
	while (*fmt != '\0') {
		if (*fmt =='%') {
			switch (*++fmt) {
				case '%':
					uart_putc(u,'%');
					break;
				case 'c':
					ch = va_arg(ap, char);  	//On récupère le paramètre corespondant à "%c" (caractère)
					uart_putc(u,ch); 			//Envoie du caractère
					break;
				case 's':
					tmp_string = va_arg(ap,char*);
					uart_puts(u,tmp_string); //On récupère le pointeur associé à la chaine de caractère
					break;
				case 'd':
					tmp_s_int = va_arg(ap,int);
					if(tmp_s_int < 0) { //Si valeur négative
						
						uart_putc(u,'-'); 
						tmp_s_int = tmp_s_int *(-1);//Remet en valeur positive
					}//End if
					num2str(s,(unsigned int)tmp_s_int,10);	//on convertit le nombre en chaine de caractère base 10
					uart_puts(u,s);						//On envoie la chaine
					break;
				case 'u':
					tmp_u_int = va_arg(ap,unsigned int);	//On récupère le paramètre corespondant à "%u" (unsigned int)
					num2str(s,(unsigned int)tmp_u_int,10);	//on convertit le nombre en chaine de caractère base 10
					uart_puts(u,s);							//On envoie la chaine
					break;
				case 'x':
					tmp_u_int = va_arg(ap,unsigned int);	//On récupère le paramètre corespondant à "%x" (hexa)
					num2str(s,(unsigned int)tmp_u_int,16);	//on convertit le nombre en chaine de caractère base 16
					uart_puts(u,s);							//On envoie la chaine
					break;
				default:
				    uart_putc(u, *fmt);
			}
		} else uart_putc(u, *fmt);
		fmt++;
	
	va_end(ap);
\end{Cpp}

\section{Tests}

Pour tester le code, nous allons exécuter le MAIN4.

\begin{Cpp}{Définition du MAIN4}

static void on_rx_cb(char c)
{
	char  s[34];
	num2str(s,c,16);
	uart_puts(_USART2, " 0x");
	uart_puts(_USART2, s);
}

int main()
{
	int a = 5, b = 8;

	uart_init(_USART2, 115200, UART_8N1, on_rx_cb);

	uart_puts(_USART2, "\x1B[2J\x1B[H");

	uart_printf(_USART2, "la somme de %d et %d est %d\n", a, b, a+b);
	uart_printf(_USART2, "\x1B[%u;%uHle pointeur _USART2 = 0x%x\n", 20,5,_USART2);

	while(1) {
	}

	return 0;
}
\end{Cpp}

Voici le résultat : 

\img{\rootImages/gtk.png}{Résultat du MAIN4}{0.6}

