

 \chapter{Conclusion}

 \section{Ce qu'il faut retenir}

 
 Nous avons à notre disposition tout un ensemble de technologies pour contrôler la partie puissance.

 Les transistors ne sont pas adaptés pour commuter une charge sur secteur (230V), cette partie sera donc réservée aux relais.\\
 En revanche, pour toutes les tensions continues, les transistors sont adaptés et prennent moins de place en encombrement.\\

 \section{Les fiches techniques}

 L'intégralité des informations disponibles pour un transistor sont disponibles dans un document complet appelé \bold{Datasheet}.

 Ce document détaille les broches, les caractéristiques électriques, propose des schémas d'exemples....\\

 Par exemple, voici quelques extraits de la documentation du transistors IRF520\footnote{Transistor de puissance} :

 \img{\rootImages/irf520.png}{Extrait n°1 du IRF520}{0.4}

 \img{\rootImages/rdson.png}{Extrait n°2 du IRF520}{0.5}

 On retrouve sur cette figure la valeur de $R_{DS_{on}}$ et de $V_{GS_{th}}$

