%\addPartText{Théorie sur les interfaces de puissance et applications pratiques avec Arduino}
\partImg{Les interfaces de puissance}{\rootImages/power.jpg}{0.5}
\chapter{Introduction}

Pour certains projets plus évolués, on souhaite utiliser des composants tels que des moteurs ou des résistances (chauffage, ventilation).\\

Or, on constate rapidement que le branchement direct de ces éléments sur une carte Arduino va se révéler impossible. \\

En effet, la carte Arduino est prévu pour délivrer de \bold{faibles courants} et \bold{faibles tensions}. Nous allons donc créer un circuit où la puissance et la commande sont dissociés.
