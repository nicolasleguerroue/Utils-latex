\chapter{Annexes}

\section{Liste des bibliothèques}

\begin{items}{orange}{\faFile}
    \item \lib{UART}
    Cette bibliothèque gère la communication série avec des périphériques en liaison série émulée. La bibliothèque utilise en interne la bibliothèque \lib{SoftwareSerial} pour émuler une liaison série.
    \item \lib{RS232}
    Cette bibliothèque gère la communication série avec des périphériques en liaison série native.
    \item \lib{Gyro}
    Cette bibliothèque gère le capteur MPU9250 afin de récupérer les données de la centrale IMU.
    \item \lib{SpaceViewer}
    Cette bibliothèque transforme les valeurs des angles obtenus par la centrale IMU en chaîne de caractère pouvant être transmise sur une liaison série.
    \item \lib{Rover}
    Cette bibliothèque gère le contrôle du Rover UQAC
    \item \lib{LED}
    Cette bibliothèque gère les LED.
    \item \lib{MusicPlayer}
    Cette bibliothèque gère les tonalités pouvant être générées sur un buzzer.
    \item \lib{LIDARLite}
    Cette bibliothèque gère les modules LIDAR.
\end{items}




\section{Installation de XCTU (Linux)}
\subsection{Lien de téléchargement}

\begin{enumerate}
 \item Se rendre à la page suivante : 
\url{https://hub.digi.com/support/products/xctu/?path=/support/asset/xctu-v-655-linux-x64/}

\item Puis Cliquer sur \bold{download}
\end{enumerate}

\img{\rootImages/download.png}{Image à télécharger}{0.5}


Un fichier au format \bold{.run} se télécharge à ce moment là. Il est appelé \italic{40002881\_AC.run}

\subsection{Don des permissions}

\begin{enumerate}
    \item Se placer dans le répertoire contenant le fichier téléchargé \\
    \item Ouvrir un terminal et lancer la commande \\
\end{enumerate}

\begin{Bash}{Don des droits}
chmod +x 40002881_AC.run
\end{Bash}
\subsection{Lancement de l'installateur}

\begin{enumerate}

    \item Exécuter le programme en saisissant 
    
\begin{Bash}{Lancement de l'installation de XCTU}
./40002881_AC.run
\end{Bash}

    \item Veuillez éviter d'installer le logiciel dans le répertoire `/opt/`, des problèmes de droits d'accès pourraient survenir. \\
    Il est recommandé de l'installer dans son répertoire personnel (/home/user)

    \item Valider toutes les étapes. 

\end{enumerate}

\subsection{Première ouverture du logiciel}

\begin{enumerate}

    \item Se placer dans le répertoire où est installé le logiciel (choisi par vos soins lors de l'installation) 

\item Lancer la commande :
\begin{Bash}{Lancement de XCTU}
./app
\end{Bash}

(Si la commande ne se lance pas, répéter l'opération 1.2 avec le fichier \file{app}
    
\end{enumerate}

\subsection{Création du raccourci}

Il est possible que le raccourci sous Linux pour lancer XCTU ne se créer pas. Pour cela, éditer le fichier \file{bash\_aliases} via les commandes suivantes : 

\begin{Bash}{Édition des alias}
cd ~
sudo nano .bash_aliases
\end{Bash}

Puis ajouter la ligne 
\begin{Bash}{Ajout de l'alias}
alias xctu=Emplacement_dossier_XCTU/XCTU-NG/app
\end{Bash}
(Ctrl+X permet de sauvegarder le script)

puis 
\begin{Bash}{Mise à jour de l'alias}
source .bashrc
\end{Bash}

\subsection{Lancement du logiciel}

Dans un terminal, saisir
\begin{Bash}{Lancement de XCTU}
xctu &
\end{Bash}


\section{Reconnaissance des modules Xbee}

\subsection{Ouverture de la liaison}

\begin{enumerate}
\item  Brancher le module XBee coté ordinateur au module FTDI 

\item Cliquer sur l'image avec la loupe pour ajouter un appareil 


\img{\rootImages/loop.png}{Ajouter un appareil}{1}


Une liste avec un nombre de port série trouvés apparaît.

\img{\rootImages/select_usart.png}{Afficher les ports}{0.4}

\item Il suffit de sélectionner celui du module puis cliquer sur \bold{Next}.

Un scan pour détecter le type de module se lance.

\img{\rootImages/found.png}{Scan des modules}{0.7}

Lorsque qu'il a trouvé le module, faire \bold{Next}

\item Une fenêtre de paramétrage apparaît.

\img{\rootImages/default.png}{fenêtre XCTU}{0.5}

On ne touche pas aux paramètres par défaut qui sont : 

\begin{items}{gray}{\faGear}
    \item Aucun bit de parité
    \item Vitesse de communication à 9600
    \item 8 bits de données
    \item 1 bit de stop
\end{items}

Puis \bold{Next}

Les modules ajoutés de cette manière seront visibles sur le menu latéral gauche.

\img{\rootImages/left_menu.png}{Un module sélectioné}{0.5}


En cliquant dessus, le menu de paramétrage du module devient visible sur le coté droit du logiciel.

L'onglet des paramétrages ressemble à ceci : 

\img{\rootImages/right.png}{L'onglet de paramétrage}{0.7}

C'est ici que nous allons définir les paramètres de communication \bold{entre 2 modules Xbee}

\end{enumerate}

\label{xbee}
\section{Paramétrage des modules Xbee}

\subsection{Choix des paramètres}

Pour pouvoir communiquer, les modules doivent être sur le même canal (channel) et avoir le même identifiant PAN (Personnal Area Network, qui peut être vu comme un sous-réseau).\\


Utilisons le canal \bold{C} (via \bold{CH Channel}) et  \bold{PAN ID} à $2000$.\\
Chaque module possède une adresse propre. Mettons là à $1001$ via l'onglet \bold{MY}.
Ce module communiquera avec un module ayant une adresse de $1000$.
 Nous mettons donc le champ \bold{DL} à $1001$.

En résumé, nous avons ces paramètres :

\img{\rootImages/settings.png}{paramètres du Xbee}{0.7}


\messageBox{Remarque}{orange}{white}{
Il faut bien vérifier que le mode API est désactivé, ce qui permet au module XBee de renvoyer toutes les données reçues.\\
Il faut mettre le mode \bold{API Mode à 0} dans la partie \bold{Serial interfacing}
}{black}

\img{\rootImages/api_mode.png}{Désactivation du mode API}{0.7}


Le module est configuré, il ne reste plus qu'à transférer les paramètres

\subsection{Transfert des paramètres}

Nous allons envoyer les paramètres au module Xbee via le logiciel.

Il suffit de cliquer sur le bouton \bold{Write}

\img{\rootImages/write.png}{Transfert des paramètres}{0.8}

Nous allons procéder aux mêmes opérations pour le deuxième module Xbee.

Celui-ci aura pour paramètres : 

\begin{items}{gray}{\faGear}
\item CHANNEL : C
\item PAN ID : 2000
\item DL : 1001
\item MY : 1000
\end{items}


\section{Communication entre deux modules Xbee}

Une fois les deux modules correctement configurés, nous allons mettre en évidence les échanges de données.\\

Pour cela, nous pouvons utiliser :

\begin{items}{green}{\faLeaf}
 \item XCTU
\item  GTKterm
\item  Minicom
\item  Putty
\item  Screen
\end{items}

Par simplicité, nous utiliserons le logiciel \ib{XCTU}

\subsection{Configuration minimale}

Il faut au préalable que les deux modules soit ajoutés dans le logiciel XCTU comme suivant

\img{\rootImages/all.png}{}{0.5}


\subsection{Lancement de la console XCTU}

Le logiciel XCTU propose une console pour échanger des données.

La console est disponible pour chaque module.

\begin{enumerate}
    
    \item Cliquer sur un des modules

    \item Faire \shortcut{Alt+c} ou bien cliquer sur l'image de la console en haut à droite

\img{\rootImages/console.png}{Afficher la console}{0.5}


La console occupe désormais la page centrale 

\img{\rootImages/console_2.png}{Console XCTU}{0.3}

\item Cliquer sur le bouton \bold{Open}

\img{\rootImages/open.png}{Bouton Open}{0.8}

\item La console passe de gris clair à blanc. 

\item Procéder de la même façon pour le second module (de l'étape 1 à la 4)
\end{enumerate}
\subsection{Écriture-lecture}

\begin{enumerate}
    
    \item Cliquer sur une des console ouverte et saisir un texte dans la page de gauche.\\
    On constate à ce moment-là que le texte est envoyé via la liaison série vers le module Xbee. Celui ci va envoyer le message à l'autre module, caractère par caractère.

\img{\rootImages/send.png}{Réception et écriture}{0.15}

\item  Les données envoyés sont en bleu, celles reçues sont en rouge.

\item En cliquant sur la console du second module, on constate que le texte envoyé (en bleu) est bien reçu car en rouge.

\img{\rootImages/receive.png}{message reçu}{0.3}

La communication est fonctionnelle entre les modules.
\end{enumerate}


\label{code}
\section{Code principal}

\begin{Cpp}{Code principal}
#include "UART.h"
#include "RS232.h"
#include "Gyro.h"
#include "Rover.h"
#include "SpaceViewer.h"
#include "LED.h"
#include "MusicPlayer.h"
#include "LIDARLite.h"
 
//#define SENDER  
#define RECEIVER

#define XBEE_UART
//#define XBEE_RS232 

//Xbee
#define RX_XBEE 10   //Rx pin of Xbee
#define TX_XBEE 8  //Tx pin of Xbee
#define BAUDRATE_XBEE 9600    //Baudrate of serial port
#define BAUDRATE_DEBUG 9600   //baudrate of xBee device


//#define NO_XBEE_UART 

#ifdef XBEE_UART
UART xBee = UART(RX_XBEE, TX_XBEE);                     //New instance of UART class, provides tools to communicate between microcontroleur and Xbee
#endif


#ifdef XBEE_RS232
RS232 xBee = RS232();                     //New instance of UART class, provides tools to communicate between microcontroleur and Xbee
#endif

#ifdef SENDER

  //Gyro
  #define SAMPLING_PERIOD_MS 100
  Gyro gyroscop = Gyro(ADDRESS_GYRO);                         //New instance of Gyro class, provides tools to read gyro
  //Led
  #define PIN_LED 13
  LED led = LED(PIN_LED);                                 //New instance of Led class

#endif


#ifdef RECEIVER

  //Rover
  #define ROVE_PWM_FORWARD 3
  #define ROVER_PWM_TURN 9

  #define BUZZER 5

  #define SECURITY_DISTANCE 70
  
  Rover rover = Rover(ROVE_PWM_FORWARD, ROVER_PWM_TURN);  //New instance of Rover class, provides tools to control rover
  MusicPlayer music = MusicPlayer(BUZZER);
  LIDARLite lidarLite;
 

#endif


/*!
  Setup function
  Used to setting up all devices and init instance
  This function is called one time
*/

void setup() {

#ifdef XBEE_UART
  Serial.begin(BAUDRATE_DEBUG);
	Serial.println(">>> Start");
#endif
	xBee.begin(XBEE_9600);
  xBee.setExitTime(200);


#ifdef SENDER
  
  gyroscop.setAlgorithm(Algorithm::Mahony);

#ifdef XBEE_UART
  Serial.println(">>> Init MPU9250");
  Serial.println(">>> Send calibrating order");
#endif

  xBee.send("START");

  gyroscop.begin();  
  delay(10);

#ifdef XBEE_UART  
  Serial.println(">>> End init MPU9250");
#endif
  led.on();  
  gyroscop.setMagneticDeclination(+15.9);
  gyroscop.calibrate(50);
  led.off();
  xBee.send("END");
#ifdef XBEE_UART
  Serial.println(">>> Send end calibrating order");
#endif
  
  
#endif


#ifdef RECEIVER

  rover.setDebug(true);

#ifdef XBEE_UART
  rover.addDebugChannel(&Serial, BAUDRATE_DEBUG);
#endif

  lidarLite.begin(0, true);   //set config to default and I2C to 400kHz, starts I2C
  lidarLite.configure(0);     //there are 6 different configs available
  
  rover.stop();
  delay(1000);
  music.addUARTDevice(&xBee, "START", "END");
  music.play();   //Wait until calibration
#endif

}//End setup

void loop() {

#ifdef SENDER

  gyroscop.readAll();

  float gx, gy, gz;

  float relativeRoll = gyroscop.relativeRoll();
  float relativePitch = gyroscop.relativePitch();

  float temp = gyroscop.getTemperature();
  
  String spacePosition = SpaceViewer::getSpacePosition(relativeRoll, relativePitch);
  //Serial.println("Pitch = "+String(relativePitch)+"\t Roll = "+String(relativeRoll)+"\t"+String(spacePosition)+"\t"+String(temp));
  xBee.send(spacePosition);
  delay(SAMPLING_PERIOD_MS);

#endif

#ifdef RECEIVER
  
	String message = xBee.read(WAIT_DATA);  //NO_WAIT_DATA

  
#ifdef XBEE_UART
	Serial.println(message);
#endif

  if(message.length()==6) 
  {
    //Gestion des commandes de base => un seul ordre à la fois    
    if(message[0]=='0' && message[1]=='0' && message[2]=='0' && message[3]=='0')
    {
      rover.stop(); //arreter
    }
    else if(message[0]=='1' && message[1]=='0' && message[2]=='0' && message[3]=='0')
    {
      rover.left(); //gauche
    }
    else if(message[0]=='0'&& message[1]=='1' && message[2]=='0' && message[3]=='0')
    {
      rover.right(); //droite
    }
    else if(message[0]=='0' && message[1]=='0' && message[2]=='1' && message[3]=='0')
    {
      if(lidarLite.distance() < SECURITY_DISTANCE)
      {
        rover.stop();
      }
      else
      {
        rover.forward(); //avancer
      }       
    }
    else if(message[0]=='0' && message[1]=='0' && message[2]=='0' && message[3]=='1')
    {
       rover.backward(); //reculer
    }

    //Gestion des diagonales => deux ordres en même temps    
    else if(message[0]=='1' && message[1]=='0' && message[2]=='1' && message[3]=='0') //Avancer à gauche
    {
    if(lidarLite.distance() < SECURITY_DISTANCE)
      {
        rover.stop();
      }
      else
      {
        //Serial.println("diagonale gauche");
       rover.left_forward();
      }             
    }

    //Gestion des diagonales => deux ordres en même temps    
    else if(message[0]=='0' && message[1]=='1' && message[2]=='1' && message[3]=='0') //Avancer à droite
    {
      if(lidarLite.distance() < SECURITY_DISTANCE)
      {
        rover.stop();
      }
      else
      {
        //Serial.println("diagonale gauche");
       rover.right_forward();
      }   
    }    

    else {}
     
  }
  else if(message.length()==0)
  {
    Serial.println(">>> Xbee disconnected");
    rover.stop();
  }
  else 
  {
    Serial.println(">>> Bad frame input");
  }//End else

  
#endif

}//End loop

\end{Cpp}

