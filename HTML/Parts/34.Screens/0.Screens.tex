\part{Les écrans}

\chapter{Les écrans OLED}

\subsection{Les caractéristiques}
Les écrans \glossary{OLED} sont des afficheurs graphiques 
compacts avec une résolution de 128×64 pixels ou 64x32 pixels.\\
Leur résolution plus élevée que des écrans \glossary{LCD} permet d’afficher du texte, des images, et des figures 
graphiques. Ces écrans intègrent généralement un contrôleur, permettant de faire l’interface entre l’écran
en lui même et la carte de commande (Arduino, ESP32...) et utilisent le protocole \glossary{I2C} pour communiquer, avec les broches \glossary{SDA} et \glossary{SCL}.

\img{\rootImages/oled.jpg}{Un écran OLED}{0.25}

La plupart des écrans utilisent le contrôleur SSD1306 et généralement, ces modules OLED s’alimentent de 3,3 à 5V.


\subsection{Installation de la bibliothèque}

Plusieurs bibliothèques sont disponibles pour utiliser l'écran dont la bibliothèque \lib{adafruit\_SSD1306.h}.\\

Pour changer, nous utiliserons la bibliothèque de rinky-dink-electronics à l'adresse suivante : 
\link{OLED\_I2C.zip},  bien documentée et simple d’utilisation

La figure suivante rappelle comment installer rapidement une bibliothèque depuis une archive ZIP.
En l’occurrence, le fichier \file{OLED\_I2C.zip}

\img{\rootImages/library.jpg}{Installation d'une bibliothèque depuis l'archive}{0.6}
L'installation plus détaillé des bibliothèques Arduino est disponible en annexe du document.



\subsection{Branchements}

\begin{items}{green}{\Circle}
    \item La broche \outputPin{+5V} de l'Arduino vers la broche \inputPin{VCC} de l'écran
    \item La broche \outputPin{GND} de l'Arduino vers la broche \inputPin{GND} de l'écran
    \item La broche \outputPin{A4} de l'Arduino vers la broche \inputPin{SDA}de l'écran
    \item La broche \outputPin{A5} de l'Arduino vers la broche \inputPin{SCL} de l'écran
\end{items}

\img{\rootImages/branchementOled.png}{Le branchement de l'écran}{0.45}


\messageBox{Remarque}{orange}{white}{
    Les broches SDA et SCL sont sur des broches différentes pour les carte Arduino Leonardo et Arduino Mega : \\
    \begin{items}{orange}{\Triangle}
        \item Arduino Mega
        \begin{items}{orange}{\Triangle}
            \item SDA : Broche 20
            \item SCL : Broche 21
        \end{items}
        \item Arduino Leonardo
        \begin{items}{orange}{\Triangle}
            \item SDA : Broche 20
            \item SCL : Broche 21
        \end{items}
    \end{items}
}{white}

Plusieurs accès aux broches s'offrent à vous : 
\begin{enumerate}
\item Utiliser les broches I2C du shield
\img{\rootImages/shield1.jpg}{Le branchement sur le SensorShield avec les broches I2C}{0.08}

\item Utiliser les broches A4 et A5 du shield
\img{\rootImages/shield2.jpg}{Le branchement sur le SensorShield avec les broches A5 et A4}{0.08}

\end{enumerate}


\subsection{Programme de l'écran}
\begin{Cpp}{Création de l'objet MyOLED}
    // ----------ppr début ------------------------
    // écran OLED
    // --------------------------------------------
    //
    #include <OLED_I2C.h>
    OLED  myOLED(SDA, SCL, 8);
    extern uint8_t SmallFont[];
    extern uint8_t MediumNumbers[];
    extern uint8_t BigNumbers[];
    // ----------ppr fin   ------------------------
    //
\end{Cpp}

\begin{Cpp}{Démarrage de l'écran}
    // 
----------ppr début ------------------------
// écran OLED démarrage
// --------------------------------------------
//
myOLED.begin();
myOLED.setFont(SmallFont);
// ----------ppr fin ------------------------
//
\end{Cpp}

\begin{Cpp}{Coeur de l'affichage}
    // ----------ppr début ------------------------
    // écran OLED affichage
    // --------------------------------------------
    //myOLED.drawLine(0,0,127,63); 
    // Draw a line from the upper left 
    // to the lower right corner
      myOLED.drawLine(0,20,127,20);  // ligne horizontale 
      myOLED.drawLine(65,0,65,40);   // ligne verticale
      myOLED.setFont(MediumNumbers);      
  
      myOLED.printNumI(HG, LEFT, 0);
      myOLED.printNumI(HD, RIGHT, 0);
      myOLED.printNumI(BG, LEFT, 20);
      myOLED.printNumI(BD, RIGHT, 20);
  
      myOLED.setFont(BigNumbers);
      myOLED.printNumI(consigneInclinaison, LEFT, 40);
      myOLED.printNumI(consigneRotation, RIGHT, 40);
      myOLED.update();
    // ----------ppr fin --------------------------
    //
  
\end{Cpp}



\chapter{Les écran à encre électronique}

Parmi les supports d'affichage disponibles, en plus des écrans, cristaux liquides 
(\glossary{LCD}), OLED, une révolution est apparue : le papier électronique \bold{e-paper} 
bien connu des utilisateurs de liseuses type Kindle ou Kobo (Fnac).

L'objectif de ces écrans est d'imiter l'apparence d'une feuille imprimée.\\

\img{\rootImages/example.png}{Un écran e-paper}{1}


L'affichage est obtenu par l'orientation de sphères colorées (noir et blanche ou noir, blanche, rouge) sous l'effet d'un champ électrique.\\ 
Une fois que l'affichage a été actualisé, l'écran n'a plus besoin d'énergie pour maintenir l'affichage de l'écran.

\img{\rootImages/ink2.png}{Principe d'affichage}{0.45}
\img{\rootImages/ink3.png}{La dalle d'affichage}{0.45}
\img{\rootImages/ink4.png}{Un zoom sur les billes}{0.45}


Quelques utilisations : \\
\begin{items}{green}{\Bullet}
    \item Des posters
    \item Des liseuses
    \item Les étiquettes des prix dans les grandes surfaces
\end{items}


La principale utilisation est l'étiquette électronique de gondole / EEG Electronic Shelf Labels 
qui est vendu actuellement comme un système intégré de gestion géré par un ordinateur central.\\
Cela permet d'automatiser les prix en temps réel (cela évite de faire des centaines d'étiquettes 
papier à mettre à jour) mais surtout de synchroniser les prix dans un magasin en ligne et 
hors ligne sur plusieurs sites.\\ 
Des bornes interactives permettent de trouver un produit dans un hypermarché.\\
Équipé de puces NFC ou par QR code, des informations sont disponibles sur le smartphone du client 
en rayon.\\
Actuellement un système de gestion d'inventaire et de stock est en cours de développement.


\chapter{Utilisation des sous-programmes Arduino}

On améliore la lisibilité du code en découpant le programme Arduino en plusieurs onglets. \\
La méthode la plus simple et d'abord, d'écrire un sous-programme/sous-routine, en langage 
Arduino une fonctions pour structurer le programme.\\ 
Ensuite, on place dans un onglet la fonction (sous-programme/sous-routine) : le nom de 
l'onglet sera sauvegardé en fichier .ino placé dans le même répertoire que le programme principal.\\

L'environnement de développement intégré \glossary{IDE} Arduino va modifier l'ensemble en 
un seul fichier type cpp qui sera envoyé dans le compilateur puis le linker 
(enfin vers la carte Arduino pour exécution).

Concrètement : 

\img{\rootImages/sub1.jpg}{Nom de la fonction}{0.4}
\img{\rootImages/sub2.png}{Les variables "partagées"}{0.3}
\img{\rootImages/sub3.png}{Fonction setup}{0.5}
\img{\rootImages/sub4.jpg}{Fonction d'affichage}{0.9}
\img{\rootImages/sub5.png}{Suite fonction d'affichage}{0.3}

\subsection{Code Arduino}

\begin{Cpp}{Code OLED avec routine}

    // ----------ppr début ------------------------
// écran OLED
// --------------------------------------------
//
#include <OLED_I2C.h>
OLED  myOLED(SDA, SCL, 8);
extern uint8_t SmallFont[];
extern uint8_t MediumNumbers[];
extern uint8_t BigNumbers[];
// ----------ppr fin   ------------------------
//

//*
//******************************
//*
void setup_oled() { 
  // ----------ppr début ------------------------
  // écran OLED démarrage
  // --------------------------------------------
  //
  myOLED.begin();
  myOLED.setFont(SmallFont);
  // ----------ppr fin   ------------------------
  //
 }  
  
 //*
 //*************************
 //*
  
 void oled_Affiche() {
  // ----------ppr début ------------------------
  // écran OLED affichage
  // --------------------------------------------
    myOLED.clrScr();   // efface la mémoire de l'écran (évite vieux affichage)

    // affiche les valeurs des consignes Inclinaison et rotation
    myOLED.setFont(MediumNumbers);   // 1ere ligne jaune
    myOLED.printNumI(consigneInclinaison, 20, 0);   // 20,40
    myOLED.printNumI(consigneRotation, RIGHT, 0);   // right,40

    // affiche les valeurs HG,HD,BG, BD 
    myOLED.setFont(BigNumbers);  // gros chiffres
    myOLED.printNumI(HG, LEFT, 16);
    myOLED.printNumI(HD, RIGHT, 16);
    myOLED.printNumI(BG, LEFT, 41);
    myOLED.printNumI(BD, RIGHT, 42);
    
    // mise en page 
    myOLED.drawLine(0,40,127,40);  // ligne horizontale 
    myOLED.drawLine(65,20,65,80);   // ligne verticale   
    
    myOLED.setFont(SmallFont);
    myOLED.print("INC", 0, 8);   // 
    myOLED.print("ROT",70, 8);   //  

    myOLED.print("HG",50, 30);    // 
    myOLED.print("HD",70, 30);    // 
    myOLED.print("BG",50, 50);    // 
    myOLED.print("HD",70, 50);    // 

    myOLED.update();
  // ----------ppr fin --------------------------
  //

}
\end{Cpp}