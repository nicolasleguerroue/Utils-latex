\part{Annexes}
\chapter{Configuration de l'ESP12 sous Arduino}

\img{\rootImages/nodemcu.png}{ESP12 NodeMCU}{0.8}

\section{Installation des bibliothèques et cartes ESP8266}

La carte ESP12 NodeMCU est prévue pour être programmée directement via l'\glossary{IDE} Arduino.\\
Cette carte fait partie de la grande famille des ESP8266.\\

\messageBox{\faviconInfo}{green}{green}{L'installation est strictement identique avec la version 1.X ou 2.X d'Arduino, que ce soit sous Windows, Linux ou MAC}{black}
Pour installer les bibliothèques et cartes sur le logiciel Arduino, il faut réaliser les étapes suivantes : 


\begin{items}{blue}{\Triangle}

    \item Ouvrir les préférences du logiciel Arduino dans \lbl{blue}{KEY}{Fichiers - Préférences}
    \img{\rootImages/preference.png}{Préférences Arduino}{0.5}

    \item Dans le champ \bold{URL de gestionnaire de cartes supplémentaires}, mettre le lien suivant : \\

    \link{\url{http://arduino.esp8266.com/stable/package_esp8266com_index.json}}
    
    \messageBox{\faviconWarning}{orange}{orange}{Veuillez vérifier l'URL après le copier-coller car les underscores ("tirets du 8") peuvent disparaître.}{black}
    \img{\rootImages/url.png}{Lien pour les cartes ESP8266}{0.5}
    
    Puis faire \lbl{blue}{KEY}{OK}

    \item Fermer le logiciel Arduino

    \item Lancer le logiciel Arduino
    \item Allez dans \lbl{blue}{KEY}{Outils - Type de carte - Gestionnaire de carte} 
    \img{\rootImages/boardManager.png}{Gestionnaire des cartes ESP8266}{0.5}
    
    et faire une recherche avec le mot clé \lbl{blue}{KEY}{esp8266}

    \img{\rootImages/install.png}{Installation des bibliothèques ESP8266}{0.5}

    Il ne vous reste plus qu'à cliquer sur \lbl{blue}{KEY}{Installer} et redémarrer le logiciel Arduino.
\end{items}

\section{Installation du driver CH340 sous Windows}

La plupart des nouvelles cartes ESp12 (et Arduino) utilisent le module CH340 pour communiquer entre la carte et l'ordinateur.
Installons les pilotes. Tout d'abord, se rendre à l'adresse suivante pour télécharger le pilote :
\link{\url{https://www.arduined.eu/ch340-windows-10-driver-download/}}

\img{\rootImages/link.png}{téléchargement du driver CH340}{0.6}

En cliquant sur \bold{Driver CH340 for Windows 10}, un fichier compressé au format ZIP se télécharge. Il ne vous reste plus 
qu'à le décompresser \footnote{Clic-droit sur le fichier et "extraire ici"}.
Les fichiers suivants devraient apparaître dans le dossier : 

\img{\rootImages/dir.png}{La liste des fichiers pour le driver}{0.8}

Il faut clicker sur le fichier \file{SETUP.exe} et ensuite, l'interface suivante apparaît : 

\img{\rootImages/installCH340.png}{L'installation du driver CH340}{0.6}

Faites \bold{INSTALL} et après quelques secondes, la fenêtre suivante apparaît : 

\img{\rootImages/installed.png}{L'installation effectuée du driver CH340}{0.8}

Vous pouvez revenir au logiciel Arduino.

\section{Recherche des cartes ESP8266}

\bold{A cette étape, ne pas oublier de brancher la carte ESP12 à l'ordinateur avec un câble USB-micro}

\img{\rootImages/micro.jpg}{Un câble USB-micro}{0.6}
\messageBox{\faviconWarning}{red}{red}{Certains câbles USB sont prévus uniquement pour alimenter un appareil, pas pour communiquer. En cas d'erreur par la suite, n'hésitez pas à changer de câble si vous avez un doute.}{white}
Lors de la programmation d'une carte ESP8266 NodeMCU, il faudra donc aller dans \\
\lbl{blue}{KEY}{Outils - Type de carte - ESP8266 Boards NodeMCU X.X (ESP12 Module)} 

\img{\rootImages/espFull.png}{Sélection de la carte ESP12}{0.4}


Afin de tester le bon fonctionnement, nous vous invitons à tester le programme \bold{Blink} disponible dans les exemples.

\img{\rootImages/blink.png}{Emplacement de l'exemple Blink}{0.4}

La led bleue de l'ESP12 devrait clignoter si l'installation s'est correctement effectuée.

\section{Recherche des cartes Arduino}

Pour la programmation des cartes Arduino, il suffira de sélectionner \\
\lbl{blue}{KEY}{Outils - Type de carte - Arduino AVR Boards - Carte X} en fonction du modèle de votre carte.

