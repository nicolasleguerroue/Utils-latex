\newcommand{\wel}{Wellington~}



\chapter{Introduction}

La bataille de Waterloo est l'une des batailles de l'Empire les plus étudiée car elle marque l'écroulement 
définitif du régime de Napoléon. \\
Ce document a pour but de synthétiser les principales erreurs de Napoléon effectuées durant la campagne 
de Belgique.


\epigraph{Cette bataille sera l'affaire d'un déjeuner}{\textit{Napoléon}}

\chapter{Contexte}



Nous sommes en juin 1815.\\
Dès le retour de Napoléon le 1er avril 1815, les Alliés, c'est à dire l'Angleterre, la Prusse, 
l'Autriche et la Russie ont de nouveau réunis le Congrès de Vienne afin de mettre définitivement 
Napoléon à genoux. Ils mettent donc en place plusieurs armées qui convergent vers la France. 
Le premier rang est formé par les Anglais et les prussiens, viennent ensuite les russes et les autrichiens.
La première vague est formée par 210 000 anglais et prussiens.
Les autrichiens et les russes sont environ 300 000 hommes.\\

Dans un premier temps, avec l'analyse des forces en présence et les différentes chaînes de commandement, nous verrons qu'un ensemble d'erreurs incombe directement à Napoléon, le reste du moins implique ses subordonnées.\\

\subsection{L'Armée du Nord}

Sous le ministre de la Guerre Davout, l'Armée du Nord est constitué principalement de français, 
les régiments étrangers ayant retrouvés leur territoire national (notamment les polonais et les italiens). \\
Cette armée, forte de 133.000 hommes est une armée formée de vétérans, certains ayant été rapatriés des 
pontons espagnols et gardant par ce fait une rancune tenace contre les anglais. 
Ils sont motivés à en découdre.\\


L'Armée du Nord est composée des Corps d'armée suivant :  \\


- Le 1er Corps du comte Drouet-D'Erlon.\\

N'ayant pas participé à la bataille de Ligny le 16 juin, il s'agit du corps d'armée le plus à même 
d'effectuer une offensive sur le centre anglais, étant le plus frais et le plus nombreux en terme de troupe. il est constitué de 20 000 hommes et du 4ème corps de cavalerie de Milhaud.\\
La cavalerie de la Garde lui est directement rattachée sous le commandement de  Lefebvre-Desnouettes.\\


- Le 2ème Corps d'Armée est celui de Reille (17 000 h)\\

- Le VIème Corps est celui de Lobau (10 0000).\\

- La Garde forme la réserve et la cavalerie lourde sera sous le commandement de Kellerman.\\

Suite au Cents-Jours, de nombreux maréchaux se sont ralliés à Louis XVIII et Napoléon ne peut faire le 
difficile au sein du haut commandement \footnote{Certains maréchaux ont été mis en retraite comme Jourdan, Kellerman ou bien disgracié comme Masséna}. Il confie donc l'aile gauche de l'armée au Maréchal Ney, rallié 
récemment à Napoléon après de nombreuses hésitations.\\
L'aile droite de l'armée est sous la responsabilité du nouveau Maréchal Grouchy. 
Ce choix semble étonnant dans la mesure où ce dernier n'a jamais commandé des effectifs aussi importants. 
Étant responsable pendant des années de la cavalerie française, il a donc commandé des divisions de 
cavalerie mais jamais de Corps d'armée. \\


Napoléon possédait cependant d'excellents divisionnaires comme Gérard, Saint-Hilaire ou Guyot qui 
connaissent les rouages du commandement. \\
Le choix le plus dérangeant est sans doute la présence de Davout en tant que ministre de la Guerre. 
Bien que très compétant et ayant réussi à mettre sur pied une armée en quelques mois après les coupes 
des effectifs pendant les Cents-Jours, sa présence sur le terrain eut été un avantage décisif.\\
Il reste le seul Maréchal invaincu de Napoléon et ses talents de manœuvriers ne sont plus à démontrer. \\
De plus, la présence de Suchet, seul maréchal ayant gagné son bâton de maréchal en Espagne (fait rare) 
a été laissé dans les Alpes avec un corps \\d'Observation pour éviter la menace venant d'Italie.\\
Enfin, et non une des moindres, la perte de Berthier en juin 1815 au poste de chef d'État-Major de l'Armée 
se fera cruellement ressentir dans la chaîne de commandement. Cette responsabilité incombe désormais à 
Soult qui est plus manœuvrier que bureaucrate.\\
Berthier occupait ce poste depuis les débuts de la Grande Armée et savait parfaitement traduire les 
ordres de l'Empereur en lettre précises et concises pour les subordonnées.\\

Napoléon compte appliquer sa technique habituelle, c'est à dire appliquer la manœuvre en position centrale :

- Percer le centre anglais\\

- Battre séparément les deux ailes ennemies\\

N'ayant jamais affronté Wellington dans une bataille rangée, Napoléon essayait de faire taire les 
remarques de Soult sur les qualité défensives de ce dernier. \\
Il avait pourtant raison, Wellington est un général préférant rester sur ses positions pour attendre le 
moment opportun d'une contre-attaque, comme en Espagne.\\
De plus, au vue de la situation géographique des deux armées alliées, le plan de Wellington visait à 
temporiser pour attendre les prussiens. Comme l'a résumé Wellington, "je livrerai bataille si je peux espérer le soutien ne serait que d'un seul corps prussien".
Il en aura 3 à la place, le 4ème (celui de Bulow, chef d'état major de Blucher) étant retenu par le corps de Grouchy.\\



Un élément dérange Napoléon. La présence de la ferme fortifiée de la Haye-Sainte. 
Située en plein centre du champ de bataille, cette ferme empêche de déployer les bataillons 
français dans une vaste offensive. La prise de cette ferme constituera donc l'objectif central pour 
Napoléon, le reste devant découler naturellement une fois le centré enfoncée. \\

Toute la journée, Wellington tenta de préserver son centre en faisant parvenir de nombreux renforts. 
Il y aura donc une bataille dans la bataille au vu des effectifs engagés des deux cotés, 8000 h pour 
les français contre 4000 coté anglais.\\

Nous reviendrons sur cette offensive, qui étonnamment, a été mené par l'un des frères de Napoléon, 
Jérôme. \footnote{N'ayant pas particulièrement brillé lors de la campagne de Russie en se détachant 
du commandement de Davout, il sera mis de côté jusqu'à ce moment là.}

Toute l'offensive prussiennes reposera donc sur le corps de Lobau à partir de 18 h.


\subsection{L'Armée Alliée}


Les troupes anglaises sont formées de contingents anglais (470000 h), irlandais et comptent parmi eux la 
Légion Hanovrienne \footnote{King's German Legion}.\\
Ces derniers auront pour mission de tenir la Haye-Sainte, la position clé du centre britannique.\\

L'ensemble des forces britanniques sont sous le Duc de Wellington, ayant gagné son titre en Espagne en 
1808 après la bataille de Vittoria.\\
L'objectif de \wel est de tenir le plus longtemps possible sur ses postions afin d'attendre le corps 
de Blucher. \\
Cette stratégie payera dans la mesure où sentant son centre menacé, \wel fera intervenir ses réserves 
afin de combler le centre menacé.\\


Les troupes prussiennes quand à elles sont dirigées par le maréchal Blucher, alors âge de 71 ans.
Cet habitué des guerre napoléoniennes \footnote{Prise de Lubeck en 1806 alors tenue sous ses ordres}, 
c'est un militaire qui a également participé à la guerre de 7 ans (1756-1763). Il a donc pu assister à 
la réorganisation de l'armée prussienne suite à la débâcle de Iéna et Austerdaest en 1806.
Cette évolution s'est faite sous l'impulsion de son chef d'état major Von Bulow et du célèbre 
théoricien Karl Von Clausewitz.\\

Les prussiens se méfiaient des anglais, ces derniers n'ayant pas encore fait de combats directs avec 
Napoléon. Ils souhaitaient donc se rapprocher de leur ligne de communication (Wavre) afin de panser 
leurs plaies suite à la bataille de Ligny effectuée le 16 juin.\\


\subsection{Disposition des troupes}

\section{Entrée en campagne}

Le 14 juin 1815, Napoléon décide de passer à l'attaque, en se disant que la meilleure stratégie était de battre d'abord les anglo-prussiens puis de se retourner vers les russes et les prussiens.

Il fonde donc sa stratégie sur une campagne éclair.


Il confie donc au Maréchal Ney l'attaque de la position aux Quatres-bras. Il aura sous son commandement le corps de Drouet d'Erlon, Lobau et Reille. 

la timidité du maréchal conduira Wellington à rester sur ses positions. De plus, le corps de Drouet-d'Erlon jonglera en d'incessant allers-retours entre le corps de Ney et l'armée principales sous les ordres de Napoléon.\\ Il ne participera donc à aucun des deux affrontements de la journée.

\section{Bataille de Ligny}

Le 16 juin 1815, Napoléon décide de passer à l'attaque, en se disant que la meilleure stratégie était de battre d'abord les anglo-prussiens puis de se retourner vers les russes et les prussiens.

Il fonde donc sa stratégie sur une campagne éclair.


Du coté de Napoléon, l'armée prussienne accepte le combat et durant toute la journée, ses troupes sont envoyées successivement au feu, ce qui amoindrie leur puissance offensive.

Ayant donc subi les assauts des français, ils se voient dans l'obligation de se retirer.

Une des clés de la victoire des alliées résidera à ce moment là au choix de la direction de repli des anglais.

les prussiens veulent se retirer sur Namur mais les anglais leur conseille plutôt vers Wavre pour lier leur communication. Si les prussiens s'étaient repliés vers leur ligne de repli classique, les prussiens ne seraient pas intervenus sur le champ de bataille le 18, cela aurait permis aux français de mener leur offensive principale plus sereinement.













\section{Ouverture des hostilités}

Ayant plu toute la nuit du 17 juin, le sol est détrempé. or à cette époque, la présence de boue dans le sol empêche les boulets pleins de rebondir et de faire des ricochets sur le sol. l'efficacité des salves en est grandement réduite car les boulets ne ricochent pas.

Les rares boulets explosifs n'ont que peu d'impact.

De plus, le rôle de la cavalerie sous l'Empire, du moins la cavalerie lourde est la rupture sur une section du front. Or la boue amoindrie cette efficacité.


Sous l'avis du général Drouet, responsable de l'Artillerie, les canons de la Garde ouvrent le feu à 11 h 30. \\

Il s'agit incontestablement d'une erreur car à cette époque, le batailles commençaient au lever du jour afin de profiter du maximum d'heure avec de la lumière. \footnote{Les combats de nuits étaient rares.}. nous pouvons citer la bataille de la Moskova, Wagram, Austerlitz, toutes ces batailles ont commencée au lever du jour.






\section{Le Subterfuge de Bulow}
\section{Trahison}

A 18 h 30, Napoléon joue sa dernière carte. La Veille Garde entre en action sur le centre anglais.\\
Les bonnets d'ours se dirigent vers le centre mais arrivés à 200 m, ils subissent un feu roulant des anglais qui se sont levés des champs de blé. La première ligne ets fauché, la seconde prend le relais et tire une salve.


Lors de la 5ème salve, les grognards hésitent puis commencent à reculer. A ce moment là, le cri de Trahison circule et se propage au sein de l'Armée. L"effort demandé par l'armée est trop fort, les régiments commencent à refluer en désordre.


\section{Le dernier carré}

A ce moment là, \wel agite son chapeau, c'est le signal convenu de l'assaut général.

Seuls au mileu des fuyards, la Vielle Garde forme 3 carrés, Napoléon se réfugie parmi un des carré.

La légende aura forgé le Dernier Carré où Cambronne répond aux anglais "Merde !" quand ces derniers lui demandent de se rendre.\\
\section{La fuite de l'Aigle}
