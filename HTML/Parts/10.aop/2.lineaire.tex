

\chapter{Étude en mode linéaire}
\section{Intérêt de l’étude }

Les AOP en fonctionnement linéaire permettent de réaliser les opérations mathématiques :

d\begin{items}{blue}{\Triangle}
  \item \bold{amplification} : $Vs=A_0 \cdot Ve$
  $A_0$ est le coefficient d’amplification du montage (A ne pas confondre avec $A_d$ le coefficient d’amplification différentiel 
  imposé par le constructeur)
  $A_0$ peut être positif ou négatif
  \item \bold{addition algébrique} : $V_s=\sum_{k=0}^{n} V_k$
  \item \bold{intégration et dérivation} (avec des condensateurs) à une constante près
  \item \bold{logarithme et exponentielle}
\end{items}

\section{Méthode de résolution}
La \colors{red}{réaction négative} (liaison entre la sortie et l’entrée inverseuse) impose un fonctionnement stable et linéaire, 
d'où \colors{red}{$\varepsilon=0, E_+=E_-$} \\

L’hypothèse de la résistance d’entrée de l’AOP implique que $I_+=I_-=0$

Afin de déterminer $Vs$, il faut exprimer \bold{$E_+$et $E_-$ en fonction des éléments du montage}. \\

\bold{En égalisant les deux équations obtenues ($E_+=k$ et $E_-=k'$)}, on obtient une relation de type $V_s = f(V_e)$


\messageBox{Remarque}{orange}{white}{$I_s$ est issu d’une source de tension, il n’y a donc pas de loi simple permettant de déterminer sa valeur algébrique. Il ne faut pas avoir d'à priori sur son sens}{black}
