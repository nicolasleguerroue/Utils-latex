%\addPartText{Théorie sur les amplificateurs opérationnels}
\partImg{Les amplificateurs opérationnels}{\rootImages/aop.jpeg}{0.4}

 \chapter{Introduction}
 
\section{Généralités}

  \index{AOP}

Un AOP (Amplificateur Opérationnel) est un composant actif qui permet de réaliser des opérations mathématiques (addition, soustraction, intégration, dérivation, etc) et du fait de sa miniaturisation et de sa fiabilité, on le rencontre aujourd’hui dans de nombreuses applications comme l’audio, la radio, l’asservissement…

\img{\rootImages/pinoutAOP.png}{Les entrées et sorties de l'AOP}{0.5}

\index{Alimentation symétrique}
Un AOP possède deux entrées notées appelées \colors{red}{entrée non inverseuse} et \colors{red}{entrée inverseuse}, une 
\colors{red}{sortie} et deux broches d’\colors{red}{alimentation}.
L’AOP dispose souvent d’une alimentation symétrique ($Vcc+$ et $Vcc-$) avec comme référence de tension le point milieu 
(GND) des alimentations.


\img{\rootImages/power.png}{L'alimentation d'un AOP}{0.6}


\section{Conventions}


Afin de simplifier les calculs sur les AOP, quelques conventions ont été adoptées :

\begin{items}{blue}{\Triangle}
  \item La tension de sortie de l’AOP est notée $Vs$
  \item La tension sur l’entrée inverseuse est appelée $E_-$
  \item La tension sur l’entrée non inverseuse est appelée $E_+$
  \item La tension différentielle \colors{red}{$(E+-E-)$ est appelée $\varepsilon$}
  \item Le gain d’amplification différentiel de l’AOP est appelé $Ad$
  Ce gain est variable entre différentes familles d’AOP mais reste constant dans le temps
  \item Le gain d’amplification du montage est appelé $A_0$ et varie en fonction des différents montages possibles 
\end{items}

\messageBox{Remarque}{green}{white}{L’alimentation des montages suivants ne sera pas représentée par souci de clarté.}{black}
