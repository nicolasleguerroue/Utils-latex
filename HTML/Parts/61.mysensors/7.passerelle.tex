\chapter{Programmation passerelle}

Nous allons configurer le programme de la passerelle(\bold{Passerelle\_MySensors.ino})pour la réception des données.

\subsection{Paramétrage du NRF24}

Il faut mettre le même canal de communication que pour la sonde.\\
Un module recevant sur le canal 84 ne pourra pas recevoir des données en provenance d'un canal 83 ou 85.

Pour rappel, voici le tableau des canaux: 
\begin{figure}[!h]
    \centering
    \begin{tabular}{|l|r|}
        \hline
\bold{Prénom} & \bold{CANAL\_NRF24}\\
    \hline
André P. & 84 \\
\hline
Florian M. & 85 \\
\hline
Guy D. & 86 \\
\hline
Marcel R. & 87 \\
\hline
Michel T. & 88 \\
\hline
Nicolas L.G. & 89 \\
\hline
Patrice G. & 90 \\
\hline
Patrick P. & 91 \\
\hline
Patrick Z. & 92 \\
\hline
Philippe C. & 93\\
\hline
Pierre G. & 94\\
\hline
Yvon & 95\\
\hline
    \end{tabular}
    \caption{Répartition des canaux pour les utilisateurs}
    \end{figure}

    Voici un extrait du code \bold{Passerelle\_MySensors.ino} : 

\begin{Cpp}{Paramétrage du NRF24}
/*!
 * ************************************************************
 * PARAMETRES NRF24 (exemple canal 86)
 * Légende: (*) = A changer pour chaque personne
 * ************************************************************
 */
 #ifndef MY_RF24_CHANNEL
 #define MY_RF24_CHANNEL    86         //Le canal doit être le même que celui de la sonde
 #endif             

\end{Cpp}


\section{Envoi des programmes}

Une fois les deux programmes modifiés avec les bonnes valeurs, il ne vous reste plus qu'à envoyer le programme de la sonde sur la carte Pro-Mini et celui de la passerelle sur la carte Nano.\\
Toutes les informations pour programmer les deux cartes sont disponibles au chapitre 2 (\bold{Programmation}).

Lorsque les deux cartes sont programmées, occupons nous maintenant de Domoticz.



