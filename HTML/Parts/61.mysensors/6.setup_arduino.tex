\chapter{Programmation sonde}

Nous allons configurer le programme de la sonde (\bold{Sonde\_MySensors.ino})pour l'envoi des données des capteurs. 

\subsection{Paramétrage du NRF24}

Pour éviter les interférences entre les modules NRF24 dans une même salle, nous allons sélectionner un canal de communication pour chaque personne.\\
Un canal correspond à une fréquence précise d'émission et de réception pour le module NRF24.\\

Voici le tableaux des canaux attribué aux personnes :
\begin{figure}[!h]
    \centering
    \begin{tabular}{|l|r|}
        \hline
\bold{Prénom} & \bold{CANAL\_NRF24}\\
    \hline
André P. & 84 \\
\hline
Florian M. & 85 \\
\hline
Guy D. & 86 \\
\hline
Marcel R. & 87 \\
\hline
Michel T. & 88 \\
\hline
Nicolas L.G. & 89 \\
\hline
Patrice G. & 90 \\
\hline
Patrick P. & 91 \\
\hline
Patrick Z. & 92 \\
\hline
Philippe C. & 93\\
\hline
Pierre G. & 94\\
\hline
Yvon & 95\\
\hline
    \end{tabular}
    \caption{Répartition des canaux pour les utilisateurs}
    \end{figure}


    Voici un extrait du code \bold{Sonde\_MySensors.ino} : 

\begin{Cpp}{Paramétrage du NRF24}
/*!
 * ************************************************************
 * PARAMETRES NRF24 (exemple canal 86)
 * Légende : (*) = A changer pour chaque personne
 * ************************************************************
 */
//Mode debug activé
#define MY_DEBUG

// Enable and select radio type attached 
#define MY_RADIO_RF24

#ifndef MY_RF24_PA_LEVEL
#define MY_RF24_PA_LEVEL     RF24_PA_MAX
#endif

#ifndef MY_RF24_CHANNEL
#define MY_RF24_CHANNEL    86 //(*)  A CHANGER
#endif

#ifndef MY_RF24_DATARATE
#define MY_RF24_DATARATE RF24_250KBPS
#endif

\end{Cpp}

\subsection{Paramétrage de Domoticz}

Comme vue dans le chapitre de présentation, chaque sonde possède un identifiant et chaque capteur rattaché à la sonde possède lui aussi un identifiant.\\

Pour éviter toute confusion, nous allons également attribuer un identifiant à notre sonde et à nos capteurs, ces identifiants sont définis pour chaque personne dans le tableau suivant :\\


\begin{figure}[!h]
    \centering
    \begin{tabular}{|l|r|r|r|r|}
        \hline
\bold{Prénom} & \bold{MY\_NODE\_ID} & \bold{ID batterie} & \bold{ID température} & \bold{ID Humidité} \\
    \hline
André P. & 10 & 11 & 12 & 13 \\
\hline
Florian M. & 20 & 21 & 22 & 23 \\
\hline
Guy D. & 30 & 31 & 32 & 33 \\
\hline
Marcel R. & 40 & 41 & 42 & 43 \\
\hline
Michel T. & 50 & 51 & 52 & 53 \\
\hline
Nicolas L.G. & 60 & 61 & 62 & 63 \\
\hline
Patrice G. & 70 & 71 & 72 & 73 \\
\hline
Patrick P. & 80 & 81 & 82 & 83 \\
\hline
Patrick Z. & 90 & 91 & 92 & 93 \\
\hline
Philippe C. & 100 & 101 & 102 & 103 \\
\hline
Pierre G. & 110 & 111 & 112 & 113 \\
\hline
Yvon G. & 120 & 121 & 122 & 123 \\
\hline
    \end{tabular}
    \caption{Répartition des identifiants pour les utilisateurs}
    \end{figure}

\begin{Cpp}{Paramétrage de Domoticz}

/*!
 * ************************************************************
 * PARAMETRES DOMOTICZ (Exemple avec l'ID 30)
 * ************************************************************
 * Légende : (*) = A changer pour chaque personne
 */
#define MY_NODE_ID 30         //Noeud dans Domoticz (*)

//Voir le tableau
#define CHILD_ID_BATT 31      //(*) Identifiant Domoticz pour le niveau de batterie 
#define CHILD_ID_TEMP 32      //(*) Identifiant Domoticz pour la température
#define CHILD_ID_HUM 33       //(*) Identifiant Domoticz pour l'humidité

\end{Cpp}

ici, il nous reste à définir le temps entre deux envois de données par la sonde.

\begin{Cpp}{Temps de mise à jour}

//Temps entre deux envois de données
static const uint64_t UPDATE_INTERVAL = 10000;   
\end{Cpp}

Ensuite, on peut éventuellement changer le nombre de mesure au bout duquel la sonde envoie les données même si elles n'ont pas changés

\begin{Cpp}{Nombre de lectures forcée}
    //Nombre au bout duquel la sonde envoie les données même si elles n'ont pas changés
    static const uint8_t FORCE_UPDATE_N_READS = 10;
\end{Cpp}



\subsection{Paramétrage du DHT}

Il faut préciser si le capteur est un DHT11 ou DHT22 et indiquer la broche du capteur

\begin{Cpp}{Paramétrage du DHT}
    /*!
    * ************************************************************
    * DHT22/DHT11
    * ************************************************************
    */
   //Broche de données du DHT22 (ou DHT11)
   #define DHT_DATA_PIN 3
   
   #define DHT_TYPE DHT22 //use DHT11 or DHT22
   
   //définir une valeur si le capteur à un offset permanent par rapport à la température réelle
   #define SENSOR_TEMP_OFFSET 0
   
   float lastTemp;   //Dernière valeur de température lue
   float lastHum;    //dernière valeur d'humidité lue
   
\end{Cpp}


\subsection{Paramétrage de la mesure de la batterie}

Il suffit de renseigner les valeurs des résistances (en $\Omega$) formant le pont diviseur de tension.\\
Dans mon cas, les résistances ont une valeur de 330 $k\Omega$ et $1 M\Omega$

\bold{Le calcul pour afficher la tension réel de la batterie est détaillé en annexe.}

\begin{Cpp}{Paramétrage du multimètre}
    /*!
    * ************************************************************
    * BATTERIE
    * ************************************************************
    */
   //Partie pour mesure Batterie
   int BATTERY_SENSE_PIN = A0;  //ne pas toucher
   
   float oldBatteryV = 0;//Sauvegarde du niveau de batterie
   
   const float r1_value = 1000000.0; //Valeur de la résistance la plus proche de l'alimentation de la batterie
   const float r2_value = 330000.0; //Valeur de la résistance la plus proche de la masse

\end{Cpp}




\chapter{Branchements du DHT}

\img{\rootImages/dht.jpg}{Branchement du DHT sur la sonde}{0.07}

Voici les connexions pour brancher le DHT : 

\begin{items}{orange}{\Triangle}
  \item \inputPin{GND\_DHT} est représenté par le câble verts (droite)
  \item \inputPin{VCC\_DHT} est représenté par le câble orange (gauche)
  \item \inputPin{OUT\_FHT} est représenté par le câble jaune (D3) (centre)
\end{items}

\img{\rootImages/connection.png}{Schéma du circuit imprimé}{0.5}

