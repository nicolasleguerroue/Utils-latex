\chapter{Bibliothèques Arduino}

\section{Installation des bibliothèques}

Lors de la première compilation d'un programme, il se peut que des bibliothèques soient manquantes.
C'est ce que nous allons voir. Pour cela, ouvrez le programme de la sonde (\bold{Sonde\_MySensors.ino}) sans brancher de carte Arduino.

Ensuite, cliquez sur le bouton \bold{Vérifier} (bouton de gauche) et patientez quelques secondes.\\

\img{\rootImages/verifier.png}{Bouton de vérification}{0.5}

Si la bibliothèque MySensors est manquante, vous obtiendrez l'erreur suivante: 

\img{\rootImages/error.png}{La bibliothèque MySensors manquante}{0.5}

Pour installer la bibliothèque, il suffit d'aller dans \bold{Croquis > Inclure une bibliothèque >Ajouter la bibliothèque .ZIP}

\img{\rootImages/place.png}{Ajout d'une bibliothèque}{0.5}

Il ne reste qu'à trouver le fichier \bold{Bibliothèque\_MySensors.zip} et à faire \bold{OK}

\img{\rootImages/lib.png}{Sélection du fichier ZIP}{0.5}

Un message de confirmation d'ajout est affichée en bas de la page du logiciel Arduino.

\img{\rootImages/message.png}{La bibliothèque MySensors est ajoutée}{0.5}

On clique à nouveau sur le bouton \bold{Vérifier} pour afficher les éventuelles erreurs.

\messageBox{Important}{orange}{white}{Dans certains cas, la bibliothèque \bold{Adafruit\_sensor} est manquante, il faut installer le fichier \bold{Bibliothèque\_Adafruit\_sensor.zip} disponible en annexe.}{white}
 

On refait ensuite le bouton \bold{Vérifier} et si la bibliothèque \bold{DHT} n'est pas installé, on obtient de nouveau : 

\img{\rootImages/dht.png}{La bibliothèque DHT manquante}{0.5}

On procède de la même façon, on va importer le fichier \bold{Bibliothèque\_DHT.zip} dans \bold{Croquis > Inclure une bibliothèque >Ajouter la bibliothèque .ZIP}

Une fois toutes les bibliothèques installées, le message suivant apparaît: 

\img{\rootImages/compilation.png}{La compilation est terminée}{0.5}

On va ensuite s'occuper de la vérification de la communication entre les cartes Arduino et l'ordinateur.