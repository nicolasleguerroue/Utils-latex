\chapter{Contenu}

%######################################################
\section{Partie I - Les bases}

\begin{items}{red}{\Circle}
\item Présentation des périphériques (schéma)
\item Le bureau Windows
\item Le menu Démarrer
\item La barre des tâches
\item Quizz Ordinateur Débutant ou exercice concret
\end{items}



%######################################################
\section{Partie II - Souris et clavier}

\begin{items}{red}{\Circle}
\item Clic, double clic, clic droit
\item Présentation d’un clavier (chercher touche exotique, clavier numérique verrouillé)
\messageBox{Remarque}{orange}{white}{Désactiver les pavés numériques avant la session ! }{black}
\item Ecrire et modifier un texte (curseur, saisie de texte)
\item Les raccourcis clavier (evocation copier coller, équivalent clic-droit)



\item Quizz souris et clavier
\end{items}


\section{Partie III - Les fichiers et dossiers}

\begin{items}{red}{\Circle}
 \item Arborescence des dossiers (Espace personnel) 
 \messageBox{Remarque}{orange}{white}{Exo : télécharger un document et l'enregistrer dans un dossier nommé "..."}{black}
 \item Gestion des dossiers et fichiers (suppression, copie , déplacement)
\item Les icônes
\item Les fenêtres (retour, fermeture, gestuelle Windows, déplacement des fenêtres)
\item Clé USB, carte SD, disque dur externe
\end{items}



\section{Partie IV - Les mails et Internet}

\subsection{Organisation boîte mail}

\begin{items}{red}{\Circle}
\item Boîte de réception: tous vos nouveaux messages arriveront dans ce dossier
\item Boîte d’envoi : les messages qui sont en train d’être envoyés par Internet
\item Brouillons : les messages que vous avez commencé à rédiger mais pas envoyé
\item Éléments envoyés : un historique de vos messages envoyés
\item Éléments supprimés : les e-mails, reçus ou envoyés, que vous avez supprimé
\item Courrier indésirable : également appelé spam, ce sont des messages à ignorer. 

\end{items}

\subsection{Écrire un mail}

\begin{items}{red}{\Circle}
\item Objet : c’est le titre du message. Il doit être concis
\item Expéditeur : C’est la personne qui envoie le message
\item Destinataire(s) : La ou les personnes recevant le message
\item Corps du message : le contenu de l’e-mail
\item Pièces jointes : Les fichiers attachés au message
\item Répondre : la zone pour écrire votre réponse
\end{items}

\subsubsection{Ajouter un fichier joint dans son mail}

\begin{items}{red}{\Circle}

\item Repérez le bouton Joindre ou l’icône en forme d’agrafe lorsque vous tapez votre message.
\item Une fenêtre s’ouvre alors, vous demandant de choisir le ou les fichiers voulus.
\item Cliquez sur Ouvrir ou Joindre pour joindre le message

Une nouvelle ligne apparaît dans votre e-mail : votre fichier est joint et prêt à être envoyé

\end{items}






\subsection{Recherche Internet}
\begin{items}{red}{\Circle}
\item Ouvrir un navigateur
\img{\rootImages/Capture.PNG}{Vue globale du navigateur}{0.8}
\item Naviguer sur une page internet
\img{\rootImages/navigation.PNG}{Navigation sur une page}{0.9}
\end{items}

\subsubsection{Rechercher une information utile}
\begin{items}{red}{\Circle}

\item Un horaire d'ouverture, se renseigner sur un sujet (foot, wikipedia, informations...)
\item Rechercher une réponse à un problème

\end{items}

\newpage

\subsection{Bureautiques}
\messageBox{Remarque}{red}{white}{Vérifier que l'on a bien Libre office à l'ENIB }{black}



\begin{items}{red}{\Circle}
\item Comparaison entre Libre Office et Microsoft Office
\item utilisation du traitement de texte. (libreOffice Writer) :\\
élements de base (Titre, sous titre pour le formalisme), mettre un passage en gras/italique/ en couleur...
\item Element un peu plus technique\\
Insérer un tableau, une image (changer l'adaptation d'une image)

\item Libre Office Impress
\item Insérer une image, nouvelle diapositive, mise en page
\end{items}

Manipulation pour les plus expérimentés:

imprim écran.

paint extrait le contour (image voulu).

copier coller dans un document writer.

Puis pourquoi pas l'imprimer.


\subsection{Périphériques}

\begin{items}{red}{\Circle}
\item Imprimantes
\end{items}
\section{Fin de séance}

Petit goûter 
Demande des thèmes futurs



A la fin de la séance on leur demande sur quoi ils veulent travailler la prochaine fois, s'il y a des demandes particulières, d'autre thèmes, un approfondissement... 


\section{Organisationnel}

- Pauses sur demande, quand on le souhaite, et que l'on sent qu'ils ont du mal à se concentrer. Pause d'environ un quart d'heure.


\section{mails}

- mail pour service réception.
- Relance Anne-Marie, appel dès réception de la convention


\section{Proposition de la permanence}

\begin{items}{red}{\Circle}
\item Bureautique
\item Multimédia
\item Imprimante
\item Bonne pratiques (fichiers temporaires sur le bureau....)
\end{items}



\section{Devoirs}

On peut leur demander de réaliser quelques actions chez eux, pour ceux qui le souhaitent.

- 
- envoyer un mail
- écrire dans un bloc note leur retour d'expérience et nous l'envoyer par mail



\section{Répartition des rôles}

Nicolas : Souris et claviers

Théo : Fichier et dossier

Romain : Mails et Internet

Corentin : Les Bases


\chapter{Lettre}

\subsection{L'énergie}

Tout d'abord nous voyons l'avenir comme une époque d'avancées technologiques dans les domaines de l'énergie renouvelable et de l'hydrogène. Dans notre groupe nous misons particulièrement sur l'hydrogène. On a pu également aborder le sujet de la fusion nucléaire comme autre avancée majeure.\\

Pour ce qui est des ressources,  certaines matières premières n’existent quasiment plus, telles que le pétrole. Il y a donc des alternatives au pétrole, telles que les plastiques d’origine végétale ou encore des carburants synthétiques (c'est le tout début en ce moment avec Porsche), il reste pour nous encore du cuivre car on a amélioré notre manière de concevoir pour prendre en compte le recyclage de ces métaux.
Le recyclage est omniprésent. On ne jette rien sans le recycler. Tout est démonté pour ensuite être réutilisé et cela presque à l’infini.\\



Chaque régions du globe est auto suffisante en terme d’énergie. Chacun utilise l’énergie la plus adéquate. Le solaire pour les pays chaud, l’hydraulique et l’éolien dans certaines régions. Les technologies utilisées sont bien dimensionnées. Utilisation du low tech quand pas besoin de plus et vice versa. Ce qui a permis de sauver la planète, mais au dernier moment (OUF).


\subsection{La société}

Pour ce qui est de l'aspect sociétal, nous ne voyons pas une société comme \underline{1984}\footnote{George Orwell}\\
En Europe, il n'y a que des démocraties (des vrais pas comme en Russie), même les royautés fonctionnent sous un régime démocratique. Les titres royaux existent toujours mais sont plus honorifiques, par exemple l'Espagne actuelle. 
Le Moyen Orient reste toujours une zone instable.\\


La Bretagne est la région la plus prisée de France, elle est très touristique mais du coup le coût de la vie est plus chère. En effet, il fait aussi chaud que dans le sud de la France (de notre époque). Pour ce qui est du monde en général les évènements climatiques sont de plus en plus fréquents et de plus en plus forts. Résultat du réchauffement climatique, mais les choses s'améliorent.

\subsection{Les loisirs}

Pour ce qui est des loisirs, le foot est toujours prédominant, néanmoins les salaires ont diminué.\\ 
Pour l'aspect culturel, les domaines classiques (Opéra, théatre, etc) ont disparu pour laisser place à des expositions (histoire, technologie ...).\\

Pour ce qui est du 7ème art nous espérons de très bon films à l'instar de Star Wars, Avatar, Interstellar, le Seigneur des Anneaux ou encore Iron Man. Les films sont passés en réalité virtuelle, le spectateur devient le héros du film. Les salles de cinéma deviennent alors de plus en plus prisées de pars cette évolution technologique.

\newpage
\section{Les missions de l’ingénieur}
%
%La société attend toujours plus d’innovations de la part des ingénieurs et tous les domaines. Ceux-ci sont donc sous-pression et sont désignés comme étant les responsables des échecs et des réussites de toute la société. \\
%
%
%Quelques ingénieurs fous osent se risquer dans le domaine de la politique, tentant, souvent en vain, d’apporter un peu de raison et de réalisme dans les décisions prises. Encore une fois, les rares s’y risquant endossent les responsabilités du politique associés à celles de l’ingénieur.\\
%
%Il y a un retour aux écoles d’ingénieurs spécialisées dans certaines industries. Les gros secteurs de l’industrie créer leur propre école d’ingénieurs (style Amazon). La spécialisation du travail de l’ingénieur aamené la création de nombreuses écoles récentes. Certaines créés par les descendants directs des GAFAM.\\
%
%Elles forment les ingénieurs à ne réfléchir que d’une manière et pour une seule chose.
%Cependant, les ingénieurs style ENIB se démarquent par leur vision du monde plus globale et construisent la société de demain. Malheureusement le profit et l’argent reste au cœur de la société.\\
% 
%Le rôle de l’ingénieur généraliste est dévalorisé car souvent perçus comme moins "utiles" à la société et moins "utilisables" par les employeurs. Ceux-ci sont pourtant souvent les plus avisés et responsables dans leurs décisions car ils possèdent une bien meilleure vision de l’état du monde actuel.\\
%
%Dans 40 ans, les entreprises prennent conscience de l’importance de l’ingénieur, qui devient le pilier des entreprises, on ne forme plus de manager ou commercial sans un minimum de connaissance technique, cela permet une meilleur communication dans l’entreprise. En effet les managers connaissent désormais les contraintes rencontrées par les techniciens et ouvriers et prennent des décisions en tenant compte de la réalité du terrain. \\
%
%Dans l’entreprise, il n’y a plus de séparation aussi stricte entre les ingénieurs, les techniciens et les ouvriers. De grandes salles avec 4 ou 5 personnes, regroupées par service, existent.\\
%Le télétravail est de mise et est vu comme un atout pour les entreprises. En effet il permet au personne loin géographiquement de prendre part aux activités de plusieurs entreprises.
%
%
%Définir et approfondir les mots : 
%\begin{items}{blue}{\Triangle}
%\item \bold{Responsabilité} : 
%\item \bold{Communication} : il y a de plus en plus de communication entre service, ce qui facilite le travail de groupe.
%\item \bold{Innovation} : Toutes les nouvelles méthodes sont testés et encouragés par les entreprise
%\item \bold{Intégrité} : Toutes les décisions sont prises en connaissance de cause et assumées. L'honnêteté et la transparence sont primordiaux pour les postes à responsabilité.
%\item \bold{Résilience} : Les ingénieurs de 2061 font preuve de beaucoup de résilience par rapport aux erreurs commises auparavant. Ils arrivent à trouver des solutions pour résoudre des problèmes qui étaient mal engagés tels que ceux liés à l'environnement.
%\end{items}

%\newpage

Mon très cher ami,

Je conçois parfaitement que vous soyez assailli de doutes lors de votre entrée à l'ENIB par ces temps incertains.

Il relève donc de mon devoir d'essayer de vous aider et ainsi de vous apporter mes maigres connaissances et ma vision de votre futur.

Il ne vous sera pas rare d'entendre que nous vivons les heures les plus sombres de l'humanité, que votre génération doit composer avec les erreurs de
toutes les précédentes dans une société ravagée, sur une planète à l'agonie.
Laissez-moi vous rassurer, ce discours est plus ou moins similaire depuis plusieurs dizaines d'années. A tort, ceci dit.

En effet, selon moi, les heures les plus sombres de l'humanité ne surviendront que lorsque celle-ci n'aura plus d'espoir, plus de solution.
C'est justement là qu'intervient le métier d'ingénieur tel que vous vous apprêtez à le découvrir.

Ma génération a assistée à la métamorphose de notre métier. Notre simple rôle de concepteur technique a finalement aussi endossé le rôle de politique.
Lors de la crise de 2040, la société s'est reposée sur les ingénieurs généralistes et leur vision globale de ce monde pour le reconstruire.
Ceux-ci ont alors jouis du prestige des réussites mais également subis la honte des échecs.

Les ingénieurs étaient désignés d'office responsables de la société. Cette situation n'a depuis, jamais été remise en cause.

Cela a bouleversé toute l'organisation mondiale. Il était alors inacceptable que dans une entreprise, un rôle de manager, de responsable soit confié à
une personne sans aucune formation technique. Celle-ci était et reste considérée comme obligatoire pour que la hiérarchie ait conscience des contraintes
imposées à une équipe et de leurs conséquences.

Paradoxalemenent, le rôle de l'ingénieur généraliste a été peu à peu dévalorisé au profit des ingénieurs spécialisés.
Ces derniers sont issus des "grandes écoles" fondées par les géants de l'industrie (que l'on appelait alors encore les GAFAM). C'est dans cette période
que de telles formations ont pris leur essors. Faciles d'accès, ammenant systèmatiquement à un travail et un bon salaire, elles ne sont cependant pas
conçues pour apprendre aux jeunes citoyens à réfléchir et à aborder le monde avec curiosité. En réalité, je me permettrai de les qualifier de "stériles".
Ne servant que les intérêts des géants, leur unique but est de fournir de la main d'oeuvre facilement utilisable et ne cherchant pas à remettre en cause
les institutions en place.

Alors pourquoi ceux-ci sont-ils valorisés malgré ce portrait bien peu flatteur ? La raison est très simple : ils sont présentés comme plus utiles à la
société. Réalisant concrétement des actions. Ils conçoivent, prototypent et concrétisent des demandes de la société.
Leur vision très limitée de notre monde ne leur permettent cependant, de ne créer que des projets à court terme et sur un domaine bien précis.

J'estime que cette présentation historique, cette explication du cheminement parcouru pour en arriver à notre situation est essentielle pour vous parler
de votre futur.
Comme vous aurez pu le constater, nous avions l'espoir que, enfin, la société se repose sur des personnes ayant conscience des problèmatiques à résoudre
et capables d'apporter des solutions.
Cela s'est réalisé, sur une période très courte. Puis nous sommes de nouveau retombé dans une vision à court terme.
C'est ici que vous, futur ingénieur généraliste, intervenez. Ce n'est pas parce que des écoles comme l'ENIB sont en perditions que vous devez perdre
ce qui fait leur force.
Ces écoles ont toujours visés à former des ingénieurs conscients et responsables.

Le futur de votre métier, selon moi, sera donc de faire profiter le monde de votre formation devenue si rare. Simplement partager votre point de vue,
amener à la réflexion sur la société en place, lier le technique et l'humain pour parvenir à créer un système fiable sur le long terme.
Votre vision globale de ce monde vous permettra d'obtenir le meilleur de vous même et des ressources que vous avez à votre disposition.

Ces possibilités sont à prendre avec les responsabilités qui les accompagnent. C'est à vous que l'on attribuera les futurs échecs et réussites de la société.

Plus que jamais, votre rôle est primordial. Vous avez, littéralement, la possibilité de changer le monde.

J'espère de tout coeur ne pas vous avoir découragé. Beaucoup de chose restent à accomplir.
Je vous souhaite bon courage et de réussir, quel que soit la voie que vous choisirez.

Amicalement vôtre,



\chapter{Rapport d'étonnement}

\newcommand{\pd}{Petits Débrouillards }

\section{La Fresque du Climat}

Probablement l'une des activités les plus marquantes de cet intersemestre.\\

Bien que je doutais d'apprendre beaucoup plus de choses que ce qui était déjà de l'ordre de ma connaissance, j'en suis ressorti profondément marqué. En effet, cet atelier m'a appris et fait réalisé énormément de choses.
Il était par exemple intéressant de savoir que l'impact de l'Homme sur le changement climatique a pu être mesuré. Cela m'a permis de me rendre compte à quel point nos activités ont un impact. Ce lien entre les "on dit que" et la réalité a été extrêmement puissant.\\

L'apprentissage sous forme de cartes avec une interaction active entre les participants nous a permis de confronter nos avis et nos connaissances sur le sujet. \\

Les remarques de la part de l'intervenant à la fin de l'atelier, bien que peu joyeuses de premier abord, nous offrent un moyen de quantifier approximativement notre impact individuel et nous ont offertes un peu d'espoir quant à l'avenir.\\

J'en ressors avec beaucoup d'idées de progrès que je peux faire dans mon quotidien pour améliorer les choses à mon échelle.

\section{Les \pd}

\subsection{L'expérience avec les enfants}

J'abordais cet atelier avec autant d'appréhension que d'envie. Cette activité de médiation scientifique était pour moi l'occasion de partager mon goût des sciences aux générations futures en espérant susciter des vocations.\\
C'était l'occasion pour moi de me plonger dans l'animation et la médiation scientifique auprès d'un jeune public. J'espère avoir réussi à transmettre un peu de savoir et beaucoup d'envie de s'intéresser aux sciences à et l'expérimentation.\\

Nous avions à charge de présenter le thème "Mélanges et solutions" à une classe de CM2 et de leur présenter le principe de densité à travers diverses manipulations mettant en scène des réactions chimiques.\\

Ce fut pour moi intéressant de réfléchir à la manière la plus appropriée de présenter à des enfants des concepts scientifiques qui sont beaucoup plus compliqué que ce qui est normal d'aborder à leur âge.\\
Cette capacité à rendre intelligible par le plus grand nombre des éléments complexes compte parmi les plus importante pour un ingénieur généraliste qui sera plus tard amener à travailler sur différents domaines et à faire le lien entre ceux-ci.\\

En revanche, l'organisation a, d'une manière générale, été déplorable. Bien que les conditions sanitaires n'aient sans doute pas aidées, cela n'excuse pas tout.\\
En effet, il me semble inacceptable que l'enseignant n'ait eu à sa disposition que une soirée (en dehors de ses horaires habituels) pour organiser vaguement l'intervention du lendemain.\\
Même si il est possible de rejeter une partie de la responsabilité de ce problème sur la communication interne de l'école qui ne semblait pas avoir transmise correctement les informations, la livraison du kit de manipulation la veille dans l'après midi empêche une bonne organisation et une bonne intervention auprès des enfants.\\
Par exemple, les fluides fournis ne se trouvaient que dans un seul gros contenant.\\ Les \pd nous ont ensuite gentiment expliqués après l'intervention que nous devions trouver un moyen de nous organiser avec l'enseignant (qui s'est trouvé être particulièrement aidant ceci dit) pour distribuer ces liquides. Ce qui était parfaitement impossible étant donné que nous avions cours en présentiel tout l'après midi et que l'intervention se tenait le lendemain matin.\\
Tout ceci sachant que la première journée d'intervention a eu une utilité relativement faible et qu'il aurait été préférable de commencer à travailler sur celle-ci dès le début.\\

Tout cela pour dire que cette intervention qui s'annonçait intéressante et constructive s'est terminée en une intervention bâclée, faite avec les moyens du bords, stressante pour l'enseignante qui semblait baigner dans l'inconnu et d'une qualité moyenne pour les enfants.

\subsection{Le Grand Jeu}

Suite à cette première expérience, nous devions créer et organiser un grand jeu dans le bassin de Brest pour un public donné. Dans notre cas, il s'agissait d'une famille avec deux enfants. En conséquent, les activités devaient être adaptées à ce public.\\

Je devais, avec mon groupe, faire découvrir un lieu historique. Nous avons choisi les Ateliers des Capucins et à travers les différentes machines, nous leur avons fait un jeu de piste se basant sur des énigmes et des photos. A chaque énigme résolue, quelques informations étaient dispensées à l'aide d'une page internet que nous avions rédigés au préalable.\\

En parallèle, mon groupe devait également participer à un grand jeu organisé par d'autres étudiants de l'ENIB. Le jeu était fort sympathique, nous devions découvrir un ensemble de scientifiques par l'intermédiaire des rues et places de la technopôle. Nous avons cependant regretté que l'aspect éducatif du jeu passe uniquement par des liens redirigeant vers des pages wikipédia.\\

J'ai eu les même regrets que lors de la première activité. On nous a toujours présentés ce jeu comme étant une activité que nous aurions eu le temps de préparer. Or, celui-ci était en réalité un hackaton avec une seule journée pour préparer de A à Z en sachant qu'il fallait faire du repérage, installer et tester le jeu dans un délai très court.\\
Cependant de pars les contraintes choisis, le jeu s'est transformé en balade avec quelques informations dispensées maladroitement, pour ma part cette activité ne m'a rien apporté si ce n'est de la frustration.\\

Pourquoi nous faire faire deux activités avec les \pd et donc réaliser deux actions "baclées" au lieu d'en faire une seule organisée, préparée et réfléchie, ce qui aurait pu donner un très bon rendu.\\

\addQuote{Bonne chance !}{L'équipe des \pd}




\section{Négociations}

L'art de la négociation est très vaste et l'intervention m'a permis de me rendre compte à quel point il peut être difficile d'argumenter dans certaines conditions.\\
Cependant, j'ai compris quelque chose d'essentiel, c'est que dans la négociation il n'y a ni gagnant ni perdant. L'objectif est de trouver un accord pour trouver satisfaction et trouver la solutions la moins contraignante.\\

En effet, dans notre métier nous serons amené à travailler en équipe et donc à se mettre d'accord.
Nous avons les cinq techniques de négociation qui sont le \bold{compromis}, le \bold{consensus}, l'\bold{abandon}, la \bold{fuite}, et le fait d'\bold{imposer son point de vue} en cas d'urgence. Nous avons aussi vu la méthode de Fisher et Ury pour réussir sa négociation.\\

Les notions étaient toujours introduites par des mises en situations afin de voir comment on réagit.\\ Ensuite, une méthode nous était présentée afin de voir comment nous aurions pu procédé.\\
Je pense que cet atelier m'a été bénéfique et me sera profitable autant dans ma vie quotidienne que dans mon futur métier.


\section{Collapsologie}

J'ai trouvé cet atelier très intéressant, notamment le fait de parler des problèmes liés aux croisements des enjeux.\\
Pour illustrer ce fait, nous avons parlé de l'obsolescence programmé, ce qui selon moi résume bien la manière de penser de notre société.\\
En effet, nous sommes partagé entre l'envie de faire des efforts au point de vue écologique et l'envie de créer de l'emploi. Nous avons vu que la création ou le maintient d'emploi a toujours été privilégié.\\
Ce fait peut s'illustrer par les fabricants d'ampoule, qui se sont mis d'accord pour limiter la durée de vie des ampoules afin de pouvoir vendre plus, et donc créer des emplois. L'ultra-consumérisme est un phénomène intéressant à traiter car il soulève beaucoup de question.\\

Le rappel des neuf frontières planétaires était aussi très intéressant. Cet atelier, tout comme celui sur la fresque du climat nous a amené à se poser beaucoup de questions sur notre mode de vie et sur ce que l'on doit faire pour améliorer les choses.



\section{Management d'Équipe}

Cette activité m'a permis de me rendre compte des différentes étapes de la création d'une entreprise jusqu'à sa gestion.

Le jeu interactifs m'a permis de réaliser que la gestion d'une entreprise repose beaucoup sur la communication. \\
Sans cette dernière, que ce soit en interne ou bien pour les intervenants externes (fournisseurs, clients), la viabilité d'une entreprise est compromise.\\

De plus, elle nous a permis de découvrir certains aspects auxquels il est nécessaire de penser quand on créer notre entreprise, notamment en ce qui concerne les frais liées à celle-ci.





\section{Le Court-Métrage}

Après avoir visionné des courts-métrages durant notre première matinée, nous avons été mis dans le bain car je devais réaliser un scénario avec mon groupe de tournage. \\ Après de nombreuses délibérations, nous avons pu partir sur une histoire de vol de matériel à l'ENIB.\\

Les jours suivants furent encore plus intéressants car nous avons commencé le tournage sur différents lieux de tournage. La partie plus pénible concernait tous les plans avec des sons car la mise en place fut beaucoup plus longue et la prise de sons étaient souvent compliquée avec les conditions en extérieur.\\
Cependant, l'ambiance était très bonne au sein du groupe et cela nous a permis de nous amuser tout en étant efficace.\\

L'utilité de l'intervenant lors de cet atelier pourrait être questionnée. Même si regarder des courts métrages réalisés auparavant peut apporter quelque chose, on peu soulever la question de la pertinence de nous faire faire ceci en amphi. Mais ne pas nous apprendre à filmer, je trouve ça dommage. Mettre a dispositions des élèves des guides et des astuces tant pour le montage que pour la prise son et vidéo aurait été une bonne chose.



\section{Délibération et Débats Éthiques}

J'ai trouvé que chacun des débats était très intéressant, car il y avait toujours deux groupes de personnes qui ne voyaient pas les choses de la même manière.\\


De plus, lorsque le débat traite de l'éthique, il convient de peser le pour et le contre bien attentivement car les situations peuvent s'avérer dangereuses.\\
La notion de débat amène bien souvent à une longue phase de délibération au sein du groupe afin de réfléchir à la meilleure proposition. \\

L'incertitude morale a lieu en présence d'un groupe ou ce dernier est obligé de prendre une décision collective afin de résoudre un problème. Lorsque ce dernier semble insoluble, cette décision tourne au dilemme et il convient de choisir la "meilleure solution".
Cependant, la meilleure solution est-elle celle de la majorité ?\\

Nous avons cherche ensemble des solutions parfois insolubles, ce qui a permis de mettre en place un dialogue constructif, même si au début, lors des premiers exemples, chaque personne restait sur ses positions ce qui formait un débat avec des arguments "creux".\\

La prise de conscience du type de discussion m'a fait réfléchir à ma façon de discuter au sein d'un groupe, ce qui m'a permis de moins me braquer sur mes positions.\\


Pour ma part, cette activité s'est révélée décevante. En effet, quelques soucis techniques nous ont malheureusement empêchés de profiter pleinement de cette intervention. Outre ceci, bien qu'étant intéressante, j'ai regretté que cette intervention ne nous laisse pas plus de temps pour débattre entre nous de différents sujets (je pense notamment à la fabrication d'arme(s) par exemple).

\section{Cyberdéfense}

Même s'il me paraissait, au départ, inadéquat de proposer cet atelier en présentiel, il s'est au final avéré que Olivier BALD a sut tirer profit de ce privilège pour apporter une interactivité plaisante.\\
Il m'a rarement été donné d'assister à une conférence aussi intéressante et utile, je me suis même dit que cela pourrait être pertinent d'en faire une matière enseignée à l'ENIB. En effet la cyberdéfense est pour moi un des enjeux majeurs de notre société actuelle auquel nous serons confronté en tant que ingénieur.



\section{SST}


La formation que j'ai effectué était intéressante pour connaître les responsabilités au sein de l'entreprise. 

L'avantage de l'auto-formation est que je pouvais la faire en dehors des temps dédiés afin d'assouplir mon emploi du temps d'intersemestre. \\Je pouvais donc me former pendant les temps creux. \\

La pertinence du distanciel est d'autant plus justifié que l'on apprend tout autant avec un intervenant présent physiquement que sur un écran.


\section{Sensibilisation à la différence}

Nous étions répartis en groupe de 4-5 et nous devions rédiger une charte pour améliorer le comportement des élèves à l'ENIB sur le sexisme, discrimination, etc.\\

Après avoir proposé différentes actions pour améliorer nos comportements, l'intervenant a fait une synthèse et nous avons accepté ou non la charte.\\

Pour notre dernière journée d'intersemestre, nous avons pu assister à plusieurs représentations de mini-pièces afin de sensibiliser aux discriminations. Les intervenants ont réussi à amener beaucoup d'interactions lors des interventions

Enfin, nous avons pu mettre en scène une pièce et nous avons choisi le thème "État de santé" où nous avons exprimé un problème lié aux vacances et aux allergies.\\

Ces interventions m'ont permises de me rendre compte que même aujourd'hui, à l'ENIB, il y avait malheureusement encore des comportements discriminatoires et/ou blessants.


\section{Test Sportif UBO}

Le test sportif est un bon moyen de voir si notre forme physique est bonne. En tout qu'étudiant et surtout en période de confinement, nous passons beaucoup de temps assis derrière notre bureau.

\section{Atelier Nutrition}

J'ai trouvé cet atelier très intéressant car étant étudiant tout conseil nutritionnel est le bienvenu. On avait la possibilité de poser des questions à des professionnels ce qui est pour moi une très bonne chose. Pour ma part j'ai appris des choses et les conseils donnés étaient les bienvenus.

\newpage
\section{Divers}

La multitude des supports d'expressions et de liens de communications a été un calvaire pour cet intersemestre. En effet, le nombre de supports dont voici un extrait : 

\begin{itemize}
    \item Zoom
    \item BigBlueButton
    \item Jitsi
    \item RocketChat
    \item Mail
    \item Moodle
    \item Wiki(s)
\end{itemize}


\chapter{Conclusions}

Dans l'ensemble, cet intersemestre  qui s'annonçait prometteur et que j'attendais avec enthousiasme s'est révélé entaché par des problèmes d'organisations.\\

Si certains n'étaient que peu gênants, les déboires avec Les \pd, qui constituaient une partie très importante de nos activités, m'ont très sérieusement refroidis. Ceux-ci ne nous ont au final apportés que peu de choses par rapport au stress engendré et au temps demandé.\\

Il reste important de noter que de nombreuses choses ont été vues dont je me souviendrais pendant longtemps et qui me seront sans aucun doute d'une grande utilité.