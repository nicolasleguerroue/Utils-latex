%\addPartText{Introduction aux réseaux}

%\part{Les réseaux}

\chapter{Les réseaux}

Un réseau est un ensemble de machines  \footnote{Serveurs, PC, Tablettes, téléphones, imprimantes} reliées entre elles. 
Dans notre cas, la liaison sur le réseau se fera en Ethernet ou WiFi.\\

Pour communiquer entre elles, il faut définir des adresses \glossary{IP} exactement comme une \\
adresse postale pour transmettre un courrier.


\subsection{Les adresses MAC}

Une adresse \glossary{MAC} est une adresse unique qui désigne la machine
\footnote{Plus précisément l'adresse de la carte réseau de la machine}.\\ 
Elle est de la forme \lbl{orange}{MAC}{XX:XX:XX:XX:XX:XX} 
(six octets usuellement exprimés et hexadécimal) et reste invariante dans le temps.\\

Contrairement aux adresses IP, les adresses MAC ne sont normalement pas attribuées explicitement par 
configuration ; elles sont au contraire attribuées à la production de la carte réseau par son fabriquant

La table \glossary{ARP} d'une machine sert à associer des adresses IP à des 
adresses MAC.


\subsection{Les adresse IP}

Une adresse IP est une adresse donnée à une machine qui rejoint le réseau. \\


\subsubsection{A quoi ressemble une adresse IP ?}

\begin{items}{blue}{\Bullet}
	\item Version 4 (Ipv4) : Les adresses sont codées sur 32 bits \\
		
	\lbl{green}{IP}{192.168.0.1}	= \lbl{orange}{BIN}{1100000.10101000.0000000.00000001} (En Binaire)
	\item version 6 (Ipv6)	: les adresses sont codées sur 128 bits \\
	\lbl{green}{IP}{FE80:0000:0000:0000:020C:76FF:FE21:1C3B}. 
	L’adresse de version 4 (IPv4) est encore actuellement la plus utilisée. 
\end{items}

Dans un réseau, chaque machine possède une adresse IP fixée par l'administrateur du réseau. \\
Il est interdit de donner la même adresse à 2 machines différentes sous peine de dysfonctionnement.\\

Une adresse IPv4 est une suite de 32 bits (4 octets)  notée en général a.b.c.d avec a, b, c, et d des entiers « décimal » 
compris entre 0 et 255.
 Chaque valeur a, b, c ou d représente dans ce cas une suite de 8 bits. 

 \begin{exemple}
Une machine qui a comme adresse IP 134.214.80.12 (En décimal) :

\begin{items}{blue}{\Bullet}
\item a vaut 134 soit (1000 0110) en binaire.
\item b vaut 214 soit (1101 0110) en binaire. 
\item c vaut 80 soit (0101 0000) 
\item d vaut 12 vaut (00001100).
\end{items}

En binaire, l'adresse IP s'écrit  \lbl{green}{IP}{10000110.1101 0110.0101 0000.0000 1100} et puisque le codage se fait sur 
8 bits les valeurs seront obligatoirement comprises entre 0 et 255.
\end{exemple}

\subsubsection{Le net-id et le host-id}

Au sein d'un même réseau IP, toutes les adresses IP commencent par la même suite de bits.
L’adresse IP d’une machine va en conséquence être composée de 2 parties : 

\begin{items}{blue}{\Bullet}
	\item Net-id : La partie fixe de l'adresse IP
	\item Host-id : La partie réservée aux machines qui viennent sur le réseau
\end{items}


\subsubsection{Masque de réseau IP}

\index{Masque de sous-réseau}
Le masque du réseau permet de connaître le nombre de bits du net-id. On appelle N ce nombre. Il s’agit d’une suite de 32 bits 
composée en binaire de N bits à 1 suivis de 32-N bits à 0.\\
C’est le masque du réseau qui définit la taille d’un réseau IP : c’est-à-dire la plage d’adresses assignables aux machines du réseau.

Ainsi, pour connaître le net-id, on va faire un ET (\& ou .) logique entre le masque et l'adresse IP.

\begin{exemple}

	Prenons l'adresse IP \lbl{green}{IP}{192.168.1.0} et le masque \lbl{green}{MASK}{255.255.255.0}. \\ Que vaut le net-id ? \\
	Pour rappel : 
	\begin{items}{blue}{\Triangle}
		\item 0.0 = 0
		\item 0.1 = 0
		\item 1.0 = 0
		\item 1.1 = 1
	\end{items}

\end{exemple}

	
\begin{solution}
	
	L'adresse \lbl{green}{IP}{192.168.1.0} vaut \lbl{orange}{BIN}{1100000.10101000.0000000.00000001} et \\
L'adresse \lbl{green}{MASK}{255.225.255.0} vaut \lbl{orange}{BIN}{11111111.11111111.11111111.00000000} \\


~~~01100000.10101000.00000000.00000001 \\
ET
~~11111111.11111111.11111111.00000000 \\
=
~~~01100000.10101000.00000000.00000000 ~~ = ~~\lbl{green}{BIN}{01100000.10101000.00000000.00000000} \\

L'adresse \lbl{green}{IP}{192.168.0.0} obtenue représente donc \bold{l'adresse du réseau}.\\

Les autres bits à 0 sont donc réservés aux périphériques du réseau. Étant sur 8 bits, il y a donc $2^8-2$ adresses 
disponibles sur ce réseau.

\end{solution}


\subsection{Les adresses interdites}

Il est interdit d’attribuer à une machine d’un réseau l’adresse du réseau et l’adresse de broadcast (diffusion)

\subsubsection{L'adresse de diffusion}

Cette adresse permet à une machine d’envoyer une info à toutes les machines d’un réseau. 
Cette adresse est celle obtenue en mettant tous les bits de l’host-id à 1. \\
Dans notre cas c'est l'adresse \lbl{green}{IP}{192.168.0.255}

\subsubsection{L'adresse de réseau}

Cette adresse est celle obtenue après avoir fait le ET entre l'adresse IP et le masque de sous-réseau.
Dans notre cas c'est l'adresse \lbl{green}{IP}{192.168.0.0}


\section{Choix des adresses IP}


Sur un réseau local, les adresses sont choisies par un serveur \glossary{DHCP}.

Les adresses peuvent être :

\begin{items}{blue}{\Bullet}
	\item Statiques : Un appareil sur le réseau possède la même adresse après connexion puis déconnexion.
	\item Dynamiques : l'adresse IP varie dans le temps au bout d'une période d'expiration après une déconnexion 
	(bail d'une journée par exemple)
\end{items}

A de rares exceptions, vous n'êtes pas censé attribuer une adresse IP vous-même à une machine.\\


\section{Quelques exemples de type de réseau}

Un réseau IP peut avoir une taille très variable :

\begin{items}{green}{\Bullet}
	\item Une entreprise moyenne aura un réseau comportant une centaine de machines.
	\item Un campus universitaire aura un réseau comportant de quelques milliers à quelques dizaines de milliers de machines.
	\item Le réseau d'une multinationale (un grand fournisseur d'accès par exemple) peut comporter des millions de postes.
\end{items}

\img{\rootImages/network.png}{Un réseau plus évolué}{0.4}

Il existe 2 types de réseau :

\begin{items}{green}{\Bullet}
\item Les réseaux publics Internet où chaque équipement connecté doit posséder une adresse unique et enregistrée au niveau mondial. 
\item Les réseaux privés, dans ce cas le choix des adresses est libre et ne doivent être uniques que dans ce réseau. 
\end{items}
Si un réseau privé doit être inter-connecté avec le réseau Internet, il faudra alors utiliser des adresses privées qui 
ne puissent correspondre à des adresses publiques utilisées sur Internet. Des plages d’adresses réservées à usage privé 
existent et elles ne sont donc pas acheminées par les routeurs Internet, ce qui supprime tout risque de conflit (cf. document annexe). 


\section{Récupération des adresse IP}

Voici 2 commandes pour récupérer l'adresse IP et le masque de sous-réseau.

\subsubsection{Pour Windows}
Sous un terminal Windows \footnote{Saisir \bold{cmd} dans le gestionnaire de programme}
\begin{Bash}{Récupération des informations IP sous Windows}
ipconfig
\end{Bash}
\img{\rootImages/ipconfig.png}{La commande ipconfig}{0.7}
	

\subsubsection{Pour Linux}
Sous un terminal Linux, on peut utiliser la commande \lbl{purple}{LIB}{ifconfig}\footnote{Si non installée sous Linux, 
saisir \lbl{purple}{CMD}{sudo apt-get install net-tools}}
\begin{Bash}{Récupération des informations IP sous Linux}
ifconfig
\end{Bash}

\img{\rootImages/ifconfig.png}{La commande ipconfig}{0.5}
	
	


