%\part{Processus Approvisionnement}

\chapter{Mise à jour des informations devis}

\section{Présentation du besoin}

Entre la création d'un devis et la commande, il peut se passer plusieurs jours voir plusieurs semaines. Il convient donc de mettre à jour les informations du devis car le service Approvisionnement se base sur les données Devis.

\section{Pistes suggérée}

En faisant une recherche en ligne avec les API, il est possible de récupérer les prix et informations des composants du devis.\\
En cette période ou les composants change de prix et les disponibilité fluctuent rapidement, il se peut que certains composants n'existent plus entre le devis et la commande.

\section{Fréquence de rafraîchissement}

La fréquence de rafraîchissement maximale dépend du nombre de requêtes possibles par API. En moyenne, ce nombre vaut 1000.

Deux politiques de rafraîchissement envisagées : 

\begin{items}{orange}{\Triangle}
\item Une mise à jour automatique tous les X jours.
\item Une mise à jour sur demande.
\end{items}


\subsection{Mise à jour automatique}

Ce choix semble compliqué car au vu du nombre maximal de requêtes, nous pouvons avoir le scénario suivant : 


\begin{items}{blue}{\Triangle}
\item Lundi : devis de 45 lignes réalisé, cela implique 45 requêtes.
\item Mardi : 3 devis de 25 lignes. Si mise à jour du devis de lundi, cela fait 120 requêtes à la journée
\item Mardi : 2 devis de 30 lignes. Si mise à jour du devis de lundi et mardi, cela fait 180 requêtes à la journée
\item Mardi : 4 devis de 20 lignes. Si mise à jour du devis de lundi,mardi mardi et mercredi, cela fait 260 requêtes à la journée
\end{items}
En suivant ce rythme, on voit bien qu'au bout d'une semaine ou deux, le nombre de requêtes par jour dépasse les 1000.

\subsection{Mise à jour sur demande}

Avant de faire une commande pour une BOM, il faudrait mettre à jour la BOM spécifique.

Avec une moyenne de XX lignes de commande par jour, cela représente XXX requêtes par jour, ce qui est plus acceptable.

