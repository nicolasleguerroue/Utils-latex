

\chapter{Processus global}


Lien du planning  :\url{https://docs.google.com/spreadsheets/d/1A7SzA6MrE79ad0Wu_JOUoUsbwYkRI9X0kK3oco4b4y0/edit?usp=sharing}

Voici les différentes étapes et axes de travail du projet de stage. Toutes ces étapes sont détaillées par la suite.

\section{Processus Devis}

\begin{items}{orange}{\Bullet}

\item Extraction des données clients (OCR). 
Depuis plusieurs supports clients (PDF, mails, Excel),pouvoir extraire les données pour les manipuler plus facilement (codification, recherche des prix en ligne...)

\item Récupération des prix, des délais de livraison et des durées de vie des composants par les fournisseurs et distributeurs.

\item Gestion des devis PCB

\item (Gestion des données documentaires (accès simplifié des plans, datasheets pour les services suivants))

\end{items}


\section{Processus Codification}

\begin{items}{orange}{\Bullet}

\item Reduction de l’utilisation d’Excel via la mise en place d’outils de traitement plus automatiques. Ces outils visent à importer automatiquement les données en provenance des fournisseurs (API) pour faire une codification automatique.

\item En découle une gestion attentive des doublons de référence lors de la codification automatique.

\end{items}

\section{Processus Approvisionnement}

\begin{items}{orange}{\Bullet}
\item Mise à jour régulière des prix et délai pour les commandes.
\end{items}

\section{Synthèse du processus}

\img{\rootImages/processus.PNG}{Processus de traitement global}{0.9}

\section{Les limites actuelles}

Voici la liste des limites et des contraintes pour la mise en place des outils :
\begin{items}{red}{\Triangle}

\item Il y a une limitation du nombre de licences X3, actuellement, tout le monde ne peux pas
utiliser X3
\item Taille du serveur pour stocker des fichiers (datasheets et plans peuvent poser un problème
de taille, 7000 fichiers. Avec une moyenne de 10 Mo par fichier, cela fait plus de 70 Go de stockage nécessaires. Ce point sera abordé dans la gestion des datasheets lors de la récupération des données en ligne.
\item Pour chaque API, nous sommes limités à 1000 requêtes par jour.
\end{items}
