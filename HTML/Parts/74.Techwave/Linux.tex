%
%\part{Architecture}

\chapter{Machine linux}
	
Le composant technique le plus important de ce projet est un serveur/VM linux où serons hébergées l’intégralité des fonctionnalités du projet. Ce serveur devra :

\begin{items}{orange}{\Bullet}
\item Pouvoir être accédé depuis le réseau interne de l’entreprise (Accès à un service web)
\item Pouvoir avoir accès à Internet (Pour aller interroger les API fournisseurs)
\item Pouvoir avoir accès au serveur interne brzadx01 (serveur host du système X3 / BDD X3) 
\end{items}

\begin{tabular}{|p{1.5cm}|p{6cm}|p{6cm}|p{2cm}|}
  \hline
  \bold{N°} & \bold{Tâche} & \bold{Description} & \bold{Nécessite} \\
  \hline
 1-1 & Déploiement du serveur /VM linux & La machine linux est déployée & - \\
 \hline
1-2 &	Accès par le réseau local de l’entreprise &	La machine linux doit être joignable par le réseau interne de l’entreprise & - \\
  \hline
1-3	& Accès à Internet & La machine linux doit avoir accès à Internet & - \\
\hline
1-4	& Accès à brzadx01 & La machine linux doit avoir accès au serveur brzadx01 & - \\
\hline
\end{tabular}

\section{Composant de lecture de la BDD X3}


Notre application a besoin d’accéder en lecture à la base de données de X3. Pour cela, nous allons utiliser une liaison directe avec la BDD Oracle avec un petit programme codé en (Java/PHP/Python ? TBD). Ce programme prendra en entrée :

\begin{items}{orange}{\Bullet}
\item Une requête SQL (Nous utiliserons principalement les vues fournies par Yves)
\item Des arguments pour la requête (Dépendant de la requête en question)
\end{items}

Et donnera en sortie : \\

\begin{items}{blue}{\Bullet}
\item Le résultat de la requête sous la forme d’un tableau d’objets JSON
\item Un statut OK/KO et un message d’erreur s’il y en a une
\end{items}

\begin{tabular}{|p{1.5cm}|p{6cm}|p{6cm}|p{2cm}|}
  \hline
  \bold{N°} & \bold{Tâche} & \bold{Description} & \bold{Nécessite} \\
  \hline
 2-1 & Récupération des données & La récupération des données par le programme doit être fonctionnelle & - \\
 \hline
2-2 &	Gestion des erreurs &	Le programme doit correctement renvoyer le statut KO et le message d’erreur s’il en est survenue & - \\
  \hline
\end{tabular}


\section{Composant d’écriture de la BDD X3}

Notre application a besoin d’écrire et de mettre à jour des données dans la BDD d’X3 (Notamment des articles). Pour cela, nous écrivons les données que nous voulons importer dans des fichiers CSV avant d’importer ces fichiers CVS dans le système X3.

Par mesure de sécurité et de simplicité d'utilisation de cet utilitaire, seul des commandes prédéfinies correspondantes aux différents options d'importation vers la BDD que nous offrira Yves. \\

\begin{tabular}{|p{1.5cm}|p{6cm}|p{6cm}|p{2cm}|}
  \hline
  \bold{N°} & \bold{Tâche} & \bold{Description} & \bold{Nécessite} \\
  \hline
 3-1 & Écriture des données & L’écriture des données par le programme doit être fonctionnelle & - \\
 \hline
3-2 &	Gestion des erreurs &	Le programme doit correctement renvoyer le statut KO et le message d’erreur s’il en est survenue & - \\
  \hline
\end{tabular}


\section{Base de données MySQL}

Afin de stocker les noms des fiches des composants pour permettre de le retrouver facilement, notre projet a besoin d’une base de données. \\

Cette base de données sera basée sur MySQL et sera host sur la même machine linux que le serveur web. Cette DBB servant uniquement pour cette fonctionnalité et dans le but de limiter un maximum les données redondantes avec la BDD d’X3, le MCD est très simple :\\

\begin{tabular}{|p{12cm}|}
  \hline
  \bold{Fiche} \\
  \hline
ref\_fab VARCHAR(30) PK \\
desc VARCHAR(100) \\
fiche VARCHAR(30) \\
  \hline
\end{tabular} \\

Et les tâches à réaliser : \\

\begin{tabular}{|p{1.5cm}|p{6cm}|p{6cm}|p{2cm}|}
  \hline
  \bold{N°} & \bold{Tâche} & \bold{Description} & \bold{Nécessite} \\
  \hline
 4-1 & Installation de MySQL-server & Installation de la BDD MySQL sur le serveur linux & 1-1 \\
 \hline
3-2 &	Création de la BDD &Création de la BDD décrite plus haut & - \\
  \hline
\end{tabular}

\section{Composants généraux serveur WEB}

Un serveur web va être nécessaire pour fournir une interface utilisateur pour chacun des éléments de ce projet. Nous allons donc pour cela :

\begin{items}{orange}{\Bullet}
\item Déployer un serveur web basé sur apache2 et PHP sur la machine linux
\item Développer un ensemble de composants JS spécifique aux besoins de ce projet :

\begin{items}{blue}{\Triangle}
\item Un composant permettant l’affichage et un peu de manipulation de données en tableau exemples : BOM, Résultats API, …
\item Un composant permettant de faire un appel direct en JS au composant de lecture de la BDD X3 sur le serveur.
\item Un composant permettant de faire un appel direct en JS au composant d’écriture dans la BDD X3 sur le serveur.
\end{items}

\end{items}


\begin{tabular}{|p{1.5cm}|p{6cm}|p{6cm}|p{2cm}|}
  \hline
  \bold{N°} & \bold{Tâche} & \bold{Description} & \bold{Nécessite} \\
  \hline
 5-1 & Déploiement du serveur WEB & Déployer le serveur web sur la machine linux & 1-1 \\
 \hline
5-2 &	Composant JS accès lecture BDD & Réalisation d’un composant JS permettant de faire le lien avec le composant de lecture de la BDD X3 sur le serveur & - \\
  \hline
5-3 & Composant JS accès écriture BDD & Réalisation d’un composant JS permettant de faire le lien avec le composant d’écriture de la BDD X3 sur le serveur & - \\
  \hline
5-4 &	Composant JS Tableau BDD & Réalisation d’un composant JS permettant l’affichage de données en tableau (BOM, Résultats API, …) & - \\
  \hline
5-5 &	Page d’accueil & Création d’une page d’accueil au site pour naviguer à travers ses différentes fonctionnalités & - \\
  \hline
\end{tabular}\\


Pour pouvoir faire le devis d’une BOM, le service achats a besoin des prix de chacun des composants de la BOM. La recherche de ces prix est coûteuse en temps. Or les fournisseurs nous donnes accès a des API permettant de récupérer des informations de prix, MOQ, etc relatives à ces composants automatiquement. Il serait donc intéressant d’aller rechercher ces informations automatiquement afin de réduire la charge de travail du service achats.\\

Pour cela, il nous semble pertinent de fournir au service achat une page web sur laquelle :


\begin{items}{orange}{\Bullet}
\item L’utilisateur doit entrer ces informations sur la page :
\begin{items}{blue}{\Triangle}
\item Une BOM sous format CSV composé de : (Ref Fabricant, Quantité par carte, Description).
\item Un nombre de cartes à produire
\end{items}

\item La page doit prendre ces données et :
\begin{items}{blue}{\Triangle}
\item Comparer les composants de la BOM à la BDD X3 existante pour voir s’ils y figurent
\item Aller vérifier grâce aux API des fournisseurs :
\begin{items}{black}{\Triangle}
\item Vérifier la présence des composants de la BOM, et si on les trouve, les trouver au meilleur prix, en prenant en compte des infos comme les MOQ.
\item Si on ne trouve pas ces composants, vérifier si les fournisseurs on des équivalences. Ces équivalences seront affichées sur la page web et devront être vérifiées à la main via la page web avant d’être validées.
\end{items}
\end{items}
\begin{items}{blue}{\Triangle}
\item Vérifier les stocks potentiels des composants de la BOM
\end{items}
\item Une fois ce traitement valider l’utilisateur peut :
\begin{items}{blue}{\Triangle}
\item Récupérer ces données sous la forme d’un fichier CSV
\item Déclencher une codification automatique de tout les composants découverts grâce aux API des fournisseurs
\end{items}
\end{items}

\begin{tabular}{|p{1.5cm}|p{6cm}|p{6cm}|p{2cm}|}
  \hline
  \bold{N°} & \bold{Tâche} & \bold{Description} & \bold{Nécessite} \\
  \hline
 6-1 &Recherche automatique des nouveau composants & Récupération des informations des API à propos d’un composant & -  \\
 \hline
6-2 & Recherche dans la base X3 des composants connus &Récupération des informations des composants connus par la BDD X3  & 2-1 \\
  \hline
6-3 & Front Web Page Entrée des informations & Réalisation de la page HTML / CSS d’entrée des informations & - \\
  \hline
6-4 &	Front Web Page Résultats & Réalisation de la page HTML / CSS d’affichage des résultats & - \\
  \hline
6-5 &Back web & Réalisation de la gestion de la requête AJAX au niveau du serveur (PHP) & - \\
  \hline
6-6 & Importation des CSV &L’utilisateur doit pouvoir importer un CSV contenant la BOM simplifiée sur la page WEB  & 4-4 \\
  \hline
6-7 & Téléchargement des résultats &	L’utilisateur doit pouvoir télécharger les résultats sous forme d’un CSV & 4-4 \\
  \hline
6-8 & Codification automatique des résultats API &L’utilisateur doit pouvoir déclencher la codification automatique des résultats découverts par API  & 3-1 \\
  \hline
\end{tabular}\\


\section{Organisation des fiches de composants}

Afin de faciliter l’accès aux fiches des composants à tous les utilisateurs, notre projet intègre un page de recherche des fiches composants. A partir de cette page l’utilisateur peut :
\begin{items}{black}{\Triangle}
\item Rechercher la fiche d’un composant à partir soit de sa référence fabricateur soit de sa référence AODE, soit de sa description.
\item Ajouter manuellement une fiche
\item Supprimer manuellement une fiche
\end{items}

L’avantage d’incorporer cette fonctionnalité à notre application est, en plus que d’être plus pratique à recherchée qu’un dossier, que des fiches peuvent êtres rajoutés automatiquement au moment des recherches par API.\\


\begin{tabular}{|p{1.5cm}|p{6cm}|p{6cm}|p{2cm}|}
  \hline
  \bold{N°} & \bold{Tâche} & \bold{Description} & \bold{Nécessite} \\
  \hline
 7-1 & Recherche des fiches à partir de la référence fabriquant ou la description &L’utilisateur peut rechercher une fiche précise à partir de la référence fabricant ou la description du composant & -  \\
 \hline
7-2 & Recherche des fiches à partir de la référence AODE &L’utilisateur peut rechercher une fiche précise à partir de la référence AODE  & 2-1 \\
  \hline
7-3 &Ajout manuel d’une fiche &L’utilisateur peut rajouter une fiche à la BDD manuellement & - \\
  \hline
7-4 & Suppression manuelle d’une fiche & L’utilisateur peut supprimer une fiche de la BDD manuellement & - \\
  \hline
\end{tabular}\\


\section{Utilitaire de gestion des Relances}

Afin de simplifier la gestion des relances, notre projet prévois un petit composant en permettant leur gestion. Le but de se composant est de regarder automatiquement dans la BDD d’X3, et d’envoyer un mail de relance aux fournisseurs n’ayant pas envoyés leur commande dans les temps. Pour cela, nous comptons mettre en place un programme python s’exécutant en continu en arrière-plan sur la machine linux. De plus il nous semblerait intéressant de créer avec cet utilitaire une interface web montrant les relances envoyées récemment. 
\\ \\
\begin{tabular}{|p{1.5cm}|p{6cm}|p{6cm}|p{2cm}|}
  \hline
  \bold{N°} & \bold{Tâche} & \bold{Description} & \bold{Nécessite} \\
  \hline
 8-1 & Déploiement du composant & Le composant doit tourner en permanence en arrière-plan sur la machine linux & 1-1  \\
 \hline
8-2 & Envois de mails &Le composant peut envoyer des mails automatiquement via le serveur SMTP d’AODE & - \\
  \hline
8-3 & Récupération des commandes en retard & Le composant arrive à récupérer les commandes en retard ainsi qu’une adresse mail de contact liée à cette commande & - \\
  \hline
8-4 & Interface WEB & Création d’une interface web liée à ce composant pour permettre aux utilisateurs de tracker les messages envoyés & 8-2 \\
  \hline
\end{tabular}\\


