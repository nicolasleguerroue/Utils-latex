%\part{Processus Devis}

\chapter{Gestion des formats client}

Les données des clients proviennent de différents supports (PDF, Excel...)

En ce qui concerne les fichiers PDF, il est souhaitable de lire les données du document pour les mettre dans un format exploitable (Fichier CSV, JSON...).


\chapter{Gestion des composants de la BOM}
\section{Présentation du besoin}

Il faut saisir et mettre à jour régulièrement le prix des composants. Le temps de recherche est long et fastidieux. Les composants sont recherchés sur les fournisseurs/distributeurs suivants  :

\begin{items}{orange}{\Bullet}
\item Farnell (1000 requêtes par jour, clé disponible, en attente de la \bold{secretKey} et \bold{custormerID})
\item Mouser ( 1000 requêtes par jour, clé disponible, pas de réduction pour Techwave)
\item Digi-key (1000 requêtes par jour, clé disponible, prix réduits)
\item Arrow (clé disponible, prix réduits)
\item Future (clé disponible, prix réduits)
\item Avnet (en attente de clé)
\end{items}


 RS-Component ne possède pas d'API et TTI met en place une API pour l'Europe cet été.
 
\section{Piste suggérée}

En récupérant les données fournisseurs avec des API\footnote{Interface de Programmation d'Application}, il est possible de connaitre la quantité, le prix et le temps de livraison estimé en temps réel.Une recherche d'un composant est considéré par la suite comme étant une requête.\\

Comme énoncé dans les limites, chaque API limite à 1000 requêtes par jour.

L’objectif est d’obtenir des prix fiables car ces derniers sont en provenance des fournisseurs.\\

Pour les composants obsolètes, recherche dans le stock (référence, type [CMS]...) et en cas de non correspondance, recherche en ligne.


\section{Gestion des API}

\subsection{format d'entrée des API}

Lors d'une recherche d'un composant pour une API, il convient de donner les éléments suivants : 

\begin{items}{blue}{\Bullet}
\item Référence fabricant souhaitée
\item Quantité souhaitée
\end{items}

\subsection{Format de sortie des API}
Pour une référence de composant données, chaque API va nous retourner une liste de composants ayant les attributs suivants : 

\begin{items}{blue}{\Bullet}
\item Référence fabricant
\item Catégorie
\item Lien de la documentation
\item Cyle de vie (Obsolète, nouveau produit...)
\item Description
\item Fabricant
\item Disponibilité
\item MOQ
\item Group Order\\
\end{items}

Les données en provenance des API sont au format JSON ou XML.


Les données sont récupérées via les API en multithreading\footnote{Se reporter à la page \pageref{multithrading} pour plus d'informations}, c'est à dire que les n fournisseurs/distributeurs sont solicités en même temps pour réduire le temps d'attente.\\

Une fois que les données brutes ont été reçues, ces données sont traitées et mises en forme sous forme de liste de composants dans un format spécifique (JSON)

\img{\rootImages/LectureAPI.png}{Utilisation des API}{1}

\subsection{Données composées}

Ces données peuvent avoir un format non conforme mais après traitement, nous pouvons en déduire les attributs suivants :\\

\begin{items}{blue}{\Bullet}
\item Prix unitaire
\item Quantité disponible
\item Etat du composant (obsolète, nouveau produit...)
\end{items}


\subsection{Traitement des données}


Les composants sont traités de la façon suivante :

\begin{items}{blue}{\Triangle}
\item On recherche tous les composants disponibles en quantités suffisante et on met de coté les composants oboslètes et similaires
\item On cherche le meilleur prix parmi les composants trouvés en tenant compte des MOQ\footnote{Minimal Order Quantity}
\item Si aucun composant n'est trouvée, on retourne un composant "inexistant"
\end{items}

En analysant ces données, nous pouvons définir des statuts pour chaque composant : 

\begin{items}{orange}{\Bullet}
\item Composant existant / inexistant
\item Composant obsolète/ non obsolète
\item Composants similaire / non similaire
\item Composant au meilleur prix
\end{items}

Un composant est défini come non existant si aucune référence pour le composant demandé n'est trouvée.\\

Si des composants sont trouvés mais sont tous en quantité insuffisante, le programme retourne le composant avec le meilleur prix.\\

Un composant est défini comme "similaire" lorsque la référence demandée n'est pas strictement incluse dans la référence trouvée.\\
Par exemple, si on cherche un composant \bold{BC337} et que l'API retourne une référence \bold{BC537}, le composant est considéré comme "similaire".

\img{\rootImages/MethodeAPI.png}{Processus de traitement des données}{1.1}

\subsubsection{Gestion des composants obsolètes}
Les composants avec la référence exacte mais obsolètes sont supprimés de la liste principale mais apparaissent dans une liste "obsolète"


\subsubsection{Gestion  des composants similaires}

Les composants similaires sont traités de la même manière que les composants avec la référence exacte

\subsubsection{Quelques cas d'usage}

Cas d'usage 1 : \\


Pour un composant demandé, on peut avoir : 

\begin{items}{orange}{\Bullet}
\item 0 composant avec la référence exacte trouvée
\item 1 composants avec la référence exacte obsolète
\item 2 composants similaires
\end{items}

\bold{Le choix devra donc se porter sur le meilleur composant similaire}\\

Cas d'usage 2 : \\

\begin{items}{orange}{\Bullet}
\item 1 composant avec la référence exacte trouvée
\item 0 composant avec la référence exacte obsolète
\item 5 composants similaires
\end{items}
\bold{Le choix devra se porter entre le meilleur prix du composant exact ou la référence similaire.}



\chapter{Devis PCB}

\section{Présentation du besoin}

Actuellement, les demandes de prix sont envoyé par mail aux fabricants des PCB (SAve, ICAP, Wurth PCB). L'entreprise est donc dépendante du temps de réponse des fabricants.

\section{Piste suggérée}

En fonction du type de PCB demandée, il convient d'aller sur le site du fabricant et d’utiliser l'outil de demande de prix en ligne.

Cet outils est une page WEB ou il faut saisir les paramètres du PCB (technologie multicouche, flex, hyper-fréquence)
.

Nous avons essayé de faire le processus avec une APi (Safe en l’occurrence) mais en dehors des circuits double-face, le nombre de paramètre est beaucoup trop important pour pouvoir être renseigné par notre site internet.


Les fabricants avec les outils de demande de prix sont les suivants : 

\begin{items}{orange}{\Bullet}
\item Safe
\item ICAP
\item Wurth PCB
\end{items}



\chapter{Gestion de la documentation}
\section{Présentation du besoin}

Le service Devis a parfois besoin de la datasheet ou d’un plan, le service Codification aussi
pour visualiser le composant. Il est donc souhaitable d’éviter le doublon de l’étape "recherche de datasheet" sur le Web.


\section{Piste suggéré}

Une première piste est la récupération des datasheets par certaines API fournisseurs. Il ets possible de récupérer le lien des datasheets et de les afficher sur le site\footnote{Recherceh par référence fabricant}


Idéalement, il faudrait supprimer les datasheets considérées comme obsolètes.\\


L’outil X3 n'est pas capable de supporter l’importation de fichier PDF.

Nous pourrions utiliser un CMS pour la gestion des Documents (GED) mais ceux si sont parfois payants et la taille des données manipulées (lien HTML) ne justifie pas l'utilistion d'un GED.

Parmi les GED existants, notons : 

\begin{items}{orange}{\Bullet}
\item Xoops
\item Nuxeo
\item Alfresco
\item KnowledgeTree
\item Drupal
\end{items}
