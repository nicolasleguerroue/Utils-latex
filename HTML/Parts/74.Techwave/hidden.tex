
\chapter{Gestion des relances}


\section{Présentation du besoin}

Il faut relancer régulièrement les fournisseurs lorsque le temps de la réponse pour la validation de la commande est trop long.

\section{Piste suggérée}

En se connectant à la base de données, on peut récupérer les "gap ARC" et envoyer le mail approprié en conséquent via des bibliothèques Python. Le tout encapuslé dans un script, avec un fichier logs pour la tracabilité des mails.\\

L’envoi se ferait à partir d’un serveur SMTP Outlook basé sur une adresse mail de l’entreprise (adresse mail d'Aline)

\section{Vérifications}

Le serveur SMTP s’occupe de l’envoi des mails mais normalement, les mails non envoyés sont renvoyés dans la messagerie d’envoi, ce qui permet de vérifier le bon envoi des mails.




\section{Structure du code}

Sur l'interface principale, le bouton "Mettre à jour" se contente d'appeler un script python qui, lui, va lire la base de données et sauvegarder le contenu dans un fichier au format JSON.

Respectivement, pour :


- les relances ARC : rawARC.json

- les relances retards : rawLate.json


Ensuite, l'interface Web se contente d'afficher le fichier et de pré-cocher les relances sans erreur (c'est à dire avec une adresse mail, principale ou non).

Une fois les postes cochés, l'utilisateur valide avec le bouton "Valider" et la phase de relecture peut commencer. Il suffit juste à l'utilisateur de vérifier si une adresse mail est bien présente.

Enfin, il ne lui reste plus qu'à envoyer les mails en cliquant sur "Envoyer"

Un mail génère une erreur quand l'adresse d'envoi est vide ou bien il y a un espace. 
