
\chapter{Quelques pistes d’améliorations}
\section{Ajustement de la fréquence du signal}

Par relevés sur l’oscilloscope, nous pouvons constater que la fréquence en sortie n’est pas la fréquence théorique. 


Ici, ci contre, nous avons joué un LA\# (octave 1)
La fréquence théorique pour cette note est de $466 Hz$ et sur l’oscilloscope, nous relevons 497 Hz. Il y a un écart de $32 Hz$.




Ces facteurs d’écarts sont nombreux mais nous pouvons essayer d’en éliminer, ou du moins les réduire. 

Tout d’abord, nous avons mesuré la valeur réelle du condensateur \\
Ne pouvant pas mesurer la valeur du condensateur en se basant sur la constante de charge RC avec nos appareils de projet (résistances trop faibles, inférieures à 10 M), nous avons utilisé un appareil plus précis : au lieu des 15nF, nous avons relevés 14.46nF (4.6\% d’erreur).
De plus, nous avons mesuré précisément les valeurs de toutes les résistances du circuit.\\

Ainsi, la fréquence du signal dépend donc directement de la constante de temps RC du circuit intégrateur. \\
Plus cette constante sera élevée, plus le coefficient directeur du signal de l’intégrateur sera faible.
Or, un coefficient plus faible représente un signal de fréquence plus faible car pour atteindre les 6 volts d’amplitudes (seuil de basculement du circuit 3), 
le signal mettra plus de temps à atteindre cette valeur.\\


Afin de modifier la fréquence, nous allons modifier la valeur de la résistance R4 (plus facile que de mettre une capacité variable) en utilisant un potentiomètre couplé à la résistance R4 dont la valeur sera à modifier.\\



Notre contrainte est de concevoir une résistance équivalente R4 afin d’augmenter ou de diminuer la valeur théorique de la résistance R4. 
On a calculé que R4=6080 Ohms (théorie).
Cependant, avec une résistance R4 de 5990 Ohms (5.53K+460), la fréquence du signal ci dessus est supérieur de 31 Hz par rapport à la fréquence théorique.
Pour former une résistance équivalente similaire, légèrement plus grande, nous allons prendre une résistance de 5.6K et ajouter un potentiomètre de 1K afin de former une résistance allant jusqu’à 6.6KOhms.

Une fois le potentiomètre ajouté et valant 579, nous obtenons les fréquences suivantes :\\

Pour obtenir ces valeurs, nous avons réalisé le schéma suivant :\\

Une fois le branchement effectué, nous avons mesuré la fréquence de sortie du système en fonction de la touche du piano :

On constate que la fréquence réelle est plus proche de la fréquence théorique avec le potentiomètre.













2 - Ajustement du cycle d’hystérésis
Après les différents relevés, nous avons constaté que la fréquence était correcte avec l’ajout du potentiomètre en série avec R4. Cependant, le signal triangulaire n’est pas centré en 0V 
et Vseuil1Vseuil2.


Ici, nous avons joué un LA 440 Hz. La fréquence mesurée est proche de la théorique avec le potentiomètre POT1 mais le signal n’est pas symétrique. Nous avons 
Vmax=6.08V et
Vmin=-5.44V











Afin de pallier à ce souci, nous pouvons utiliser deux méthodes pour compenser les dérives du circuit dues notamment au fait qu’en réalité, Vsat+Vsat- (respectivement 13.8V et 13.2V)
Tout d’abord, avec un potentiomètre, nous pouvons modifier la valeur de basculement afin d’obtenir un signal crête-à-crête de 12V.

Nous pouvons injecter une tension à la base du pont diviseur formé par R5 et R6 afin de recentrer le cycle d’hystérésis

En observant l’équation du seuil de basculement, nous voyons clairement que plus R5 augmente, plus l'écart entre les deux tensions de seuils sera important.
Nous allons former une résistance R5 équivalente plus précise.
Nous avons mesuré les tensions Vsat+ et Vsat-réelles afin de tenir compte de cette différence. 
Pour être plus précis, nous allons modifier la tension d’alimentation de l’AOP N°3 afin qu’en simulation, nous obtenions Vsat+=13.8Vet Vsat-=-13.2V
Ces deux valeurs ont été mesurée avec un voltmètre.
Sur LTSpice, la tension de saturation vaut toujours Vcc-1.53
Nous allons donc alimenter (en simulation) l’AOP en +15.33 et en -14.73V

Après calcul avec la résistance R6 de valeur exacte (9.73k), la nouvelle résistance est  comprise entre7.48k et 8.10k(calculs réalisés en tenant compte des deux valeurs de saturation différentes).
Nous prendrons une résistance de 6.8kcouplée à un potentiomètre de 1k
Voici le schéma :


A l’aide de LTSpice, nous faisons varier le potentiomètre et avec un potentiomètre (monté en résistance variable) valant 1k, nous obtenons un signal de valeur crête-à-crête de 12V.


A l’avenir, il faudra donc augmenter légèrement R5 car le potentiomètre est en butée. De ce fait, il perd tout son intérêt en résistance variable.

Une fois que la tension crête-à-crête est de 6V, nous allons ajouter une source de tension à la base du pont diviseur R5/R6.






Nous allons faire le montage suivant : 



Les résistances R8 et R9 forment un potentiomètre POT3. En faisant varier le potentiomètre, nous décalons le cycle d'hystérésis, ce qui est le but recherché.
Dans cette configuration, le potentiomètre n’est pas monté en résistance variable mais bel et bien en source de tension variable. Le suiveur est ajouté afin que le pont formé par le potentiomètre ne débite pas de courant ailleur que dans les deux résistance R8 et R9.
Considérons représentant le déplacement du potentiomètre POT (01)
En testant quelques valeurs pour , on constate que si =0.49 (R9=4.9k et R8=5.1k)
on obtient un cycle d”hystérésis centré en 0V, de valeur crête-à-crête de 12V et de fréquence f=445Hz(au lieu de 466Hz).
Ainsi, pour obtenir ce cycle, il faut injecter une tension valant -0.7V (Vsat-0.05)à la base du pont diviseur

Obtenir un signal parfait est très compliqué, c’est pourquoi, au final, en faisant certains compromis, nous pouvons néanmoins valider le schéma suivant :





Ainsi, à chaque calibration du système : 
on déplace le potentiomètre POT1 afin d’obtenir la fréquence désirée et en accord avec la tension d’entrée
on régle le potentiomètre POT1 afin d’obtenir un signal de tension crête-à-crête de 12 V
on recentre ensuite le cycle avec le potentiomètre POT3 simulé par les des deux résistances.
