\chapter{Alimentation}
\section{Régulateur de Tension }


Le régulateur intégré de tension (RIT) est un régulateur linéaire permettant de baisser une tension d’alimentation à une valeur fixe.\\

La famille des composants HCT doit être alimentée en 5V [TTL]. Nous utiliserons un 78L05 qui fournit du 5V régulé. Le L « low power » indique que le composant peut être traversé par un courant de 100mA.\\
La tension d’entré concernant notre montage est de 15V, ce qui permet le bon fonctionnement du composant, le dataSheet indique « Ve>7V ».  \\

Le circuit global consommant du courant, nous ne pouvons pas utiliser de pont diviseur de tension.\\
 
Des appels de courant importants peuvent entraîner une chute de tension en sortie du régulateur. \\
Pour éviter cela nous insérons un condensateur de découplage, c’est à dire  en sortie du régulateur et relié à la masse. 
Le modèle équivalent de l’ensemble est un filtre passe bas.\\

Plus la valeur du condensateur est élevée,plus la tension de sortie sera stable.\\


 Figure 3.1 : Régulateur de tension 5V et son condensateur de découplage
