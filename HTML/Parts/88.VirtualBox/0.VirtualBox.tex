\chapter{VirtualBox}


Ce document a pour objectif de configurer le logiciel VirtualBox et de récupérer une image Linux 
afin de faire fonctionner Linux sous une machine Windows.\\
Ainsi, cela permet de travailler sur une machine Linux tout en évitant de modifier les fichiers 
sensibles de l’ordinateur (secteur d’amorçage…).\\
En revanche, ce système est gourmand en ressource, cela se traduit donc par une légère perte de 
performance sur la machine Virtuelle Linux (appelée MVL dans ce document).\\

Veuillez lire attentivement les prérequis car en cas de manquement à une obligation d’installation 
citée dans la section suivante, la MVL ne fonctionnera pas ou mal au pire des cas.\\



\section{Prérequis}


Avant de se lancer dans l’installation de la MVL, il est important de vérifier si l’ordinateur 
hôte pourra faire tourner cette MVL.\\
Pour cela, plusieurs conditions sont nécessaires :

\begin{items}{green}{\faviconCheck}
\item Être sous Windows 7, Windows 8, Windows 10 ou Windows 11
\item Avoir une connexion internet
\item Un ordinateur ayant au moins 6 Gigaoctets de mémoire vive (6 Go de RAM)
\item Cet ordinateur doit pouvoir supporter l’hébergement des machines virtuelles
\end{items}

Cette étape est primordiale, sans quoi la machine virtuelle ne pourra pas fonctionner.



\section{Bios}

Sur les ordinateurs pouvant héberger des machines virtuelles, le programme permettant 
d’activer cette fonctionnalité se trouve la plupart du temps dans le BIOS de l’ordinateur.\\

Le BIOS (Basic Input Output System) est un microprogramme qui se lance au démarrage de l’ordinateur 
avant que les systèmes d’exploitation ne démarrent. Ce microprogramme a pour but de vérifier 
l’intégrité des composants de l’ordinateur (mémoire, disque dur, ports USB…) et est accessible au 
démarrage de l’ordinateur en appuyant sur une touche du clavier lors de l’apparition de la marque 
de l’ordinateur.\\

Cependant, en fonction des ordinateurs, les touches peuvent changer…\\
Par exemple, voici les touches (les plus répandues) lors du démarrage d’un ordinateur.\\
Avant de faire une action, veuillez lire la section “Configuration du BIOS” en intégralité.

TOSHIBA
F2
DELL
F9
ASUS
F2
LENOVO
F2
HP
F10


Si jamais vous n’arrivez pas à accéder au BIOS avec les touches ECHAP, F2,F5, F10,F12 ou SUPPR, 
veuillez regarder la documentation en ligne à propos de votre ordinateur.\\


Le moment où il faut presser la touche :
(affichage de la marque de l’ordinateur)
(Je ne possède pas d’actions chez Dell…)

Une fois la touche pressée plusieurs fois dès le démarrage, le menu suivant devrait apparaître :

Il faut trouver la section Advanced (la plupart des cas) :
Dans les options, on cherche “Virtualization” ou “Virtualization Technology” que l’on sélectionne à « Enable » pour activer les machines virtuelles.

Il ne faut pas toucher aux autres options sous peine de ne plus voir l’ordinateur démarrer correctement
Ensuite, on presse F10 pour quitter le BIOS en sauvegardant les changements.


\section{Choix de la distribution}

Linux est un système d’exploitation basé sur un noyau UNIX.\\
Il existe une multitude de versions de Linux que l’on appelle distribution.\\
Ces distributions s’appellent Debian, Ubuntu, Fedora, CentOs… et se déclinent sous plusieurs 
interfaces graphiques.\\

Ainsi, pour chaque distribution Linux, le noyau est commun (gestion des composants…), 
seuls les logiciels annexes sont différents.\\
Le choix de la distribution se fait donc selon plusieurs critères comme : \\

\begin{items}{green}{\faviconLeaf}
\item La communauté autour de la distribution (Nombre d’utilisateurs et documentation fournie)
\item La variété des logiciels déjà présents lors de l’installation (peu de manipulation pour rendre 
le système fonctionnel)
\item La consommation de la mémoire vive (liée au choix de l’interface graphique)
\end{items}

Ainsi, je vous propose d’installer Xubuntu, une distribution basée sur Ubuntu et qui possède une 
interface graphique légère (interface Xfce). De plus, la communauté de Xubuntu est active et de nombreux 
logiciels sont disponibles.\\


Pour ceux qui le souhaitent, ils peuvent également installer une autre dérivée d'Ubuntu comme :

- Kubuntu  (interface gourmande mais évoluée)
- Lubuntu  (interface ultra-légère)

Pour ceux qui veulent plus d’informations, ils peuvent comparer les différentes interfaces graphiques :

- Kde			(Kubuntu...)
- Gnome		(Ubuntu...)
- Xfce			(Xubuntu...)
- Lxde			(Lubuntu...)
- Cinnamon 		(Linux Mint…)
- …


Ensuite, il vous faudra connaître votre architecture d’ordinateur. Si jamais vous ne l’avez pas, 
ce n’est pas grave mais c’est mieux de la connaître.\\
Pour cela, il vous faudra chercher dans les paramètres système de votre ordinateur.\\


Si vous avez un doute, prenez l’architecture 32 bits car cette dernière est compatible avec 
les 64 bits. Enfin, si votre ordinateur est récent, il y a de fortes chances pour qu’il soit en 64 bits.\\


Après avoir trouvé l'architecture, rendez vous sur le site officiel de la distribution en question 
(Exemple : http://xubuntu.fr/ )\\

Pour un ordinateur 32 bits avec Xubuntu, copier ce lien dans un navigateur : http://cdimage.ubuntu.com/xubuntu/releases/18.04.2/release/xubuntu-18.04.2-desktop-i386.iso
Pour un ordinateur 64 bits avec Xubuntu, copier ce lien dans un navigateur : http://cdimage.ubuntu.com/xubuntu/releases/18.04.2/release/xubuntu-18.04.2-desktop-amd64.iso

En cliquant sur un des ces liens, un fichier au format .iso va être téléchargé.\\
Il vous faudra noter l’emplacement du lieu de téléchargement (souvent dans “Téléchargements”)\\
Une fois le téléchargement de l’image ISO terminé, nous pouvons configurer VirtualBox.


\section{“Configuration}

Pour installer VirtualBox, téléchargez le logiciel sur le site officiel à l’adresse suivante :
https://www.virtualbox.org/wiki/Downloads 
Dans cette section, sélectionnez “Windows hosts”

Un exécutable sera téléchargé, exécutez le et laissez vous guider par l’installation (paramètres par défaut). Un logo similaire apparaît sur votre bureau




    b) Lancement
Lors de l’ouverture de VirtualBox, commencer par aller dans le menu supérieur et dans “Machine”, cliquez sur “Nouvelle”.



Le menu suivant s’affiche : il faut nommer la machine et sélectionner le système d’exploitation.
Dans le menu déroulant, sélectionnez “Linux/2.6/3.x” (dernière version).
Sélectionnez l’architecture 64 bits ou 32 bits en fonction de votre PC.


    c) Allocation de la mémoire
L’étape suivante permet d’allouer de la mémoire à la MVL et à VirtualBox



Par défaut, VirtualBox recommande une certaine quantité de mémoire (ici, 1Go)
Cependant, pour être à l’aise, il est recommandé de mettre 3 Go (3000 Mo)



Dans cette configuration de VirtualBox, l’ordinateur hôte n’est pas prévu pour faire beaucoup de tâches sur Windows. Si vous êtes sur la MVL, pensez à réduire les activités sous Windows, (Jeux, gros logiciels en arrière plan...).
Après quelques essais, vous pourrez diminuer la mémoire allouée pour la machine si vous le souhaitez.














    d) Création du disque dur

Nous allons créer un disque dur virtuel lors de l’installation.

Ensuite, nous allons définir le type de fichier de disque (par défaut)



Choisissez un fichier de disque dur à taille fixe afin de gagner en rapidité lors de l’utilisation de la MVL. (page suivante)





En faisant cette étape, nous allons pouvoir créer un disque dur pouvant contenir le système Linux. Pour être confortable, nous allons octroyer 20 Go de stockage pour la MVL.
Bien évidemment, cette capacité dépend de la taille de votre disque dur réel.

Après validation, cette fenêtre apparaît :


A la fin de l’animation, le disque dur virtuel est créé.





    e) Sélection de l’image ISO
Une fois sur la page principale, sélectionnez votre machine 

et cliquez sur Configuration


Cette interface apparaît :


Allez dans le menu Stockage


Puis cliquez sur “Ajouter un lecteur optique”


puis “Choisir un disque”

Cliquez sur “Ajouter”


Il ne vous reste plus qu'à trouver votre image ISO téléchargée.








Une fois ce résultat obtenu, la configuration de VirtualBox est terminée.



\section{Installation}


Une fois revenu sur le menu principal, sélectionnez votre machine et cliquez sur “Démarrer”


Après quelques instants, le logo de la distribution s’affiche suivi d’un menu gris :




Veuillez tout d’abord sélectionnez la langue d’installation : 


Puis cliquez sur “Installer Xubuntu”








Sélectionnez la langue de manière plus précise :



Pour la section “Mises à Jours”, sélectionnez les deux cases, surtout la deuxième car sans cette dernière, il vous faudra installer manuellement les pilotes et codecs multimédias (étape pénible…)
En revanche, si la première option n’est pas cochée, ce n’est pas grave car on pourra faire la mise à jour plus tard.
A partir de cette étape, le réseau internet n’est plus obligatoire.



Nous allons effacer le disque alloué (20 Go) afin de mettre Linux dessus.

Il faut ensuite valider ce choix…

Il ne vous reste plus qu'à renseigner vos informations personnelles.


Vous devez indiquer votre nom, votre identifiant et votre mot de passe.
Le nom de l’ordinateur importe peu.


Une fois validé, l’installation du noyau Linux peut débuter.


Après un petit temps, la fenêtre suivante s’affiche :

Faite “Redémarrer maintenant”.
Ensuite, à la vue de cette fenêtre, faites “Entrée”

La machine redémarre et affiche de nouveau la marque de la distribution (Xubuntu).

Vous pouvez maintenant vous connecter sous votre nom d’utilisateur
et avec votre mot de passe. Pour information, le pavé numérique n’est pas activé par défaut sous VirtualBox.


Une fois connecté, vous arrivez sur un menu semblable :


Le menu (équivalent Windows) se situe ici :



Pour éteindre la machine proprement, il suffit d’aller dans le menu et de cliquer sur le bouton 

Une fois éteinte, pour allumer la machine, il suffit de cliquer sur Démarrer



Ce tutoriel est désormais terminé, n’hésitez pas à faire des retours afin que je puisse l’améliorer et pour corriger les erreurs éventuelles.