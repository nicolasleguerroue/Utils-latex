 \chapter{Installation de Discord}
\section{Connexion au serveur}


Une fois le lien cliqué,vous tombez sur une interface similaire : 

\img{\rootImages/discord_welcome.png}{Accueil du site Discord}{0.2}

Ensuite, veuillez cliquer sur "\bold{Ouvrir dans votre navigateur}"

\img{\rootImages/discord_browser.png}{Lancement de Discord}{0.6}


Veuillez choisir un identifiant, qui sera votre nom visible par les membres de la session vocale, sans oublier de cocher la case "\bold{J'ai lu et accepte les Conditions Générales d'Utilisation}"

\img{\rootImages/discord_login.png}{Choix de votre identifiant Discord}{0.35}

Ensuite, le site va vous demander si vous êtes un robot. Cochez la case "\bold{Je ne suis pas un robot}" \footnote{Parfois, une sélection d'images diverses va vous être imposée.} \\
\img{\rootImages/discord_captcha.png}{Vérification Captcha}{0.5}

Votre date de naissance va vous être demandée.

\img{\rootImages/discord_birth.png}{Date de naissance}{0.5}



On va vous demander le type de serveur que vous voulez créer. Dans notre cas, sur la fenêtre qui s'affiche, veuillez cliquer sur \bold{"Rejoindre un serveur"}



\img{\rootImages/discord_join.png}{Rejoindre un serveur}{0.5}

\messageBox{Adresse du serveur}{orange}{white}{A ce moment, veuillez renseigner l'adresse du serveur : https://discord.gg/Fm97K3Se. Attention, ce lien est valable 24 h et est est actif uniquement le 21 novembre 2020. Lorsque vous essayez ce tutoriel à une autre date, veuillez envoyer un SMS au 06.28.88.75.12 en demandant une nouvelle invitation sur Discord. A partir du moment où vous recevrez le lien, vous aurez de nouveau 24 h pour vous connecter}{black}

\img{\rootImages/discord_enter.png}{Saisir l'adresse du serveur, celle dans l'encadré orange}{0.6}

Cliquer ensuite sur \bold{Rejoindre} \\


Une fenêtre avec un fond noir apparaît. Cette fenêtre demande à enregistrer votre compte. \\

\bold{Il faut renseigner une adresse mail et un mot de passe.}

\img{\rootImages/discord_account.png}{Saisir une adresse mail et un mot de passe}{0.6}



\bold{Veuillez noter précieusement l'adresse mail et le mot de passe utilisé. Cela vous permettra de vous connecter au serveur à tout moment.} \\

Une fenêtre de confirmation s'affiche. \\



\img{\rootImages/discord_email.png}{Confirmation de l'adresse mail}{0.5}

En cliquant sur le bouton \bold{Télécharger l'application de bureau}, un fichier va se télécharger.

Le format du fichier va dépendre du système d’exploitation.

\begin{items}{blue}{\Triangle}
	\item Pour Windows, un fichier executable (.exe) va se télécharger. 


	Il suffit de cliquer dessus pour lancer l'installation du logiciel.

	\item Pour Linux, un fichier compressé se télécharge. Il suffit de le décompresser, de se rendre dans le dossier décompressé puis dans le dossier Discord. \\

	\img{\rootImages/discord_zip.png}{Dossier compressé}{0.6}

	Vous trouvez normalement un fichier appelé Discord :

	\img{\rootImages/discord_down.png}{Emplacement de l'éxécutable}{0.6}

	Puis click-droit sur le fichier \bold{Discord > Permissions}  et cocher la case \bold{Autoriser l'éxécution du fichier comme un programme}

	\img{\rootImages/discord_droits.png}{Droit d'éxécution du fichier}{0.6}

	Puis click-gauche sur le fichier \bold{Discord} pour le lancer.
\end{items}

\subsection{Enregistrement du compte}

je vous invite à consulter votre messagerie pour recevoir le mail de confirmation. \\


Une fois sur votre messagerie, vous recevrez un message similaire.

Veuillez cliquer sur \bold{Vérifier l'adresse e-mail}

\img{\rootImages/discord_check.png}{Vérification de l'adresse mail}{0.6}

Un message de confirmation devrait apparaître 

\img{\rootImages/discord_link.png}{Confirmation de l'adresse mail}{0.5}

En cliquant sur \bold{Continuer vers Discord}, la page suivante s'affiche. Il s'agit de la page d'accueil avec votre compte permanent.\\

\bold{Nous n'avons plus besoin du navigateur internet. Vous pouvez le fermer.}


\bold{Dorénavant, pour se connecter, il suffira d'aller dans la barre de recherche de vos logiciels et de saisir "Discord"}. \\

\img{\rootImages/discord_s.png}{lancement du logiciel}{0.5}


La page suivante apparaît et vous permet de vous connecter avec vos identifiants.

\img{\rootImages/discord_new_login.png}{Connexion avec vos identifiants}{0.5}


\newpage


\section{Navigation et utilisation du serveur}

\subsection{Première vue}

Une fois que vous êtes connecté sur le serveur, voici l'interface que vous devez avoir : 

\img{\rootImages/discord_home_own.png}{Accueil du serveur}{0.3}


\subsection{Accéder à l'Atelier Arduino}

Pour faire partie de l'Atelier Arduino, il faut cliquer sur le bouton \bold{AA} dans le menu de gauche. \\

\img{\rootImages/discord_aa.png}{logo de l'Atelier Arduino non actif}{0.6}


Le logo "AA" en bleu vous indique que vous êtes bien dans le salon "Atelier Arduino"

\img{\rootImages/discord_workshop.png}{logo de l'Atelier Arduino actif}{0.6}

\subsection{Présentation}

Discord se décompose en trois parties. 

\begin{items}{blue}{\Triangle}
	\item L'accès aux salons du serveur [menu de gauche]

	\img{\rootImages/discord_salons.png}{Les types de salons}{0.6}

	Le serveur est composé de deux types de salons : 

	\begin{items}{blue}{\Bullet}
		\item Les salons textuels pour noter et faire partager des informations \bold{écrites}.
		\item Les salons vocaux pour discuter de \bold{vive voix}
	\end{items}


	\item Le contenu du salon
	\img{\rootImages/discord_content.png}{Le contenu du salon}{0.3}
	\item Les membres connectés
	\img{\rootImages/discord_members.png}{Les membres du salon}{0.6}
\end{items}



\subsection{Entrer dans un salon}

Pour pouvoir discuter avec les autres membres, il suffit de cliquer sur le bouton \bold{Salon} parmi les salons vocaux.

\img{\rootImages/discord_vocal.png}{L'accès à un salon vocal}{0.8}

Une fois rentré dans ce salon, votre identifiant apparaît, signifiant que vous pouvez discuter avec les autres membres si ces derniers sont dans le même salon.

\subsection{Activer votre microphone}

Par défaut, quand vous rentrez dans un salon vocal, le microphone est désactivé. Pour l'activer, il faut cliquer sur le petit microphone barré 

\img{\rootImages/discord_micro.png}{Activer le microphone}{0.8}



\subsection{Partager son écran}.

La première condition pour partager son écran est de rejoindre un salon vocal. \\

Pour partager son écran, il faut cliquer sur le bouton \bold{Ecran} en bas à gauche.

\img{\rootImages/discord_screen.png}{Le partage d'écran}{0.8}

A partir de cette fenêtre, vous pouvez choisir ce que vous voulez partager : 

	\begin{items}{blue}{\Triangle}
		\item Votre écran courant
		\item Une application en particulier (et seulement cette application)
	\end{items}

	Vous cliquez sur l'image correspondante à votre choix puis \bold{Partager}

\img{\rootImages/discord_share.png}{Le partage d'écran - validation}{0.4}

\section{Accéder aux paramètres}

Les paramètres de microphone, des écouteurs (ou casque) et de divers éléments sont accessibles avec le bouton en forme d'engrenage en bas à gauche, à coté de votre identifant.

\img{\rootImages/discord_setup.png}{Accès aux paramètres}{0.8}

Les paramètres pour le matériel audio est disponible à la section suivante (page de paramètres)

\img{\rootImages/discord_setup_song.png}{Accès aux paramètres sonores}{0.8}


\section{Déconnexion du serveur}

Pour quitter le serveur proprement, il suffit de se rendre sur la page \bold{Paramètres} comme indiqué précédemment et de se rendre en bas de la page pour cliquer sur \bold{{\color{red}Déconnexion}}.

\img{\rootImages/discord_logout.png}{Déconnexion du serveur}{0.8}
\newpage
\addcontentsline{toc}{section}{Table des figures}
