\chapter{Cahier des Charges}
\newcommand{\constrain}[1]{\lbl{blue}{CON}{#1}}

Ce logiciel permettra de gérer efficacement les composants électroniques.

\begin{items}{blue}{\Triangle}
    \item Une bibliothèque commune de composants
    \item Une bibliothèque personnelles
\end{items}


\begin{items}{green}{\Bullet}
    \item Un identifiant unique \\
    \constrain{PRIMARY KEY}
    \item Une référence unique \\
    \constrain{ALPHANUM} \constrain{UPPERCASE}
    \item Une désignation 
    \item Une catégorie
    \item Une sous-catégorie
    \item Un boîtier
    \item Une plage d'alimentation
    \item Une documentation technique
    \item Une image
\end{items}

Les images et documents ne seront pas stockés au format binaire dans la base de données, cela prend trop de place et ralentis considérablement le temps d'accès aux données.



\section{Cas d'usages}
\begin{items}{blue}{\Triangle}

    \item L'utilisateur souhaite ajouter un composant non existant dans la bibliothèque
    \item L'utilisateur souhaite ajouter un composant dans sa base de données personnelle.
    \begin{items}{blue}{\Bullet}
        \item Le composant existe, il l'ajoute dans son espace personnel.
        \item Le composant n'existe pas dans la bibliothèque, il doit l'ajouter manuellement dans la bibliothèque puis l'ajouter dans son espace personnel.
    \end{items}
    \item L'utilisateur souhaite consulter un composant dans la bibliothèque
    \item L'utilisateur souhaite consulter un composant dans son espace personnel
\end{items}