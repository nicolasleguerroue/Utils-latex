%\addPartText{Théorie sur les servo-moteur et applications pratiques avec Arduino}
\partImg{Les servomoteur}{\rootImages/img.jpg}{0.4}
\chapter{Présentation}

\newcommand{\servo}{servo-moteurs } 

\index{Servo-moteurs}

Les \servo sont utilisés lorsqu'on souhaite un asservissement en position d'un axe de rotation.



\section{Asservissement}

Un asservissement est un processus de correction pour maintenir une consigne. \\
Par exemple, un régulateur de vitesse dans une voiture est un système asservi car la vitesse 
doit être constante quelle que soit la pente.


\section{Architecture}

\img{\rootImages/servoInput.png}{Constitution d'un servo-moteur}{1.6}

\section{Domaines d'application}

\begin{items}{blue}{\Bullet}
    \item Modélisme
    \item Robotique
\end{items}

\section{Commande des \servo}

Les \servo ont besoins d'être contrôlés via un signal \glossary{PWM} ou \glossary{MLI}. %\footnote{Pulse Width Modulation : Modulation par Largeur d'Impulsion \index{PWM}}

\subsection{Principe de la PWM}

La PWM est la création d'un signal numérique dont le temps à l'état haut est variable.\\
On fait varier le rapport cyclique (appelé $r$) qui est compris entre 0 et 1.

$$ r = \frac{T_{on}}{T_{signal}} $$


\img{\rootImages/PWM.png}{Différents rapports cycliques}{0.5}

\subsection{Trame de commande servomoteurs}

Pour indiquer une consigne de position, il faut faire varier le rapport cyclique 
du signal envoyé au servomoteur.

La période des signaux est de 20 ms et le temps à l'état Haut du signal représente la position : 

\begin{items}{blue}{\Triangle}
\item 1 ms : 180$^\circ$
\item 1.5 ms : 90$^\circ$
\item 2 ms : 0$^\circ$
\end{items}

\img{\rootImages/TRAME.png}{Une trame PWM}{0.5}

\section{Branchement d'un servo-moteur}

\begin{items}{blue}{\Bullet}
  \item Câble noir ou marron : GND 
  \item Câble rouge : +5V
  \item Câble blanc ou jaune : Signal Arduino (9)
\end{items}


\img{\rootImages/pinout.png}{Branchement d'un servomoteur}{0.3}

\subsection{Code Arduino}


\begin{Cpp}{Variation de la position d'un \servo}

  #include <Servo.h>      //Inclusion de la bibliothèque Servo
  Servo myservo;  // Création d'un objet Servo
  int pos = 0;    //Angle du servomoteur
  
  void setup() {

    myservo.attach(9);  //Choix de la broche du servo moteur

  }
  
  void loop() {

    for (pos = 0; pos <= 180; pos += 1) { //Parcours la plage angulaire [0-180] degré par degré

      myservo.write(pos);              //Actualise la position 
      delay(15);                       //Attend 15 ms avant l'actualisation

    }//Fin for

    for (pos = 180; pos >= 0; pos -= 1) {     //Parcours la plage angulaire [0-180] degré par degré

      myservo.write(pos);              //Actualise la position 
      delay(15);                       //Attend 15 ms avant l'actualisation

    }//Fin for
  }//Fin loop

\end{Cpp}


\section{Caractéristiques}


\subsection{Electriques}


\begin{items}{blue}{\Bullet}
  \item Tension de commande et d'alimentation : \~5V
\end{items}

\subsection{Mécaniques}

\begin{items}{blue}{\Bullet}
  \item Couple de sortie (Nm)
  \item Vitesse de rotation (temps pour parcourir 60°)
\end{items}




\section{Pour aller plus loin}

\subsection{Une autre application de la PWM}

En faisant varier la tension de sortie dans le temps rapidement ($>$50Hz), on peut simuler une 
tension analogique.\\

Voici un code d'exemple pour faire varier la luminosité d'une LED.

\begin{Cpp}{Variation de la luminosité d'une LED}

  const int pin_led = 11; //Selection d'une broche PWM

  float duty_cyle[11] = {0, 0.1, 0.2, 0.3, 0.4, 0.5, 0.6, 0.7, 0.8, 0.9, 1.0};//Création d'un tableau avec les différents 
  rapports cycliques
  
  void setup() {
  
      pinMode(pin_led, OUTPUT);  //Mise en sortie de la broche LED
  
  }//Fin setup
  
  void loop() {
  
      for(int i=0;i<11;i++) 
      {
          int value_r = duty_cyle[i]*255.0; //Conversion d'une valeur entre 0 et 1 en une valeur entre 0 et 255
          analogWrite(pin_led, value_r); //Change le rapport cyclique pendant 3 s
          delay(600);        //Attend 0.6s
      }
      
  
  }//Fin loop

\end{Cpp}


%https://howtomechatronics.com/how-it-works/how-servo-motors-work-how-to-control-servos-using-arduino/