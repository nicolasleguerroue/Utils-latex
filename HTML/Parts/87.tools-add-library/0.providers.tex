\chapter{Installations de bibliothèques}

  Lors de nouveaux projets, certaines bibliothèques peuvent être manquantes.\\
  On s'en aperçoit quand on clique sur le bouton de vérification (bouton tout à gauche) du code :


  \img{\rootImages/verifier.png}{Bouton de vérification}{0.5}

  \img{\rootImages/dht.png}{La bibliothèque DHT manquante}{0.5}
  Une erreur de ce type nous indique que la bibliothèque \lib{DHT} est manquante.\\

  
  Il existe deux façons d'installer des bibliothèques Arduino.

  \section{Ajout via le gestionnaire de bibliothèques}

  Tout d'abord, veuillez vous rendre dans le menu \bold{Outils - Gérer les bibliothèques}.

  \img{\rootImages/lib-handler.png}{Le gestionnaire de bibliothèques}{0.6}

  Vous tombez sur une interface similaire :

  \img{\rootImages/lib-all.png}{Les bibliothèques existantes}{0.5}

  Cette page affiche toutes les bibliothèques disponibles via le gestionnaire de bibliothèques.\\
  Dans la barre de saisie en haut à droite, il faut indiquer le nom de la bibliothèque désirée. \\
  Prenons par exemple la bibliothèque DHT :

  \img{\rootImages/lib-install.png}{Ajouter une bibliothèque DHT}{0.5}

  Il faut parcourir la liste et trouver la bibliothèque \lib{DHT sensor library} de chez \bold{Adafruit}.
  Il ne reste plus qu'à sélectionner la version puis cliquer sur \shortcut{Installer}.

  Pour que les changements soit pris en compte, il faut redémarrer l'IDE \glossary{IDE} Arduino.


  \section{Ajout via un fichier ZIP}

  Cette méthode est un peu plus longue mais parfois, pour certaines bibliothèques non gérées par le gestionnaire de bibliothèques, nous n'avons pas le choix.\\

  En premier lieu, il faut trouver la bibliothèque sur Internet. Par exemple, la bibliothèque \lib{DHT} de chez \bold{Adafruit} est disponible à l'adresse suivante : 
  \url{https://github.com/adafruit/DHT-sensor-library}

  Il ne reste plus qu'à cliquer sur \shortcut{Code - Download ZIP} et le dossier compressé va se placer dans vos téléchargements.\\

  \img{\rootImages/git.png}{Téléchargement de la bibliothèque DHT}{0.4}


  Enfin, pour installer la bibliothèque, il suffit d'aller dans \bold{Croquis - Inclure une bibliothèque - Ajouter la bibliothèque .ZIP}
  
  \img{\rootImages/place.png}{Ajout d'une bibliothèque}{0.5}
  
  Il ne reste qu'à trouver le fichier \bold{DHT\_sensor-master.zip} et à faire \bold{OK}
  