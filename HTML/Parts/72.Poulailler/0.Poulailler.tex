\chapter{Poulailler}

\section{Calculs des condensateurs}

\chapter{Cahier des Charges}

\section{Contrainte de contrôle}

Il y a différentes solutions pour activer la séquence du poulailler 
Voici les avantages et les inconvénients :

\begin{items}{blue}{\Triangle}
    \item Contrôle par télécommande infrarouge
    \begin{items}{green}{\Bullet}
        \item Système assez fiable
    \end{items}
    \begin{items}{red}{\Bullet}
        \item Présence obligatoire
        \item Distance faible
    \end{items}
    \item Contrôle par smartphone
    \begin{items}{green}{\Bullet}
        \item Système ergonomique
    \end{items}
    \begin{items}{red}{\Bullet}
        \item Présence obligatoire
        \item Distance faible
    \end{items}
    \item Contrôle par la lumière ambiante
    \begin{items}{green}{\Bullet}
        \item Système autonome
    \end{items}
    \begin{items}{red}{\Bullet}
        \item Moins souple
    \end{items}
\end{items}


Le système par la lumière ambiante permet donc d'avoir un système autonome sans caliberer une horloge pour régler le lever et le coucher des poules.

\section{Contrainte d'ouverture/fermeture}

Un système de herse mobile sera utilisé pour ouvrir et fermer le poulailler.
Un moteur entraînera une poulie afin d'effectuer la séquence.

\section{Contrainte d'autonomie}


Il y a différentes solutions pour alimenter le poulailler
Voici les avantages et les inconvénients :


\begin{items}{blue}{\Triangle}
    \item Alimentation par prise secteur
    \begin{items}{green}{\Bullet}
        \item Système rapide à mettre en place
    \end{items}
    \begin{items}{red}{\Bullet}
        \item Peu pratique
        \item Dépend du courant dans la maison
    \end{items}
    \item Alimentation par batteries
    \begin{items}{green}{\Bullet}
        \item Autonomie
    \end{items}
    \begin{items}{red}{\Bullet}
        \item Système de recharge 
    \end{items}
\end{items}

L'alimentation va reposer sur des supercondensateurs qui vont jouer le rôle de réservoir d'energie.
En journée, les supercondensateurs vont se charger et à la tombée de la nuit, une partie de leur énergie ira au moteur pour fermer la porte.
Ayant stocké assez d’énergie, il restera suffisamment d’énergie au lever du jour pour effectuer une ouverture de la porte.

