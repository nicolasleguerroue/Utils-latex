% \iffalse
% makeindex -s gglo.ist -o eforms.gls eforms.glo
% makeindex -s gind.ist -o eforms.ind eforms.idx
%<*copyright>
%%%%%%%%%%%%%%%%%%%%%%%%%%%%%%%%%%%%%%%%%%%%%%%%%%%%%%%%%%
%% eForms package,                                      %%
%% Copyright (C) 2002-2021  D. P. Story                 %%
%%   dpstory@uakron.edu                                 %%
%%                                                      %%
%% This program can redistributed and/or modified under %%
%% the terms of the LaTeX Project Public License        %%
%% Distributed from CTAN archives in directory          %%
%% macros/latex/base/lppl.txt; either version 1 of the  %%
%% License, or (at your option) any later version.      %%
%%%%%%%%%%%%%%%%%%%%%%%%%%%%%%%%%%%%%%%%%%%%%%%%%%%%%%%%%%
%</copyright>
%<package>\NeedsTeXFormat{LaTeX2e}
%<package>\ProvidesPackage{eforms}
%<package> [2021/05/10 v2.4.3 Provides general eforms support (dps)]
%<*driver>
\documentclass{ltxdoc}
\usepackage[colorlinks,hyperindex=false]{hyperref}[2012/10/12]
\usepackage{fancyvrb}
\pdfstringdefDisableCommands{\let\\\textbackslash}
\OnlyDescription
\EnableCrossrefs
\CodelineIndex
\RecordChanges
\bgroup\ttfamily
\gdef\brpr#1{\char123\relax#1\char125\relax}\egroup
\let\darg\brpr
\let\env\texttt
\let\opt\texttt
\let\app\textsf
\let\key\texttt
\let\uif\textsf
\def\visispace{\symbol{32}}
\def\ameta#1{\ensuremath{\langle\textit{\texttt{#1}}\rangle}}
\def\meta#1{\textsl{\texttt{#1}}}
\def\SUB#1{\ensuremath{{}_{\mbox{\scriptsize\ttfamily#1}}}}
\def\ltag{<}\def\rtag{>}
\let\app\textsf\let\pkg\textsf
\let\tops\texorpdfstring
\makeatletter
\let\@latex@warning\@gobble
\makeatother
\InputIfFileExists{aebdocfmt.def}{\PackageInfo{eforms}{Inputting aebdocfmt.def}}
    {\def\IndexOpt{\DescribeMacro}\def\IndexKey{\DescribeMacro}\let\setupFullwidth\relax
     \PackageInfo{eforms}{aebdocfmt.def cannot be found}}
\begin{document}
  \GetFileInfo{eforms.sty}
  \title{\textsf{eforms}: PDF Form support for \LaTeX}
  \author{D. P. Story\\
    Email: \texttt{dpstory@uakron.edu}}
  \date{processed \today}
  \maketitle
  \tableofcontents
  \let\Email\texttt
  \DocInput{eforms.dtx}
\IfFileExists{\jobname.ind}{\newpage\setupFullwidth\PrintIndex}{\paragraph*{Index} The index goes here.\\Execute
    \texttt{makeindex -s gind.ist -o eforms.ind eforms.idx} on the command line and recompile
    \texttt{eforms.dtx}.}
\IfFileExists{\jobname.gls}{\PrintChanges}{\paragraph*{Change History} The list of changes goes here.\\Execute
    \texttt{makeindex -s gglo.ist -o eforms.gls eforms.glo} on the command line and recompile
    \texttt{eforms.dtx}.}
\end{document}
%</driver>
% \fi
%
% \MakeShortVerb{|}
% \InputIfFileExists{aebdonotindex.def}{\PackageInfo{eforms}{Inputting aebdonotindex.def}}
%    {\PackageInfo{eforms}{aebdonotindex.def cannot be found}}
%
%    \begin{macrocode}
%<*package>
%    \end{macrocode}
%
%    \section{Introduction}
%
% \textbf{Form Response Files.}
% The \pkg{eforms} package supports the creation of \app{Acrobat} \hyperref[formsSupport]{form fields},
% as well as the creation of \hyperref[linkSupport]{link annotations}. Most all features of fields and links are supported, though
% for some features, such as signatures and tabbing, the \app{Acrobat/\allowbreak Distiller} pair is needed.
% Inserting PDF form fields into your document creates a \emph{Form Response File}, to use the latest
% nomenclature of \app{Adobe}.
%
%    \section{Package Options}
%
% The package options consist mostly of driver options.
%
%    \subsection{Driver Options}
%
%    The \textsf{web} package passes these driver options to \textsf{exerquiz}.
%    These options are needed is \textsf{exerquiz} is used without
%    \textsf{web}; in this case, the options below must explicitly included.
%    Set the driver dependent code for the |quiz| environments.
%
% \changes{v2.4.3}{2021/05/10}{Changed date for sync with \string\pkg{insdljs}}
% \changes{v2.2v2.2}{2019/06/12}{Added fields to fields dict, conforming to \string\pkg{hyperref}}
% \changes{v2.13}{2019/06/10}{Added various fields to the fields dictionary so they
% can get appearances in non-conforming PDF viewers.}
% \changes{v2.5o}{2012/06/18}{Added required package \string\pkg{ifpdf}}
% \changes{v2.9m}{2017/09/03}{Added the \string\pkg{ifluatex} package}
%    \begin{macrocode}
\RequirePackage{ifpdf}[2006/02/20]
\RequirePackage{ifxetex}[2006/08/21]
\RequirePackage{ifluatex}
%    \end{macrocode}
% The \pkg{calc} package is now required (2014/02/18).
% \changes{v2.6c}{2014/02/18}{Added the \string\pkg{calc} package as required package.}
%    \begin{macrocode}
\RequirePackage{calc}
%    \end{macrocode}
% Set the driver for \texttt{dvipsone}\IndexOpt{dvipsone}
%    \begin{macrocode}
\let\ef@driver\@empty
\DeclareOption{dvipsone}{\def\eq@drivernum{0}%
  \def\eq@drivername{0}\def\ef@driver{dvipsone}%
  \def\eq@drivercode{epdfmark.def}%
  \PassOptionsToPackage{dvipsone}{insdljs}%
  \PassOptionsToPackage{dvipsone}{hyperref}%
}
\def\eq@drivername{2}
%    \end{macrocode}
% Set the driver for \texttt{dvips}\IndexOpt{dvips}
%    \begin{macrocode}
\DeclareOption{dvips}{\def\eq@drivernum{0}%
  \def\eq@drivername{1}\def\ef@driver{dvips}%
  \def\eq@drivercode{epdfmark.def}%
  \PassOptionsToPackage{dvips}{insdljs}%
  \PassOptionsToPackage{dvips}{hyperref}%
}
%    \end{macrocode}
% Set the driver for \texttt{pdftex}\IndexOpt{pdftex}
% \changes{v2.9b}{2016/07/22}{Do not pass pdftex driver to \string\pkg{insdljs} or \string\pkg{hyperref}}
%    \begin{macrocode}
\DeclareOption{pdftex}{%
  \def\eq@drivernum{1}\def\eq@driver{pdftex}%
  \def\eq@drivercode{epdftex.def}\def\ef@driver{pdftex}%
}
%    \end{macrocode}
% Added \textsf{luatex} option
% \changes{v2.9m}{2017/09/03}{Add \string\opt{luatex} option}
%    \begin{macrocode}
\DeclareOption{luatex}{%
  \def\eq@drivernum{1}\def\eq@driver{luatex}%
  \def\eq@drivercode{epdftex.def}\def\ef@driver{luatex}%
}
%    \end{macrocode}%
% Set the drivers for \texttt{dvipdfm}\IndexOpt{dvipdfm}. \texttt{dvipdfmx}\IndexOpt{dvipdfmx},
% and \texttt{xetex}\IndexOpt{xetex}.
% \changes{v2.9b}{2016/07/22}{Do not pass xetex driver to insdljs or hyperref}
%    \begin{macrocode}
\DeclareOption{dvipdfm}{%
  \def\eq@drivernum{2}\def\eq@driver{dvipdfm}%
  \def\eq@drivercode{edvipdfm.def}\def\ef@driver{dvipdfm}%
  \PassOptionsToPackage{dvipdfm}{insdljs}
  \PassOptionsToPackage{dvipdfm}{hyperref}
}
\DeclareOption{dvipdfmx}{%
  \def\eq@drivernum{2}\def\eq@driver{dvipdfmx}%
  \def\eq@drivercode{edvipdfm.def}\def\ef@driver{dvipdfmx}%
  \PassOptionsToPackage{dvipdfmx}{insdljs}
  \PassOptionsToPackage{dvipdfmx}{hyperref}
}
\DeclareOption{xetex}{%
  \def\eq@drivernum{2}\def\eq@driver{xetex}%
  \def\eq@drivercode{edvipdfm.def}\def\ef@driver{xetex}%
}
%    \end{macrocode}
%    \leavevmode\IndexOpt{textures}^^A%
% This option, and testing are due to Ross Moore  3/6/02
%    \begin{macrocode}
\DeclareOption{textures}{%
  \def\eq@drivernum{3}\def\eq@driver{textures}%
  \def\eq@drivercode{epdfmark.def}\def\ef@driver{textures}%
  \PassOptionsToPackage{textures}{insdljs}%
  \PassOptionsToPackage{textures}{hyperref}
}
%    \end{macrocode}
% If no driver is passed, assume it is
% \textsf{dvipsone} or \textsf{dvips}---\textsf{hyperref} defines the specials.
% Default driver dvipsone/dvips
%    \begin{macrocode}
\def\eq@driver{dvipsone/dvips}
\def\eq@drivercode{epdfmark.def}
%    \end{macrocode}
%
% \subsection{Other Options}
%
% The \texttt{preview}\IndexOpt{preview} option displays the bounding box of each form field with a frame box.
% Useful for laying out field with a dvi previewer.
% \changes{v2.5n}{2012/04/30}{Changed \string\cs{ifpreview} to a conditional
% definition; this allows the \string\pkg{spdef} package to control the state
% of \cs{ifpreview}.}
%    \begin{macrocode}
\DeclareOption{preview}{\previewtrue}
\let\insjs@opts\@empty
\DeclareOption*{\PassOptionsToPackage{\CurrentOption}{insdljs}}
\@ifundefined{ifpreview}{\newif\ifpreview\previewfalse}{}
%    \end{macrocode}
%    (2017/01/01) Added two convenience commands, these are \DescribeMacro\previewOn\cs{previewOn}
%    and \DescribeMacro\previewOff\cs{previewOff}. Beginning with 2019/05/24, `enhanced preview'
%    is introduced, see comments \hyperref[pmpv]{below}.
%    \changes{v2.9f}{2017/01/01}{Added \string\cs{previewOn} and \string\cs{previewOff}}
%    \changes{v2.11}{2019/05/24}{Introduce `enhanced preview' targeted at users of non-conforming
%    PDF viewer.}
%    \begin{macrocode}
\providecommand{\previewOn}{\previewtrue}
\providecommand{\previewOff}{\previewfalse}
%    \end{macrocode}
%    Use\IndexOpt{useui} the \textsf{xkeyval} package to specify the options for the links and
% forms, key-values are enclosed in the \cs{ui} command inside the option list.
% \changes{v2.0}{2008/03/14}
% {%
%   Added the \string\texttt{useui} option, which inputs \string\textsf{xkeyval} package,
%   and defines a user-friendly interface to the option arguments.
% }
%    \begin{macrocode}
\DeclareOption{useui}{\AtEndOfPackage{\ef@InputUIeForms}}
\def\ef@InputUIeForms{\InputIfFileExists{uieforms.def}%
  {\PackageInfo{eforms}{Inputting code for useui option}}%
  {\PackageInfo{eforms}{Cannot find uieforms.def.\MessageBreak
    Reinstall or refresh your file name database.}}}
%    \end{macrocode}
%    The \texttt{setcorder} option\IndexOpt{setcorder} is used to set the calculation order
%    in a forms document. Normally, the calculation order is the order the fields are created.
%    \begin{macrocode}
\DeclareOption{setcorder}{\def\inputCalcOrderJS{%
  \InputIfFileExists{setcorder.def}%
  {\PackageInfo{eforms}{Inputting code for setcorder option}}%
  {\PackageWarning{eforms}{Cannot find setcorder.def.\MessageBreak
    Reinstall or refresh your file name database.}}}}
\let\inputCalcOrderJS\relax
%    \end{macrocode}
% \changes{v2.5o}{2012/06/18}{Added automatic test for \string\textsf{pdftex}.}
% \changes{v2.9m}{2017/09/03}{Added a check for \string\textsf{lualtex}}
%    \begin{macrocode}
\@ifpackageloaded{web}{%
    \ExecuteOptions{\eq@driver@name}%
}{%
    \@ifpackageloaded{exerquiz}{%
        \ExecuteOptions{\eq@driver}%
    }{%
        \ifluatex\ExecuteOptions{luatex}\else
        \ifpdf\ExecuteOptions{pdftex}\else
        \ifxetex\ExecuteOptions{xetex}\else
        \@ifundefined{l@tex@@@@driver}{\ExecuteOptions{dvips}}
            {\ExecuteOptions{dvipsone}}\fi\fi\fi
    }%
}
%    \end{macrocode}
%    \begin{macrocode}`
\ProcessOptions
\ifx\ef@driver\@empty
  \PackageError{eforms}%
    {You have not specified dvips, dvipsone, pdftex,\MessageBreak
    dvipdfm, dvipdfmx, or xetex in the option list of\MessageBreak
    the eforms package}
    {Place one of the drivers dvips, dvipsone, pdftex,  dvipdfm,
    dvipdfmx, or xetex
    \MessageBreak in the option list of the eforms package.}%
}{}
\fi
\edef\ef@CatChngs{\the\catcode`\$}
\@makeother\$\relax
\newlength\eflength
%    \end{macrocode}
%    To support a variations of \app{dvipdfm}, we change the logic from
%    \cs{ifxetex} to |\ifnum\eq@drivernum=2\relax|, a driver number of 2
%    includes \app{dvipdfm} and all its variants.
%    \changes{v2.3.4}{2020/07/05}{Change logic for \string\cs{ifxetex}}
%    \begin{macrocode}
\@ifundefined{ifpdfmarkup}{\newif\ifpdfmarkup}{}\pdfmarkupfalse
\ifpdf\else\ifnum\eq@drivernum=\tw@\else\pdfmarkuptrue\fi\fi
\RequirePackage{hyperref}
%    \end{macrocode}
%    Changed \textsf{hyperref} definitions/parameters to add \texttt{1bp}
%    rather than \texttt{1pt} around a form field or link.
%    \changes{v2.9e}{2016/12/22}{Changed hyperref to add 1bp rather than 1pt}
%    \begin{macrocode}
\ifnum\eq@drivernum=2\relax
\else
  \ifpdf\pdflinkmargin1bp\relax
  \else
    \g@addto@macro\Hy@FirstPageHook{%
    \headerps@out{/HyperBorder {1.00375 PDFToDvips} def}}
  \fi
\fi
%    \end{macrocode}
%    Changed the requirement for \textsf{insdljs}
% \changes{v2.9a}{2016/06/09}{Change in \string\textsf{insdljs} package}
% \changes{v2.9g}{2017/01/03}{Change in \string\textsf{insdljs} package}
%    \begin{macrocode}
\RequirePackage{insdljs}[2019/02/11] % incl conv-xkv
%    \end{macrocode}
% \changes{v1.0a}{2006/10/03}
% {%
%   If \string\pkg{exerquiz} is not loaded, we do an automatic begin and end of Form.
%   Also, if \string\textsf{exerquiz} is loaded, then we use the driver declared in
%   \string\pkg{exerquiz}; otherwise, we set the default to 0
%   (\string\texttt{dvipsone}/\string\texttt{dvips}).
% }
% If \pkg{exerquiz} is not loaded, we insert |\begin{Form}| and |\end{Form}|,
% and if undefined, we set the default driver.
%    \begin{macrocode}
\@ifpackageloaded{exerquiz}{}{%
  \AtBeginDocument{\Form}%
  \AtEndDocument{\@nameuse{endForm}}
}
\@ifpackageloaded{aeb_pro}{}{%
  \newcommand{\taggedPDF}{%
    \ifnum\eq@drivernum=0\relax
    \literalps@out{\AEB@psMrk{Catalog} <<%
        /MarkInfo<</Marked true>>%
    >> /PUT pdfmark}\fi}%
}
\let\ef@YES=y \let\ef@NO=n
\let\ef@One=1 \let\ef@Zero=0
\ifnum\eq@drivername<2\relax
  \let\to@usepdfmark\ef@One
  \def\reqPkg{\RequirePackage[structure]{taborder}}\else
  \let\to@usepdfmark\ef@Zero
  \def\reqPkg{\RequirePackage{taborder}}\fi
\reqPkg
%    \end{macrocode}
%    \begin{macrocode}
%</package>
%    \end{macrocode}
% When the \texttt{preview} option is used, draw a frame box
% around the \textit{inner} bounding rectangle.
%
% \paragraph*{Utility commands, boxes, dimension and token registers}
%    \begin{macrocode}
%<*package>
\def\csarg#1#2{\expandafter#1\csname#2\endcsname}
\@ifundefined{eq@tmpbox}{\newsavebox{\eq@tmpbox}}{}
\@ifundefined{eq@tmpdima}{\newdimen\eq@tmpdima}{}
\@ifundefined{eq@tmpdimb}{\newdimen\eq@tmpdimb}{}
%    \end{macrocode}
% Define new dimension of \cs{b@} which is assigned a value of 1bp.
% \changes{v2.3.6}{2020/10/10}{Defined new dimension \string\cs{b@} (1bp)}
%    \begin{macrocode}
\newdimen\b@ \b@=1bp
\newlength\ef@dimena
\newtoks\ef@scratchtoks
%    \end{macrocode}
%    (2016/12/22) Added switches \cs{ifmakeXasPD} and \cs{ifmakePDasX}, if true, the form fields created
%    by \app{xetex} (\app{pdflatex/Distiller})
%    are adjusted in dimension to conform to the fields produced by \app{pdftex}/\app{Distiller} (\app{xetex}).
%    Four convenience commands are defined, \DescribeMacro{\makeXasPDOn}\cs{makeXasPDOn},
%    \DescribeMacro{\makeXasPDOff}\cs{makeXasPDOff}, \DescribeMacro{\makePDasXDOn}\cs{makePDasXDOn}, and
%    \DescribeMacro{\makePDasXDOff}\cs{makePDasXDOff} to set the switch to true and false, respectively.
%    \changes{v2.9d}{2016/12/22}{Added switches \string\cs{ifmakeXasPD} and \string\cs{ifmakePDasX}}
%    \begin{macrocode}
\newif\ifmakeXasPD \makeXasPDtrue
\newif\ifmakePDasX \makePDasXfalse
\def\makeXasPDOn{\makeXasPDtrue\makePDasXfalse}
\def\makeXasPDOff{\makeXasPDfalse}
\def\makePDasXOn{\makePDasXtrue\makeXasPDfalse}
\def\makePDasXOff{\makePDasXfalse}
%    \end{macrocode}
% \changes{v2.8a}{2015/06/06}{Added \string\cs{previewColor}}
% \DescribeMacro\previewColor sets the color of the preview bounding rectangle. The
% default is black. Used mostly by the \textsf{eqexam} package with the \texttt{online}
% option.
% \changes{v2.9o}{2017/10/10}{Removed \string\cs{previewColor} in favor of
% \string\cs{ckboxColor}}
%    \begin{macrocode}
\providecommand\ckboxColor[1]{\def\@rgi{#1}\ifx\@rgi\@empty
  \let\ckbox@Color\relax\else
  \def\ckbox@Color{\color{#1}}\fi}\let\ckbox@Color\relax
%    \end{macrocode}
%   \leavevmode
%   \DescribeMacro\ef@Bbox\hskip-\marginparsep\texttt{\darg{\ameta{width}}\darg{\ameta{height}}}
%    places a rule of width
%    \DescribeMacro\efPreviewOnRule\cs{efPreviewOnRule}. When the specified
%    dimensions of the bounding rectangle is less than \texttt{2\cs{fboxrule}},
%    we \emph{do not} reduce the dimensions; otherwise, we reduce the width and height
%    by \texttt{2\cs{fboxrule}}, this is an attempt to have the preview dimensions
%    to correspond to the dimensions of the form field. Following this definition
%    \cs{ef@Bbox} is let to \cs{Bbox}.
%    \begin{macrocode}
\def\efPreviewOnRule{0.4pt}
%    \end{macrocode}
%    \leavevmode
%    \DescribeMacro\PMPV\hskip-\marginparsep\texttt{\darg{\ameta{text}}} provides \ameta{text}
%    as preview text for a field. Its argument is used to define  the text macro \cs{@PMPV}
%\changes{v2.11}{2019/05/24}{Define \string\cs{PMPV} to enhance preview}
%    \begin{macrocode}
\def\PMPV#1{\def\@rgi{#1}\ifx\@rgi\@empty
  \let\@PMPV\@empty\else\edef\@PMPV{\noexpand
  \hb@xt@\noexpand\z@{\hss#1\hss}}\fi}
\let\@PMPV\@empty
\let\pmpvFmt\@empty
\let\pmpvFmtCtrl\relax
%    \end{macrocode}
%    The preview bounding box, visible when \cs{previewOn} is expanded. Incorporated into the center
%    of the box is \cs{pmpvFmt\darg{\cs{@PMPV}}}. \cs{pmpvFmt} can be used to format its argument (\cs{@PMPV});
%    while \cs{@PMPV} is a text macro defined by \cs{PMPV} that displays, normally, nothing, or \cs{eq@V} or \cs{eq@CA}.
%    \begin{macrocode}
\def\ef@Bbox#1#2{\hbox{\ifpreview
  \setlength\fboxrule{\efPreviewOnRule}\setlength\fboxsep{0pt}%
  \@tempdima=#1\relax
  \ifdim\@tempdima<2\fboxrule\else
    \advance\@tempdima by-2\fboxrule\fi
  \@tempdimb=#2\relax
  \ifdim\@tempdimb<2\fboxrule\else
    \advance\@tempdimb by-2\fboxrule\fi
  \ckbox@Color\fbox{\parbox[b][\@tempdimb][c]{\@tempdima}%
    {\vfil\hfil\pmpvFmtCtrl\pmpvFmt{\@PMPV}\hfil\vfil}}\else
  \parbox[b][#2][c]{#1}{\vfil\hfil\hfil\vfil}\fi}%
}
%    \end{macrocode}
%   The public version of \cs{ef@Bbox} is simply \DescribeMacro\Bbox\cs{Bbox}.
%    \begin{macrocode}
\let\Bbox\ef@Bbox
%    \end{macrocode}
% The following two commands were created as a convenience to changing the definition of certain
% JavaScript actions. Use \cs{efsave} to save the original definition of the action, then redefine the action,
% the later, restore the JS action to its original definition.\par\medskip\noindent
% \DescribeMacro{\efsave}\hskip-\marginparsep\texttt{\meta{\cs{cmd1}}\cs{as}\meta{\cs{cmd2}}}
% \cs{let}s \meta{\cs{cmd2}} to \meta{\cs{cmd1}} (that is, saves \meta{\cs{cmd1}} `as' \meta{\cs{cmd2}}).
% Arguments may contain \texttt{@}.
% \changes{v2.3.1}{2019/10/14}{Define \string\cs{efsave} and \string\cs{efrestore}}
%    \begin{macrocode}
\def\efsave{\bgroup\let\as\relax\makeatletter\efsavei}
\def\efsavei#1\as#2{\global\let#2#1\egroup}
%    \end{macrocode}
% \leavevmode\DescribeMacro{\efrestore}\hskip-\marginparsep\texttt{\meta{\cs{cmd1}}\cs{from}\meta{\cs{cmd2}}}
% \cs{let}s \meta{\cs{cmd1}} to \meta{\cs{cmd2}} (that is, restores \meta{\cs{cmd1}} `from' \meta{\cs{cmd2}}).
% Arguments may contain \texttt{@}.
%    \begin{macrocode}
\def\efrestore{\bgroup\let\from\relax\makeatletter\efrestorei}
\def\efrestorei#1\from#2{\global\let#1#2\egroup}
%    \end{macrocode}
% For example, expanding |\efsave\x\as\y| (eg, |\let\y\x|), the definition of the macro |\y|
% has the same definition as |\x| does. Now |\x| is free to be redefined. Later in the document,
% we can say |\efrestore\x\from\y| (eg, |\let\x\y|). Now the
% definition of |\x| has its original definition (at the time when it was saved).\medskip
%
% \noindent The two commands \DescribeMacro\txtRef\cs{txtRef} and \DescribeMacro\labelRef\cs{labelRef}
% return pure text: The first returns the value of \cs{ref} as plain text; similarly, \cs{labelRef} returns
% the true destination name of a reference.
% \changes{v2.4.2}{2021/03/11}{Defined \string\cs{txtRef}, also defined in \string\pkg{exerquiz};
% moved \string\cs{labelRef} from the \string\textsf{userinterface} section of this DTX.}
%    \begin{macrocode}
\def\txtRef#1{\@ifundefined{r@#1}
  {??}{\aeb@exiii\@firstoffive\csname r@#1\endcsname}}
\def\labelRef#1{\@ifundefined{r@#1}{Doc-Start}
  {\aeb@exiii\@fourthoffive\csname r@#1\endcsname}}
\def\aeb@exiii{\expandafter\expandafter\expandafter}
\def\noexpandiii{\noexpand\noexpand\noexpand}
%    \end{macrocode}
%    \begin{macrocode}
%</package>
%    \end{macrocode}
%    \section{\tops{\protect\app}{}{Adobe} Forms Support}\label{formsSupport}
%    \begin{macrocode}
%<*package>
%    \end{macrocode}
% Listed below are the various types of form objects available in PDF.
%\par\medskip\noindent
%\begin{minipage}{.67\linewidth}\raggedcolumns
%\begin{multicols}{2}
%\begin{itemize}
% \item \hyperref[button]{Button}
%   \begin{itemize}
%       \item \hyperref[pushbutton]{Push buttons}
%       \item \hyperref[checkbox]{Check boxes}
%       \item \hyperref[radiobutton]{Radio Buttons}
%   \end{itemize}
% \item \hyperref[textfield]{Text Fields}
% \item \hyperref[choice]{Choice Fields}
%   \begin{itemize}
%       \item \hyperref[listbox]{list box}
%       \item \hyperref[combobox]{combo box}
%   \end{itemize}
%\item \hyperref[sigfield]{Signature Fields}
%\end{itemize}
%\end{multicols}
%\end{minipage}
%
%\bigskip
%
%
% \subsection{Process Key-Value Pairs: Main Macro}\label{procArgs}
%
% The following macro, \cs{processAppArgs},  is  due in part to
% Dan Luecking. He proposed a very nice modification of my original
% macros.
%
% The macro \cs{processAppArgs} takes an \textit{even} number of arguments; it
% picks off two at a time, processes them, then picks off two more. The macro is meant to
% process the optional arguments of a form field.
%
% All legal arguments (see \nameref{eformvariables} for a detailed
% listing) are of the form \cs{\ameta{name}\darg{\ameta{arg}}}.
% The macro takes two tokens at a time and constructs a macro
% \cs{@eq\ameta{name}\darg{\ameta{arg}}}. Each of the macros \cs{@eq\ameta{name}} must be
% defined. Such a macro defines another macro as follows
% \cs{def\string\eq@\ameta{name}\darg{\ameta{arg}}}. For example if the user enters the token
% pair  |\TU{|\ameta{text}|}|, \cmd{\processAppArgs} will construct
% |\@eqTU|, with argument |{|\ameta{text}|}|, this macro will be
% executed, which expands to |\def\eq@TU{/TU(|\ameta{text}|)}| (a simplified definition).  The macro
% \cmd{\eq@TU} is then be used within the construction of the
% widget object.\medskip
%
% \noindent
% \DescribeMacro\proccessAppArgs\hskip-\marginparsep\texttt{\darg{\ameta{key}}\darg{\ameta{value}}}
% The macro takes pairs of key-values and constructs auxiliary macros, as described above.
% It also does an addition operation for \textsf{\textbf{F}} (\cs{F}) and \textsf{\textbf{Ff}} (\cs{Ff}) entries.
% Initially, it is called by |\processAppArgs|\texttt{\ameta{KV-pairs}}|\end\@nil|, \cs{end} being the detected
% end of the stream of key-values.
%    \begin{macrocode}
\let\ef@passedArgs\@empty
\def\processAppArgs#1#2{%
  \ifx\end#1% if #1=\end, #2=\@nil.
    \let\ef@next\relax
  \else
%    \end{macrocode}
% If a token has a value of \cs{@empty} then it has been protected. It is skipped
% and there is no user redefinition of that form field attribute allowed. Normally,
% this is done for \cs{A} and \cs{AA} to prevent overwriting critical functionality.
%    \begin{macrocode}
    \ifx#1\@empty\def\ef@next{\processAppArgs}\else
%    \end{macrocode}
% This is the user interface to the new optional argument of links and forms. If
% the key is \cs{ui}, we pass its argument to |\setkeys{eforms}{#2}| to process
% the key values of \textsf{xkeyval} style. If one of the keys is \texttt{annotflags} or
% \texttt{fieldflags}, we pass those back to this stream to be analyzed the special cases
% that follow for |\F| and |\Ff|.
%    \begin{macrocode}
      \@getCmdName{\ui}\edef\arg@ui{\@CmdName}%
      \@getCmdName{#1}%
      \ifx\arg@ui\@CmdName
        \@ifundefined{@equi}{\PackageError{eforms}%
        {The user interface '\string\ui' is not defined!%
        \MessageBreak Use the useui option of eforms
        and try again}{I said, use the useui option of
        eforms and try again!}}{}%
        \def\ef@next{\setkeys{eforms}{#2}%
        \processAppArgs\presets{\ef@passedArgs}}%
      \else
%    \end{macrocode}
% If current key is |\Ff|, we add its value to the current value of |\Ff|.
% We basically are `or-ing' the new value with the old value in the bit field.
%    \begin{macrocode}
        \@getCmdName{\Ff}\edef\arg@Ff{\@CmdName}%
        \@getCmdName{#1}%
        \ifx\arg@Ff\@CmdName  % if \Ff, let's  add arguments
          \ifx\eq@Ff\@empty\def\eq@FfValue{0}\else
            \expandafter\getFfValue\eq@Ff\@nil\fi
          \@tempcnta=\eq@FfValue
          \def\eq@arg{#2}%
          \ifx\eq@arg\@empty\else
            \def\x{\FfRadiosInUnison}%
            \ifx\eq@arg\x\let\isRadiosInUnison\ef@YES
            \else\let\isRadiosInUnison\ef@NO\fi
          \advance\@tempcnta by#2\fi
          \edef\eq@Ff{/Ff \the\@tempcnta}%
          \def\ef@next{\processAppArgs}%
        \else
%    \end{macrocode}
% If current key is |\F|, we add its value to the current value of |\F|.
% We basically are `or-ing' the new value with the old value in the bit field.
%    \begin{macrocode}
          \@getCmdName{\F}\edef\arg@F{\@CmdName}%
          \@getCmdName{#1}%
          \ifx\arg@F\@CmdName  % if \F, let's  add arguments
            \ifx\eq@F\@empty\def\eq@FValue{0}\else
              \expandafter\getFValue\eq@F\@nil\fi
            \@tempcnta=\eq@FValue
            \def\eq@arg{#2}%
            \ifx\eq@arg\@empty\else
%    \end{macrocode}
%    Something included in for `enhanced previews' is we test if this field is hidden
%    (has a flag of 2 or 32), if yes we \cs{let} \DescribeMacro\ef@isHidden\cs{ef@isHidden} to \cs{ef@YES}, otherwise
%    it is \cs{let} to \cs{ef@NO}.
%    \begin{macrocode}
              \ifnum#2=2\relax
                \let\ef@isHidden\ef@YES\else
              \ifnum#2=32\relax
                \let\ef@isHidden\ef@YES\else
                \let\ef@isHidden\ef@NO
              \fi\fi
              \advance\@tempcnta by#2\fi
            \edef\eq@F{/F \the\@tempcnta}%
            \def\ef@next{\processAppArgs}%
          \else
%    \end{macrocode}
% If the key we are processing is \cs{presets}, then use \cs{expandafter} to
% expand its argument (it is assumed the argument is a macro), then put it back
% into the input stream.
%    \begin{macrocode}
            \@getCmdName{\presets}%
            \edef\arg@presets{\@CmdName}\@getCmdName{#1}%
            \ifx\arg@presets\@CmdName
              \def\ef@next{\expandafter\processAppArgs#2}%
            \else
%    \end{macrocode}
%      (2019/01/22) Here, we process the \cs{epresets} key. For \cs{epresets}, all
%      \pkg{eforms} keys should be protected, as we expand fully. This is useful when
%      the argument of \cs{presets} is a macro of the form \cs{csname}/\cs{endcsname}.
%      An ordinary \cs{presets} does not work, we must use \cs{epresets}:
%\begin{verbatim}
%   \csarg\def{fld1}#1{\protect\BG{#1}}
%   \textField[\epresets{\nameuse{fld1}{yellow}}]{mytxt}{2in}{13bp}
%\end{verbatim}
%In the above example we use \cs{nameuse} rather than \cs{@nameuse}, since in a recent
%version of \pkg{eforms}, we \cs{let} \cs{nameuse} to \cs{@nameuse}.
%We can also say,
%\begin{verbatim}
%   \protectedKeys{pKeys}{\BG{red}\BC{blue}}
%%  \nameuse is expanded within the argument below, where it is defined
%   \csarg\def{fld2}{\nameuse{pKeys}}
%   \pushButton[\CA{Push Me}\epresets{\nameuse{fld2}}]{pb1}{}{13bp}
%\end{verbatim}
%where \cs{protectedKeys} is used to protected each of the keys in its argument.
%    \begin{macrocode}
              \@getCmdName{\epresets}%
              \edef\arg@epresets{\@CmdName}\@getCmdName{#1}%
              \ifx\arg@epresets\@CmdName
                \def\ef@next{\let\protect\noexpand
                  \edef\x{#2}\set@typeset@protect\expandafter
                  \processAppArgs\x}%
              \else
%    \end{macrocode}
% This is the last, and the most frequent case. We process an ordinary key, one
% that is not |\presets|, |\epresets|, |\ui|, |\Ff| or |\F|.
%    \begin{macrocode}
                \csname @eq%
                \expandafter\@gobble\string#1\endcsname{#2}%
                \def\ef@next{\processAppArgs}%
              \fi
            \fi
          \fi
        \fi
      \fi
    \fi
  \fi
  \ef@next
}
%    \end{macrocode}
% Process the field defaults and the `every' changes. Build up the required command
% in a token list, then execute.
%    \begin{macrocode}
\def\@processEvery{\edef\eqtemp{}\toks@={}\@@processEvery}
\def\@@processEvery#1{\ifx#1\end
  \def\ef@next{\the\toks@}\else
  \edef\eqtemp{\the\toks@}%
  \toks@=\expandafter{\eqtemp\expandafter
    \processAppArgs#1\end\@nil}%
  \def\ef@next{\@@processEvery}\fi\ef@next
}
%    \end{macrocode}
%    \begin{macrocode}
\newdimen\eqcenterWidget
%    \end{macrocode}
% This macro is used to vertically center text fields and buttons on a
% line. Seems to work well.
% \changes{v2.5h}{2012/11/17}{Introduce the \string\cs{inline} key designed for
% inline form fields.}
%    \begin{macrocode}
\def\centerWidget#1{%
  \ifeq@inlineCenter
%    \end{macrocode}
% Inline form field, do a better job at centering it.
%    \begin{macrocode}
    \eqcenterWidget=#1\relax
    \eqcenterWidget=.5\eqcenterWidget
    \ifnum\eq@textSize=0\relax
    \dimen@=-\eq@textSizeDefault\b@\else
    \dimen@=-\eq@textSize\b@\fi
    \dimen@=0.9167\dimen@ % 11/12
    \dimen@=.5\dimen@
    \advance\dimen@\eq@W@value\b@
    \ifx\eq@S\@empty\else
      \def\eq@S@cmp{B}%
      \ifx\eq@S@value\eq@S@cmp
        \advance\dimen@ by \eq@W@value\b@
      \else
        \def\eq@S@cmp{I}%
        \ifx\eq@S@value\eq@S@cmp
          \advance\dimen@ by \eq@W@value\b@
        \else\advance\dimen@\b@
    \fi\fi\fi
    \advance\eqcenterWidget\dimen@
  \else
    \eqcenterWidget=#1\relax
    \eqcenterWidget=.5\eqcenterWidget
    \advance\eqcenterWidget-4\b@
  \fi
}
%    \end{macrocode}
%    \begin{macro}{\ef@optscale}
%    This macro supports the \texttt{width}, \texttt{height}, and \texttt{scale} option keys for forms.
%    We obey these three keys in the order listed above.
%    \changes{v2.3.5}{2020/09/15}{Added \string\cs{ef@optscale}}
%    \begin{macrocode}
\def\ef@optscale{%
  \ifx\eq@width\@empty
    \ifx\eq@height\@empty
      \ifx\eq@scalefactor\@empty
      \else
%    \end{macrocode}
%    Re-scale using scale factor of \cs{eq@scale}.
%    \begin{macrocode}
        \@tempdima\eq@scalefactor\p@
        \ifdim\@tempdima<\z@\edef\eq@scalefactor{-\eq@scalefactor}\fi
        \setlength{\eflength}{\eq@rectH*\real{\eq@scalefactor}}%
        \edef\eq@rectH{\the\eflength}%
        \setlength{\eflength}{\eq@rectW*\real{\eq@scalefactor}}%
        \xdef\eq@rectW{\the\eflength}%
      \fi
    \else
%    \end{macrocode}
%    Re-scale according to height.
%    \begin{macrocode}
      \setlength{\eflength}{\eq@rectH}%
      \ifdim\eflength>\z@
        \setlength{\eflength}{\eq@height*\ratio{\eq@rectW}{\eq@rectH}}%
        \edef\eq@rectW{\the\eflength}\edef\eq@rectH{\eq@height}%
      \fi
    \fi
  \else
%    \end{macrocode}
%    Re-scale according to width.
%    \begin{macrocode}
    \setlength{\eflength}{\eq@rectW}%
    \ifdim\eflength>\z@
      \setlength{\eflength}{\eq@width*\ratio{\eq@rectH}{\eq@rectW}}%
      \edef\eq@rectH{\the\eflength}\edef\eq@rectW{\eq@width}%
    \fi
  \fi
}
%    \end{macrocode}
%    \end{macro}
%
% \subsection{\textsf{eforms} key-values}\label{eformvariables}
%
% We maintain two key-value systems: (1) \pkg{eforms} key-values, (2) user-friendly
% key-value system. The latter requires the option \opt{useui}.
%
% \subsubsection{Key-Value Definitions}
%
% The following definitions are used in various field templates.
% Some convenience macros to help define the button attributes. The default
% values are defined within the button macros themselves. Use these macros
% within the optional argument of buttons and text fields to modify their
% appearance.
%
% You'll notice, for example, the macros listed are not actually defined. For example
% \cmd{\CA} is never defined, we define instead \cmd{\@eqCA} and \cmd{\eq@CA}.
% The macros \cmd{\processAppArgs} treats \cmd{\CA} as a token, and prefixes with
% \texttt{@eq} in a clever sort of way. It's done so that these macros cannot be used
% outside the optional macro arguments of the button and text field macros.
%
% \paragraph*{Entries common to all annotation dictionaries:}
% \texttt{F, BS, Border, AP, AS, T, A, AA}.
%
% \medskip\noindent\textsl{Annotation Flag Bit Field:} See \nameref{F} for values.
%    \begin{macrocode}
\def\@eqF#1{\def\eq@arg{#1}\ifx\eq@arg\@empty
    \let\eq@F\@empty\else\def\eq@F{/F #1}\fi}\def\eq@F{}
%    \end{macrocode}
% \DescribeMacro{\BS}The \textbf{Border Style} key, \texttt{BS}: \texttt{W}, \texttt{S}, \texttt{D}  (dictionary, optional)
%    \begin{macrocode}
\def\@eqBS#1{%
  \let\eq@BS=0\relax
  \ifx\eq@W\@empty\else\let\eq@BS=1\fi
  \ifx\eq@S\@empty\else\let\eq@BS=1\fi
  \ifx\eq@D\@empty\else\let\eq@BS=1\fi
  \edef\link@BS{\if\eq@BS1/BS<<\eq@W\eq@S\eq@D>>\fi}%
  \ifx\eq@W\@empty\let\link@BS\@empty\fi
}\def\link@BS{}
%    \end{macrocode}
% \noindent\DescribeMacro{\presets}\hskip-\marginparsep\texttt{\darg{\meta{\cs{cmd}}}}
% We define the \cs{presets} key. The argument of \key{presets} is a macro (\meta{\cs{cmd}}) consisting of key-value pairs.
% The macro \meta{\cs{cmd}} expanded by \cs{expandafter}
% and put back into the parsing stream.\par\medskip
% \changes{v1.0e}{2008/03/04}{%
%   Added a \string\cs{presets} key to make it easier to dynamically change options}
%
% \noindent \DescribeMacro{\epresets}\hskip-\marginparsep\texttt{\darg{\meta{\cs{cmd}}}} The \cs{epresets} key is used
% when the properties are to be expanded early. The command argument \meta{\cs{cmd}} is fully expanded within
% an \cs{edef}; as a result, the keys need to be protected, for example, |\epresets{\protect\BC{red}}|.
% The command \cs{protectedKeys} can be used to protect the keys to all the key-value pairs.
% \changes{v2.9.23}{2019/01/22}{Added \string\cs{epresets} key}\par\medskip
%
% \noindent Both \cs{presets} and \cs{epresets} are handled within the \cs{proccessAppArgs} macro;
% the definitions below actually do nothing and are never expanded.
%    \begin{macrocode}
\def\@eqpresets#1{#1}
\def\@eqepresets#1{#1}
%    \end{macrocode}
% \DescribeMacro{\W} The width of the boundary line.
%    \begin{macrocode}
\def\@eqW#1{\def\eq@arg{#1}\ifx\eq@arg\@empty
  \let\eq@W\@empty\def\eq@W@value{0}\else
  \def\eq@W@value{#1}\def\eq@W{/W #1}\fi
%    \end{macrocode}
%    (2016/12/22) Add a global value for boundary width, used to adjust the spacing between form fields
%    \changes{v2.9d}{2016/12/22}{Add a global value for boundary width}
%    \begin{macrocode}
  \xdef\g@eq@W@value@bp{\eq@W@value\b@}}
\def\eq@W{}\def\eq@W@value{0}
\def\g@eq@W@valu@bp{0bp}
%    \end{macrocode}
% \DescribeMacro{\S} Line style, values are \texttt{S} (solid), \texttt{D} (dashed),
%      \texttt{B} (beveled), \texttt{I} (inset), \texttt{U} (underlined)
%    \begin{macrocode}
\def\@eqS#1{\def\eq@S@value{#1}\ifx\eq@S@value\@empty
  \let\eq@S\@empty\else
  \def\eq@S{/S/#1}\def\eq@temp{D}%
  \ifx\eq@S@value\eq@temp
    \ifx\eq@D\@empty\def\eq@D{/D [3]}\fi
  \fi\fi}\def\eq@S{}
%    \end{macrocode}
% \DescribeMacro{\D} The dash array.
%    \begin{macrocode}
\def\@eqD#1{\def\eq@arg{#1}\ifx\eq@arg\@empty
  \let\eq@D\@empty\else
  \def\eq@D{/D [#1]}\fi}\def\eq@D{}
%    \end{macrocode}
% \DescribeMacro{\Border} Used with \emph{link annotations}, an array of three numbers and an optional dash array.
%   If all three numbers are 0, no border is drawn
%    \begin{macrocode}
\def\@eqBorder#1{\def\eq@arg{#1}\ifx\eq@arg\@empty
    \let\eq@Border\@empty\else\def\eq@Border{/Border [#1]}\fi}%
\def\eq@Border{/Border [0 0 0]}
%    \end{macrocode}
% \DescribeMacro{\AP} Appearance dictionary, used mostly with check boxes
%    to define the `On' value.
% \changes{v2.9.21}{2018/11/10}{Modified \string\cs{@eqAP}, added two internal
% commands \string\cs{eq@@On} and \string\cs{eq@@Off}}
% Within \cs{@eqAP}, \DescribeMacro\Off\cs{Off} and \DescribeMacro\On\cs{On} are
% \cs{let} to \cs{eq@@Off} and \cs{eq@On} to make
% it `easy' to assign on and off values in the case of icon appearances.
%    \begin{macrocode}
\def\@eqAP#1{\def\eq@arg{#1}\ifx\eq@arg\@empty
  \let\eq@AP\@empty\else\let\Off\eq@@Off\let\On\eq@@On
  \ifx\annot@type\annot@type@button
    \let\N\eq@pb@N\else\let\N\eq@cbrb@N\fi
  \edef\eq@AP{/AP<<#1>>}\fi}\let\eq@AP\@empty
\def\eq@pb@N#1{/N \ifpdf #1 \space 0 R\else
  \ifxetex #1\else{#1}\fi\fi}
\def\eq@cbrb@N#1{/N <<#1>>}
\def\eq@@On#1#2{/#1 \ifpdf #2 \space 0 R\else
  \ifxetex #2\else{#2}\fi\fi}
\def\eq@@Off#1{/Off \ifpdf #1 \space 0 R\else
  \ifxetex #1\else{#1}\fi\fi}
%    \end{macrocode}
% In the \texttt{AP} dictionary for checkboxes is the `On' value.
% It is introduced into \texttt{AP} by passing a TeX parameter
% normally, this variable is not used.
%    \begin{macrocode}
    \def\@eqOn#1{\def\eq@On{/#1}}\def\eq@On{/Yes}
%    \end{macrocode}
% \DescribeMacro{\AS} Appearance state, normally used with check boxes and radio buttons when there are
%  more than one appearance. Advanced techniques only.
%    \begin{macrocode}
\def\@eqAS#1{\def\eq@arg{#1}\ifx\eq@arg\@empty
    \let\eq@AS\@empty\else\ifpdfmarkup\def\eq@AS{/AS(#1) cvn }\else
    \def\eq@AS{/AS/#1}\fi\fi}\def\eq@AS{}
%\def\eq@setAS#1{\ifx\annot@type\annot@type@checkbox\@eqAS{#1}\else
%    \ifx\annot@type\annot@type@radio\@eqAS{#1}\fi\fi}
%    \end{macrocode}
% \paragraph*{The A Dictionary.} In the \texttt{A} dictionary for actions.
%\changes{v2.5a}{2009/12/22}
%{%
% Added special commands for processing the optional argument
% for links. We search for keys words in order to set the
% correct link color. \string\cs{ef@preprocessA} was defined to
% execute these search routines, this is introduced
% for in \string\cs{@eqA} (for \string\cs{rPage}). For links and
% buttons, the command \string\cs{ef@preProcDefns} is also
% inserted.
%}
% Added special commands for processing the optional argument
% for links. We search for keys words in order to set the
% correct link color. \cs{ef@preprocessA} was defined to
% execute these search routines, this is introduced
% for in \cs{@eqA} (for \cs{rPage}). For links and
% buttons, the command \cs{ef@preProcDefns} is also
% inserted.
%    \begin{macrocode}
\def\ef@gobbleToendmarker#1\ef@endmarker{}
\let\ef@endmarker\relax
%    \end{macrocode}
% \textsf{eform} definitions of \texttt{true} and \texttt{false}, used in the search
% algorithm.
%    \begin{macrocode}
\def\ef@end{\end}\def\ef@true{true}
%    \end{macrocode}
% This is the internal definition of |rPage|, a command is used
% in the destination array and the \texttt{GoToR} action. When
% we jump to page number, the number must be zero-based, so we
% take the number provided by the author (1-based), descrement
% by one, and re-define |rPage|.
%    \begin{macrocode}
\def\ef@rPage#1{\@tempcnta=#1\relax\advance\@tempcnta-1
    \edef\rPage##1{\the\@tempcnta}}
%    \end{macrocode}
% We search for |\rPage| in the argument of |\eq@A|, get the page
% number and decrement it.
%    \begin{macrocode}
\long\def\ef@searchrPage#1\rPage#2#3\@nil{\def\ef@argii{#2}%
  \ifx\ef@argii\ef@end\else\rPage{#2}\fi}
%    \end{macrocode}
% When the user specifies |\mlLink{true}| in the option list, we branch off to
% \cs{mlhypertext}.
%    \begin{macrocode}
\def\ef@searchmlLink#1\mlLink#2#3\@nil{\def\ef@argii{#2}%
  \ifx\ef@argii\ef@end\let\ef@mlLink\ef@Zero\else
  \ifx\ef@argii\ef@true\let\ef@mlLink\ef@One\else
  \let\ef@mlLink\ef@Zero\fi\fi}
%    \end{macrocode}
% Search for \texttt{/GoToR}, if found, change the link color to |\@filecolor|
%    \begin{macrocode}
\def\ef@searchGoToR#1/GoToR#2\@nil{\def\ef@argii{#2}%
  \ifx\ef@argii\ef@end\else\ifx\ef@linktxtcolor@set0%
  \def\ef@thislinkcolor{\@filecolor}\fi
  \expandafter\ef@gobbleToendmarker\fi}
%    \end{macrocode}
% Search for \texttt{/URI}, if found, change the link color to |\@urlcolor|
%    \begin{macrocode}
\def\ef@searchURI#1/URI#2\@nil{\def\ef@argii{#2}%
  \ifx\ef@argii\ef@end\else\ifx\ef@linktxtcolor@set0%
  \def\ef@thislinkcolor{\@urlcolor}\fi
  \expandafter\ef@gobbleToendmarker\fi}
\def\ef@searchCmdURI#1\URI#2\@nil{\def\ef@argii{#2}%
  \ifx\ef@argii\ef@end\else\ifx\ef@linktxtcolor@set0%
  \def\ef@thislinkcolor{\@urlcolor}\fi
  \expandafter\ef@gobbleToendmarker\fi}
%    \end{macrocode}
% Search for \texttt{/Named}, if found, change the link color to |\@menucolor|
%    \begin{macrocode}
\def\ef@searchNamed#1/Named#2\@nil{\def\ef@argii{#2}%
  \ifx\ef@argii\ef@end\else\ifx\ef@linktxtcolor@set0%
  \def\ef@thislinkcolor{\@menucolor}\fi
  \expandafter\ef@gobbleToendmarker\fi}
\def\ef@searchCmdNamed#1\Named#2\@nil{\def\ef@argii{#2}%
  \ifx\ef@argii\ef@end\else\ifx\ef@linktxtcolor@set0%
  \def\ef@thislinkcolor{\@menucolor}\fi
  \expandafter\ef@gobbleToendmarker\fi}
%    \end{macrocode}
% Search for \texttt{/Launch}, if found, change the link color to |\@runcolor|
%    \begin{macrocode}
\def\ef@searchLaunch#1/Launch#2\@nil{\def\ef@argii{#2}%
  \ifx\ef@argii\ef@end\else\ifx\ef@linktxtcolor@set0%
  \def\ef@thislinkcolor{\@runcolor}\fi
  \expandafter\ef@gobbleToendmarker\fi}
%    \end{macrocode}
% Executed by |\eq@A|, which calls the search routines defined above, at least
% in the case of links. It also searches for |\rPage|.
%    \begin{macrocode}
\def\ef@preprocessA#1{%
  \let\rPage\relax\edef\ef@argi{#1}%
  \ifx\annot@type\annot@type@link
  \expandafter\ef@searchGoToR\ef@argi/GoToR\end\@nil
  \expandafter\ef@searchURI\ef@argi/URI\end\@nil
  \expandafter\ef@searchCmdURI\ef@argi\URI\end\@nil
  \expandafter\ef@searchNamed\ef@argi/Named\end\@nil
  \expandafter\ef@searchCmdNamed\ef@argi\Named\end\@nil
  \expandafter\ef@searchLaunch\ef@argi/Launch\end\@nil
  \ef@endmarker\fi
  \let\rPage\ef@rPage
  \expandafter\ef@searchrPage\ef@argi\rPage\end\@nil
}
%    \end{macrocode}
% \DescribeMacro{\A} This is the \emph{action dictionary} (used by links and forms). If the argument
% is empty, we do nothing, otherwise, we call |\ef@preprocessA|, then define
% the \texttt{/A <<...>>} dictionary.
%    \begin{macrocode}
\def\@eqA#1{\def\eq@arg{#1}\ifx\eq@arg\@empty
  \let\eq@A\@empty\else\ef@preprocessA{#1}%
  \def\eq@A{/A <<#1>>}\fi}\def\eq@A{}
%    \end{macrocode}
% \DescribeMacro{\mlLink} This is a key for the \cs{setLink} command. If
% we say \verb!\mlLink{true}! in the \cs{setLink} option list, we use
% \cs{mlhypertext} from \texttt{aeb\_mlink}, if that package is loaded.
%    \begin{macrocode}
\def\@eqmlLink#1{\def\eq@arg{#1}\ifx\eq@arg\ef@true
  \let\ef@mlLink\ef@One\else\let\ef@mlLink\ef@Zero\fi}
%    \end{macrocode}
% \DescribeMacro{\Lock} The \cs{Lock} key is used with signature fields.
% We allow the document author to either use raw PDF markup, or the simplified
% \pkg{eforms} key-value or UF key-values. To do this, we determine if
% \texttt{/Action} appears somewhere in the argument, if yes, then it is raw;
% otherwise, it is key-value markup.
%    \begin{macrocode}
\def\ef@searchAction#1/Action#2\@nil{\def
  \ef@argii{#2}\ifx\ef@argii\ef@end\let\ef@Action\ef@One\else
  \let\ef@Action\ef@Zero\fi}
%    \end{macrocode}
%    The definition of \cs{@eqLock}, we determine if raw or KV-Pairs.
%    \begin{macrocode}
\def\@eqLock#1{\def\eq@arg{#1}\ifx\eq@arg\@empty
  \let\eq@Lock\@empty\else\def\lckArgs{#1}%
  \ef@searchAction#1/Action\end\@nil
  \ifx\ef@Action\ef@One\def\ef@next{\let\Action\lckAction#1}\else
    \let\ef@next\relax\fi\ef@next
  \def\eq@Lock{/Lock <<\lckArgs>>}\fi}
\let\eq@Lock\@empty
%    \end{macrocode}
%Various examples using \emph{raw PDF markup} follow:
%\begin{Verbatim}[xleftmargin=\parindent,fontsize=\small,
%commandchars={!~@},codes={\catcode`\%=9}]
%\Lock{/Action/All}     !% !textsf~all fields in the doc@
%\Lock{/Action/Include  !% !textsf~include all fields listed in Fields@
%      /Fields [(field1)(field2)...]}
%\Lock{/Action/Exclude  !% !textsf~exclude all fields listed in Fields@
%      /Fields [(field1)(field2)...]}
%\end{Verbatim}
%    A simplified syntax for the arguments of \cs{Lock}; sample
%    syntax for the same three examples above are,
%\changes{v2.3.6}{2020/10/10}{Addition support for the \string\cs{Lock} key}
%\begin{Verbatim}[xleftmargin=\parindent,fontsize=\small,
%commandchars={!~@},codes={\catcode`\%=9}]
%\Lock{\Action\All}
%\Lock{\Action\IncludeFields{field1,field2,...}}
%\Lock{\Action\ExcludeFields{field1,field2,...}}
%\end{Verbatim}
%The code for the simplified markup for the \cs{Lock} key.
%    \begin{macrocode}
\def\lckAll{\All}
\def\lckIncludeFields{\IncludeFields}
\def\lckExcludeFields{\ExcludeFields}
\def\lckActionTst#1{\let\X\ef@Yes#1}%
\def\lckAction#1{\def\@rgi{#1}\ifx\@rgi\lckAll
  \def\lckArgs{/Action/All}\let\ef@next\relax\else
  \ifx\@rgi\lckIncludeFields
    \def\ef@next{\lckGetFieldsFor{Include}}\else
    \ifx\@rgi\lckExcludeFields
      \def\ef@next{\lckGetFieldsFor{Exclude}}\else
      \PackageWarning{eforms}%
        {The argument of the \string\Lock key\MessageBreak
        is not correctly expressed. Review the documentation}%
    \fi
  \fi
\fi\ef@next
}
\def\lckGetFieldsFor#1#2{\@temptokena={}\@for\@fld:={#2}\do
  {\edef\@etmp{\the\@temptokena(\@fld)}%
    \@temptokena=\expandafter{\@etmp}%
  }\edef\lckArgs{/Action/#1/Fields[\the\@temptokena]}%
}
%    \end{macrocode}
% Another option that is included in the Signed tab is titled ``This script executes
% when field is signed.''
% This is an option that, through the user interface, is mutually exclusive from
% locking fields. This option appears to be implemented through the format event.
% For example,
%\begin{Verbatim}[xleftmargin=\parindent,fontsize=\small,
%codes={\catcode`\%=9}]
%\AAformat={app.alert("Thank you for signing this field.");}
%\end{Verbatim}
%\paragraph*{Additional Actions.}
% \cs{ef@searchCalc} searches for the \cs{AACalculate} token. If, for some reason,
% \cs{AACalculate} is within a pair of braces, this search will not find it,
% so don't group \cs{AACalculate} within braces, there is no reason to do so
% anyway. If we find \cs{AACalculate}, we set \cs{isCalculate} to \texttt{true}, which is normally
% \texttt{false}.
%    \begin{macrocode}
\newif\ifisCalculate\isCalculatefalse
\def\ef@searchCalc#1\AACalculate#2\@nil{%
  \ifx#2\end\else\aftergroup\isCalculatetrue\fi
}
%    \end{macrocode}
% \DescribeMacro{\AA} (02/06/09) The argument of \cs{@eqAA} is nonempty, we search for the token \cs{AACalculate}
% if found, we set \cs{ifisCalculate} to \texttt{true}. When the document author
% uses the \cs{ui}, the key \cs{AA} is not used, so this change does not affect
% this option. The user interface populates \cs{eq@AA} with a series of commands
% that are either empty or expand to the correct code.
%    \begin{macrocode}
\def\@eqAA#1{\def\eq@arg{#1}\ifx\eq@arg\@empty\let\eq@AA\@empty
  \else\begingroup\ef@searchCalc#1\AACalculate\end\@nil\endgroup
  \def\eq@AA{/AA <<#1>>}\fi}
%    \end{macrocode}
% Begin some additional action definitions for the user interface option
%    \begin{macrocode}
\def\eq@AA{/AA <<\eq@AAmouseup\eq@AAmousedown\eq@AAmouseenter%
  \eq@AAmouseexit\eq@AAonfocus\eq@AAonblur\eq@AAformat%
  \eq@AAkeystroke\eq@AAvalidate\eq@AAcalculate\eq@AApageopen%
  \eq@AApageclose\eq@AApagevisible\eq@AApageinvisible>>}
%    \end{macrocode}
%    \begin{macro}{AAmouseup}
%    \begin{macro}{AAmousedown}
%    \begin{macro}{AAmouseenter}
%    \begin{macro}{AAmouseexit}
%    \begin{macro}{AAonfocus}
%    \begin{macro}{AAonblur}
% These keys are generated internally and put into the parsing stream when
% the uses specifies actions using the user interface (|\ui|).
%    \begin{macrocode}
\def\@eqAAmouseup#1{\def\eq@arg{#1}\ifx\eq@arg\@empty
  \let\eq@AAmouseup\@empty\else\def\eq@AAmouseup{/U<<\JS{#1}>>}\fi}
\let\eq@AAmouseup\@empty
\def\@eqAAmousedown#1{\def\eq@arg{#1}\ifx\eq@arg\@empty
  \let\eq@AAmousedown\@empty\else
  \def\eq@AAmousedown{/D<<\JS{#1}>>}\fi}
\let\eq@AAmousedown\@empty
\def\@eqAAmouseenter#1{\def\eq@arg{#1}\ifx\eq@arg\@empty
  \let\eq@AAmouseenter\@empty\else
  \def\eq@AAmouseenter{/E<<\JS{#1}>>}\fi}
\let\eq@AAmouseenter\@empty
\def\@eqAAmouseexit#1{\def\eq@arg{#1}\ifx\eq@arg\@empty
  \let\eq@AAmouseexit\@empty\else
  \def\eq@AAmouseexit{/X<<\JS{#1}>>}\fi}
\let\eq@AAmouseexit\@empty
\def\@eqAAonfocus#1{\def\eq@arg{#1}\ifx\eq@arg\@empty
  \let\eq@AAonfocus\@empty\else
  \def\eq@AAonfocus{/Fo<<\JS{#1}>>}\fi}
\def\@eqAAmousefocus{\@eqAAonfocus}
\let\eq@AAonfocus\@empty
\def\@eqAAonblur#1{\def\eq@arg{#1}\ifx\eq@arg\@empty
  \let\eq@AAonblur\@empty\else
  \def\eq@AAonblur{/Bl<<\JS{#1}>>}\fi}
\def\@eqAAmouseblur{\def\@eqAAonblur}
\let\eq@AAonblur\@empty
\def\@eqAAformat#1{\def\eq@arg{#1}\ifx\eq@arg\@empty
  \let\eq@AAformat\@empty\else
  \def\eq@AAformat{/F<<\JS{#1}>>}\fi}
%    \end{macrocode}
%    \begin{macro}{AAformat}
%    \begin{macro}{AAkeystroke}
%    \begin{macro}{AAvalidate}
%    \begin{macro}{AAcalculate}
% Formatting, keystroke, validate, calculate tabs.
%    \begin{macrocode}
\let\eq@AAformat\@empty
\def\@eqAAkeystroke#1{\def\eq@arg{#1}\ifx\eq@arg\@empty
  \let\eq@AAkeystroke\@empty\else
  \def\eq@AAkeystroke{/K<<\JS{#1}>>}\fi}
\let\eq@AAkeystroke\@empty
\def\@eqAAvalidate#1{\def\eq@arg{#1}\ifx\eq@arg\@empty
  \let\eq@AAvalidate\@empty\else
  \def\eq@AAvalidate{/V<<\JS{#1}>>}\fi}
\let\eq@AAvalidate\@empty
%    \end{macrocode}
%Additional calculate code, used to add on prior to the user's code
%\changes{v2.9.23}{2019/01/22}{Added \string\cs{AddAAcalculate}}
%    \begin{macrocode}
\def\@eqAddAAcalculate#1{\def\eq@arg{#1}\ifx\eq@arg\@empty
  \else\def\eq@AAcalculate{;}\fi\def\eq@AddAAcalculate{#1}}
\let\eq@AddAAcalculate\@empty
\def\@eqAAcalculate#1{\def\eq@arg{#1}\ifx\eq@arg\@empty
  \let\eq@AAcalculate\@empty\else\isCalculatetrue
  \ifx\eq@AAcalculate\ef@semicolon
    \def\eq@AAcalculate{/C<<\JS{\eq@AddAAcalculate}>>}\else
    \def\eq@AAcalculate{/C<<\JS{\eq@AddAAcalculate#1}>>}\fi\fi
}
\let\eq@AAcalculate\@empty
%    \end{macrocode}
%    \begin{macro}{AApageopen}
%    \begin{macro}{AApageclose}
%    \begin{macro}{AApagevisible}
%    \begin{macro}{AApageinvisible}
% Page related additional actions.
%    \begin{macrocode}
\def\@eqAApageopen#1{\def\eq@arg{#1}\ifx\eq@arg\@empty
  \let\eq@AApageopen\@empty\else
  \def\eq@AApageopen{/PO<<\JS{#1}>>}\fi}
\let\eq@AApageopen\@empty
\def\@eqAApageclose#1{\def\eq@arg{#1}\ifx\eq@arg\@empty
  \let\eq@AApageclose\@empty\else
  \def\eq@AApageclose{/PC<<\JS{#1}>>}\fi}
\let\eq@AApageclose\@empty
\def\@eqAApagevisible#1{\def\eq@arg{#1}\ifx\eq@arg\@empty
  \let\eq@AApagevisible\@empty\else
  \def\eq@AApagevisible{/PV<<\JS{#1}>>}\fi}
\let\eq@AApagevisible\@empty
\def\@eqAApageinvisible#1{\def\eq@arg{#1}\ifx\eq@arg\@empty
  \let\eq@AApageinvisible\@empty\else
  \def\eq@AApageinvisible{/PI<<\JS{#1}>>}\fi}
\let\eq@AApageinvisible\@empty
%    \end{macrocode}
%    \end{macro}
%    \end{macro}
%    \end{macro}
%    \end{macro}
%    \end{macro}
%    \end{macro}
%    \end{macro}
%    \end{macro}
%    \end{macro}
%    \end{macro}
%    \end{macro}
%    \end{macro}
%    \end{macro}
%    \end{macro}
%
% This ends the definitions use by the user interface option for additional actions.
%
% \paragraph*{Additional entries common to fields containing variable text:} \texttt{DR, DA, Q,
% DS, RV}.
%    \begin{macrocode}
%    \end{macrocode}
% \DescribeMacro{\DA} Default appearance (required)
%    \begin{macrocode}
\def\@eqDA#1{\def\eq@DA{#1}}
\def\eq@DA{\eq@textFont\space\eq@textSize\space Tf \eq@textColor}%
%    \end{macrocode}
% \DescribeMacro{\textFont} PDF form font. This is a simple assignment; we define
% \DescribeMacro{\textFontDefault}\cs{textFontDefault} to conveniently change the default font over
% all form fields, with the exception of checkboxes and radio button fields. The default
% is \uif{Helv}.
% \changes{v2.4.1}{2020/12/14}{Defined \string\cs{eqtextFontDefault}}
%    \begin{macrocode}
\def\@eqtextFont#1{\def\eq@textFont{/#1}}
\def\textFontDefault#1{\def\eq@textFontDefault{#1}
  \def\eq@textFont{/#1}}
\textFontDefault{Helv}
%    \end{macrocode}
% \DescribeMacro{\textSize} PDF form text size. This is a simple assignment; we define
% \DescribeMacro{\textSizeDefault}\cs{textSizeDefault} to conveniently change the default font size for all
% form fields, including checkboxes and radio buttons. The default is 9 (points).
% \changes{v2.4.1}{2020/12/14}{Defined \string\cs{eqtextSizeDefault}}
%    \begin{macrocode}
\def\@eqtextSize#1{\def\eq@textSize{#1}}
\def\textSizeDefault#1{\def\eq@textSizeDefault{#1}
  \def\eq@textSize{#1}}
\textSizeDefault{9}
%    \end{macrocode}
% \DescribeMacro{\RV} Rich text value
%\changes{v2.5l}{2011/01/28}{%
%   Wrap the \string\cs{RV} key in an XHTML \string\texttt{{\string\ltag}body\string\rtag} element, part of adding
%   better support for rich text strings for form fields.
%}
%    \begin{macrocode}
\def\eq@RV@Body{<?xml version="1.0"?><body %
  xfa:APIVersion="Acroform:2.7.0.0" %
  xfa:contentType="text/html" %
  xfa:spec="2.1" xmlns="http://www.w3.org/1999/xhtml" %
  xmlns:xfa="http://www.xfa.org/schema/xfa-data/1.0/">}
\def\eq@RV@endBody{</body>}
\def\@eqRV#1{\def\eq@arg{#1}\ifx\eq@arg\@empty
  \let\eq@RV\@empty\else\def\eq@RV{/RV(\eq@RV@Body#1%
  \eq@RV@endBody)\fi}}\def\eq@RV{}
%    \end{macrocode}
% \DescribeMacro{\DS} Rich text default style
%    \begin{macrocode}
\def\@eqDS#1{\def\eq@arg{#1}\ifx\eq@arg\@empty
  \let\eq@DS\@empty\else\def\eq@DS{/DS(#1)\fi}}\def\eq@DS{}
%    \end{macrocode}
% \DescribeMacro{\textColor} Text color
%    \begin{macrocode}
\def\@eqtextColor#1{\ef@parsePDFColor{#1}%
  \HyColor@IfXcolor{%
      \expandafter\HyColor@FieldColor%
      \expandafter{\ef@colorSpec@out}{\eq@textColor}{}{}%
  }{\edef\eq@textColor{\ef@colorSpec@out}}%
}
\def\eq@textColor{0 g}
%    \end{macrocode}
% \DescribeMacro{\Q} Quadding for text fields: \texttt{Q=0} left-justified, \texttt{Q=1} centered
%   \texttt{Q=2} right-justified.
%    \begin{macrocode}
\def\@eqQ#1{\def\eq@arg{#1}\ifx\eq@arg\@empty
  \let\eq@Q\@empty\else\def\eq@Q{/Q #1}\fi}\def\eq@Q{}
%    \end{macrocode}
% \paragraph*{Entries common to all fields:} \texttt{TU, Ff, V, DV, A, AA}
% \par\medskip\noindent
% \DescribeMacro{\DV}  The DV key sets the value of the field when the form field is reset.
% When the unicode option is taken (\cs{ifHy@unicode is \texttt{true}},
% we pass the argument through \cs{pdfstringdef} to get the octal encoding,
% which is the method hyperref uses.
% \changes{v2.7}{2014/07/08}{Removed the use of \string\cs{ifHy@unicode}, now pass all PDF text strings
% through \string\cs{pdfstringdef}.}
% \changes{v2.8a}{2015/07/12}{Added \string\cs{ef@isunicode} to automatically detect
% unicode}
% (2015/07/12) Added \cs{ef@isunicode} to automatically detect
% unicode. When the first token of the argument of \cs{@eqDV}
% and \cs{@eqV} is \cs{unicodeStr}, we bifurcate to \cs{@equDV}
% and \cs{@equV}, respectively.
%    \begin{macrocode}
\def\ef@isunicode#1\unicodeStr#2\@nil{\def\argi{#1}%
  \ifx\argi\@empty\def\ifbool@ef{iftrue}\else
  \def\ifbool@ef{iffalse}\fi}
\def\@eqDV#1{\ef@isunicode#1\unicodeStr\@nil
  \expandafter\csname\ifbool@ef\endcsname\@equDV{#1}\else
  \def\eq@arg{#1}\ifx\eq@arg\@empty
  \let\eq@DV\@empty\else
  \ef@pdfCRLFTABDefns\pdfstringdef\ef@uni@temp{#1}%
  \edef\eq@DV{/DV(\ef@uni@temp)}%
%    \end{macrocode}
%   Google Chrome uses the \textsf{\textbf{DV}} entry to display as the initial value
%   of the combo box (only), the code below \cs{let}s \cs{eq@DV} to \cs{@empty} in enhanced
%   preview mode, for combo boxes only.
%    \begin{macrocode}
  \makespecialJS\fi\fi}\def\eq@DV{}
%    \end{macrocode}
% \DescribeMacro{\nuDV} \cs{@eqnuDV} is the old definition of DV. This version does not
% use hyperref's unicode option. This version comes in handy
% in the acroflex package, where it is undesirable to unicode
% the default (and initial values).
%    \begin{macrocode}
\def\ef@pdfCRLFTABDefns{%
  \def\r{\textCR}\def\t{\textHT}\def\n{\textLF}}
\def\@eqnuDV#1{\def\eq@arg{#1}\ifx\eq@arg\@empty
  \let\eq@DV\@empty\else\def\eq@DV{/DV(#1)}\fi}
%    \end{macrocode}
% \DescribeMacro{\uDV} Unicoded DV, used in \textsf{acroflex.dtx}
%    \begin{macrocode}
\def\@equDV#1{\def\eq@arg{#1}\ifx\eq@arg\@empty
  \let\eq@DV\@empty\else\def\eq@DV{/DV<#1>}\fi}
%    \end{macrocode}
% \DescribeMacro{\V} |\V| is the field value (optional). Beginning with 2019/05/24,
% the preview is enhanced to show the value in PDF previewers such as \app{sumatraPDF};
% when \cs{pmpvOn} is expanded, the value of the field is typeset into the document.
% \cs{eq@Vpv} shall hold the enhanced preview value for \cs{V}.
% \changes{v2.11}{2019/05/24}{Modified \string\cs{@eqV} to enhance preview}
%    \begin{macrocode}
\let\pmpvV\@empty
\let\eq@VSAVE\@empty
%    \end{macrocode}
% Finer control over enhanced preview: for the \cs{V} entry, we can turn the enhanced
% preview on with \DescribeMacro\pmpvVOn\cmd{\pmpvVOn} and off again with
% \DescribeMacro\pmpvVOff\cmd\pmpvVOff.
% \changes{v2.12}{2019/06/07}{Added finer control: \string\cs{pmpvVOff} and
% \string\cs{pmpvVOn}}
%    \begin{macrocode}
\let\ef@Vpv\ef@YES
\def\pmpvVOff{\let\ef@Vpv\ef@NO\let\pmpvFmtCtrl\@gobble}
\def\pmpvVOn{\let\ef@Vpv\ef@YES\let\pmpvFmtCtrl\relax}
\def\noexpand@iii{\noexpand\noexpand\noexpand}
\def\@eqV#1{\Hy@pdfstringfalse
%    \end{macrocode}
%    We define \DescribeMacro\pmpvV\cmd{\pmpvV} to hold the preview value of the field.
%    \changes{v2.3}{2019/06/14}{Special definition of \string\cs{protect} to suppress
%    expansion within first argument of \string\cs{tops} within \string\cs{@eqV}}
%    \begin{macrocode}
  \let\x\protect
  \let\protect\noexpand@iii
  \edef\pmpvV{#1}\let\protect\x
%    \end{macrocode}
%    There is a problem with detecting unicode, if the author wants to use unicode,
%    he should use \cs{unicodeStr}, which we try to detect, but we not allow
%    the use of \cs{tops} (\cs{texorpdfstring}), so we must first remove the
%    tex string, if there is one. We save the definition of \cs{unicodeStr}
%    then let it to \cs{relax} to prevent its expansion. \cs{x} holds the
%    pdf string part of the argument of \cs{texorpdfstring}.
%    \begin{macrocode}
  \let\unicodeStrSAVE\unicodeStr
  \Hy@pdfstringtrue\let\unicodeStr\relax
  \edef\x{#1}\let\unicodeStr\unicodeStrSAVE
  \expandafter\ef@isunicode\x\unicodeStr\@nil
  \expandafter\csname\ifbool@ef\endcsname\Hy@pdfstringtrue
%    \end{macrocode}
%     At this point, we have a unicode string. As a design decision, if
%     \cs{ifefpmpv} is true, the value of the fields will be empty,
%     otherwise, it is what the value of \cs{V} key.
%    \begin{macrocode}
    \ifefpmpv\def\x{FEFF}\let\eq@V\@empty\else\edef\x{#1}\fi
    \@equV{\x}%
  \else
    \def\eq@arg{#1}%
    \ifx\eq@arg\@empty
      \let\eq@V\@empty
    \else
      \Hy@pdfstringtrue
      \ef@pdfCRLFTABDefns\pdfstringdef\ef@uni@temp{#1}%
      \Hy@pdfstringtrue\edef\eq@V{/V(\ef@uni@temp)}%
      \let\eq@VSAVE\eq@V
      \ifefpmpv\let\eq@V\@empty\fi
      \makespecialJS
    \fi
  \fi
}\def\eq@V{}
%    \end{macrocode}
% \DescribeMacro{\nuV} No Unicode V
%    \begin{macrocode}
\def\@eqnuV#1{\def\eq@arg{#1}\ifx\eq@arg\@empty
    \let\eq@V\@empty\else\def\eq@V{/V(#1)}\fi}
%    \end{macrocode}
% \DescribeMacro{\uV} Unicode version of V
%    \begin{macrocode}
\def\@equV#1{\def\eq@arg{#1}\ifx\eq@arg\@empty
    \let\eq@V\@empty\else\edef\eq@V{/V<#1>}\fi}%
%    \end{macrocode}
% \DescribeMacro{\TU} Tool tip (optional, PDF 1.3)
%    \begin{macrocode}
\def\@eqTU#1{\ef@isunicode#1\unicodeStr\@nil
  \expandafter\csname\ifbool@ef\endcsname\@equTU{#1}\else
  \def\eq@arg{#1}\let\r@save\r\let\r\textCR
  \ifx\eq@arg\@empty\let\eq@TU\@empty\else
  \ef@pdfCRLFTABDefns\pdfstringdef\ef@uni@temp{#1}%
  \edef\eq@TU{/TU(\ef@uni@temp)}\makespecialJS\fi\fi
  \let\r\r@save}\def\eq@TU{}
%    \end{macrocode}
% \DescribeMacro{\uTU} Tool tip (optional, PDF 1.3), unicode version
%    \begin{macrocode}
\def\@equTU#1{\def\eq@arg{#1}\ifx\eq@arg\@empty
  \let\eq@TU\@empty\else\def\eq@TU{/TU<#1>}\fi}
%    \end{macrocode}
% \DescribeMacro{\Ff} The Field flags bit field, these values are listed below.
%    \begin{macrocode}
\def\@eqFf#1{\def\eq@arg{#1}\ifx\eq@arg\@empty
  \let\eq@Ff\@empty\else\def\eq@Ff{/Ff #1}\fi}
\def\eq@Ff{}
%    \end{macrocode}
% \goodbreak
% \paragraph*{Keys specific to text fields:} The following keys are specific to
% text fields.
%    \begin{macrocode}
%    \end{macrocode}
% \DescribeMacro{\MaxLen} text fields only. Restricts number of characters
% input. Required if a comb field.
%    \begin{macrocode}
\def\@eqMaxLen#1{\def\eq@arg{#1}\ifx\eq@arg\@empty
  \let\eq@MaxLen\@empty\else\def\eq@MaxLen{/MaxLen #1}\fi}%
  \def\eq@MaxLen{}
%    \end{macrocode}
% \DescribeMacro{\H} Highlight, used in button fields and link annotation. Possible values
% are None, Push, Outline, Invert (respectively, |\H{N}|, |\H{P}|,
% |\H{O}|, |\H{I}|)
% The default highlighting is invert (\texttt{I}).
%    \begin{macrocode}
\def\@eqH#1{\def\eq@arg{#1}\ifx\eq@arg\@empty
  \let\eq@H\@empty\else\def\eq@H{/H/#1}\fi}\def\eq@H{}
%    \end{macrocode}
% \paragraph*{Appearance characteristics dictionary:}
% \texttt{MK\,=\,R,\,BC,\,BG,\,CA,\,RC,\,AC,\,I,\,RI,\allowbreak\,IX,\,IF,^^A
% \,TP}
%    \begin{macrocode}
%    \end{macrocode}
% \DescribeMacro{\MK} A dictionary containing other keys
%    \begin{macrocode}
\def\@eqMK#1{\def\eq@arg{#1}\ifx\eq@arg\@empty
  \let\eq@MK\@empty\else\def\eq@MK{/MK << #1 >> }\fi}%
\def\eq@MK{}
%    \end{macrocode}
% \DescribeMacro{\R} Rotation of field, values 0, 90, 180, 270.
%    \begin{macrocode}
\let\@vertRotate=0
\def\@eqR#1{\def\eq@R@value{#1}\ifx\eq@R@value\@empty
  \let\eq@R\@empty\else
%    \end{macrocode}
% Determine if we are rotating 90 or 270, if so, set the weak switch
% \cs{@vertRotate} to~1
%    \begin{macrocode}
  \@tempcnta=\eq@R@value\relax
  \ifnum\@tempcnta<0 \@tempcnta=-\@tempcnta\fi
  \ifnum\@tempcnta=0 \else\ifnum\@tempcnta=180 \else
    \let\@vertRotate\ef@One\fi\fi
  \def\eq@R{/R #1}\fi}
\def\eq@R{}
%    \end{macrocode}
%   \DescribeMacro{\BC} Boundary color
% \changes{v2.5j}{2011/01/18 }{%
%    Changed \string\cs{def}\string\cs{eq@arg} to \string\cs{edef}\string\cs{eq@arg} in the definition of
%    \string\cs{@eqBC} and \string\cs{@eqBG}. When the argument is a macro that is empty,
%    we can exceed {\string\TeX} capacity.
%}
%    \begin{macrocode}
\def\@eqBC#1{\edef\eq@arg{#1}\ifx\eq@arg\@empty
  \let\eq@BC\@empty\else
  \expandafter\ef@isitnamed\eq@arg\ef@nil
  \ifx\ef@latex@color\ef@y\expandafter
    \HyColor@XZeroOneThreeFour\expandafter{\eq@arg}{\eq@BC}{}{}%
    \edef\eq@BC{/BC [\eq@BC]}\else
    \def\eq@BC{/BC [#1]}\fi
  \fi
}\def\eq@BC{}
%    \end{macrocode}
% \DescribeMacro{\BG} Background color
% \changes{v2.11}{2019/05/24}{Added test for transparency for \string\cs{BG}}
%    \begin{macrocode}
\def\@eqBG#1{\edef\eq@arg{#1}\ifx\eq@arg\@empty
    \let\eq@BG\@empty\let\ef@isBGtransparent\ef@YES
  \else\let\ef@isBGtransparent\ef@NO
    \expandafter\ef@isitnamed\eq@arg\ef@nil
    \ifx\ef@latex@color\ef@y\expandafter
      \HyColor@XZeroOneThreeFour\expandafter{\eq@arg}{\eq@BG}{}{}%
      \edef\eq@BG{/BG [\eq@BG]}\else
      \def\eq@BG{/BG [#1]}\fi
  \fi
}\def\eq@BG{}
\let\pmpvCA\@empty
%    \end{macrocode}
% \DescribeMacro{\CA} normal appearance text. Beginning with 2019/05/24 we have
% enhanced preview, described above in the definition of \cs{@eqV}.
% \changes{v2.11}{2019/05/24}{Modified \string\cs{@eqCA} to enhance preview}
%    \begin{macrocode}
%    \end{macrocode}
% Finer control over enhanced preview: for the \cs{CA} entry, we can turn the enhanced
% preview on with \DescribeMacro\pmpvCAOn\cmd{\pmpvCAOn} and off again with
% \DescribeMacro\pmpvCAOff\cmd\pmpvCAOff.
% \changes{v2.12}{2019/06/07}{Added finer control: \string\cs{pmpvCAOff} and
% \string\cs{pmpvCAOn}}
% \changes{v2.3}{2019/06/14}{Special definition of \string\cs{protect} to suppress
% expansion within first argument of \string\cs{tops} within \string\cs{@eqCA}}
%    \begin{macrocode}
\let\ef@CApv\ef@YES
\def\pmpvCAOff{\let\ef@CApv\ef@NO\let\pmpvFmtCtrl\@gobble}
\def\pmpvCAOn{\let\ef@CApv\ef@YES\let\pmpvFmtCtrl\relax}
\def\@eqCA#1{\let\unicodeStrSAVE\unicodeStr
  \Hy@pdfstringtrue\let\unicodeStr\relax
  \edef\x{#1}\let\unicodeStr\unicodeStrSAVE
  \expandafter\ef@isunicode\x\unicodeStr\@nil
  \expandafter\csname\ifbool@ef\endcsname\@equCA{#1}\else
    \def\eq@arg{#1}\ifx\eq@arg\@empty
    \let\eq@CA\@empty\let\ef@kvCA\@empty
    \else\ef@pdfCRLFTABDefns
      \pdfstringdef\ef@uni@temp{#1}\Hy@pdfstringfalse
%    \end{macrocode}
% We define \DescribeMacro\pmpvCA\cs{pmpvCA} as a local macro to hold
% the caption. It should be something that can be typeset, if not use
% \cs{tops} to offer an alternative.
%    \begin{macrocode}
      \let\x\protect\let\protect\noexpand@iii
      \edef\pmpvCA{#1}\def\eq@CA{#1}\let\protect\x
      \edef\ef@kvCA{/CA(\ef@uni@temp)}%
      \makespecialJS
    \fi
  \fi
}\def\eq@CA{}\def\ef@kvCA{}
%    \end{macrocode}
% \DescribeMacro{\uCA} normal appearance text, unicode version
%    \begin{macrocode}
\def\@equCA#1{\def\eq@arg{#1}\ifx\eq@arg\@empty
  \let\eq@CA\@empty\let\ef@kvCA\@empty
  \else\def\eq@CA{#1}\def\ef@kvCA{/CA<#1>}\fi}
%    \end{macrocode}
% \DescribeMacro{\RC} Roll over text
%    \begin{macrocode}
\def\@eqRC#1{\ef@isunicode#1\unicodeStr\@nil
  \expandafter\csname\ifbool@ef\endcsname\@equRC{#1}\else
    \def\eq@arg{#1}\ifx\eq@arg\@empty
      \let\eq@RC\@empty\let\ef@kvRC\@empty
    \else
      \ef@pdfCRLFTABDefns\pdfstringdef\ef@uni@temp{#1}%
      \def\eq@RC{#1}\edef\ef@kvRC{/RC(\ef@uni@temp)}%
      \makespecialJS
    \fi
  \fi
}\def\eq@RC{}\def\ef@kvRC{}
%    \end{macrocode}
% \DescribeMacro{\uRC} Roll over text, unicode version
%    \begin{macrocode}
\def\@equRC#1{\def\eq@arg{#1}\ifx\eq@arg\@empty
  \let\eq@RC\@empty\let\ef@kvRC\@empty
  \else\def\eq@RC{#1}\def\ef@kvRC{/RC<#1>}\fi}
%    \end{macrocode}
% \DescribeMacro{\AC} Push text
%    \begin{macrocode}
\def\@eqAC#1{\ef@isunicode#1\unicodeStr\@nil
  \expandafter\csname\ifbool@ef\endcsname\@equAC{#1}\else
    \def\eq@arg{#1}\ifx\eq@arg\@empty
      \let\eq@AC\@empty\let\ef@kvAC\@empty
    \else\ef@pdfCRLFTABDefns
      \pdfstringdef\ef@uni@temp{#1}%
      \def\eq@AC{#1}\edef\ef@kvAC{/AC(\ef@uni@temp)}%
      \makespecialJS
    \fi
  \fi
}\def\eq@AC{}\def\ef@kvAC{}
%    \end{macrocode}
% \DescribeMacro{\uAC} Push text, unicode version
%    \begin{macrocode}
\def\@equAC#1{\def\eq@arg{#1}\ifx\eq@arg\@empty
  \let\eq@AC\@empty\let\ef@kvAC\@empty
  \else\def\eq@AC{#1}\def\ef@kvAC{/AC<#1>}\fi}
%    \end{macrocode}
% Other keys of \texttt{MK} include: \texttt{I}, \texttt{RI}, \texttt{IX}, \texttt{IF} and \texttt{TP}
% If I haven't covered everything, use this macro to insert
% material into \texttt{MK}\par\medskip\noindent
% (02/07/09) We begin support for the \texttt{MK} dictionary for a push
% button with an icon appearance, the entries in the \texttt{MK} dictionary that
% effect an icon of the push button has this form:
%\begin{verbatim}
% \ifx\eq@AP\@empty
%   /MK <<\eq@R\eq@BC\eq@BG%
%     \ef@kvCA\ef@kvRC\ef@kvAC\eq@IconMK\eq@mkIns>>
% \else
%   \eq@AP
% \fi
%\end{verbatim}
% The \cs{eq@IconMK} macro inserts the appropriate code for an icon appearance.
%    \begin{macrocode}
%    \end{macrocode}
% \cs{eq@define@IconMK} defines the elements of the \texttt{MK} dictionary,
% used only if there is an icon appearance. The default definition of
% \cs{eq@IconMK} is empty.
%    \begin{macrocode}
\def\eq@define@IconMK{\def\eq@IconMK{\eq@I\eq@RI\eq@IX\eq@TP
  /IF<<\eq@SW\eq@ST\eq@PA\eq@FB>>}}
\let\eq@IconMK\@empty
%    \end{macrocode}
% \DescribeMacro{\I} an indirect reference to a form XObject defining the
% buttons's \emph{normal icon}
% \changes{v2.9}{2016/05/09}{Modified \string\cs{I}, \string\cs{RI}, and
% \string\cs{IX} to accommodate pdftex for null argument.}
% \changes{v2.9.21}{2018/11/10}{Modified \string\cs{@eqI}, \string\cs{@eqRI},
% and \string\cs{@eqIX} in the case of pdftex}
%    \begin{macrocode}
\def\eq@relRef@null#1{0 0 R}
\ifluatex\def\eq@relRef#1{#1 \space 0 R}\else
  \ifpdf\def\eq@relRef#1{#1\space 0 R}\else
    \ifxetex\def\eq@relRef#1{#1}\else
      \def\eq@relRef#1{{#1}}\fi\fi\fi
\def\@eqimportIcons#1{\ifpdfmarkup\ifx\annot@type\annot@type@button
  \def\ef@arg{#1}\ifx\ef@arg\ef@y
  \ifx\eq@I\@empty\@eqI{null}\fi\fi\fi\fi}
\def\ef@null{null}
\def\@eqI#1{%
  \ifx\annot@type\annot@type@button
    \def\eq@arg{#1}%
    \ifx\eq@arg\@empty
      \let\eq@I\@empty\else
      \ifx\eq@arg\ef@null
        \def\eq@I{/I \ef@null}\else
        \def\eq@I{/I \eq@relRef{#1}}%
      \fi
    \fi
    \eq@define@IconMK
  \fi
}\def\eq@I{}
%    \end{macrocode}
% \DescribeMacro{\RI} an indirect reference to a form XObject defining
% the buttons's \emph{rollover icon}
%    \begin{macrocode}
\def\@eqRI#1{%
  \ifx\annot@type\annot@type@button
    \def\eq@arg{#1}%
    \ifx\eq@arg\@empty
      \let\eq@RI\@empty\else
      \ifx\eq@arg\ef@null
        \def\eq@RI{/RI \ef@null}\else
        \def\eq@RI{/RI \eq@relRef{#1}}%
      \fi
    \fi
    \eq@define@IconMK
  \fi
}\def\eq@RI{}
%    \end{macrocode}
% \DescribeMacro{\IX} an indirect reference to a form XObject defining
% the buttons's \emph{down icon}
%    \begin{macrocode}
\def\@eqIX#1{%
  \ifx\annot@type\annot@type@button
    \def\eq@arg{#1}%
    \ifx\eq@arg\@empty
      \let\eq@IX\@empty\else
      \ifx\eq@arg\ef@null
        \def\eq@IX{/IX \ef@null}\else
        \def\eq@IX{/IX \eq@relRef{#1}}%
      \fi
    \fi
    \eq@define@IconMK
  \fi
}\def\eq@IX{}
%    \end{macrocode}
% \DescribeMacro{\TP} A code indicating the \texttt{layout} of the text and icon; these codes are
%        0 (label only); 1 (icon only); 2 (label below icon); 3 (label above icon); 4 (label to the right of icon);
%        5 (label to the left of icon); 6 (label overlaid on the icon). The default is 0.
%    \begin{macrocode}
\def\@eqTP#1{\def\eq@arg{#1}\ifx\eq@arg\@empty
  \let\eq@TP\@empty\else\def\eq@TP{/TP #1}\fi}
\def\eq@TP{/TP 0} % default 0
%    \end{macrocode}
% \DescribeMacro{\SW} The \emph{scale when key}. Permissible values are \texttt{A} (always scale),
%   \texttt{B} (scale when icon is too big), \texttt{S} (scale when icon is too small), \texttt{N}
%   (never scale). The default is \texttt{A}.
%    \begin{macrocode}
\def\@eqSW#1{\def\eq@arg{#1}\ifx\eq@arg\@empty
  \let\eq@SW\@empty\else\def\eq@SW{/SW/#1}\fi}
\def\eq@SW{/SW/A} % the default, always scale
%    \end{macrocode}
% \DescribeMacro{\ST} The \emph{scaling type.} Permissible values are \texttt{A}
%    (anamorphic scaling); \texttt{P} (proportional scaling). The default is \texttt{P}.
%    \begin{macrocode}
\def\@eqST#1{\def\eq@arg{#1}\ifx\eq@arg\@empty
  \let\eq@ST\@empty\else\def\eq@ST{/S/#1}\fi}
\def\eq@ST{/S/P} % the default, proportional scaling
%    \end{macrocode}
% \DescribeMacro{\PA} The \textit{position array.} An array of two numbers, each
%  between 0 and 1 indicating the fraction of left-over space to allocate at the left and bottom
%  of the annotation rectangle. The two numbers should be separated by a space. The default value, \verb!\PA{.5 .5}!
%    \begin{macrocode}
\def\@eqPA#1{\def\eq@arg{#1}\ifx\eq@arg\@empty
  \let\eq@PA\@empty\else\def\eq@PA{/A [#1]}\fi}
\def\eq@PA{/A [0.5 0.5]} % the default
%    \end{macrocode}
% \DescribeMacro{\FB} The \emph{fit bounds} Boolean. If \texttt{true}, the button appearance
%   is scaled to fit fully within the bounds of the annotation without taking into consideration
%   the line width of the border. The default is \texttt{false}.
%    \begin{macrocode}
\def\@eqFB#1{\def\eq@arg{#1}\ifx\eq@arg\@empty
  \let\eq@PA\@empty\else\def\eq@FB{/FB #1}\fi}
\def\eq@FB{/FB false} % the default
%    \end{macrocode}
% \DescribeMacro{\mkIns} used for miscellaneous entries for \texttt{MK} dictionary.
%    \begin{macrocode}
\def\@eqmkIns#1{\def\eq@mkIns{#1}}\def\eq@mkIns{}
%    \end{macrocode}
% \paragraph*{Additional entries specific to choice fields:} \texttt{Opt, TI, I}\par\medskip\noindent
% An array of options in the list
%    \begin{macrocode}
\def\@eqOpt#1{\def\eq@arg{#1}\ifx\eq@arg\@empty
  \let\eq@Opt\@empty\else\ifefpmpv\let\eq@Opt\@empty\else
    \def\eq@Opt{/Opt [#1]}\fi\fi}\def\eq@Opt{}
%    \end{macrocode}
% For scrollable list boxes, the top index.
%    \begin{macrocode}
\def\@eqTI#1{\def\eq@arg{#1}\ifx\eq@arg\@empty
  \let\eq@TI\@empty\else\def\eq@TI{/TI #1}\fi}
\def\eq@TI{}
%    \end{macrocode}
% When all else fails, use the \cs{rawPDF} command to modify the widget.
%    \begin{macrocode}
\def\@eqrawPDF#1{\def\eq@rawPDF{#1}}\def\eq@rawPDF{}
%    \end{macrocode}
% An experimental key, used to insert (localalized) definitions or declarations
% into the key-value processor
% \changes{v2.10}{2019/03/16}{added the key \string\cs{cmd}}
%    \begin{macrocode}
\def\@eqcmd#1{#1}
%    \end{macrocode}
% The following is in support for multi-line links
%    \begin{macrocode}
%    \end{macrocode}
% \DescribeMacro{\QuadPoints} Used by \texttt{aeb\_mlink}, used internally by that
% package to create multi-line links.
%    \begin{macrocode}
\def\@eqQuadPoints#1{\def\eq@arg{#1}\ifx\eq@arg\@empty
  \let\eq@QuadPoints\@empty\else
  \def\eq@QuadPoints{/QuadPoints {#1}}\fi}
\def\eq@QuadPoints{}
%    \end{macrocode}
% \DescribeMacro{\Color} Changed |\def\eq@arg| to |\edef\eq@arg| (01/18/11)
%    \begin{macrocode}
\def\@eqColor#1{\edef\eq@arg{#1}\ifx\eq@arg\@empty
  \let\eq@Color\@empty\else
  \HyColor@XZeroOneThreeFour{#1}{\eq@Color}{}{}%
  \edef\eq@Color{/C[\eq@Color]}\fi}
\def\eq@Color{}
%    \end{macrocode}
% \DescribeMacro{linktxtcolor} key to set the color of the link through
% the option list.
%    \begin{macrocode}
\def\@eqlinktxtcolor#1{%
  \def\ef@argi{#1}\ifHy@colorlinks
    \ifx\ef@argi\@empty\let\ef@colorthislink\normalcolor\else
    \let\ef@linktxtcolor@set=1\def\ef@thislinkcolor{#1}\fi\fi
}\let\ef@linktxtcolor@set=0
%    \end{macrocode}
% \paragraph*{Specialized, non-PDF Spec, commands}
%    \begin{macrocode}
%    \end{macrocode}
%    \begin{macro}{\rectH}
%    \begin{macro}{\rectW}
% Used to set the height and width of a widget, useful when the width and height arguments
% of the widget is not accessible.
%    \begin{macrocode}
\def\@eqrectH#1{\def\eq@rectH{#1}\ifx\eq@rectH\@empty\else
  \setlength\eflength\eq@rectH\edef\eq@rectH{\the\eflength}\fi}
\def\@eqrectW#1{\def\eq@rectW{#1}\ifx\eq@rectW\@empty\else
  \setlength\eflength\eq@rectW\edef\eq@rectW{\the\eflength}\fi}
%    \end{macrocode}
%    \end{macro}
%    \end{macro}
%    \begin{macro}{\width}
%    Re-scale the widget to have a specified width, keeping the height in proportion
% \changes{v2.3.5}{2020/09/15}{Added \string\pkg{eforms} keys \string\texttt{width}, \string\texttt{height},
% and \string\texttt{scale}}
%    \begin{macrocode}
\def\@eqwidth#1{\def\eq@width{#1}}
\let\eq@width\@empty
%    \end{macrocode}
%    \begin{macro}{\height}
%    Re-scale the widget to have a specified height, keeping the height in proportion
%    \begin{macrocode}
\def\@eqheight#1{\def\eq@height{#1}}
\let\eq@height\@empty
%    \end{macrocode}
%    \begin{macro}{\scale}
%    Re-scale the widget by a scale factor.
%    \begin{macrocode}
\def\@eqscalefactor#1{\def\eq@scalefactor{#1}}
\let\eq@scalefactor\@empty
%    \end{macrocode}
%    \end{macro}
%    \end{macro}
%    \end{macro}
% \DescribeMacro{\objdef} Insert an indirect reference (ps only drivers). The value of this
% key must be unique throughout the whole document.
% This is a \textbf{pdfmark} feature that inserts a references to this COS object,
% used with setting the tab order using the structure.
%    \begin{macrocode}
\def\@eqobjdef#1{\def\ef@arg{#1}\ifx\ef@arg\@empty
  \let\eq@objdef\@empty\else\def\eq@objdefName{#1}%
  \def\eq@objdef{/_objdef {#1}}\fi
}
\let\eq@objdef\@empty
%    \end{macrocode}
% \DescribeMacro{\taborder} The \cs{taborder} key is used for the
% \texttt{dvipsone}/\texttt{dvips} options on a page where the tab order is
% determined through structure.
%    \begin{macrocode}
\def\@eqtaborder#1{\def\ef@arg{#1}\ifx\ef@arg\@empty
  \let\eq@taborder\@empty\else
  \def\eq@taborder{#1}\fi
}
\let\eq@taborder\@empty
%    \end{macrocode}
% \DescribeMacro{\autoCenter} Auto-center feature, values are
% |\autoCenter{y}| (the default) or |\antoCenter{n}|.
%    \begin{macrocode}
\def\ef@y{y}\def\ef@n{n}
\def\@eqautoCenter#1{\def\ef@arg{#1}\ifx\ef@arg\ef@y
  \let\autoCenter\ef@y\else\ifx\ef@arg\ef@n
  \let\autoCenter\ef@n\else\let\autoCenter\ef@y
  \PackageWarning{eforms}{The value of '#1' is not a
  supported value\MessageBreak for \string\autoCenter.\MessageBreak
  Using the default of 'y'}\fi
\fi}
\let\autoCenter\ef@y
%    \end{macrocode}
% \DescribeMacro{\inline} If |\inline{y}|, then we attempt to
% get a better vertical positioning. Designed for inline form fields.
% \changes{v2.5h}{2012/11/17}{Introduce the \string\cs{inline} key designed for
% inline form fields.}
%    \begin{macrocode}
\newif\ifeq@inlineCenter \eq@inlineCenterfalse
\let\inlineCenter=n
\def\@eqinline#1{\eq@inlineCenterfalse
  \def\ef@arg{#1}\ifx\ef@arg\ef@y
  \let\inlineCenter\ef@y\eq@inlineCentertrue\else
  \ifx\ef@arg\ef@n\let\inlineCenter\ef@n\else\let\inlineCenter\ef@n
  \PackageWarning{eforms}{The value of '#1' is not a
  supported value\MessageBreak for \string\inline.\MessageBreak
  Using the default of 'n'}\fi
\fi}
%    \end{macrocode}
% \DescribeMacro{\symbolchoice} The symbol used for a check box or radio button.
% Elsewhere, we have defined,
%\begin{verbatim}
%   \def\eq@check{4}
%   \def\eq@circle{l}
%   \def\eq@cross{8}
%   \def\eq@diamond{u}
%   \def\eq@square{n}
%   \def\eq@star{H}
%\end{verbatim}
% Possible values for this key are \texttt{check}, \texttt{circle},
% \texttt{cross}, \texttt{diamond}, \texttt{square}, and \texttt{star}.
%    \begin{macrocode}
\def\@eqsymbolchoice#1{\expandafter\ifx\csname eq@#1\endcsname\relax
 \typeout{exerquiz: `#1' is not an acceptable option
 for \string\symbolechoice, inserting default, `check'.}
 \edef\symbol@choice{\eq@check}\else
%    \end{macrocode}
% We take \texttt{\#1} and form the command \cs{eq@\#1}, to match one of
% the definitions listed above.
%    \begin{macrocode}
  \edef\symbol@choice{\csname eq@#1\endcsname}\fi
}
%    \end{macrocode}
%\DescribeMacro{\mlfix} When set to \texttt{y}, contiguous boxes are combined. This is for
%multi-line hyperlinks. Recognizable values are \texttt{y} and \texttt{n}. contiguous boxes
%are combined by default. There are command versions of \cs{mlfix}, these are
%\DescribeMacro{\mlfixOn} and \DescribeMacro{\mlfixOff}
%\changes{v2.9.16}{2018/03/08}{Added \string\cs{mlfix} for multi-line links}
%\changes{v2.9.19}{2018/03/22}{Added \string\cs{relax} following width dimension
% in \string\cs{@eqmlstrut}}
%    \begin{macrocode}
\newif\iffixmlinks \fixmlinkstrue
\def\mlfixOn{\fixmlinkstrue}
\def\mlfixOff{\fixmlinksfalse}
\def\@eqmlfix#1{\def\ef@arg{#1}\ifx\ef@arg\ef@y
  \mlfixOn\else\ifx\ef@arg\ef@n
  \mlfixOff\else\mlfixOn
  \PackageWarning{eforms}{The value of '#1' is not a
  supported value\MessageBreak for \string\mlfix.\MessageBreak
  Using the default of 'y'}\fi
\fi}
\newbox\mlstrutbox
%    \end{macrocode}
%     \DescribeMacro{\mlstrut}\hskip-\marginparsep\texttt{\darg{\ameta{strut-amt}}}
%     (2018/03/22) \cs{mlstrut} is used to adjust the height of a multi-line link,
%     e.g., \verb!\mlstrut{\large\strut}!
%     \changes{v2.9.19}{2018/03/22}{Added \string\cs{mlstrut} used
%     to adjust the height of a multi-line link}
%    \begin{macrocode}
\def\@eqmlstrut#1{\setbox\mlstrutbox\hbox{#1}%
  \def\ml@strut{\vrule \@height\ht\mlstrutbox
    \@depth\dp\mlstrutbox
    \@width\z@\relax}}
\def\ml@strut{\relax\ifmmode\copy\mlstrutbox\else
  \unhbox\mlstrutbox\fi}
\@eqmlstrut{\strut}
%    \end{macrocode}
%     \DescribeMacro{\mlcrackat}\hskip-\marginparsep\texttt{\darg{\ameta{num}}}
%     (2018/03/22) \cs{mlcrackat} is used to break a multi-line link across a page boundary;
%     specifying \verb~\mlcrackat{3}~ breaks the link after the 3rd syllable. The \pkg{aeb\_mlink}
%     package then creates two links consisting of the text up to and including the crack-at value and
%     the second link consisting of the rest of the hypertext link (or url) string.
%     \changes{v2.9.19}{2018/03/22}{Added \string\cs{mlcrackat}}
%    \begin{macrocode}
\def\@eqmlcrackat#1{\def\eq@mlcrackat{#1}}
\let\eq@mlcrackat\@empty
%    \end{macrocode}
%     \DescribeMacro{\mlhyph}\hskip-\marginparsep\texttt{\darg{}}
%      (2018/03/22) \cs{mlhyph} is used to add a hyphen when a multi-line link is cracked
%    using \cs{mlcrackat}. The default is that no hyphen is produced.
%   \changes{v2.9.19}{2018/03/22}{Added \string\cs{mlhyph}}
%    \begin{macrocode}
\def\@eqmlhyph#1{\def\ef@arg{#1}\ifx\ef@arg\ef@y
  \def\eq@mlhyph{-}\else\ifx\ef@arg\ef@n
  \let\eq@mlhyph\@empty\else\let\eq@mlhyph\@empty
  \PackageWarning{eforms}{The value of '#1' is not a
  supported value\MessageBreak for \string\mlhyph.\MessageBreak
  Using the default of 'n'}\fi
\fi}\let\eq@mlhyph\@empty
%    \end{macrocode}
%     \leavevmode
%     \DescribeMacro{\mlignore}\hskip-\marginparsep\texttt{\darg{\upshape0\string|1\string|empty}}
%     (2018/03/22) \cs{mlignore} is an internal option used when breaking apart
%     two multi-line links; not used with urls. The argument is used to identify
%     whether, when breaking a link or annot apart, we are working on the first or
%     second part. The flag \cs{eq@mlignore} is set to \cs{ef@YES}.
%     \changes{v2.9.19}{2018/03/22}{Added \string\cs{mlignore}}
%     \changes{v2.9.20}{2018/08/16}{Added \string\cs{eq@mlchunk} to definition
%     of \string\cs{@eqmlignore}}
%    \begin{macrocode}
\def\@eqmlignore#1{\def\eq@mlchunk{#1}\ifx\eq@mlchunk\@empty
  \def\eq@mlchunk{0}\fi\let\eq@mlignore\ef@YES}
\def\eq@mlchunk{0}
\let\eq@mlignore\ef@NO
%    \end{macrocode}
%    \leavevmode\DescribeMacro{\mlcrackinsat}
%    \hskip-\marginparsep\texttt{\darg{\ameta{latex-content}}}
%    introduces \ameta{latex-content} just after \cs{eq@mlhyph}.
%     \changes{v2.9.20}{2018/08/16}{Added \string\cs{eq@mlcrackinsat}}
%    \begin{macrocode}
\def\@eqmlcrackinsat#1{\def\eq@mlcrackinsat{#1}}
\let\eq@mlcrackinsat\@empty
%    \end{macrocode}
% \DescribeMacro{\protect} A key for protecting a key from begin changed
% by the user through the optional arguments.
%    \begin{macrocode}
\def\@eqprotect#1{\eq@protect{#1}}
\def\eq@protect#1{\let#1\@empty}
%    \end{macrocode}
%    \leavevmode
%    \DescribeMacro{\protectedKeys}\hskip-\marginparsep\texttt{*\darg{\ameta{cmd-name}}\darg{\ameta{KV-pairs}}}
%    is a command that protects each key in a \pkg{eforms} key-value list; eg, it replaces
%    \cs{BC\darg{red}} with \cs{protect\cs{BC\darg{red}}}. The results are saved under the command
%    name \cs{\ameta{cmd-name}}. Designed to be used with the \cs{epresets} key, which expands its arguments
%    early; the keys are not defined so we prevent them from expanding prior to being passed to
%    the parsing mechanism. \cs{protectedKeys} is used in the \pkg{bargraph-js} package. If \cs{\ameta{cmd-name}}
%    is already defined, a warning message is written to the TEX LOG, unless the \texttt*-option is taken.
%    \changes{v2.10}{2019/03/16}{Added \string\cs{protectedKeys}}
%    \begin{macrocode}
\def\ef@stop{\relax}
\let\isst@r\ef@NO
\def\protectedKeys{\@ifstar
  {\let\isst@r\ef@YES\protectedKeys@i}
  {\let\isst@r\ef@NO\protectedKeys@i}%
}
\def\protectedKeys@i#1#2{\ef@scratchtoks={}%
  \@ifundefined{#1}{}{\ifx\isst@r\ef@NO\PackageWarning{eforms}
    {Be aware command name #1 is already\MessageBreak
    in use}\fi}\protectedKeys@gettwo#2\ef@stop\relax
   \csarg\edef{#1}{\the\ef@scratchtoks}}
\def\protectedKeys@gettwo#1#2{%
  \ifx#1\ef@stop\else\ef@scratchtoks=\expandafter
    {\the\ef@scratchtoks\protect#1{#2}}\expandafter
    \protectedKeys@gettwo\fi}
%    \end{macrocode}
% \DescribeMacro{\multigroupradios} Declares whether there are multiple sets of
% radio button groups with the same names and values,
%    \begin{macrocode}
\def\@eqmultigroupradios#1{\let\ef@multigroupradios\ef@YES}
\let\ef@multigroupradios\ef@NO
%    \end{macrocode}
%
% \subsubsection{Support for Hex escapes in PDF names}
% \changes{v2.9p}{2017/10/10}{rework the support for Hex escapes in PDF names}
%    \begin{macrocode}
\begingroup\catcode`\#=12 \catcode`*=6
  \ifpdfmarkup
    \gdef\ef@Hx*1*2{\@nameuse{efHex*1*2}}\else
    \gdef\ef@Hx*1*2{\@nameuse{efHex*1*2}}\fi
\endgroup
\def\efHxError{\PackageError{eforms}{The glyph is not supported}{}}
\def\HGERROR{efHxError}
\ifpdfmarkup
  \def\HexGlyph#1#2{\def\arg@ii{#2}\ifx\arg@ii\HGERROR
  \global\@namedef{efHex#1}{\csname#2\endcsname}\else
  \global\@namedef{efHex#1}{\expandafter
    \string\csname#2\endcsname}\fi}
\else
  \begingroup\catcode`\#=12 \catcode`*=6
    \gdef\HexGlyph*1*2{\def\arg@ii{*2}\ifx\arg@ii\HGERROR
    \global\@namedef{efHex*1}{\csname*2\endcsname}\else
    \global\@namedef{efHex*1}{#*1}\fi}
  \endgroup
\fi
\def\ef@inputPDFHEX{\InputIfFileExists{pdfdochex.def}
    {\PackageInfo{eforms}{Inputting pdfdochex.def}}{}}
\AtEndOfPackage{\ef@inputPDFHEX}
%</package>
%<*hexoctcodes>
% begin C0 Controls (U000.pdf) http://www.unicode.org/charts/PDF/
\HexGlyph{00}{efHxError}%{000}
\HexGlyph{01}{efHxError}%{001}
\HexGlyph{02}{efHxError}%{002}
\HexGlyph{03}{efHxError}%{003}
\HexGlyph{04}{efHxError}%{004}
\HexGlyph{05}{efHxError}%{005}
\HexGlyph{06}{efHxError}%{006}
\HexGlyph{07}{efHxError}%{006}
\HexGlyph{08}{efHxError}%{008}
\HexGlyph{09}{efHxError}%{009}
\HexGlyph{0A}{efHxError}%{012}
\HexGlyph{0B}{efHxError}%{013}
\HexGlyph{0C}{efHxError}%{014}
\HexGlyph{0D}{efHxError}%{015}
\HexGlyph{0E}{efHxError}%{016}
\HexGlyph{0F}{efHxError}%{017}
\HexGlyph{10}{efHxError}%{020}
\HexGlyph{11}{efHxError}%{021}
\HexGlyph{12}{efHxError}%{022}
\HexGlyph{13}{efHxError}%{023}
\HexGlyph{14}{efHxError}%{024}
\HexGlyph{15}{efHxError}%{025}
\HexGlyph{16}{efHxError}%{026}
\HexGlyph{17}{efHxError}%{027}
% end C0 Controls (U000.pdf)
\HexGlyph{18}{030}% U+02D8 BREVE
\HexGlyph{19}{031}% U+02c7 CARON
\HexGlyph{1A}{032}% U+02c6 MODIFIER LETTER CIRCUMFLEX ACCENT
\HexGlyph{1B}{033}% U+02D9 DOT ABOVE
\HexGlyph{1C}{034}% U+02DD DOUBLE ACUTE ACCENT
\HexGlyph{1D}{035}% U+02DB OGONEK
\HexGlyph{1E}{036}% U+02DA RING ABOVEZ
\HexGlyph{1F}{037}% U+02DC SMALL TILDE
\HexGlyph{20}{040}% U+0020 SPACE
\HexGlyph{21}{041}% U+0021 EXCLAMATION MARK
\HexGlyph{22}{042}% U+0022 QUOTATION MARK
\HexGlyph{23}{043}% U+0022 NUMBER SIGN
\HexGlyph{24}{044}% U+0023 DOLLAR SIGN
\HexGlyph{25}{045}% U+002 PERCENT SIGN
\HexGlyph{26}{046}% U+0026 AMPERSAND
\HexGlyph{27}{047}% U+0027 APOSTROPHE
\HexGlyph{28}{050}% U+0028 LEFT PARENTHESIS
\HexGlyph{29}{051}% U+0029 RIGHT PAENTHESIS
\HexGlyph{2A}{052}% U+002A ASTERISK
\HexGlyph{2B}{053}% U+002B PLUS SIGN
\HexGlyph{2C}{054}% U+002C COMMA
\HexGlyph{2D}{055}% U+002D HYPHEN-MINUS
\HexGlyph{2E}{056}% U+002E FULL STOP
\HexGlyph{2F}{057}% U+002F SOLIDUS
\HexGlyph{30}{060}% U+0030 ZERO
\HexGlyph{31}{061}% U+0031 ONE
\HexGlyph{32}{062}% U+0032 TWO
\HexGlyph{33}{063}% U+0033 THREE
\HexGlyph{34}{064}% U+0034 FOUR
\HexGlyph{35}{065}% U+0035 FIVE
\HexGlyph{36}{066}% U+0036 SIX
\HexGlyph{37}{067}% U+0037 SEVEN
\HexGlyph{38}{070}% U+0038 EIGHT
\HexGlyph{39}{071}% U+0039 NINE
\HexGlyph{3A}{072}% U+003A COLON
\HexGlyph{3B}{073}% U+003B SEMICOLON
\HexGlyph{3C}{074}% U+003C LESS-THAN SIGN
\HexGlyph{3D}{075}% U+003D EQUALS SIGN
\HexGlyph{3E}{076}% U+003E GREATER-THAN SIGN
\HexGlyph{3F}{077}% U+003F QUESTION MARK
\HexGlyph{40}{100}% U+0040 COMMERCIAL AT
\HexGlyph{41}{101}% U+0041 CAPITAL LETTER A
\HexGlyph{42}{102}% U+0042 B
\HexGlyph{43}{103}% U+0043 C
\HexGlyph{44}{104}% U+0044 D
\HexGlyph{45}{105}% U+0045 E
\HexGlyph{46}{106}% U+0046 F
\HexGlyph{47}{107}% U+0047 G
\HexGlyph{48}{110}% U+0048 H
\HexGlyph{49}{111}% U+0049 I
\HexGlyph{4A}{112}% U+004A J
\HexGlyph{4B}{113}% U+004B K
\HexGlyph{4C}{114}% U+004C L
\HexGlyph{4D}{115}% U+004D M
\HexGlyph{4E}{116}% U+004E N
\HexGlyph{4F}{117}% U+004F O
\HexGlyph{50}{120}% U+0050 P
\HexGlyph{51}{121}% U+0051 Q
\HexGlyph{52}{122}% U+0052 R
\HexGlyph{53}{123}% U+0053 S
\HexGlyph{54}{124}% U+0054 T
\HexGlyph{55}{125}% U+0055 U
\HexGlyph{56}{126}% U+0056 V
\HexGlyph{57}{127}% U+0057 W
\HexGlyph{58}{130}% U+0058 X
\HexGlyph{59}{131}% U+0059 Y
\HexGlyph{5A}{132}% U+005A Z
\HexGlyph{5B}{133}% U+005B LEFT SQUARE BRACKET
\HexGlyph{5C}{134}% U+005C REVERSE SOLIDUS (BACKSLASH)
\HexGlyph{5D}{135}% U+005D RIGHT SQUARE BRACKET
\HexGlyph{5E}{136}% U+005E CIRCUMFLEX ACCENT
\HexGlyph{5F}{137}% U+005F LOW LINE
\HexGlyph{60}{140}% U+0060 GRAVE ACCENT
\HexGlyph{61}{141}% U+0061 LATIN SMALL LETTER a
\HexGlyph{62}{142}% U+0062 b
\HexGlyph{63}{143}% U+0063 c
\HexGlyph{64}{144}% U+0064 d
\HexGlyph{65}{145}% U+0065 e
\HexGlyph{66}{146}% U+0066 f
\HexGlyph{67}{147}% U+0067 g
\HexGlyph{68}{150}% U+0068 h
\HexGlyph{69}{151}% U+0069 i
\HexGlyph{6A}{152}% U+006A j
\HexGlyph{6B}{153}% U+006B k
\HexGlyph{6C}{154}% U+006C l
\HexGlyph{6D}{155}% U+006D m
\HexGlyph{6E}{156}% U+006E n
\HexGlyph{6F}{157}% U+006F o
\HexGlyph{70}{160}% U+0070 p
\HexGlyph{71}{161}% U+0071 q
\HexGlyph{72}{162}% U+0072 r
\HexGlyph{73}{163}% U+0073 s
\HexGlyph{74}{164}% U+0074 t
\HexGlyph{75}{165}% U+0075 u
\HexGlyph{76}{166}% U+0076 v
\HexGlyph{77}{167}% U+0077 w
\HexGlyph{78}{170}% U+0078 x
\HexGlyph{79}{171}% U+0079 y
\HexGlyph{7A}{172}% U+007A z
\HexGlyph{7B}{173}% U+007B LEFT CURLY BRACKET
\HexGlyph{7C}{174}% U+007C VERTICAL LINE
\HexGlyph{7D}{175}% U+007D RIGHT CURLY BRACKET
\HexGlyph{7E}{176}% U+007E TILDE
\HexGlyph{7F}{efHxError}% 177 UNDEFINED IN PDFDOCENC
\HexGlyph{80}{200}% U+2022 BULLET
\HexGlyph{81}{201}% U+2020 DAGGER
\HexGlyph{82}{202}% U+2021 DOUBLE DAGGER
\HexGlyph{83}{203}% U+2026 HORIZONTAL ELLIPSIS
\HexGlyph{84}{204}% U+2014 EM DASH
\HexGlyph{85}{205}% U+2013 EN DASH
\HexGlyph{86}{206}% U+0192 LATIN SMALL LETTER F WITH HOOK
\HexGlyph{87}{207}% U+2044 FRACTION SLASH
\HexGlyph{88}{210}% U+2039 SINGLE LEFT-POINTING ANGLE QUOTE MARK
\HexGlyph{89}{211}% U+203A SINGLE RIGHT-POINTING ANGLE QUOTE MARK
\HexGlyph{8A}{212}% U+2212 MINUS-SIGN
\HexGlyph{8B}{213}% U+2030 PER MILL SIGN
\HexGlyph{8C}{214}% U+201E DOUBLE LOW-9 QUOTE MARK
\HexGlyph{8D}{215}% U+201C LEFT DOUBLE QUOTE MARK
\HexGlyph{8E}{216}% U+201C RIGHT DOUBLE QUOTE MARK
\HexGlyph{8F}{217}% U+2018 LEFT SINGLE QUOTE MARK
\HexGlyph{90}{220}% U+2019 RIGHT SINGLE QUOTE MARK
\HexGlyph{91}{221}% U+201A SINGLE LOW-9 QUOTE MARK
\HexGlyph{92}{222}% U+2122 TRADE MARK SIGN
\HexGlyph{93}{223}% U+FB01 LATIN SMALL LIGATURE FI
\HexGlyph{94}{224}% U+FB02 LATIN SMALL LIGATURE FL
\HexGlyph{95}{225}% U+0141 LATIN CAPITAL LETTER L WITH STROKE
\HexGlyph{96}{226}% U+0152 LATIN CAPITAL LIGATURE OE
\HexGlyph{97}{227}% U+0160 LATIN CAPITAL LETTER S WITH CARON
\HexGlyph{98}{230}% U+0178 LATIN CAPITAL  LETTER Y WITH DIAERESIS
\HexGlyph{99}{231}% U+017D LATIN CAPITAL LETTER Z WITH CARON
\HexGlyph{9A}{232}% U+0131 LATIN SMALL LETTER DOTLESS I
\HexGlyph{9B}{233}% U+0142 LATIN SMALL LETTER L WITH STROKE
\HexGlyph{9C}{234}% U+0153 LATIN SMALL LIGATURE OE
\HexGlyph{9D}{235}% U+0161 LATIN SMALL LETTER S WITH CARON
\HexGlyph{9E}{236}% U+017E LATIN SMALL LETTER Z WITH CARON
\HexGlyph{9F}{efHxError}% 237 UNDEF IN PDFDOCENC
\HexGlyph{A0}{240}% U+20AC EURO SIGN
\HexGlyph{A1}{241}% U+00A1 INVERTED EXCLAMATION MARK
\HexGlyph{A2}{242}% U+00A2 CENT SIGN
\HexGlyph{A3}{243}% U+00A3 POUND SIGN
\HexGlyph{A4}{244}% U+00A4 CURRENCY SIGN
\HexGlyph{A5}{245}% U+00A5 YEN SIGN
\HexGlyph{A6}{246}% U+00A6 BROKEN BAR
\HexGlyph{A7}{247}% U+00A7 SECTION SIGN
\HexGlyph{A8}{250}% U+00A8 DIAERESIS
\HexGlyph{A9}{251}% U+00A9 COPYRIGHT SIGN
\HexGlyph{AA}{252}% U+00AA FEMININE ORDINAL INDICATOR
\HexGlyph{AB}{253}% U+00AB LEFT-POINTING DOUBLE ANGLE QUOTE MARK
\HexGlyph{AC}{254}% U+00AC NOT SIGN
\HexGlyph{AD}{efHxError}% 255 UNDEFINED IN PDFDOCENC
\HexGlyph{AE}{256}% U+00AE REGISTERED SIGN
\HexGlyph{AF}{257}% U+00AF MACRON
\HexGlyph{B0}{260}% U+00B0 DEGREE SIGN
\HexGlyph{B1}{261}% U+00B1 PLUS-MINUS SIGN
\HexGlyph{B2}{262}% U+00B2 SUPERSCRIPT 2
\HexGlyph{B3}{263}% U+00B3 SUPERSCRIPT 3
\HexGlyph{B4}{264}% U+00B4 ACUTE ACCENT
\HexGlyph{B5}{265}% U+00B5 MICRO SIGN
\HexGlyph{B6}{266}% U+00B6 PILCROW SIGN (PARAGRAPH SIGN)
\HexGlyph{B7}{267}% U+00B7 MIDDLE DOT
\HexGlyph{B8}{270}% U+00B8 CEDILLA
\HexGlyph{B9}{271}% U+00B9 SUPERSCRIPT ONE
\HexGlyph{BA}{272}% U+00BA MASCULINE ORDINAL INDICATOR
\HexGlyph{BB}{273}% U+00BB RIGHT-POINTING DOUBLE ANGLE QUOTE MARK
\HexGlyph{BC}{274}% U+00BC VULGAR FRACTION ONE QUARTER
\HexGlyph{BD}{275}% U+00BD VULGAR FRACTION ONE HALF
\HexGlyph{BE}{276}% U+00BE VULGAR FRACTION THREE QUARTERS
\HexGlyph{BF}{277}% U+00BF INVERTED QUESTION MARK
\HexGlyph{C0}{300}% U+00C0 LATIN CAPITAL LETTER A WITH GRAVE
\HexGlyph{C1}{301}% U+00C1 LATIN CAPITAL LETTER A WITH ACUTE
\HexGlyph{C2}{302}% U+00C2 LATIN CAPITAL LETTER A WITH CIRCUMFLEX
\HexGlyph{C3}{303}% U+00C3 LATIN CAPITAL LETTER A WITH TILDE
\HexGlyph{C4}{304}% U+00C4 LATIN CAPITAL LETTER A WITH DIAERESIS
\HexGlyph{C5}{305}% U+00C5 LATIN CAPITAL LETTER A WITH RING ABOVE
\HexGlyph{C6}{306}% U+00C6 LATIN CAPITAL LETTER AE
\HexGlyph{C7}{307}% U+00C7 LATIN CAPITAL LETTER C WITH CEDILLA
\HexGlyph{C8}{310}% U+00C8 LATIN CAPITAL LETTER E WITH GRAVE
\HexGlyph{C9}{311}% U+00C9 LATIN CAPITAL LETTER E WITH ACUTE
\HexGlyph{CA}{312}% U+00CA LATIN CAPITAL LETTER E WITH CIRCUMFLEX
\HexGlyph{CB}{313}% U+00CB LATIN CAPITAL LETTER E WITH DIAERESIS
\HexGlyph{CC}{314}% U+00CC LATIN CAPITAL LETTER I WITH GRAVE
\HexGlyph{CD}{315}% U+00CD LATIN CAPITAL LETTER I WITH ACUTE
\HexGlyph{CE}{316}% U+00CE LATIN CAPITAL LETTER I WITH CIRCUMFLEX
\HexGlyph{CF}{317}% U+00CF LATIN CAPITAL LETTER I WITH DIAERESIS
\HexGlyph{D0}{320}% U+00D0 LATIN CAPITAL LETTER ETH
\HexGlyph{D1}{321}% U+00D1 LATIN CAPITAL LETTER D WITH TILDE
\HexGlyph{D2}{322}% U+00D2 LATIN CAPITAL LETTER O WITH GRAVE
\HexGlyph{D3}{323}% U+00D3 LATIN CAPITAL LETTER O WITH ACUTE
\HexGlyph{D4}{324}% U+00D4 LATIN CAPITAL LETTER O WITH CIRCUMFLEX
\HexGlyph{D5}{325}% U+00D5 LATIN CAPITAL LETTER O WITH TILDE
\HexGlyph{D6}{326}% U+00D6 LATIN CAPITAL LETTER O WITH DIAERESIS
\HexGlyph{D7}{327}% U+00D7 MULTIPLICATION SIGN
\HexGlyph{D8}{330}% U+00D8 LATIN CAPITAL LETTER O WITH STROKE
\HexGlyph{D9}{331}% U+00D9 LATIN CAPITAL LETTER U WITH GRAVE
\HexGlyph{DA}{332}% U+00DA LATIN CAPITAL LETTER U WITH ACUTE
\HexGlyph{DB}{333}% U+00DB LATIN CAPITAL LETTER U WITH CIRCUMFLEX
\HexGlyph{DC}{334}% U+00DC LATIN CAPITAL LETTER U WITH DIAERESIS
\HexGlyph{DD}{335}% U+00DD LATIN CAPITAL LETTER Y WITH ACUTE
\HexGlyph{DE}{336}% U+00DE LATIN CAPITAL LETTER THORN
\HexGlyph{DF}{337}% U+00DF LATIN CAPITAL LETTER SHARP S (Eszett)
\HexGlyph{E0}{340}% U+00E0 LATIN SMALL LETTER A WITH GRAVE
\HexGlyph{E1}{341}% U+00E1 LATIN SMALL LETTER A WITH ACUTE
\HexGlyph{E2}{342}% U+00E2 LATIN SMALL LETTER A WITH CIRCUMFLEX
\HexGlyph{E3}{343}% U+00E3 LATIN SMALL LETTER A WITH TILDE
\HexGlyph{E4}{344}% U+00E4 LATIN SMALL LETTER A WITH DIAERESIS
\HexGlyph{E5}{345}% U+00E5 LATIN SMALL LETTER A WITH RING ABOVE
\HexGlyph{E6}{346}% U+00E6 LATIN SMALL LETTER AE
\HexGlyph{E7}{347}% U+00E7 LATIN SMALL LETTER C WITH CEDILLA
\HexGlyph{E8}{350}% U+00E8 LATIN SMALL LETTER E WITH GRAVE
\HexGlyph{E9}{351}% U+00E9 LATIN SMALL LETTER E WITH ACUTE
\HexGlyph{EA}{352}% U+00EA LATIN SMALL LETTER E WITH CIRCUMFLEX
\HexGlyph{EB}{353}% U+00EB LATIN SMALL LETTER E WITH DIAERESIS
\HexGlyph{EC}{354}% U+00EC LATIN SMALL LETTER I WITH GRAVE
\HexGlyph{ED}{355}% U+00ED LATIN SMALL LETTER I WITH ACUTE
\HexGlyph{EE}{356}% U+00EE LATIN SMALL LETTER I WITH CIRCUMFLEX
\HexGlyph{EF}{357}% U+00EF LATIN SMALL LETTER I WITH DIAERESIS
\HexGlyph{F0}{360}% U+00F0 LATIN SMALL LETTER ETH
\HexGlyph{F1}{361}% U+00F1 LATIN SMALL LETTER N WITH TILDE
\HexGlyph{F2}{362}% U+00F2 LATIN SMALL LETTER O WITH GRAVE
\HexGlyph{F3}{363}% U+00F3 LATIN SMALL LETTER O WITH ACUTE
\HexGlyph{F4}{364}% U+00F4 LATIN SMALL LETTER O WITH CIRCUMFLEX
\HexGlyph{F5}{365}% U+00F5 LATIN SMALL LETTER O WITH DIAERESIS
\HexGlyph{F6}{366}% U+00F6 LATIN SMALL LETTER C WITH ETH
\HexGlyph{F7}{367}% U+00F7 DIVISION SIGN
\HexGlyph{F8}{270}% U+00F8 LATIN SMALL LETTER O WITH STROKE
\HexGlyph{F9}{271}% U+00F9 LATIN SMALL LETTER U WITH GRAVE
\HexGlyph{FA}{272}% U+00FA LATIN SMALL LETTER U WITH ACUTE
\HexGlyph{FB}{273}% U+00FB LATIN SMALL LETTER U WITH CIRCUMFLEX
\HexGlyph{FC}{374}% U+00FC LATIN SMALL LETTER U WITH DIAERESIS
\HexGlyph{FD}{375}% U+00FD LATIN SMALL LETTER Y WITH ACUTE
\HexGlyph{FE}{376}% U+00FE LATIN SMALL LETTER C THORN
\HexGlyph{FF}{377}% U+00FF LATIN SMALL LETTER Y WITH DIAERESIS
%</hexoctcodes>
%<*package>
%    \end{macrocode}
%
% \subsubsection{Parsing PDF Color}
%
% The command is called by the \cs{textColor} key of a form field,
% the return value, \cs{ef@colorSpec@out}, is then used in the color
% specification of the text. If \textsf{xcolor} is loaded, we pass
% the named color or latex color specification with color model back,
% and is turned processed by the \textsf{hycolor} command \cs{HyColor@FieldColor}.
%    \begin{macrocode}
\def\ef@semicolon{;}
\def\ef@stripsemi#1;\@nil{\def\ef@colorSpec@out{#1}}
%    \end{macrocode}
%    \begin{macrocode}
\def\ef@isitnamed{\let\ef@latex@color\ef@y
    \@ifnextchar[{\ef@gobbletonil}{%
    \@tfor\mytok:=.0123456789\do{%
        \if\mytok\@let@token
        \let\ef@latex@color\ef@n
        \@break@tfor\fi}\ef@gobbletonil}}
%    \end{macrocode}
%    \begin{macrocode}
\def\ef@gobbletonil#1\ef@nil{}
%    \end{macrocode}
%    \begin{macro}{\ef@parsePDFColor}
% \cs{ef@parsePDFColor} tries to guarantee backward compatibility, and tries to
% assure the color passed to \textsf{hycolor} fits its design expectations.
% We begin by completely expanding the argument, for the argument may be in
% macro form.
%    \begin{macrocode}
\def\ef@parsePDFColor#1{\edef\ef@color@arg{#1}\ef@parsePDFColori}
\def\ef@parsePDFColori{%
    \expandafter\ef@@parsePDFColor\ef@color@arg; ; ; ; ; ;\\}
\def\ef@@parsePDFColor#1 #2 #3 #4 #5 #6\\{%
%    \end{macrocode}
% \paragraph*{Test of gray or model or named.} We test whether this is either a named color, or gray color space. This is
% the case if \texttt{\#3} is either `\texttt{g;}' or `\texttt{;}'.
%    \begin{macrocode}
    \def\argii{#2}\def\ef@cmp{g;}%
    \ifx\argii\ef@cmp
%    \end{macrocode}
% It is a PDF color \texttt{<num> g}, if \textsf{xcolor} is loaded we pass it as
% \verb![gray]{#1}!; otherwise, we return \texttt{\#1 g}.
%    \begin{macrocode}
        \expandafter\ifx\csname convertcolorspec\endcsname\relax
            \def\ef@colorSpec@out{#1 g}\else
            \def\ef@colorSpec@out{[gray]{#1}}%
        \fi
    \else\ifx\argii\ef@semicolon
%    \end{macrocode}
% Either a named color or a single number
%    \begin{macrocode}
        \expandafter\ifx\csname convertcolorspec\endcsname\relax
            \ef@isitnamed#1\ef@nil
            \ifx\ef@latex@color\ef@n
                \ef@stripsemi#1\@nil
                \edef\ef@colorSpec@out{\ef@colorSpec@out\space g}%
            \else
                \ef@stripsemi#1\@nil
                \PackageWarning{eforms}{Color specification
                `\ef@colorSpec@out' not supported\MessageBreak
                without xcolor, using a black color}
                \def\ef@colorSpec@out{0 g}%
            \fi
        \else  % xcolor
%    \end{macrocode}
% If \textsf{xcolor} is not loaded, we do nothing; if \texttt{xcolor} is loaded
% we determine if the first token is a `\texttt{[}' indicating a color space
% specification; or if the first token is a number, indicating that this is a number.
% If neither cases are detected, we assume a named color.
%    \begin{macrocode}
            \ef@isitnamed#1\ef@nil
            \ifx\ef@latex@color\ef@n
                \ef@stripsemi#1\@nil
                \edef\ef@colorSpec@out{[gray]{\ef@colorSpec@out}}%
            \else
                \ef@stripsemi#1\@nil
                \edef\ef@colorSpec@out{\ef@colorSpec@out}%
            \fi
        \fi
    \else % not semicolon
        \def\argiv{#4}\def\ef@cmp{rg;}%
%    \end{macrocode}
% \paragraph*{RGB test.} If \texttt{\#4} is either `\texttt{rg;}' or `\texttt{;}', we are RGB
%    \begin{macrocode}
        \ifx\argiv\ef@cmp
            \expandafter\ifx\csname convertcolorspec\endcsname\relax
                \def\ef@colorSpec@out{#1 #2 #3 rg}\else
                \def\ef@colorSpec@out{[rgb]{#1,#2,#3}}\fi
        \else\ifx\argiv\ef@semicolon
            \expandafter\ifx\csname convertcolorspec\endcsname\relax
            \ef@stripsemi#1 #2 #3\@nil
            \edef\ef@colorSpec@out{\ef@colorSpec@out\space rg}\else
            \ef@stripsemi#3\@nil
            \edef\ef@colorSpec@out{[rgb]{#1,#2,\ef@colorSpec@out}}\fi
%                Help!: {#1,#2,\ef@colorSpec@out}%
        \else
            \def\argv{#5}\edef\ef@cmp{k;}
%    \end{macrocode}
% \paragraph*{CMYK test.} If \texttt{\#5} is either `\texttt{k;}' or `\texttt{;}', we are CMYK.
%    \begin{macrocode}
        \ifx\argv\ef@cmp
            \expandafter\ifx\csname convertcolorspec\endcsname\relax
                \def\ef@colorSpec@out{#1 #2 #3 #4 k}\else
                \def\ef@colorSpec@out{[cmyk]{#1,#2,#3,#4}}\fi
        \else
            \ifx\argv\ef@semicolon
                \ef@stripsemi#1 #2 #3 #4\@nil
                \expandafter\ifx\csname convertcolorspec\endcsname\relax
                \edef\ef@colorSpec@out{\ef@colorSpec@out\space k}\else
                \ef@stripsemi#4\@nil
                \edef\ef@colorSpec@out{%
                    [cmyk]{#1,#2,#3,\ef@colorSpec@out}}\fi
        \else\ef@parseColor@iv
    \fi\fi\fi\fi\fi\fi
}
%    \end{macrocode}
%    \end{macro}
% Error messages for color spec parsing
%    \begin{macrocode}
\def\ef@parseColor@iv{\PackageError{AeB}{%
    The number of arguments
    is incorrect.\MessageBreak I was expecting
    1, 3, or 4 components of color}{Specify the correct number of
    components for the color space.}}
%    \end{macrocode}
%
% \subsection{Support for setting the calculation order}
%
% The command \cs{calcOrder}\DescribeMacro{\calcOrder} set the order of calculation
% of all calculation fields listed in this comma delimited list.
%\begin{verbatim}
%   \calcOrder{field3,field2,field1}
%\end{verbatim}
%    \begin{macrocode}
\def\calcOrder#1{\let\efCalcOrder\@gobble
    \@for\coi:=#1\do{\edef\efCalcOrder{\efCalcOrder,"\coi"}}%
    \edef\efCalcOrder{[\efCalcOrder]}}
\@onlypreamble\calcOrder
\def\efCalcOrder{[]}
%    \end{macrocode}
%
%\subsection{Symbol Definitions}
%
% Some definitions for radio fields and checkboxes
%    \begin{macrocode}
\def\eq@check{4}
\def\eq@circle{l}
\def\eq@cross{8}
\def\eq@diamond{u}
\def\eq@square{n}
\def\eq@star{H}
%    \end{macrocode}
%    \begin{macro}{\symbolchoice}
% Use this macro to change the symbol used in radio and
% checkboxes.  The default is \cmd{\eq@check}.  This macro takes one
% argument: permissible values are: check, circle, cross, diamond,
% square, and star. The definition of \cs{symbolchoice} is given
% above in the definition of \cs{@eqsymbolchoice}.
%    \begin{macrocode}
\let\symbolchoice\@eqsymbolchoice
%    \end{macrocode}
% Set the default to `check'.
%    \begin{macrocode}
\symbolchoice{check}
%    \end{macrocode}
%    \end{macro}
%\subsection{Convenience Commands}
%
%\subsubsection{Writing Actions}
%
% Writing actions for \pkg{eforms} requires certain key-value combinations. The following
% commands provides the correct syntax, the code is inserted via the required
% argument of each.
%
%    \begin{macro}{\JS}
%    \begin{macro}{\Named}
%    \begin{macro}{\Next}
%    \begin{macro}{\toggleAttachmentsPanel}
% Convenience commands for writing JavaScript and for executing named events.
% \changes{v2.3.5}{2020/11/20}{Remove \string\cs{URI} from \string\pkg{eforms}, already
% defined in \string\pkg{insdljs}. (Jason G.)}
%    \begin{macrocode}
\providecommand{\JS}[1]{/S/JavaScript/JS(#1)}
%\newcommand{\URI}[1]{/S/URI/URI(#1)}
\providecommand{\Named}[1]{/S/Named/N/#1}
\newcommand{\Next}[1]{/Next<<#1>>}
\providecommand{\toggleAttachmentsPanel}[2]{%
    \setLink[\Border{0 0 0}\A{\Named{ShowHideFileAttachment}}]
    {\textcolor{#1}{#2}}}%
%    \end{macrocode}
%    \end{macro}
%    \end{macro}
%    \end{macro}
%    \end{macro}
% When entering triggers into the AA dictionary, use these commands (all the ones of
% the form \cs{AA<Name>}. These all have one argument, the action to take, usually
% a JavaScript action.  There is now a routine in the parsing of the AA dictionary
% that searches for \cs{AACalculate}, if found, sets the switch \cs{ifisCalculate}.
% For this search to succeed, these helper macros must be used.
%    \begin{macro}{\AAMouseUp}
%    \begin{macro}{\AAMouseDown}
%    \begin{macro}{\AAMouseEnter}
%    \begin{macro}{\AAMouseExit}
%    \begin{macro}{\AAOnFocus}
%    \begin{macro}{\AAOnBlur}
% This set of six appear in the Action tab of the field properties dialog box.
% Here, we don't assume the actions are JavaScript actions. The \cs{AAMouseUp}
% key is normally not used, rather, the mouse up action can be define using
% the \cs{A} key, such as \verb!\A{\JS{app.beep(0)}}!
%    \begin{macrocode}
\newcommand{\AAMouseUp}[1]{/U<<#1>>}
\newcommand{\AAMouseDown}[1]{/D<<#1>>}
\newcommand{\AAMouseEnter}[1]{/E<<#1>>}
\newcommand{\AAMouseExit}[1]{/X<<#1>>}
\newcommand{\AAOnFocus}[1]{/Fo<<#1>>}
\newcommand{\AAOnBlur}[1]{/Bl<<#1>>}
%    \end{macrocode}
%    \end{macro}
%    \end{macro}
%    \end{macro}
%    \end{macro}
%    \end{macro}
%    \end{macro}
%    \begin{macro}{\AAFormat}
%    \begin{macro}{\AAKeystroke}
%    \begin{macro}{\AAValidate}
%    \begin{macro}{\AACalculate}
% These four triggers  are JavaScript only, so we use \cs{JS}
% to insert the appropriate code. In the user interface, they appear
% in the Format, Validate, and Calculate tabs of the text field, and the
% combo box.
%    \begin{macrocode}
\newcommand{\AAFormat}[1]{/F<<\JS{#1}>>}
\newcommand{\AAKeystroke}[1]{/K<<\JS{#1}>>}
\newcommand{\AAValidate}[1]{/V<<\JS{#1}>>}
\newcommand{\AACalculate}[1]{/C<<\JS{#1}>>}
%    \end{macrocode}
%    \end{macro}
%    \end{macro}
%    \end{macro}
%    \end{macro}
%    \begin{macro}{\AAPageOpen}
%    \begin{macro}{\AAPageClose}
%    \begin{macro}{\AAPageVisible}
%    \begin{macro}{\AAPageInvisible}
% Page actions associated with a form field (PDF 1.5 or later)
% Originally designed for multimedia annotations, but can be
% use for form fields as well. There is no UI for these JavaScripts. The
% order of execution of these is Page 1: PV, Page 1: PO, Page 2: PV,
% Page 1: PC, Page 2: PO.
%\changes{v1.0b}{2006/10/14}
%{
%    Added page actions for annotations, PDF 1.5 or later
%}
%    \begin{macrocode}
\newcommand{\AAPageOpen}[1]{/PO<<\JS{#1}>>}
\newcommand{\AAPageClose}[1]{/PC<<\JS{#1}>>}
\newcommand{\AAPageVisible}[1]{/PV<<\JS{#1}>>}
\newcommand{\AAPageInvisible}[1]{/PI<<\JS{#1}>>}
%    \end{macrocode}
%    \end{macro}
%    \end{macro}
%    \end{macro}
%    \end{macro}
%
%\subsubsection{Saving Paths}
%
%    \begin{macro}{\definePath}
% A convenience command for saving a path or an URL, usage,
%\begin{verbatim}
%\definePath{\myPath}{http://www.example.edu/~dpstory}
% ...
%\setLink[\A{\URI{\myPath}}]{Go There!}
%\end{verbatim}
%    \begin{macrocode}
\newcommand{\definePath}[1]{\def\ef@ctrlName{#1}%
    \hyper@normalise\ef@definePath}
\def\ef@definePath#1{\expandafter\xdef\ef@ctrlName{#1}}
%    \end{macrocode}
%    \end{macro}
%
% \subsection{Annotation and Field Flags}
%
% \subsubsection{Annotation Flag \texttt{/F} Definitions}\label{F}
%
%    \begin{macrocode}
\def\FHidden{2}         % bit 2: hidden field
\def\FPrint{4}          % bit 3: print (we set this bit by default)
\def\FNoPrint{-4}       % bit 3: -print (this clears the bit)
\def\FNoView{32}        % bit 6: no view
\def\FLock{128}         % bit 8: locked field (PDF 1.4)
%    \end{macrocode}
%\paragraph*{Notes:}\par
%\noindent\begin{tabular}{ll}
% Visible (and printable)      &|\F\FPrint|\\
%Hidden but printable          &|\F\FNoView|\\
%Visible but doesn't print     &|\F\FNoPrint|\\
%Hidden (and does not print)   &|\F\FHidden|
%\end{tabular}
%
%\subsubsection{Field Flags \texttt{/Ff} Definitions}\label{Ff}
%    \begin{macrocode}
\def\FfReadOnly{1}                   % all
\def\FfRequired{2}                   % all
\def\FfNoExport{4}                   % all
\def\FfMultiline{4096}               % text
\def\FfPassword{8192}                % text
\def\FfNoToggleToOff{16384}          % radio
\def\FfRadio{32768}                  % radio
\def\FfPushButton{65536}             % Push button
\def\FfCombo{131072}                 % choice
\def\FfEdit{262144}                  % combo
\def\FfSort{524288}                  % choice
\def\FfFileSelect{1048576}           % text (PDF 1.4)
\def\FfMultiSelect{2097152}          % choice (PDF 1.4)
\def\FfDoNotSpellCheck{4194304}      % text, combo (PDF 1.4)
\def\FfDoNotScroll{8388608}          % text (PDF 1.4)
\def\FfComb{16777216}                % text (PDF 1.5)
\def\FfRadiosInUnison{33554432}      % radio (PDF 1.5)
\def\FfCommitOnSelChange{67108864}   % choice (PDF 1.5)
\def\FfRichText{33554432}            % radio (PDF 1.5)
%    \end{macrocode}
% The keys \texttt{/F} and \texttt{/Ff} will be additive, that is,
% for example, \verb+\F\FHidden\F\FPrint+ will get \texttt{/F 6},
% a field that is both printable and hidden. These are the only
% flags that are additive this way. The following to macros
% are supportive of the additivity.
%    \begin{macrocode}
\def\getFfValue/Ff#1\@nil{\def\eq@FfValue{#1}}
\def\getFValue/F#1\@nil{\def\eq@FValue{#1}}
\def\@getCmdName#1{\edef\@CmdName{\expandafter\@gobble\string#1}}
%    \end{macrocode}
%
% \subsection{The \texorpdfstring{\protect\cs{every\dots}}{\textbackslash{every...}} Commands}
%
%    \begin{macro}{\everyTextField}
%    \begin{macro}{\everySigField}
% Insert optional arguments for every \cs{textField}.
%    \begin{macrocode}
\newcommand{\everyTextField}[1]{\def\every@TextField{#1}}
\def\every@TextField{}
\newcommand{\everySigField}[1]{\def\every@sigField{#1}}
\def\every@sigField{}
%    \end{macrocode}
%    \end{macro}
%    \end{macro}
%    \begin{macro}{\everyCheckBox}
% Here, you can control the appearance of all the standard checkboxes, also
% effects radio fields of shortquiz and quiz.
%    \begin{macrocode}
\newcommand{\everyCheckBox}[1]{\def\every@CheckBox{#1}}
\def\every@CheckBox{}
%    \end{macrocode}
%    \end{macro}
%    \begin{macro}{\everyRadioButton}
% Pass key-values to every radio button through this key.
%    \begin{macrocode}
\newcommand{\everyRadioButton}[1]{\def\every@RadioButton{#1}}
\def\every@RadioButton{}
%    \end{macrocode}
%    \end{macro}
%    \begin{macro}{\everyButtonField}
%    \begin{macro}{\everyPushButton}
%    \begin{macro}{\everyListBox}
%    \begin{macro}{\everyComboBox}
%    \begin{macro}{\everyLink}
% Here, you can control the appearance of all the standard buttons.
%    \begin{macrocode}
\newcommand{\everyButtonField}[1]{\def\every@ButtonField{#1}}
\def\every@ButtonField{}
\newcommand{\everyPushButton}[1]{\def\every@PushButton{#1}}
\def\every@PushButton{}
\newcommand{\everyListBox}[1]{\def\every@listBox{#1}}
\newcommand{\everyComboBox}[1]{\def\every@comboBox{#1}}
\def\every@listBox{}\def\every@comboBox{}
\newcommand{\everyLink}[1]{\def\every@Link{#1}}
\def\every@Link{}
%    \end{macrocode}
%    \end{macro}
%    \end{macro}
%    \end{macro}
%    \end{macro}
%    \end{macro}
%
% \subsection{Set Field Properties}
%
% These are the two commands that process the field properties. This is
% done by calling \cs{processAppArgs} to loop through pairs of tokens
% and executing them, as explained in the discussion of \cs{processAppArgs}.
% \medskip\par\noindent
% The \DescribeMacro{\ef@djXPD}\cs{ef@djXPD} command attempts to reduce the size of a form field created by \app{xelatex} so its
% dimensions are consistent with what \app{pdflatex}/\app{Distiller} produce.
%    \begin{macrocode}
\def\ef@adjrectWH#1{\dimen@ii#1\relax
    \ifx\eq@rectW\@empty\else
        \eflength\eq@rectW\relax
        \advance\eflength\dimen@ii
        \edef\eq@rectW{\the\eflength}%
        \eflength\eq@rectH\relax
        \advance\eflength\dimen@ii
        \edef\eq@rectH{\the\eflength}%
    \fi
}
\def\ef@djXPD{\ifxetex\ifmakeXasPD\ef@adjrectWH{2\b@}\fi
    \else\ifmakePDasX\ef@adjrectWH{-2\b@}\fi\fi}
%    \end{macrocode}
%    \begin{macro}{\eq@setButtonProps}
% This macro measure the width of the largest text on defined for the
% button, and then passes the this info on to a driver specific macro
% \#1, the first parameter.
%\begin{verbatim}
% #1 is the driver specific macro to build the button widget
% #2 are the button properties
%\end{verbatim}
%\DescribeMacro\btnSpcr\hskip-\marginparsep\texttt{\darg{\ameta{content}}} Adds space on either
%side of a pushbutton caption; the amount of the spacing is the width of \ameta{content} in the current
%current font, the fontsize is the one determined by the \cs{textSize} key.
%\changes{v2.5m}{2012/01/27}{Added \string\cs{ef@btnspcr} to give user
% ability to adjust the spacing around text for a button.}
% \changes{v2.3.5}{2020/09/15}{Added \string\cs{tbnSpcr} to populate \string\cs{eq@btnspcr}}
%    \begin{macrocode}
\def\btnSpcr#1{\def\ef@btnspcr{#1}}
\def\ef@btnspcr{}
%    \end{macrocode}
%    \subsubsection{Enhanced preview}\label{pmpv}
%\changes{v2.10}{2019/03/16}{Added \string\cs{pmpvOn} and \string\cs{pmpvOff}, changed from
% \string\cs{pmcaOn} and \string\cs{pmcaOff}}
%\changes{v2.10}{2019/03/16}{Alias for \string\cs{pmcaOn} and \string\cs{pmpvOn}, etc.}
%    \begin{macrocode}
\@ifundefined{ifefpmpv}{\newif\ifefpmpv\efpmpvfalse}{}
\def\pmpvOn{\efpmpvtrue\PackageInfo{eforms}
  {Turning on enhanced preview (\string\pmpvOn)}}\let\pmcaOn\pmpvOn
\def\pmpvOff{\efpmpvfalse\PackageInfo{eforms}
  {Turning off enhanced preview (\string\pvpmOff)}}\let\pmcaOff\pmpvOff
%    \end{macrocode}
%    Enhanced preview attempts to typeset into the document the value of the \cs{CA} key (for push buttons)
%    and the \cs{V} key for text fields and choice fields. This is useful for those using a non-conforming
%    PDF reader such as \app{sumatraPDF}. The enhanced preview is activated by expanding \cs{pmpvOn}
%    (\textbf{p}oor \textbf{m}an's \textbf{p}re\textbf{v}iew). A summary of the effects is describe below.\medskip
%    \begin{description}
%    \item[push buttons] The \cs{CA} entry is always
%    displayed; however, when the \emph{background color \emph{(\cs{BG})} is transparent}, the key-value entry generated by \cs{CA}
%    \emph{is removed}. This is to avoid two overlaying captions, one typeset into the document, the other part
%    of the button appearance.
%    \item[text and choice fields] The \cs{V} entry \emph{is set to empty} (when \cs{pmpvOn} is active),
%    but the value of the \cs{V} key is typeset into the document; this is to avoid two overlaying
%    values within the field. There is one special case, when the field is \emph{hidden}; in this case,
%    the value of the field is restored. Hidden text fields are used by the \pkg{acrotex} packages to hold
%    information that can later be retrieved.
%    \item[check box and radio button fields] These two cases are handled similar to \textbf{choice fields}. For these
%    types of fields, the values is typically a mark: a check, an cross, a star, and so on. For preview purposes,
%    \pkg{eforms} defines the declarative command \cs{pmpvMrk} that takes one argument, the mark to be used. The
%    package declares |\pmpvMrk{X}|, another good choice is |\pmpvMrk{$\checkmark$}|.
%    \end{description}
%    With respect to the enhanced preview, the local command \cs{tops}, used within the argument of \cs{V} or \cs{CA}, is
%    \cs{let} to \cs{texorpdfstring}. Use \DescribeMacro\tops\cs{tops} to offer an alternate text to the value of \cs{CA}
%    of \cs{V}. The \cs{V} key of radio button and check box fields do not handle the \cs{tops} command.
%
%    For example,
%    \begin{quote}\small
%       |\pushButton[\CA{\tops{Tap Me}{Push Me}}]{pbFld}{}{11bp}|
%    \end{quote}
%    This button will preview with the caption as `Tap Me', but will appear within
%    a conforming PDF reader as `Push Me'; however, if the background color is
%    transparent (|\BG{}|), `Tap Me' will be the (typeset) caption even in a
%    conforming PDF reader. (This is to avoid overlaying captions.) It is
%    important\marginpar{\raggedleft \textbf{Important!}} to say that the final
%    document should be compiled with \cs{previewOff} and \cs{pmpvOff} opened in
%    \app{Adobe Reader} and saved to obtain proper appearances of the form
%    fields.
%
%    The arguments of \cs{CA} and \cs{V} should be text, and must note have any {\TeX} formatting
%    like \cs{textbf}, \cs{textit}, and so on. That's not to say such formatting cannot be obtained.
%    There is an undocumented macro, \DescribeMacro\pmpvFmt\cs{pmpvFmt} that takes one argument, the argument being the
%    enhanced preview text. To illustrate its usage, we use the above example,
%    \begin{quote}\small
%       |\pushButton[\cmd{\let\pmpvFmt\textbf}|\\
%       \null\quad|\CA{\tops{Tap Me}{Push Me}}]{pbFld}{}{11bp}|
%    \end{quote}
%    Now, the enhanced preview will read `\textbf{Tap Me}'. Pass |\let\pmpvFmt\textbf| through
%    the \cs{cmd} key, the changes are local to the field.
%    \changes{v2.11}{2019/05/24}{Enhanced preview introduced}
%
%    \subsubsection{Set push button properties}
%    \leavevmode\DescribeMacro\ef@adjcapfont^^A
%    When \cs{pmpvOn} is expanded, the caption text is previewed, we try to adjust
%    the typeset caption font size to match the font size used by the push button.
%    \changes{v2.3.5}{2020/09/15}{Added \string\cs{ef@adjcapfont}}
%    \begin{macrocode}
\def\ef@adjcapfont{%
  \dimen@=\eq@textSize\b@
  \dimen@1.00375\dimen@
  \edef\eq@textSize@pt{\strip@pt\dimen@}%
  \fontsize{\eq@textSize@pt}{0}\selectfont
}
\def\eq@setButtonProps#1#2{\makeJSspecials
  \processAppArgs#2\end\@nil  % set widget properties
%    \end{macrocode}
%    If \cs{eq@rectW} is not a positive dimension, mark it as empty.
%    \changes{v2.3.5}{2020/09/15}{If \string\cs{rectW} is not a positive dimension,
%    make it empty for push buttons.}
%    \begin{macrocode}
  \ifx\eq@rectW\@empty\else
    \ifdim\eq@rectW>0pt \else\let\eq@rectW\@empty\fi\fi
%    \end{macrocode}
%    Re-scale dimensions as requested. If \cs{eq@rectW} is empty, then the width is calculated
%    from the \cs{CA} key; re-scaling is ignored in this case.
%    \changes{v2.3.5}{2020/09/15}{Added \string\cs{ef@optscale}}
%    \changes{v2.3.5}{2020/09/15}{If \string\cs{rectW} is not a positive dimension,
%    make it empty for push buttons.}
%    \begin{macrocode}
  \ifx\eq@rectW\@empty\else\expandafter\ef@optscale\fi
%    \end{macrocode}
%    Coordinate the values of \cs{BC} and \cs{W}, if one is empty
%    the other is too.
%    \changes{v2.9f}{2017/01/01}{BC=empty iff W=0 or empty}
%    \begin{macrocode}
  \ifx\eq@BC\@empty\@eqW{}\else
    \if\eq@W@value0\let\eq@BC\@empty\fi\fi
  \Hy@pdfstringfalse
  \ifx\eq@rectW\@empty
    \ifnum\eq@textSize=0 \else
%    \end{macrocode}
% If \cs{rectW} is empty and \cs{textSize} is not zero, we calculate with
% width of the caption on the button by first adjusting the font size
% to properly gauge the width of the text. This may not be really accurate
% because the font used by tex will no doubt be different from the font used
% by the button.
%    \begin{macrocode}
      \ef@adjcapfont
    \fi
%    \end{macrocode}
% If the button is beveled, we pad the width by 2 times the width of the border,
% the beveled edge taking up a width approx equal to the border.
%    \begin{macrocode}
    \dimen@\eq@W@value\b@
    \def\eq@S@B{B}\ifx\eq@S@value\eq@S@B
      \def\eq@btn@sp{\hbox to2\dimen@{\hfill}}%
    \else
      \def\eq@btn@sp{\hbox to\dimen@{\hfill}}%
    \fi
    \Hy@pdfstringtrue
    \expandafter\def\expandafter
      \ef@btnspcr\expandafter{\ef@btnspcr\eq@btn@sp}%
    \sbox{\eq@tmpbox}{\ef@btnspcr\eq@CA\ef@btnspcr}%
      \eq@tmpdima=\wd\eq@tmpbox
    \sbox{\eq@tmpbox}{\ef@btnspcr\eq@RC\ef@btnspcr}%
    \ifdim\eq@tmpdima>\wd\eq@tmpbox\else
      \eq@tmpdima=\wd\eq@tmpbox\fi%
    \sbox{\eq@tmpbox}{\ef@btnspcr\eq@AC\ef@btnspcr}%
    \ifdim\eq@tmpdima>\wd\eq@tmpbox\else
      \eq@tmpdima=\wd\eq@tmpbox\fi
%    \end{macrocode}
%    (2017/01/22) If X-like, increase by 2bp
%    \changes{v2.9k}{2017/01/22}{If X-like, increase by 2bp}
%    \begin{macrocode}
    \ifmakePDasX\advance\eq@tmpdima2\b@\fi
    \wd\eq@tmpbox=\eq@tmpdima
  \else % if \eq@rectW is not \@empty
    \wd\eq@tmpbox=\eq@rectW
  \fi
%    \end{macrocode}
%    (2016/12/22) \cs{ef@djXPD} adjusts the size of the field dimensions, if \cs{makeXasPD} is true.\\
%    (2019/03/16) Insert \cs{PMPV} if \cs{if@efpmpv} is true.
%    \changes{v2.10}{2019/03/16}{Insert \string\cs{PMPV} if \string\cs{if@efpmpv} is true}
%    \changes{v2.11}{2019/05/24}{Added \string\cs{ef@isBGtransparent}}
%    \begin{macrocode}
  \ifefpmpv\Hy@pdfstringfalse
%    \end{macrocode}
%    If this push button has a transparent background, we remove the \texttt{/CA} key.
%    \begin{macrocode}
    \ifx\ef@isBGtransparent\ef@YES
      \let\ef@kvCA\@empty\fi
%    \end{macrocode}
%    Adjust the font size for the preview text.
%    \begin{macrocode}
    \ifx\ef@CApv\ef@YES\ef@adjcapfont\PMPV{\eq@CA}\fi\fi
  \ef@djXPD#1%
}
%    \end{macrocode}
%    \end{macro}
%    \subsubsection{Set other properties of other fields}
%    \begin{macro}{\eq@setWidgetProps}
% Same as \cmd{\eq@setButtonProps} but does not measure the width of the
% field.  Simply lays in the optional parameters that modify the appearance
% then calls the driver specific macro to build the widget.
%\begin{verbatim}
% #1 is the driver specific macro to build the widget
% #2 are the widget properties
%\end{verbatim}
%    \begin{macrocode}
\def\eq@setWidgetProps#1#2{\makeJSspecials
  \processAppArgs#2\end\@nil  % set widget properties
%    \end{macrocode}
%    Coordinate the values of \cs{BC} and \cs{W}, if one is empty
%    the other is too. This rule does not apply to links.
%    \changes{v2.9f}{2017/01/01}{BC=empty iff W=0 or empty}
%    \begin{macrocode}
  \ifx\annot@type@link\annot@type\else
%    \end{macrocode}
%    Re-scale dimensions as requested
%    \changes{v2.3.5}{2020/09/15}{Added \string\cs{ef@optscale}}
%    \begin{macrocode}
    \ef@optscale
    \ifx\eq@BC\@empty\@eqW{}\else
    \if\eq@W@value0\let\eq@BC\@empty\fi\fi
  \fi
%    \end{macrocode}
%    (2016/12/22) \cs{ef@djXPD} adjusts the size of the field dimensions, if \cs{makeXasPD} is true.
%    \begin{macrocode}
  \ef@lateWidgetOpts
  \ifefpmpv\Hy@pdfstringfalse
%    \end{macrocode}
%     Having an enhanced preview generates several problems. Text fields are sometimes hidden and their values
%     are used to store information. So, if the field is hidden, we give no enhanced preview.
%     \changes{v2.11}{2019/05/24}{Enhanced preview for text and choice fields}
%    \begin{macrocode}
    \ifx\ef@isHidden\ef@YES
%    \end{macrocode}
%     This is a tricky part. The value \cs{eq@VSAVE} was earlier \cs{let} to the original value
%     of \cs{eq@V} (at that time, we set \cs{eq@V} to \cs{@empty}), now we restore its original
%     value now that we know this field is hidden.
%    \begin{macrocode}
      \let\eq@V\eq@VSAVE\else
      \ifx\ef@Vpv\ef@YES\PMPV{\pmpvV}\fi\fi
  \fi
  \global\let\ef@lateWidgetOpts\relax
  \ef@djXPD#1%
}
%    \end{macrocode}
%    \end{macro}
% We now begin creating various commands for creating Acrobat Form fields. These
% fall into four categories: \hyperref[choice]{Choice Fields} (list box and the combo box),
% \hyperref[button]{Button Fields} (push button, check box, and radio button), \hyperref[textfield]{Text Fields},
% and \hyperref[sigfield]{Signature Fields}.
%
% Each of these fields has an optional first parameter which is used to
% change the appearance properties of the field, to set action of the
% field, and so on.  For this optional parameter, we sanitize certain
% characters so they can be used in, for example, urls, or JavaScript strings.
%\changes{v2.5k}{2011/01/19}
%{%
%   Added double quotes as other to the \string\cs{ef@sanitize@toks} command to
%   keep Babel from changing the JavaScript string. This definition remains
%   in effect only for the optional argument for form fields and links.
%}
%    \begin{macrocode}
\def\ef@sanitize@toks{\@makeother\~\@makeother\#\@makeother\&%
    \@makeother\"\@makeother\_}
%    \end{macrocode}
%    \begin{macro}{\efKern}
% A convenience macro to adjust spacing between fields.
%\begin{itemize}
%   \item \texttt{\#1} is the spacing for non-\app{xelatex} apps
%   \item \texttt{\#2} is the spacing for \app{xelatex}
%\end{itemize}
% The \app{xelatex} application includes the border width as part of the bounding rectangle
%dimensions; all other PDF creators do not include the border as part of the dimensions
%of the bounding rectangle. As a result, this wider boundary is known to {\TeX} as it lays out the line. For
% non-\app{xelatex} applications, the boundary width is \emph{invisible} to {\TeX}. The command \cs{ef@adjHWxetex} tries to
% adjust the dimensions for \app{xelatex} so the form field dimensions are the same over all platforms. The boundary
% is still ``visible'' in {\TeX} space.
%
%\paragraph*{Sample uses:}
%\begin{enumerate}
%   \item Two contiguous fields of the same border color \verb!\efKern{1bp}{-1bp}!. The right and
%         left borders exactly overlap.
%   \item Two contiguous fields of different border colors \verb!\efKern{2bp}{0bp}!. The right
% and left border are contiguous but do not overlap.
%\end{enumerate}
%    \begin{macrocode}
\newcommand\efKern[2]{\ifxetex\kern#2\else\kern#1\fi}
%    \end{macrocode}
%    \end{macro}
%    The use of \cs{efKern} has been largely subsumed by \cs{olBdry} and \cs{cgBdry}, however.
%    \DescribeMacro{\olBdry}\cs{olBdry} and \DescribeMacro{\cgBdry}\cs{cgBdry}
%    are convenience commands to setting the gap between two contiguous fields.
%    \cs{olBdry} gives positions the fields so that the boundary lines overlap
%    (ol) and \cs{cgBdry} positions the field so the boundary lines are
%    contiguous (cg).
%    \changes{v2.9d}{2016/12/22}{Added \string\cs{olBdry} and\string\cs{cgBdry}}
%    \begin{macrocode}
\newcommand\olBdry{\bgroup\ifxetex
  \@tempdima-\g@eq@W@value@bp
  \edef\@mtkern{\the\@tempdima}\else
  \@tempdima2\b@\advance\@tempdima-\g@eq@W@value@bp
  \edef\@mtkern{\the\@tempdima}\fi\kern\@mtkern\egroup
}
%    \end{macrocode}
%    \DescribeMacro\cgBdry\hskip-\marginparsep\texttt{[\ameta{length}]}
%    This command inserts spacing between to fields so they are
%    contiguous (boundary-to-boundary); it insert additional space based in the
%    optional argument \ameta{length}. Normally, when the boundaries are the same
%    color, \cs{olBdry} is used; otherwise, \cs{cgBdry} looks best.
%    \changes{v2.9h}{2017/01/15}{Add optional argument to \string\cs{cgBdry}}
%    \begin{macrocode}
\newcommand\cgBdry[1][0\b@]{\bgroup\def\ef@rgi{#1}\ifx\ef@rgi\@empty
  \def\ef@rgi{0\b@}\fi\setlength{\@tempdima}{\ef@rgi}%
  \ifxetex\else\addtolength{\@tempdima}{2\b@}\fi
  \kern\@tempdima\egroup\ignorespaces}
%    \end{macrocode}
%    \DescribeMacro{\volBdry}\cs{volBdry} and \DescribeMacro{\vcgBdry}\cs{vcgBdry}
%    are convenience commands to setting the gap between two vertically oriented fields.
%    \cs{volBdry} gives positions the fields so that the boundary lines overlap
%    (ol) and \cs{vcgBdry} positions the field so the boundary lines are
%    contiguous (cg).
%    \changes{v2.9h}{2017/01/15}{Added \string\cs{volBdry} and\string\cs{vcgBdry}}
%    \changes{v2.9n}{2017/09/04}{Added \string\cs{efSupprIndent}}
%    \changes{v2.3.3}{2020/04/15}{Use \string\cs{toks@} in \string\cs{efSupprIndent};
%     use of \string\cs{ef@scratchtoks} proved to be a problem.}
%    \begin{macrocode}
\newcommand{\efSupprIndent}{\toks@=\expandafter{\the\everypar}%
  \everypar{{\setbox\z@\lastbox}\clubpenalty\@M
  \everypar=\expandafter{\the\toks@}}}
\newcommand\volBdry{\bgroup\parskip0pt\relax\@@par\nointerlineskip
  \olBdry\egroup\efSupprIndent}
\newcommand\vcgBdry{\@ifstar{\edef\ef@offset{\the\parskip}\vcgBdry@i}
  {\def\ef@offset{0pt}\vcgBdry@i}}
\newcommand\vcgBdry@i[1][0bp]{\bgroup
  \setlength{\ef@dimena}{#1-\ef@offset}\parskip0pt\relax
  \par\nointerlineskip\cgBdry[\ef@dimena]%
  \egroup\ignorespaces\efSupprIndent}
%    \end{macrocode}
%
% \subsection{Choice Fields}\label{choice}
%
% This is the form template used for all choice fields, list box and combo box. Within
% the optional argument, the following aliases are used: \cs{nameuse} for \cs{@nameuse}
% and \cs{tops} for \cs{texorpdfstring}.
%    \begin{macrocode}
\def\common@choiceCode{%
  /Subtype/Widget
  /T (\Fld@name)
  /FT/Ch
  \eq@Ff
  \eq@F
  \eq@TU
  \eq@TI
  /BS << \eq@W\eq@S >>
  /MK <<\eq@R\eq@BC\eq@BG\eq@mkIns>>
  /DA (\eq@DA)
  /Opt [\eq@Opt]
  \eq@DV\eq@V
  \eq@A\eq@AA
  \eq@rawPDF
}
%    \end{macrocode}
% Sets the dimensions of the fields/links based on \texttt{\#1} (width)
% and \texttt{\#2} (height).
%    \begin{macrocode}
\def\eqf@setDimens#1#2{\@eqrectW{#1}\@eqrectH{#2}}
%    \end{macrocode}
% The \cs{ef@pdfstrCLOpt} command runs the option array for list and combo box through
% \cs{pdfstringdef}. There are two forms:
% \begin{enumerate}
%   \item |[(1)(Zur Info)][(2)(Bitte um R\"{u}cksprache)]|
%   \item |(Zur Info)(Bitte um R\"{u}cksprache)|
%\end{enumerate}
% In the code below, we distinguish these two cases based on the first character:
% if \texttt{[} then it is form (1), otherwise, it is form (2). The \cs{pdfstringdef}
% will convert the accented character \verb!\"{u}! into a PD1 character. (2015/06/06)
% the token \DescribeMacro\passthruCLOpts\cs{passthruCLOpts} is used to pass through
% raw option syntax for combo and list boxes; this allows a unicode notation.
%    \begin{macrocode}
\let\ef@@nil\relax
\def\@gobbleto@@nil#1\ef@@nil{}%
%    \end{macrocode}
% When \cs{passthruCLOpts} (or \texttt{*}) is used, there may be a
% lingering space that survives after the closing right brace of the
% argument, we attempt to absorb it with \cs{@gobbleto@@nil} immediately
% after the \cs{g@addto@macro}.
%    \begin{macrocode}
\long\def\g@addto@macrogobble#1#2{\g@addto@macro{#1}{#2}\@gobbleto@@nil}
%    \end{macrocode}
% \changes{v2.8a}{2015/07/15}{New def for \string\cs{passthruCLOpts}, and modified
% \string\cs{ef@pdfstrCLOpts} and its spawn.}
% (2015/07/15) New def for \cs{passthruCLOpts}, and modified \cs{ef@pdfstrCLOpts} and its spawn.
%    \begin{macrocode}
\def\passthruCLOpts{*}
\def\ef@pdfstrCLOpt{\Hy@unicodefalse\def\eq@Opt{}\ef@pdfstrCLOpti}
\def\ef@pdfstrCLOpti{\@ifnextchar\ef@@nil{\@gobble}{\ef@pdfstrCLOptia}}
%    \end{macrocode}
% The final argument of \cs{comboBox} or \cs{listBox} is any of three forms.
%\begin{verbatim}
% {[(1)(Socks)][(2)(Shoes)][(3)(Pants)][(4)(Shirts)][(5)(Tie)]}
% {(Socks)(Shoes)(Pants)(Shirts)(Tie)}
% {\passthruCLOpts{%
%   [(Euro)<\unicodeStr(myEuro)>]%
%   [(Yen)<\unicodeStr(myYen)>]%
%   [(Sheqel)<\unicodeStr(mySheqel)>]%
%   [(Pound)<\unicodeStr(myPound)>]%
%   [(Franc)<\unicodeStr(myFranc)>]}
%\end{verbatim}
% As long as the first token is not \cs{passthruCLOpts}, the argument may also be
% a combination of the first two, such as
%\begin{verbatim}
%   {[(1)(Socks)](Shoes)[(3)(Pants)](Shirts)[(5)(Tie)]}
%\end{verbatim}
% The  trick is to identify the first token of the group (\texttt{[} or \texttt{(}).
% When the first token is \cs{passthruCLOpts}, the extensive parse does not take place,
% and the whole argument just passes to \cs{eq@Opt}.
%\medskip\noindent
% If next char is `\texttt{[}' then we have an array element of the form
% \verb![(val)(appr)]!, we \cs{@gobble} the left brace in this case. If
% the next char is not a left bracket, go to next state.
%    \begin{macrocode}
\def\ef@pdfstrCLOptia{\@ifnextchar[{\expandafter
  \ef@pdfstrOptWBii\@gobble}{\ef@pdfstrCLOptb}}
%    \end{macrocode}
% If not a left bracket, check for \cs{passthruCLOpts}, which expands to `\texttt{*}'.
% If `\texttt{*}', then we pass the whole argument to \cs{eq@Opt} and end parsing. If
% not `\texttt{*}', check for parentheses.
%    \begin{macrocode}
\def\ef@pdfstrCLOptb{\@ifstar{\g@addto@macrogobble\eq@Opt}
  {\ef@pdfstrOptWPi}}
%    \end{macrocode}
%\changes{v2.6d}{2014/04/26}{Added additional parsing, so spaces
% can occur between arguments.}
% (2014/04/26) Added additional parsing, so spaces
% can occur between arguments.
%\par\medskip\noindent
% Following a bracket, the next token can only be a left parenthesis, if not error.
%    \begin{macrocode}
\def\ef@pdfstrOptWBii{%
  \@ifnextchar({\ef@pdfstrOptWBiia}{\PackageError{eforms}
    {Left parenthesis expected here}{}}}
%    \end{macrocode}
% We process the element of the form by pushing \verb![(val)(appr)]! onto the
% \cs{eq@Opt} stack.
%    \begin{macrocode}
\def\ef@pdfstrOptWBiia(#1){%
  \g@addto@macro\eq@Opt{[(}%
  \pdfstringdef\@optTokstemp{#1}%
  \expandafter\g@addto@macro\expandafter\eq@Opt
    \expandafter{\@optTokstemp}%
  \@ifnextchar({\ef@pdfstrOptWBiib}{\PackageError{eforms}
    {Left parenthesis expected here}{}}}
%    \end{macrocode}
% Get the second element of the array, which is enclosed in parentheses.
% push its value onto the stack too.
%    \begin{macrocode}
\def\ef@pdfstrOptWBiib(#1){%
  \g@addto@macro\eq@Opt{)(}%
  \pdfstringdef\@optTokstemp{#1}%
  \expandafter\g@addto@macro\expandafter\eq@Opt
    \expandafter{\@optTokstemp}%
  \g@addto@macro\eq@Opt{)]}%
%    \end{macrocode}
% Now after removing the final right bracket, return to the beginning,
% which is the \cs{ef@pdfstrCLOpti} command.
%    \begin{macrocode}
  \expandafter\ef@pdfstrCLOpti\@gobble}
%    \end{macrocode}
% Process the purely parentheses version of the array.
%    \begin{macrocode}
\def\ef@pdfstrOptWPi{\@ifnextchar\ef@@nil{\@gobble}{\ef@pdfstrOptWPii}}
\def\ef@pdfstrOptWPii(#1){%
  \g@addto@macro\eq@Opt{(}%
  \pdfstringdef\@optTokstemp{#1}%
  \expandafter\g@addto@macro\expandafter\eq@Opt
    \expandafter{\@optTokstemp}%
  \g@addto@macro\eq@Opt{)}%
%    \end{macrocode}
% Return to the beginning, \cs{ef@pdfstrCLOpti}.
%    \begin{macrocode}
  \ef@pdfstrCLOpti}
%    \end{macrocode}
% \subsubsection{List Box}\label{listbox}
%
%\changes{v2.4}{2020/11/28}{Allow \string\cs{list@@Box} to obey \string\cs{pdfStringsOn}}
%
% The main list box code that can be used to build list box commands, such as
% \cs{listBox}, defined below. Within
% the optional argument, the following aliases are used: \cs{nameuse} for \cs{@nameuse}
% and \cs{tops} for \cs{texorpdfstring}.
%\begin{Verbatim}[xleftmargin=\parindent,fontsize=\small,
%commandchars=!(),codes={\catcode`\%=9}]
%#1 = !textsf(optional, used to enter any modification of the appearance/actions)
%#2 = !textsf(the title of the button field)
%#3 = !textsf(the width of the bounding rectangle)
%#4 = !textsf(the height of the bounding rectangle)
%#5 = !textsf(list of key-values for list/combo box)
%#6 = !textsf(not used)
%#7 = !textsf(set props and driver macros)
%#8 = !textsf(every macros)
%\end{Verbatim}
%    \begin{macrocode}
\def\annot@type@listbox{listbox}
\bgroup\obeyspaces
\gdef\list@@Box{\begingroup\global\let =\pdfSP
\ef@sanitize@toks\list@@@Box}\egroup
\newcommand\list@@@Box[8]{\endgroup\begingroup
  \let\nameuse\@nameuse\let\tops\texorpdfstring
  \edef\annot@type{\annot@type@listbox}%
  \pdfstringdef\Fld@name{#2}\dl@paramlocal
%    \end{macrocode}
% Run \texttt{\#5} through \cs{ef@pdfstrCLOpt}.
%    \begin{macrocode}
  \expandafter\ef@pdfstrCLOpt#5\ef@@nil
  \ifefpmpv\let\eq@Opt\@empty\fi
  \eqf@setDimens{#3}{#4}%
  \def\eq@DA{\eq@textFont\space\eq@textSize\space Tf \eq@textColor}%
  \@processEvery#8\end\noindent#6#7{#1}}
\bgroup\obeyspaces
\gdef\listBox{\global\let =\dl@sp@ce}\egroup
%    \end{macrocode}
%    \begin{macro}{\listBox}\hskip-\marginparsep
%    \texttt{[\ameta{options}]\darg{\ameta{fld-name}}\darg{\ameta{wd}}^^A
%    \darg{\ameta{ht}}\darg{\ameta{face-export-list}}} The user interface to creating a list box field.
%\begin{Verbatim}[xleftmargin=\parindent,fontsize=\small,
%commandchars=!(),codes={\catcode`\%=9}]
%#1 = !textsf(optional, used to enter any modification of the appearance/actions)
%#2 = !textsf(the title of the list box field)
%#3 = !textsf(the width of the bounding rectangle)
%#4 = !textsf(the height of the bounding rectangle)
%#5 = !textsf(the face values/export values of list)
%\end{Verbatim}
%\noindent Below are the default settings for the list box
%    \begin{macrocode}
\def\listBoxDefaults{\W{1}\S{I}\F{\FPrint}\BC{0 0 0}}
%    \end{macrocode}
% We \cs{let} space to \cs{pdfSP}, to convert tex space to pdf space (\cs{040})
% We sanitize the optional first parameter.
%    \begin{macrocode}
\bgroup\obeyspaces
\gdef\listBox{\begingroup\global\let =\pdfSP
\ef@sanitize@toks\ef@listbox}\egroup
%    \end{macrocode}
% Now capture the rest of the parameters, and pass them all to the low
% level \cs{list@@Box}.
%    \begin{macrocode}
\newcommand\ef@listbox[5][]{\endgroup
  \mbox{\list@@Box{#1}{#2}{#3}{#4}{#5}{}{\eq@setWidgetProps
    \eq@choice@driver}{\listBoxDefaults\every@listBox}}}
%    \end{macrocode}
%    \end{macro}
% \subsubsection{Combo Box}\label{combobox}
%
%\changes{v2.4}{2020/11/28}{Allow \string\cs{combo@@Box} to obey \string\cs{pdfStringsOn}}
%
% Within the optional argument, the following aliases are used: \cs{nameuse} for \cs{@nameuse}
% and \cs{tops} for \cs{texorpdfstring}.
%\begin{Verbatim}[xleftmargin=\parindent,fontsize=\small,
%commandchars=!(),codes={\catcode`\%=9}]
%#1 = !textsf(optional, used to enter any modification of the appearance/actions)
%#2 = !textsf(the title of the button field)
%#3 = !textsf(the width of the bounding rectangle)
%#4 = !textsf(the height of the bounding rectangle)
%#5 = !textsf(list of key-values for list/combo box)
%#6 = !textsf(not used)
%#7 = !textsf(set props and driver macros)
%#8 = !textsf(every macros)
%\end{Verbatim}
%    \begin{macrocode}
\def\annot@type@combobox{combobox}
\bgroup\obeyspaces
\gdef\combo@@Box{\begingroup\global\let =\pdfSP
\ef@sanitize@toks\combo@@@Box}\egroup
\newcommand\combo@@@Box[8]{\endgroup\begingroup
  \let\nameuse\@nameuse\let\tops\texorpdfstring
  \edef\annot@type{\annot@type@combobox}%
  \@eqFf{\FfCombo}\pdfstringdef\Fld@name{#2}\dl@paramlocal
%    \end{macrocode}
% Run \texttt{\#5} through \cs{ef@pdfstrCLOpt}.
%    \begin{macrocode}
  \expandafter\ef@pdfstrCLOpt#5\ef@@nil
  \ifefpmpv\let\eq@Opt\@empty\fi
  \eqf@setDimens{#3}{#4}%
  \def\eq@DA{\eq@textFont\space\eq@textSize\space Tf \eq@textColor}%
  \@processEvery#8\end\noindent#6#7{#1}}
%    \end{macrocode}
%    \begin{macro}{\comboBox}\hskip-\marginparsep
%    \texttt{[\ameta{options}]\darg{\ameta{fld-name}}\darg{\ameta{wd}}^^A
%    \darg{\ameta{ht}}\darg{\ameta{face-export-list}}}\\
% A general combo box command.
%\begin{Verbatim}[xleftmargin=\parindent,fontsize=\small,
%commandchars=!(),codes={\catcode`\%=9}]
%#1 = !textsf(optional, used to enter any modification of the appearance/actions)
%#2 = !textsf(the title of the radio field)
%#3 = !textsf(the width of the bounding rectangle)
%#4 = !textsf(the height of the bounding rectangle)
%#5 = !textsf(the face values/export values of combo box)
%\end{Verbatim}
%\noindent The default settings for the \cs{comboBox}
%    \begin{macrocode}
\def\comboBoxDefaults{\W{1}\S{I}\F{\FPrint}\BC{0 0 0}}
%    \end{macrocode}
% We sanitize the optional first parameter.
%    \begin{macrocode}
\bgroup\obeyspaces
\gdef\comboBox{\begingroup\global\let =\pdfSP
\ef@sanitize@toks\ef@combobox}\egroup
%    \end{macrocode}
% Close group, get rest of the arguments.
%    \begin{macrocode}
\newcommand\ef@combobox[5][]{\endgroup
  \mbox{\combo@@Box{#1}{#2}{#3}{#4}{#5}{}{\eq@setWidgetProps
    \eq@choice@driver}{\comboBoxDefaults\every@comboBox}}}
%    \end{macrocode}
%    \end{macro}
% \subsection{Button Fields}\label{button}
% Here is the field template for push button fields. Within
% the optional argument, the following aliases are used: \cs{nameuse} for \cs{@nameuse}
% and \cs{tops} for \cs{texorpdfstring}.\par\medskip\noindent
% (2018/11/10) \cs{eq@APX} determines if \cs{eq@I} is empty, if not we supply the normal appearance
% for the button to be the same appearance as set by \cs{eq@I}.
% \changes{v2.9.21}{2018/11/10}{Introduce \string\cs{eq@APX} an `intelligent' AP key}
%    \begin{macrocode}
\def\eq@APX{\ifx\eq@I\@empty\eq@AP\else
  /AP<< \expandafter\get@NIR\eq@I\@nil\space>>\fi}
\def\get@NIR/#1 #2\@nil{/N #2}
\def\common@pushButtonCode{%
  /Subtype/Widget
  /T (\Fld@name)
  /FT/Btn
  \eq@Ff
  \eq@TU
  \eq@H
  \eq@F
  /BS <<\eq@W\eq@S >>
%    \end{macrocode}
%   (2018/11/10) Remove conditional \string\cs{eq@AP}, some PDF viewers
%    use AP to build the normal appearance for buttons.
%    \changes{v2.9.21}{2018/11/10}{Remove conditional \string\cs{eq@AP}
%    in \string\cs{common@pushButtonCode}}.
%    \begin{macrocode}
  /MK <<\eq@R\eq@BC\eq@BG%
    \ef@kvCA\ef@kvRC\ef@kvAC\eq@IconMK\eq@mkIns>>
  \eq@APX
  /DA (\eq@DA)
  \eq@A\eq@AA
  \eq@rawPDF
}
%    \end{macrocode}
% Here is the field template for check boxes and radio button fields fields.
%    \begin{macrocode}
\def\radio@parent{%
  /DA (\eq@DA)%
  /FT/Btn%
  \eq@Ff%
  \eq@TU%
  \eq@DV%
\expandafter\ifx\csname kids@\Fld@name\endcsname\relax\else
  /Kids [\@nameuse{kids@\Fld@name}]%
\fi
\ifx\ef@multigroupradios\ef@YES
\expandafter\ifx\csname radio@\Fld@name\endcsname\relax\else
  /Opt[\@nameuse{radio@\Fld@name}]\fi\fi
  /T(\Fld@name)%
  \eq@V
}
%    \end{macrocode}
%    \begin{macrocode}
\def\common@RadioCode{%
  /Subtype/Widget
\ifuseNewRadios
  \expandafter\ifx\csname radio@\Fld@name\endcsname\relax\else
  /Parent \@nameuse{parent@\Fld@name}\fi
\else
  /T (\Fld@name)
  /FT/Btn
  \eq@Ff
  \eq@F
  \eq@TU
  \eq@DV\eq@V
  /DA (\eq@DA)
\fi
  /BS <<\eq@W\eq@S>>
\ifx\eq@AP\@empty
  /AP<< /N <<\eq@On<<>>>> >>
  \eq@MK
\else
    \eq@AP
\fi
  \eq@AS
%  \eq@DV\eq@V
  \eq@A\eq@AA
  \eq@rawPDF
}
%    \end{macrocode}
%    \begin{macrocode}
\def\common@CheckCode{%
  /Subtype/Widget
  /T (\Fld@name)
  /FT/Btn
  \eq@Ff
  \eq@F
  \eq@TU
  /BS <<\eq@W\eq@S>>
%    \end{macrocode}
% \changes{v1.0d}{2007/09/01}{%
%    Corrected a problem with radio buttons. The problem was created by earlier
%    changes so that the \string\textbf{AP} key could play a greater role in creating
%    different appearances.
%}
%    \begin{macrocode}
\ifx\eq@AP\@empty
  /AP<< /N <<\eq@On<<>>>> >>
  \eq@MK
\else
    \eq@AP
\fi
  /DA (\eq@DA)
  \eq@AS
  \eq@DV\eq@V
  \eq@A\eq@AA
  \eq@rawPDF
}
%    \end{macrocode}
% \subsubsection{Push Button}\label{pushbutton}
% Here is the basic command for creating a button field. This is the building block
% for all other buttons.
%\changes{v2.0b}{2008/06/19}
%{
%   Added code to take the dimensions and pass them through
%   \string\cs{setlength}, so that calc type dimensions can be used
%   when setting the dimensions of this widget. Made the same
%   changes for all AcroForm elements and for linking.
%}
%\changes{v2.4}{2020/11/28}{Rewrite of \string\cs{push@@Button} to support \string\cs{pdfSpacesOn}}
%\begin{Verbatim}[xleftmargin=\parindent,fontsize=\small,
%commandchars=!(),codes={\catcode`\%=9}]
%#1 = !textsf(optional, used to enter any modification of the appearance/actions)
%#2 = !textsf(the title of the button field)
%#3 = !textsf(the width of the bounding rectangle)
%#4 = !textsf(the height of the bounding rectangle)
%#5 = !textsf(not used)
%#6 = !textsf(set props and driver macros)
%#7 = !textsf(every macros)
%\end{Verbatim}
%    \begin{macrocode}
\def\annot@type@button{pushbtn}
\bgroup\obeyspaces
\gdef\push@@Button{\begingroup\global\let =\pdfSP
\ef@sanitize@toks\push@@@Button}\egroup
\newcommand\push@@@Button[7]{\endgroup\begingroup
  \let\nameuse\@nameuse\let\tops\texorpdfstring%
  \edef\annot@type{\annot@type@button}%
  \pdfstringdef\Fld@name{#2}\dl@paramlocal%
  \makeJSspecials\ef@preProcDefns%
  \def\eq@Ff{/Ff \FfPushButton}%
  \def\eq@DA{\eq@textFont\space\eq@textSize\space Tf \eq@textColor}%
  \eqf@setDimens{#3}{#4}%
  \@processEvery#7\end\noindent#5#6{#1}}

%    \end{macrocode}
%    \begin{macro}{\pushButton}
%    \hskip-\marginparsep\texttt{[\ameta{options}]\darg{\ameta{fld-name}}\darg{\ameta{wd}}\darg{\ameta{ht}}}
%    The user command for creating a push button form field.
%\begin{Verbatim}[xleftmargin=\parindent,fontsize=\small,
%commandchars=!(),codes={\catcode`\%=9}]
%#1 = !textsf(optional, used to enter any modification of the appearance/actions)
%#2 = !textsf(the title of the button field)
%#3 = !textsf(the width of the bounding rectangle)
%#4 = !textsf(the height of the bounding rectangle)
%\end{Verbatim}
%The defaults for \cs{pushButton}
%    \begin{macrocode}
\def\pushButtonDefaults{%
  \W{1}\S{B}\F{\FPrint}\BC{0 0 0}
  \H{P}\BG{.7529 .7529 .7529}}
%    \end{macrocode}
% We sanitize the optional first parameter.
% \changes{v2.10}{2019/03/16}{Implement pdfSP for push button fields}
%    \begin{macrocode}
\bgroup\obeyspaces
\gdef\pushButton{\begingroup\global\let =\pdfSP
\ef@sanitize@toks\ef@pushbutton}\egroup
%    \end{macrocode}
%   End sanitize group, get the rest of the parameters, pass them to the low-level \cs{push@@Button}
%    \begin{macrocode}
\newcommand\ef@pushbutton[4][]{\endgroup
  \mbox{\push@@Button{#1}{#2}{#3}{#4}{}{%
    \eq@setButtonProps\eq@Button@driver}%
    {\pushButtonDefaults\every@PushButton}}}
%    \end{macrocode}
%    \end{macro}
% \subsubsection{Check Box}\label{checkbox}
% The basic command for creating check boxes. For \emph{enhanced preview} we define
% \DescribeMacro\pmpvMrk\cs{pmpvMrk\darg{mrk}}, which defines \cs{pmpv@mrk} that is eventually
% used in the core of \cs{ef@Bbox}.
%\changes{v2.4}{2020/11/28}{Rewrite of \string\cs{check@@Box} to support \string\cs{pdfSpacesOn}}
%    \begin{macrocode}
\def\pmpvMrk#1{\def\pmpv@mrk{#1}}
\pmpvMrk{X}
%    \end{macrocode}
%\begin{Verbatim}[xleftmargin=\parindent,fontsize=\small,
%commandchars=!(),codes={\catcode`\%=9}]
%#1 = !textsf(optional, used to enter any modification of the appearance/actions)
%#2 = !textsf(the title of the button field)
%#3 = !textsf(the width of the bounding rectangle)
%#4 = !textsf(the height of the bounding rectangle)
%#5 = !textsf(the `on' value, usually Yes)
%#6 = !textsf(not used)
%#7 = !textsf(set props and driver macros)
%#8 = !textsf(every macros)
%\end{Verbatim}
%    \begin{macrocode}
\def\annot@type@checkbox{checkbox}
\bgroup\obeyspaces
\gdef\check@@Box{\begingroup\global\let =\pdfSP
\ef@sanitize@toks\check@@@Box}\egroup
\newcommand\check@@@Box[8]{\endgroup\begingroup
  \let\nameuse\@nameuse\let\tops\texorpdfstring
  \let\#\ef@Hx\edef\annot@type{\annot@type@checkbox}%
  \pdfstringdef\Fld@name{#2}\@eqAS{Off}\dl@paramlocal
  \def\@eqDV##1{\def\eq@arg{##1}\ifx\eq@arg\@empty\let\eq@DV\@empty
    \else\ifpdfmarkup\def\eq@DV{/DV(##1) cvn }\else
    \def\eq@DV{/DV/##1}\fi\fi}%
  \def\@eqV##1{\def\eq@arg{##1}\ifx\eq@arg\@empty
%    \end{macrocode}
%    Provide enhanced preview for checkboxes
%    \begin{macrocode}
    \let\eq@V\@empty\else\def\pmpvV{\pmpv@mrk}\ifpdfmarkup
      \def\eq@V{/V(##1) cvn }\else
      \def\eq@V{/V/##1}\fi\@eqAS{##1}\fi
      \ifefpmpv\let\eq@V\@empty\fi}%
  \eqf@setDimens{#3}{#4}%
  \ifpdfmarkup\def\eq@On{(#5) cvn }\else
      \def\eq@On{/#5}\fi\@eqtextFont{ZaDb}%
  \def\eq@DA{\eq@textFont\space\eq@textSize\space Tf \eq@textColor}%
  \@eqMK{\eq@R\eq@BC\eq@BG/CA(\symbol@choice)\eq@mkIns}%
  \@processEvery#8\end\noindent#6#7{#1}}
%    \end{macrocode}
%    \begin{macro}{\checkBox}
%    \hskip-\marginparsep\texttt{[\ameta{options}]\darg{\ameta{fld-name}}\darg{\ameta{wd}}^^A
%    \darg{\ameta{ht}}\darg{\ameta{on-value}}}
%    The user interface for creating a check box.
%\begin{Verbatim}[xleftmargin=\parindent,fontsize=\small,
%commandchars=!(),codes={\catcode`\%=9}]
%#1 = !textsf(optional, used to enter any modification of the appearance/actions)
%#2 = !textsf(the title of the check box field)
%#3 = !textsf(the width of the bounding rectangle)
%#4 = !textsf(the height of the bounding rectangle)
%#5 = !textsf(the `on value' or export value, the default is "Yes")
%\end{Verbatim}
%\noindent The default for \cs{checkBox}
%    \begin{macrocode}
\def\checkBoxDefaults{\F{\FPrint}\W{1}\S{S}\BC{0 0 0}}
%    \end{macrocode}
%    \begin{macrocode}
\bgroup\obeyspaces%
\gdef\checkBox{\begingroup\global\let =\pdfSP%
\ef@sanitize@toks\ef@checkbox}\egroup
%    \end{macrocode}
%    End the sanitizing group, and get the argument, pass them on to
%    the low-level command \cs{check@@Box}.
%    \begin{macrocode}
\newcommand{\ef@checkbox}[5][]{\endgroup
  \mbox{\check@@Box{#1}{#2}{#3}{#4}{#5}{}{\eq@setWidgetProps
  \eq@Check@driver}{\checkBoxDefaults\every@CheckBox}}}
%    \end{macrocode}
%    \end{macro}
% \subsubsection{Radio Buttons}\label{radiobutton}
% The basic command for creating radio button fields. As of the version dated 2019/06/14,
% radio button code was re-written so they operate in certain non-conforming PDF readers. The old code,
% which has proven to be reliable for many years,
% is  available when \DescribeMacro\useNewRadiosOff\cmd{\useNewRadiosOff} is expanded. This is the default.
%
% When \DescribeMacro\useNewRadiosOn\cmd{\useNewRadiosOn} is expanded, the new code for radio buttons is used.
% As a result, you need \emph{three compiles} to bring the AUX files up to date. The AUX files now
% contain PDF object references to radio buttons. Because this new scheme degrades the experience
% of creating radio buttons (heavy I/O usage), use this option if you and the consumers of your PDF\emph{ do not use}
% Adobe PDF viewers (on a desktop/laptop).
%
% Use the default setting (\cmd{\useNewRadiosOff}) when you will view the PDF in AA/AR and save it; otherwise, if the PDF is to be viewed in non-conforming
% PDF readers, never having been saved using AA/AR, use the \cmd{\useNewRadiosOn} setting.
% \changes{v2.3}{2019/06/14}{Rewrote radio buttons so they work in non-conforming PDF readers}
% \changes{v2.4}{2020/11/28}{Rewrite of \string\cs{radio@@Button} to support \string\cs{pdfSpacesOn}}
%
% \medskip\noindent\textbf{Discussion.} The default scheme is to create the radio buttons \begin{quote}\footnotesize
% \cs{radioButton\darg{\ameta{name}}\darg{\ameta{width}}\darg{\ameta{height}}} ...\\
% ... \cs{radioButton\darg{\ameta{name}}\darg{\ameta{width}}\darg{\ameta{height}}}\end{quote} as separate widgets.
% When loaded by AA/AR, these viewers do some internal arranging automatically; they build appearances, among other things. Saving
% the PDF from AA/AR save this appearances within the PDF file, so that those using non-conforming PDF viewers will see
% a correct representation of the fields. (The fields themselves may not be functional, depending on the PDF viewer.)
%
% \def\Odict{<<}\def\Cdict{>>}
%
% When \cs{useNewRadiosOn} is expanded, instead of creating separate widgets, we create a radio button field,
%\begin{quote}\offinterlineskip\ttfamily\obeyspaces\obeylines
%14 0 obj
%\Odict\space/DA (/ZaDb 9 Tf 0 g)/FT/Btn/Ff 32768/DV/Three
%/Kids [15 0 R 16 0 R 17 0 R ]/T(Group1)/V/Two \Cdict
%endobj
%\end{quote}
% \texttt{/FT} and \texttt{Ff} entries declare this field to be a radio button field. Note that
% the object has a \texttt{/Kids} entry that references the individual widgets. The field contains
% the default value (\texttt{/DV}), if any, and the initial value (\texttt{/V}), if any.
% A typical widget referenced by the \texttt{/Kids} entry is,
%\begin{quote}\offinterlineskip\ttfamily\obeyspaces\obeylines
%15 0 obj
%\Odict
%/Type /Annot
%/Rect [172.716 654.735 185.716 667.735]
%/Subtype/Widget /Parent 14 0 R/BS \Odict\space/W 1/S/S\Cdict
%/AP\Odict\space/N\Odict/One\Odict\Cdict\space\Cdict\space\Cdict /MK \Odict\space/BC [0 0 0]/CA(l) \Cdict
%/AS/Off
%\Cdict
%endobj
%\end{quote}
%This is a widget annotation with \texttt{/Rect} entry, as well as other entries. The \cs{/AS} key determines
%whether this widget is `on' or not. This one is off. Some of the non-conforming PDF viewers parse this
%structure better than the old scheme.
%
%The next six lines are new (2019/06/14), to support the new radio button structure.
%    \begin{macrocode}
\newif\ifuseNewRadios \useNewRadiosfalse
\def\useNewRadiosOn{\useNewRadiostrue}
\def\useNewRadiosOff{\useNewRadiosfalse}
\let\ef@OptArray\@empty
\let\ef@KidsArray\@empty
\let\ef@lateWidgetOpts\relax
\def\annot@type@radio{radiobtn}
%    \end{macrocode}
% You can create one or more copies of a group of radio buttons, whether more copies of a
% group is not known until the end of the document, so we must set some properties at
% the end of the document. This command stores information as a function of the field name
% for later use at the end of the document.
%    \begin{macrocode}
\def\ef@NewRadiosLateOpts{%
%    \end{macrocode}
%     If this is a multi-group radio button field that work independently, we adjust the `on' value, as in this case,
%     the `on' value is referenced by an index, 0, 1, 2,...
%    \begin{macrocode}
  \@nameuse{multigroup@\Fld@name}%
  \ifx\ef@multigroupradios\ef@YES
    \ifpdfmarkup
      \def\eq@On{(\@nameuse{radioindex@\Fld@name}) cvn }\else
      \def\eq@On{/\@nameuse{radioindex@\Fld@name}}\fi
    \expandafter\ifx\csname OnVal@\Fld@name\endcsname\relax
      \@eqAS{Off}\else % today
    \ifnum\@nameuse{OnVal@\Fld@name}=%
      \@nameuse{radioindex@\Fld@name}\relax
    \@eqAS{\@nameuse{radioindex@\Fld@name}}\else\@eqAS{Off}\fi\fi
  \else
    \edef\x{\@nameuse{OnVal@\Fld@name}}%
    \ifx\x\ef@thisChoice\@eqAS{\@nameuse{OnVal@\Fld@name}}\else
      \@eqAS{Off}\fi
  \fi
%    \end{macrocode}
%    If these are groups of radio button fields that light up in unison, we
%    take to get the \cs{/AS} entry right.
%    \begin{macrocode}
  \@nameuse{uniradios@\Fld@name}%
  \ifx\isRadiosInUnison\ef@YES
    \edef\x{\@nameuse{value@\Fld@name}}%
    \ifx\x\ef@thisChoice\expandafter\@eqAS
      \expandafter{\ef@thisChoice}\else\@eqAS{Off}\fi
  \fi
}
%    \end{macrocode}
%     We track the widget belonging to a given field name (\cs{Fld@name}), these radio
%     indices are used when we have multiple groups if independent radios.
%    \begin{macrocode}
\def\ef@advanceRadioIndex#1{\bgroup
  \@tempcnta\@nameuse{radioindex@#1}\relax
  \advance\@tempcnta\@ne
  \csarg\xdef{radioindex@#1}{\the\@tempcnta}\egroup}
%    \end{macrocode}
% Some utility commands that are written to the AUX file.
%    \begin{macrocode}
\def\radioChoices#1{\csarg\xdef{radio@#1}}
\def\radioKids#1{\csarg\xdef{kids@#1}}
%    \end{macrocode}
%    A warning message when not all the PDF objects have
%    been resolved. We try to emit only one message per
%    compile.
%    \begin{macrocode}
\def\ef@radioWarning{\PackageWarningNoLine{eforms}
  {Not all PDF object references have\MessageBreak
   been resolved, keep compiling}}
%    \end{macrocode}
%     The command that detects whether any object reference is not defined.
%    \begin{macrocode}
\def\ef@@radioWarning{%
  \ifx\ef@radioWarning\relax\else
    \@ifundefined{kids@\Fld@name}
    {\ef@radioWarning\global\let\ef@radioWarning\relax}{}\fi
  \ifx\ef@radioWarning\relax\else
    \@ifundefined{radio@\Fld@name}
    {\ef@radioWarning\global\let\ef@radioWarning\relax}{}\fi
  \ifx\ef@radioWarning\relax\else
    \@ifundefined{parent@\Fld@name}
    {\ef@radioWarning\global\let\ef@radioWarning\relax}{}\fi
}
%    \end{macrocode}
%    Finally, we arrive at the low-level radio button command, where
%    changes for the 2019/06/14 version have been made.
%The arguments for this command are,
%\begin{Verbatim}[xleftmargin=\parindent,fontsize=\small,
%commandchars=!(),codes={\catcode`\%=9}]
%#1 = !textsf(optional, used to enter any modification of the appearance/actions)
%#2 = !textsf(the title of the button field)
%#3 = !textsf(the width of the bounding rectangle)
%#4 = !textsf(the height of the bounding rectangle)
%#5 = !textsf(the `on' value, usually Yes)
%#6 = !textsf(not used)
%#7 = !textsf(set props and driver macros)
%#8 = !textsf(every macros)
%\end{Verbatim}
%    \begin{macrocode}
\bgroup\obeyspaces
\gdef\radio@@Button{\begingroup\global\let =\pdfSP
\ef@sanitize@toks\radio@@@Button}\egroup
\newcommand\radio@@@Button[8]{\endgroup\begingroup\let\#\ef@Hx
  \let\nameuse\@nameuse\let\tops\texorpdfstring
  \edef\annot@type{\annot@type@radio}%
  \pdfstringdef\Fld@name{#2}%
%    \end{macrocode}
%    Additional code for the new scheme.
%    \begin{macrocode}
  \ifuseNewRadios
    \ef@@radioWarning
    \@ifundefined{rad@\Fld@name}{\let\isRadioParent\ef@YES
    \global\let\ef@OptArray\@empty
    \global\let\ef@KidsArray\@empty
    \csarg\gdef{radioindex@\Fld@name}{-1}\expandafter
    \global\csarg\let{rad@\Fld@name}\@empty}%
      {\let\isRadioParent\ef@NO}%
    \edef\ef@OptArray{\@nameuse{rad@\Fld@name}}%
    \g@addto@macro\ef@OptArray{(#5)}% opt
    \csarg\xdef{rad@\Fld@name}{\ef@OptArray}%
    \ifx\isRadioParent\ef@YES
      \def\y{\expandafter\string\noexpand}%
      \edef\x{\noexpand\immediate\noexpand\write\noexpand\@auxout
        {\y\radioChoices{\Fld@name}{\noexpand
        \@nameuse{rad@\Fld@name}}}}%
      \def\z{\expandafter\AtEndDocument\expandafter{\x}}\z
      \edef\x{\noexpand\immediate\noexpand\write\noexpand\@auxout
        {\y\radioKids{\Fld@name}{\noexpand
          \@nameuse{kid@\Fld@name}}}}%
      \def\z{\expandafter\AtEndDocument\expandafter{\x}}\z
    \fi
    \ef@advanceRadioIndex{\Fld@name}%
    \@nameuse{multigroup@\Fld@name}%
  \fi
  \@eqAS{Off}\dl@paramlocal
  \def\@eqDV##1{\def\eq@arg{##1}\ifx\eq@arg\@empty\let\eq@DV\@empty
    \else\ifpdfmarkup\def\eq@DV{/DV(##1) cvn }\else
    \def\eq@DV{/DV/##1}\fi\fi}%
  \def\@eqV##1{\Hy@pdfstringfalse\edef\pmpvV{##1}%
    \Hy@pdfstringtrue
    \edef\eq@arg{##1}%
    \if$\eq@arg$\else
      \ifpdfmarkup
        \edef\eq@V{/V(##1) cvn }\else
        \edef\eq@V{/V/##1}\fi
        \@eqAS{##1}\fi
    \if$\eq@arg$%
    \else
      \csarg\xdef{OnVal@\Fld@name}{##1}\fi
    \ifefpmpv
      \gdef\ef@lateWidgetOpts{\if$\pmpvV$\else\def\pmpvV{\pmpv@mrk}\fi}%
      \let\eq@V\@empty\else\global\let\ef@lateWidgetOpts\relax\fi
    }%
  \eqf@setDimens{#3}{#4}%
  \ifpdfmarkup\def\eq@On{(#5) cvn }\else\def\eq@On{/#5}\fi
  \def\ef@thisChoice{#5}%
  \def\eq@Ff{/Ff \FfRadio}\@eqtextFont{ZaDb}%
  \def\eq@DA{\eq@textFont\space\eq@textSize\space Tf \eq@textColor}%
  \@eqMK{\eq@R\eq@BC\eq@BG/CA(\symbol@choice)\eq@mkIns}%
  \@processEvery#8\end\noindent#6#7{#1}}
%    \end{macrocode}
%    \begin{macro}{\radioButton}\hskip-\marginparsep\texttt{[\ameta{options}]\darg{\ameta{fld-name}}\darg{\ameta{wd}}^^A
%    \darg{\ameta{ht}}\darg{\ameta{on-value}}}
%    User interface to creating a radio button.
%\begin{Verbatim}[xleftmargin=\parindent,fontsize=\small,
%commandchars=!(),codes={\catcode`\%=9}]
%#1 = !textsf(optional, used to enter any modification of the appearance/actions)
%#2 = !textsf(the title of the radio button field)
%#3 = !textsf(the width of the bounding rectangle)
%#4 = !textsf(the height of the bounding rectangle)
%#5 = !textsf(the `on value' or export value, the default is "Yes")
%\end{Verbatim}
%The default for \cs{radioButton}
%    \begin{macrocode}
\def\radioButtonDefaults{\W{1}\S{S}\BC{0 0 0}\F{\FPrint}}
%    \end{macrocode}
% We sanitize the optional first parameter.
%    \begin{macrocode}
\bgroup\obeyspaces
\gdef\radioButton{\begingroup\global\let =\pdfSP
\ef@sanitize@toks\ef@radiobutton}\egroup
%    \end{macrocode}
% End sanitize group, get rest of parameters and pass to the low-level
% command \cs{radio@@Button}
%    \begin{macrocode}
\newcommand{\ef@radiobutton}[5][]{\endgroup
  \mbox{\radio@@Button{#1}{#2}{#3}{#4}{#5}{}{\eq@setWidgetProps
  \eq@Radio@driver}{\radioButtonDefaults\every@RadioButton}}}
%    \end{macrocode}
%    \end{macro}
%
% \subsection{Text Field}\label{textfield}
%
% The template for a text field. Within
% the optional argument, the following aliases are used: \cs{nameuse} for \cs{@nameuse}
% and \cs{tops} for \cs{texorpdfstring}.
%\changes{v2.4}{2020/11/28}{Rewrite of \string\cs{text@@Field} to support \string\cs{pdfSpacesOn}}
%    \begin{macrocode}
\def\common@TextFieldCode
{%
  /Subtype/Widget
  /T (\Fld@name)
  /FT/Tx
  \eq@Ff
  \eq@F
  \eq@Q
  \eq@TU
  \eq@MaxLen
  /BS <<\eq@W\eq@S>>
  /MK <<\eq@R\eq@BC\eq@BG\eq@mkIns>>
  /DA (\eq@DA)
  \eq@DV\eq@V
  \eq@RV\eq@DS
  \eq@A\eq@AA
  \eq@rawPDF
}
%    \end{macrocode}
% The basic text field macro for constructing all other text fields.
%The arguments for this command are,
%\begin{Verbatim}[xleftmargin=\parindent,fontsize=\small,
%commandchars=!(),codes={\catcode`\%=9}]
%#1 = !textsf(optional, used to enter any modification of the appearance/actions)
%#2 = !textsf(the title of the button field)
%#3 = !textsf(the width of the bounding rectangle)
%#4 = !textsf(the height of the bounding rectangle)
%#5 = !textsf(not used)
%#6 = !textsf(set props and driver macros)
%#7 = !textsf(every macros)
%\end{Verbatim}
%    \begin{macrocode}
\def\annot@type@text{textfld}
\bgroup\obeyspaces
\gdef\text@@Field{\begingroup\global\let =\pdfSP
\ef@sanitize@toks\text@@@Field}\egroup
\newcommand\text@@@Field[7]{\endgroup\begingroup
  \let\nameuse\@nameuse\let\tops\texorpdfstring
  \edef\annot@type{\annot@type@text}%
  \pdfstringdef\Fld@name{#2}\dl@paramlocal
  \def\eq@Title{#2}\eqf@setDimens{#3}{#4}%
  \def\eq@DA{\eq@textFont\space\eq@textSize\space Tf \eq@textColor}%
  \@processEvery#7\end\noindent#5#6{#1}}
%    \end{macrocode}
%    \begin{macro}{\textField}\hskip-\marginparsep
%    \texttt{[\ameta{options}]\darg{\ameta{fld-name}}\darg{\ameta{wd}}^^A
%    \darg{\ameta{ht}}} The user interface to creating a text field.
%\begin{Verbatim}[xleftmargin=\parindent,fontsize=\small,
%commandchars=!(),codes={\catcode`\%=9}]
%#1 = !textsf(optional, used to enter any modification of the appearance/actions)
%#2 = !textsf(the title of the text field)
%#3 = !textsf(the width of the bounding rectangle)
%#4 = !textsf(the height of the bounding rectangle)
%\end{Verbatim}
%\noindent The defaults for \cs{textField}
%    \begin{macrocode}
\def\textFieldDefaults{\F{\FPrint}\BC{0 0 0}\W{1}\S{S}}
%    \end{macrocode}
% We sanitize the optional first parameter.
% \changes{v2.10}{2019/03/16}{Implement pdfSP for text fields}
%    \begin{macrocode}
\bgroup\obeyspaces
\gdef\textField{\begingroup\global\let =\pdfSP
\ef@sanitize@toks\ef@textfield}\egroup
%    \end{macrocode}
% End group, pass parameters to \cs{text@@Field}
%    \begin{macrocode}
\newcommand\ef@textfield[4][]{\endgroup
  \mbox{\text@@Field{#1}{#2}{#3}{#4}{}%
    {\eq@setWidgetProps\eq@TextField}%
    {\textFieldDefaults\every@TextField}}}
%    \end{macrocode}
% Some legacy assignments.
%    \begin{macrocode}
\let\eqTextField\textField
\let\calcTextField\textField
%    \end{macrocode}
%    \end{macro}
%
% \subsection{Signature Field}\label{sigfield}
%
% The template for a signature field. Within
% the optional argument, the following aliases are used: \cs{nameuse} for \cs{@nameuse}
% and \cs{tops} for \cs{texorpdfstring}.
%\changes{v2.4}{2020/11/28}{Rewrite of \string\cs{sig@@Field} to support \string\cs{pdfSpacesOn}}
%    \begin{macrocode}
\def\common@SigFieldCode
{%
  /Subtype /Widget
  /T (\Fld@name)
  /FT/Sig
  \eq@F
  \eq@TU
  /BS <<\eq@W\eq@S>>
  /MK <<\eq@R\eq@BC\eq@BG\eq@mkIns>>
  /DA (\eq@DA)
  \eq@Lock
  \eq@A\eq@AA
  \eq@rawPDF
}
%    \end{macrocode}
% The basic text field macro for constructing all other text fields.
%The arguments for this command are,
%\begin{Verbatim}[xleftmargin=\parindent,fontsize=\small,
%commandchars=!(),codes={\catcode`\%=9}]
%#1 = !textsf(optional, used to enter any modification of the appearance/actions)
%#2 = !textsf(the title of the button field)
%#3 = !textsf(the width of the bounding rectangle)
%#4 = !textsf(the height of the bounding rectangle)
%#5 = !textsf(not used)
%#6 = !textsf(set props and driver macros)
%#7 = !textsf(every macros)
%\end{Verbatim}
%    \begin{macrocode}
\def\annot@type@sig{sigfld}
\bgroup\obeyspaces
\gdef\sig@@Field{\begingroup\global\let =\pdfSP
\ef@sanitize@toks\sig@@@Field}\egroup
\newcommand\sig@@@Field[7]{\endgroup\begingroup
  \let\nameuse\@nameuse\let\tops\texorpdfstring
  \edef\annot@type{\annot@type@sig}%
  \pdfstringdef\Fld@name{#2}\dl@paramlocal
  \def\eq@Title{#2}\eqf@setDimens{#3}{#4}%
  \def\eq@DA{\eq@textFont\space\eq@textSize\space Tf \eq@textColor}%
  \@processEvery#7\end\noindent#5#6{#1}}
%    \end{macrocode}
%    \begin{macro}{\sigField}\hskip-\marginparsep
%    \texttt{[\ameta{options}]\darg{\ameta{fld-name}}\darg{\ameta{wd}}^^A
%    \darg{\ameta{ht}}} The user interface to creating a signature field.
%\begin{Verbatim}[xleftmargin=\parindent,fontsize=\small,
%commandchars=!(),codes={\catcode`\%=9}]
%#1 = !textsf(optional, used to enter any modification of the appearance/actions)
%#2 = !textsf(the title of the text field)
%#3 = !textsf(the width of the bounding rectangle)
%#4 = !textsf(the height of the bounding rectangle)
%\end{Verbatim}
%Defaults for \cs{sigField}
%    \begin{macrocode}
\def\sigFieldDefaults{\F{\FPrint}\BC{}\BG{}\W{1}\S{S}}
%    \end{macrocode}
% We sanitize the optional first parameter.
%    \begin{macrocode}
\bgroup\obeyspaces
\gdef\sigField{\begingroup\global\let =\pdfSP
\ef@sanitize@toks\ef@sigfield}\egroup
\newcommand\ef@sigfield[4][]{\endgroup
  \mbox{\sig@@Field{#1}{#2}{#3}{#4}{}{\eq@setWidgetProps\eq@SigField}%
    {\sigFieldDefaults\every@sigField}}}
%    \end{macrocode}
%    \end{macro}
%
% \section{Additional Link Support}\label{linkSupport}
%
% The links in \textsf{hyperref} are not sufficiently general to
% allow actions other than jumping. I've included a general link
% that increases the usage of the links provided by
% \textsf{hyperref}.
%\changes{v2.4}{2020/11/28}{Rewrite of \string\cs{set@@Link} to support \string\cs{pdfSpacesOn}}
% \paragraph{Common Link Code.} All the link commands eventually come back
% to this template for constructing a link annotation.
%    \begin{macrocode}
\def\common@LinkCode
{%
  \eq@A           % Action
  \eq@H           % Highlight
  \eq@Color       % Border color
  \link@BS        % Border styles
  \eq@rawPDF      % everything else
}
%    \end{macrocode}
% \cs{set@@Link} is the low-level link command and is the building block of all the
% other, more user-friendly link commands. Takes seven parameters:
%\begin{enumerate}\sffamily
%\item[\texttt{\#1}:] Optional arguments to modify the appearance and actions of the link.
%\item[\texttt{\#2}:] The width of the bounding rectangle.
%\item[\texttt{\#3}:] The height of the bounding rectangle.
%\item[\texttt{\#4}:] The text to be enclosed by the link.
%\item[\texttt{\#5}:] Used to set link properties and link driver.
%\item[\texttt{\#6}:] I don't remember what this one was about, same as \texttt{\#5}.
%\item[\texttt{\#7}:] Used to set link defaults and to execute \cs{everyLink}.
%\end{enumerate}
%    \begin{macrocode}
\def\annot@type@link{link}
\bgroup\obeyspaces
\gdef\set@@Link{\begingroup\global\let =\pdfSP
\ef@sanitize@toks\set@@@Link}\egroup
\newcommand\set@@@Link[7]{\endgroup\begingroup
  \let\nameuse\@nameuse\let\tops\texorpdfstring
  \dl@paramlocal
  \makeJSspecials
  \edef\annot@type{\annot@type@link}%
  \ef@preProcDefns
  \eqf@setDimens{#2}{#3}%
  \ifx\eq@rectW\@empty\def\link@@Box{#4}\else
    \def\eq@arg{#4}\ifx\eq@arg\@empty
      \def\eq@content{\hfill\vfill}\else\def\eq@content{#4}\fi
    \def\link@@Box{\parbox[\eq@pos][\eq@rectH][\eq@innerpos]%
      {\eq@rectW}{\centering\eq@content}}\fi
  \@processEvery#7\end\noindent#5#6{#1}}
%    \end{macrocode}
% \paragraph{The Visibility of Border of the Link.} We use \textsf{hyperref}'s
% option \texttt{colorlinks} to determine if we
% make the bounding rectangle visible or not by default.
%    \begin{macrocode}
\def\defaultlinkcolor{\@linkcolor}
%    \end{macrocode}
% Delay default link color until beginning of document.
%\changes{v2.8f}{2016/04/05}{Delay default link color until beginning of document.}
%    \begin{macrocode}
\def\setDef@ultLinkColor{\ifHy@colorlinks
  \def\ef@thislinkcolor{\defaultlinkcolor}%
  \def\ef@colorthislink{\color{\ef@thislinkcolor}}\else
  \let\ef@colorthislink\relax\fi}
\AtBeginDocument{\setDef@ultLinkColor}
%    \end{macrocode}
%    \begin{macro}{\setLink}\hskip-\marginparsep\texttt{[\ameta{options}]\darg{\ameta{text}}}
% \cs{setLink} is the basic high-level link command, it is used to surround text
% with a link annotation.
%\begin{Verbatim}[xleftmargin=\parindent,fontsize=\small,
%commandchars={!~@},codes={\catcode`\%=9}]
%\setLink[\A{\JS{this.pageNum=6;}}\Color{1 0 0}]{Go There!}
%\end{Verbatim}
%    \begin{macro}{\setLinkText}\hskip-\marginparsep\texttt{[\ameta{options}]\darg{\ameta{text}}}
% \cs{setLinkText} is a synonym for \cs{setLink}.
%    \begin{macro}{\mlhypertext}\hskip-\marginparsep\texttt{[\ameta{options}]\darg{\ameta{text}}}
% A command defined in the \pkg{aeb\_mlink} package.
%    \begin{macro}{\mlsetLink}\hskip-\marginparsep\texttt{[\ameta{options}]\darg{\ameta{text}}}
% \cs{mlsetLink} expands to \cs{mlhypertext}
% \paragraph{Link Parameters.} These high-level links takes two parameters, one of which
% is optional.
%\begin{enumerate}
%\item[\texttt{[\#1]}:] Optional key-value pairs to change the appearance or action
% of this link.
%\item[\texttt{\#2}:] The text to be enclosed in the bounding rectangle of this link.
%\end{enumerate}
%\paragraph{Link default parameters.} We set the line style on solid, and
% the border array to \texttt{0 0 0}, no visible border.
%    \begin{macrocode}
\def\set@LinkTextDefaults{\S{S}\Border{0 0 0}}
%    \end{macrocode}
% If the \texttt{colorlinks} option is in effect with hyperref, we color links when author
% uses |\setLinkText|, a command used by many other commands defined in AeB.
% \changes{v2.10}{2019/03/16}{Implement pdfSP for link annotations}
%    \begin{macrocode}
\bgroup\obeyspaces
\gdef\setLink{\begingroup\global\let =\pdfSP
\ef@sanitize@toks\ef@setlinktext}\egroup
%    \end{macrocode}
% \changes{v2.5b}{2009/12/24}{added path to \string\cs{mlhypertext}}
% Added a path to \cs{mlhypertext} of the \textsf{aeb\_mlink} package. When the
% user specifies \verb!\mlLink{true}! in the option list, we branch off to
% \cs{mlhypertext}. For example, if we say,
%\begin{verbatim}
%\setLink[\mlLink{true}\A{\JS{app.alert("Hello World!");}}]{Hello World!}
%\end{verbatim}
% then we branch off to \cs{mlhypertext}. The above, then, is equivalent to,
%\begin{verbatim}
%\mlhypertext[\A{\JS{app.alert("Hello World!");}}]{Hello World!}
%\end{verbatim}
%If you've created a link using \cs{setLink} and the text does not wrap around to the next
%link, simply insert \verb!\mlLink{true}! as an option to declare this link is a multi-line
%link. Of course, you can also change the name from \cs{setLink} to \cs{mlsetLink} (or to \cs{mlhypertext}).
%
%In the next line, we \cs{let} \cs{setLinkText} to \cs{setLink}
%    \begin{macrocode}
\let\setLinkText\setLink
\newcommand{\ef@setlinktext}[1][]{\endgroup
  \ef@searchmlLink#1\mlLink\end\@nil
  \ifx\ef@mlLink\ef@Zero\def\ef@next{\set@LinkText[#1]}\else
  \def\ef@next{\mlhypertext[#1]}\fi\ef@next}
\newcommand\set@LinkText[2][]{%
  \set@@Link{#1}{}{}{\ef@colorthislink#2}{}%
    {\eq@setWidgetProps{\ef@postProcLinkProps\setLink@driver}}%
    {\set@LinkTextDefaults\every@Link}}
%    \end{macrocode}
% Definitions we want make locally to the options parameters; otherwise,
% these are undefined.
%    \begin{macrocode}
\def\ef@preProcDefns{%
  \def\Win##1{/Win <<##1>>}%
  \def\fitpage{\dl@fitpage}\def\actualsize{\dl@actualsize}%
  \def\fitwidth{\dl@fitwidth}\def\fitheight{\dl@fitheight}%
  \def\fitvisible{\dl@fitvisible}\def\inheritzoom{\dl@inheritzoom}%
  \let\rPage\ef@rPage
  \edef\Page##1{\ifcase\eq@drivernum
    {Page##1}\or
    \noexpand\pdfpageref##1\space\space 0 R\or
    \noexpand @page##1\fi}%
}
%    \end{macrocode}
% After the properties are processed, the flow comes to |\ef@postProcLinkProps|
% for one last chance to process the properties more finely. The flow continues
% to |\setLink@@driver|.
%    \begin{macrocode}
\def\ef@postProcLinkProps{}
%    \end{macrocode}
%    \cs{mlhypertext} is not defined unless \pkg{aeb\_mlink} is loaded; however,
%    we define \cs{mlsetLink} to expand to \cs{mlhypertext}.
%    \begin{macrocode}
\newcommand{\mlsetLink}{\mlhypertext}
\newcommand{\mlhypertext}[2][]{\@ifundefined{mlhypertext@i}
  {\PackageWarning{eforms}{The \string\mlhypertext\space command
  does nothing unless\MessageBreak the aeb_mlink package is loaded}}{}%
  #2}
%    \end{macrocode}
%    \end{macro}
%    \end{macro}
%    \end{macro}
%    \end{macro}
%    \begin{macro}{\setLinkBbox}
% Set a link around a |\parbox|.
%\begin{Verbatim}[xleftmargin=\parindent,fontsize=\small,
%codes={\catcode`\%=9}]
%\setLinkBbox [\Color{1 0 0}}]{50bp}{30bp}[b]{\centering Press Me!}
%\end{Verbatim}
% \paragraph{Link Parameters.} There are four parameters, two of which
% are optional.
%\begin{enumerate}\sffamily
%\item[\texttt{[\#1]}:] Optional key-value pairs to change the appearance or action
% of this link.
%\item[\texttt{\#2}:] The width of the \cs{parbox}.
%\item[\texttt{\#3}:] The height of the \cs{parbox}.
%\item[\texttt{[\#4]}:] The positioning parameter of the \cs{parbox} (\texttt{b}, \texttt{c}, \texttt{t}).
%\item[\texttt{[\#5]}:] The text or object to be enclosed in a \cs{parbox}
%\end{enumerate}
%\paragraph{Link default parameters.} We set the line style on solid, and
% the border array to \texttt{0 0 0}, no visible border.
%    \begin{macrocode}
\def\set@LinkBboxDefaults{\S{S}\Border{0 0 0}}
%    \end{macrocode}
% \changes{v2.10}{2019/03/16}{Implement pdfSP for link annotations}
%    \begin{macrocode}
\bgroup\obeyspaces
\gdef\setLinkBbox{\begingroup\global\let =\pdfSP
\ef@sanitize@toks\ef@setLinkbbox}\egroup
\newcommand{\ef@setLinkbbox}[3][]{\endgroup
  \@setLinkBbox{#1}{#2}{#3}}
\def\@setLinkBbox#1#2#3{\@ifnextchar[%
  {\@@setLinkBbox{#1}{#2}{#3}}%
  {\@@setLinkBbox{#1}{#2}{#3}[c]}}
\def\@@setLinkBbox#1#2#3[#4]{\@ifnextchar[%
    {\@@@setLinkBbox{#1}{#2}{#3}{#4}}%
    {\@@@setLinkBbox{#1}{#2}{#3}{#4}[#4]}}
\def\@@@setLinkBbox#1#2#3#4[#5]#6{%
  \def\eq@pos{#4}\def\eq@innerpos{#5}%
  \set@@Link{#1}{#2}{#3}{\ef@colorthislink#6}%
    {\eq@setWidgetProps\setLink@driver}{}%
    {\set@LinkBboxDefaults\every@Link}}
%    \end{macrocode}
%    \end{macro}
%    \begin{macro}{\setLinkPbox}
% This is a special link for creating a fixed box around the
% current paragraph and is used with the multi-line link commands.
%\changes{v1.0c}{2007/07/21}
%{
%  Added \string\cs{setLinkPbox} to support multi-line links for dvips
% and dvipsone. Added to \string\cs{setLinkPbox@driver} a special link
% driver that supports \string\textbf{QuadPoints}.
%}
%    \begin{macrocode}
\def\set@LinkPboxDefaults{\S{S}\Border{0 0 0}}
\bgroup\obeyspaces
\gdef\setLinkPbox{\begingroup\global\let =\pdfSP
\ef@sanitize@toks\ef@setLinkpbox}\egroup
\newcommand\ef@setLinkpbox[1]{\endgroup
  \@setLinkPbox{#1\BS{}}{}{}{\hfill\vfill}}
\def\@setLinkPbox#1#2#3{\@ifnextchar[{\@@setLinkPbox{#1}{#2}{#3}}%
  {\@@setLinkPbox{#1}{#2}{#3}[c]}}
\def\@@setLinkPbox#1#2#3[#4]{%
  \@ifnextchar[{\@@@setLinkPbox{#1}{#2}{#3}{#4}}%
    {\@@@setLinkPbox{#1}{#2}{#3}{#4}[#4]}}
\def\@@@setLinkPbox#1#2#3#4[#5]#6{%
  \def\eq@pos{#4}\def\eq@innerpos{#5}%
  \set@@Link{#1}{#2}{#3}{#6}{\eq@setWidgetProps\setLinkPbox@driver}%
    {}{\set@LinkPboxDefaults\every@Link}}
%    \end{macrocode}
%    \end{macro}
%    \begin{macrocode}
%</package>
%    \end{macrocode}
%
% \section{A User Interface to links and forms}
%
% Through the years, I've developed my own version of a ``key-value''
% system, and this system works very well (see
% Section~\ref{procArgs}, page~\pageref{procArgs}, for the main
% macro); however, in a feeble attempt to make the system more user
% friendly, I've incorporated an \textsf{xkeyval} system into the
% \textsf{eforms} package. My old key-value system---which requires less
% typing and, in my opinion, is easier to use---still works,
% and the new one is just a user interface to the old system; parameters are
% passed to the main processing macro, defined in Section~\ref{procArgs}.
%    \begin{macrocode}
%<*userinterface>
%    \end{macrocode}
% The |\ui| interface is undefined if the \texttt{useui} option is not taken. We use this fact
% to send a message to the user that using |\ui| most accompany the \texttt{useui}.
%    \begin{macrocode}
\def\@equi#1{}
%    \end{macrocode}
% The visibility of the border of a link depends on where the \texttt{colorlinks} option of hyperref has been
% taken. If the author does not take \texttt{colorlinks}, a visible rectangle appears by default. By putting
% \texttt{border=visible}, the author can override this behavior. If \texttt{colorlinks} option was taken, then the same
% link has an invisible border by default. Again, the author can override this by putting \texttt{border=visible}.
%    \begin{macrocode}
\ifHy@colorlinks\def\eq@bordervisibledefault{0}\@eqW{}\else
\def\eq@bordervisibledefault{1}\@eqW{1}\fi
%    \end{macrocode}
% \subsection{Some Utility Commands}
% Some utility commands that are defined elsewhere in the AeB, but this package does not assume AeB, so
% we'll define them here as well.
%    \begin{macrocode}
\def\getargs#1#2{\def\aeb@argi{#1}\def\aeb@argii{#2}}
\newtoks\ef@flagtoks
\newtoks\ef@jstoks
%    \end{macrocode}
%
% \subsection{The Appearance Tab}
%
% We set this user interface up to model the UI of Acrobat/Reader.
%    \IndexKey{border}
% In the case of a link, this is the Link Type: \textsf{Visible Rectangle}
% or \textsf{Invisible Rectangle}. For forms, this key has not counterpart
% in the user interface. If you set border equal to \texttt{invisible}, that
% will set border line width to zero |\W{0}|.
%    \begin{macrocode}
\define@choicekey{eforms}{border}[\val\nr]{visible,invisible}
{%
  \ifcase\nr
%    \end{macrocode}
% Author has chosen a visible border |\eq@W@buffered|.
%    \begin{macrocode}
    \def\eq@visibleborder{1}%
    \ifx\eq@W\@empty
      \ifx\eq@W@buffered\@empty\@eqW{1}\else
      \@eqW{\eq@W@buffered}\fi
    \else
      \ifnum\eq@W@value=0 \@eqW{1}\fi
      \ifx\eq@W@buffered\@empty\else
      \@eqW{\eq@W@buffered}\fi
    \fi
    \@eqBorder{0 0 \eq@W@value}%
  \or
    % author has chosen invisible border
    \def\eq@visibleborder{0}\@eqW{}%
    \@eqBorder{0 0 0}\@eqBC{}%
  \fi
}
%    \end{macrocode}
% This choice key is for the \texttt{linewidth}\IndexKey{linewidth} of a link or form. The user interface choices
% are \texttt{thin}, \texttt{medium}, and \texttt{thick}. This number is ignored if the document
% author has set the border to \texttt{invisible}.
%    \begin{macrocode}
\define@choicekey{eforms}{linewidth}[\val\nr]{thin,medium,thick}
{%
  \ifx\annot@type\annot@type@link
%    \end{macrocode}
% This is a link, so, depending on whether the color package is loaded, we show or don't
% show the boundary, depending on the explicate choices of the author
%    \begin{macrocode}
    \ifx\eq@visibleborder\@empty
%    \end{macrocode}
% Author has not committed to a border type, so we'll assume the default
%    \begin{macrocode}
      \ifnum\eq@bordervisibledefault=0
%    \end{macrocode}
% A color package is loaded, use invisible border by default.
%    \begin{macrocode}
        \edef\eq@W@buffered{\ifcase\nr 1\or2\or3\fi}%
        \@eqW{}%
      \else
%    \end{macrocode}
% No color is loaded, show the border.
%    \begin{macrocode}
        \ifcase\nr\@eqW{1}\or\@eqW{2}\or
          \@eqW{3}\else\@eqW{1}\fi
          \@eqBorder{0 0 \eq@W@value}%
      \fi
    \else
%    \end{macrocode}
% Author has set the border type before line width, we'll do what the
% author wants. If invisible, ignore |\W|, if visible obey |\W|.
%    \begin{macrocode}
      \ifnum\eq@visibleborder=1 % visible border
        \ifcase\nr\@eqW{1}\or\@eqW{2}\or
          \@eqW{3}\else\@eqW{1}\fi
        \@eqBorder{0 0 \eq@W@value}%
      \fi
    \fi
  \else
%    \end{macrocode}
% This is not a link, its a form, so things are easier. The invisible border will override
% the linewidth. If the author wants a border, s/he better get her/his act together and
% remove \texttt{border=invisible}.
%    \begin{macrocode}
    \ifx\eq@visibleborder\@empty
%    \end{macrocode}
% If the value of |\eq@visibleborder|, author has not used the border key at this time,
% so just set the \texttt{linewidth} as stated.
%    \begin{macrocode}
      \ifcase\nr \@eqW{1}\or\@eqW{2}\or
        \@eqW{3}\else\@eqW{1}\fi
      \@eqBorder{0 0 \eq@W@value}%
      \edef\eq@W@buffered{\ifcase\nr1\or2\or3\fi}%
    \else
%    \end{macrocode}
% The author has used the border key, so we don't need to save this value.
%    \begin{macrocode}
      \ifnum\eq@visibleborder>0
        \ifcase\nr \@eqW{1}\or\@eqW{2}\or
          \@eqW{3}\else\@eqW{1}\fi
        \@eqBorder{0 0 \eq@W@value}%
      \fi
    \fi
  \fi
}
\let\eq@visibleborder\@empty
\let\eq@W@buffered\@empty
%    \end{macrocode}
%
% The highlight\IndexKey{highlight} type, choices are \texttt{none}, \texttt{invert},
% \texttt{outline}, \texttt{inset} and \texttt{push}. The term \texttt{inset}
% is used with links, and \texttt{push} is used with forms. They each have the
% same key value pair.
%    \begin{macrocode}
\define@choicekey{eforms}{highlight}[\val\nr]{none,invert,outline,%
    inset,push}
{%
    \ifcase\nr
      \@eqH{}\or\@eqH{I}\or\@eqH{O}\or\@eqH{P}\or\@eqH{P}\fi
}{}
%    \end{macrocode}
%    \IndexKey{bordercolor}
% This is an rgb color, \texttt{bordercolor=1 0 0}, is the color red.
%    \begin{macrocode}
\define@key{eforms}{bordercolor}[]{%
    \ifx\annot@type\annot@type@link\@eqColor{#1}\else
% If border is invisible, we ignore bordercolor
    \ifx\eq@visibleborder\@empty\@eqBC{#1}\else
        \ifnum\eq@visibleborder>0\relax\@eqBC{#1}\fi\fi\fi
}
%    \end{macrocode}
%
% The line style\IndexKey{linestyle} of the border, possible values are \texttt{solid},\texttt{dashed},
% \texttt{underlined}, \texttt{beveled},and \texttt{inset}. Links do not support
% the \texttt{beveled} and \texttt{inset} options.
%    \begin{macrocode}
\define@choicekey{eforms}{linestyle}[\val\nr]{solid,dashed,underlined,%
    beveled,inset}
{%
    \ifcase\nr\relax
        \@eqS{S}\or\@eqS{D}\or\@eqS{U}%
        \or\@eqS{B}\or\@eqS{I}\else\@eqS{S}%
    \fi
}
%    \end{macrocode}
%
% When a line style of dashed is chosen, you can specify a dashed array\IndexKey{dasharray}.
% The default is 3, which means 3 points of line, 3 points gap.
% A value of \texttt{dashedarray=3 2} means three points of line, followed
% by two points of space.
%    \begin{macrocode}
\define@key{eforms}{dasharray}[3]{\@eqD{#1}}
%    \end{macrocode}
% Set the color of the link text. Ignored if the colorlinks option of hyperref
% has not been taken.  The value of \texttt{linktxtcolor}\IndexKey{linktxtcolor} is a named color. For example,
% \texttt{linktxtcolor=red}. The default is |\@linkcolor| from hyperref. This default
% can be changed by redefining |\@linkcolor|, or be redefining |\defaultlinkcolor|.
% If |linktxtcolor={}| (an empty argument), or simply \texttt{linktxtcolor}, no color is applied to the text.
%    \begin{macrocode}
\define@key{eforms}{linktxtcolor}[]{%
    \let\ef@linktxtcolor@set=1\@eqlinktxtcolor{#1}}
\let\ef@linktxtcolor@set=0
%    \end{macrocode}
%
% \subsection{The General and Option Tab}
% In the general and option tabs of Acrobat a variety of attributes
% of the field can be set, such visibility, printability, readonly,
% orientation, captions (for button), values, default values, etc.
% can be set.
%\par\medskip\noindent
% The \texttt{annotflags}\IndexKey{annotflags} is a bit field, possible values are \texttt{hidden}, \texttt{print},
% \texttt{-print}, \texttt{noview}, and \texttt{lock}. Multiple values can
% be specified. The values are ``or-ed'' together. Most all forms are printable
% by default. If you don't want a form field to print specify \texttt{-print}.
% For example, |annotflags={-print,lock}| makes the field not printable and is
% locked, so the field cannot be moved through the UI.
%    \begin{macrocode}
\define@key{eforms}{annotflags}[]{\ef@flagtoks={}%
  \setkeys{annotflags}{#1}}
\@tfor\ef@flagopts:={{hidden}{FHidden}}{{print}{FPrint}}{{noprint}%
  {FNoPrint}}{{-print}{FNoPrint}}{{noview}{FNoView}}{{lock}{FLock}}%
  \do{\expandafter\getargs\ef@flagopts
    \edef\temp@expand@def{\noexpand
    \define@key{annotflags}{\aeb@argi}[true]{%
      \noexpand\ef@flagtoks=\noexpand
        \expandafter{\noexpand\ef@passedArgs}%
      \noexpand\edef\noexpand\ef@passedArgs{\noexpandiii
        \F{\noexpandiii\@nameuse{\aeb@argii\noexpandiii}}%
        \noexpand\the\noexpand\ef@flagtoks}%
    }%
  }\temp@expand@def
}
%    \end{macrocode}
% A large number of fields that sets a number of properties of a field. Some of these
% are specified through the bitfield \texttt{fieldflags}\IndexKey{fieldflags}.
% These appear in the first of the paired groups listed below. Normally, a document
% author would not specify \texttt{radio}, \texttt{pushbutton} or \texttt{combo}.
% These properties are used
% by \pkg{eforms} to construct a radio button field, a push button and a combo box.
% The others can be used as appropriate.
%    \begin{macrocode}
\define@key{eforms}{fieldflags}[]{\ef@flagtoks={}%
  \setkeys{fieldflags}{#1}}
\@tfor\ef@flagopts:={{readonly}{FfReadOnly}}{{required}{FfRequired}}%
  {{noexport}{FfNoExport}}{{multiline}{FfMultiline}}%
  {{password}{FfPassword}}{{notoggleoff}{FfNoToggleToOff}}%
  {{radio}{FfRadio}}{{pushbutton}{FfPushButton}}{{combo}{FfCombo}}%
  {{edit}{FfEdit}}{{sort}{FfSort}}{{fileselect}{FfFileSelect}}%
  {{multiselect}{FfMultiSelect}}{{nospellcheck}{FfDoNotSpellCheck}}%
  {{noscrolling}{FfDoNotScroll}}{{comb}{FfComb}}%
  {{radiosinunison}{FfRadiosInUnison}}%
  {{commitonchange}{FfCommitOnSelChange}}{{richtext}{FfRichText}}\do{%
  \expandafter\getargs\ef@flagopts
  \edef\temp@expand@def{\noexpand
    \define@key{fieldflags}{\aeb@argi}[true]{\noexpand
      \ef@flagtoks=\noexpand\expandafter{\noexpand
        \ef@passedArgs}\noexpand
      \edef\noexpand\ef@passedArgs{\noexpandiii
        \Ff{\noexpandiii\@nameuse{\aeb@argii\noexpandiii}}%
        \noexpand\the\noexpand\ef@flagtoks}%
    }%
  }\temp@expand@def
}
%    \end{macrocode}
% Enter a text value to appear as a tool tip\IndexKey{tooltip}. For example,
% \texttt{tooltip={\darg{\textsf{Hi, press me and see what happens!}}}}. Enclose in
% parentheses if the text string contains a comma.
%    \begin{macrocode}
\define@key{eforms}{tooltip}{\@eqTU{#1}}
%    \end{macrocode}
% The default value\IndexKey{default}\IndexKey{defaultstyle} of a field (text, list, combo box) what is used
% when the field is reset. Example: \texttt{default=Name}
%    \begin{macrocode}
\define@key{eforms}{default}{\@eqDV{#1}}
\define@key{eforms}{defaultstyle}{\@eqDS{#1}}
%    \end{macrocode}
% The \texttt{value}\IndexKey{value}\IndexKey{richvalue}\IndexKey{apprD} of the field
% (text, list, combobox). Example: \texttt{value=AcroTeX}.
%    \begin{macrocode}
\define@key{eforms}{value}{\@eqV{#1}}
\define@key{eforms}{richvalue}{\@eqRV{#1}}
\define@key{eforms}{apprD}{\@eqAP{#1}}
%    \end{macrocode}
% Set the orientation\IndexKey{rotate} of the field, values are 0, 90, 180 and 270. If 90 or 270
% are chosen, the height and width of the field need to be reversed. This is not
% done automatically by eForms (probably should).
%    \begin{macrocode}
\define@choicekey{eforms}{rotate}[\val\nr]{0,90,180,270}
    {\expandafter\@eqR\expandafter{\val}}{}
%    \end{macrocode}
% The background color\IndexKey{bgcolor} of the field. This is an RGB value.
%    \begin{macrocode}
\define@key{eforms}{bgcolor}[]{\@eqBG{#1}}
%    \end{macrocode}
% The normal text\IndexKey{uptxt} that appears on a button. Example \texttt{uptxt=Push Me}
%    \begin{macrocode}
\define@key{eforms}{uptxt}{\@eqCA{#1}}
%    \end{macrocode}
% The (mouse) down text\IndexKey{downtxt} of the button caption. Example: \texttt{downtxt=Thanks!}
%    \begin{macrocode}
\define@key{eforms}{downtxt}{\@eqAC{#1}}
%    \end{macrocode}
% The (mouse) rollover\IndexKey{rollovertxt} caption of a button. Example: \texttt{rollovertxt=AcroTeX!}
%    \begin{macrocode}
\define@key{eforms}{rollovertxt}{\@eqRC{#1}}
%    \end{macrocode}
% The type of alignment\IndexKey{align} of a text field. Permitted values are
% \texttt{left}, \texttt{centered}, and \texttt{right}.
%    \begin{macrocode}
\define@choicekey{eforms}{align}[\val\nr]{left,centered,right}{%
    \ifx\annot@type\annot@type@text
        \expandafter\@eqQ\expandafter{\nr}\fi}{}
%    \end{macrocode}
% The text font\IndexKey{textfont} to be used with the text of the field.
%    \begin{macrocode}
\define@key{eforms}{textfont}{\@eqtextFont{#1}}
%    \end{macrocode}
% The text size\IndexKey{textsize} to be used. A value of 0 means auto size.
%    \begin{macrocode}
\define@key{eforms}{textsize}{\@eqtextSize{#1}}
%    \end{macrocode}
% The color of the text\IndexKey{textcolor} in the field. This can be in \textsf{G}, \textsf{RGB} or \textsf{CMYK} color
% space.
%    \begin{macrocode}
\define@key{eforms}{textcolor}{\@eqtextColor{#1}}
%    \end{macrocode}
% Use \texttt{maxlength}\IndexKey{maxlength} to limit the number of characters input into a text field.
% Example: \texttt{maxlength=12}.
%    \begin{macrocode}
\define@key{eforms}{maxlength}{\@eqMaxLen{#1}}
%    \end{macrocode}
% \paragraph*{Icon Appearances} We begin a section for creating icon
% appearances for a push button. We define three keys \texttt{normappr}\IndexKey{normappr},
% \texttt{rollappr}\IndexKey{rollappr}, and \texttt{downappr}\IndexKey{downappr}
%    \begin{macrocode}
\define@key{eforms}{normappr}{\@eqI{#1}}
\define@key{eforms}{rollappr}{\@eqRI{#1}}
\define@key{eforms}{downappr}{\@eqIX{#1}}
\define@choicekey{eforms}{layout}[\val\nr]{labelonly,icononly,%
  icontop,iconbottom,iconleft,iconright,labelover}{%
  \ifx\annot@type\annot@type@button\expandafter
    \@eqTP\expandafter{\nr}\fi}{}
%    \end{macrocode}
% Scaling keys, \texttt{scalewhen}\IndexKey{scalewhen} and \texttt{scale}\IndexKey{scale}
%    \begin{macrocode}
\define@choicekey{eforms}{scalewhen}[\val\nr]{always,never,%
  iconbig,iconsmall}{%
  \ifx\annot@type\annot@type@button
  \ifcase\nr\relax
    \def\eq@@SW{A}\or
    \def\eq@@SW{N}\or
    \def\eq@@SW{B}\or
    \def\eq@@SW{S}\else
    \def\eq@@SW{A}\fi
  \expandafter\@eqSW\expandafter{\eq@@SW}\fi}{}
\define@choicekey{eforms}{scale}[\val\nr]{proportional,%
  nonproportional}{%
  \ifx\annot@type\annot@type@button
  \ifcase\nr\relax
      \def\eq@@ST{P}\or
      \def\eq@@ST{A}\else
      \def\eq@@ST{P}\fi
  \expandafter\@eqST\expandafter{\eq@@ST}\fi}{}
%    \end{macrocode}
% Positioning keys, \texttt{position}\IndexKey{position} and \texttt{fitbounds}\IndexKey{fitbounds}
%    \begin{macrocode}
\define@key{eforms}{position}{\@eqPA{#1}}
\define@choicekey{eforms}{fitbounds}[\val\nr]{true,false}[true]{%
  \ifx\annot@type\annot@type@button
    \ifcase\nr\relax
      \def\eq@@FB{true}\or
      \def\eq@@FB{false}\else
      \def\eq@@FB{false}\fi
  \expandafter\@eqFB\expandafter{\eq@@FB}\fi}{}
%    \end{macrocode}
% This set of key-values\IndexKey{norm}\IndexKey{roll}\IndexKey{down} are
% designed to fill in the AP dictionary for the check box
% and the radio button field.
%\begin{verbatim}
% appr={norm={on={},off={}}},roll={on={},off={}}},down={on={},off={}}}}
% \AP{/N << /On {xO} /Off {xX} >>  /R << /On {xX} /Off {xO} >>}
%\end{verbatim}
%    \begin{macrocode}
\define@key{efappr}{norm}[]{\def\efappr@norm{#1}}
\define@key{efappr}{roll}[]{\def\efappr@roll{#1}}
\define@key{efappr}{down}[]{\def\efappr@down{#1}}
\setkeys{efappr}{norm,roll,down}
\define@key{efappr@state}{on}[]{\expandafter
  \def\csname \ef@state @on\endcsname{#1}}
\define@key{efappr@state}{off}[]{\expandafter
  \def\csname \ef@state @off\endcsname{#1}}
\let\norm@on\@empty\let\norm@off\@empty
\let\roll@on\@empty\let\roll@off\@empty
\let\down@on\@empty\let\down@off\@empty
%    \end{macrocode}
% The key \texttt{appr}\IndexKey{appr} corresponds to the \textbf{AP} dictionary.
% and the radio button field.
%\begin{macrocode}
\define@key{eforms}{appr}{\setkeys{efappr}{#1}%
  \def\ef@state{norm}%
  \edef\ef@edef@tmp{\noexpand\setkeys{efappr@state}{\efappr@norm}}%
  \ef@edef@tmp\def\ef@state{roll}%
  \edef\ef@edef@tmp{\noexpand\setkeys{efappr@state}{\efappr@roll}}%
  \ef@edef@tmp\def\ef@state{down}%
  \edef\ef@edef@tmp{\noexpand\setkeys{efappr@state}{\efappr@down}}%
  \ef@edef@tmp\def\eq@AP{%
    /AP<<%
      \ifx\efappr@norm\@empty\else/N <<%
        \eq@On\space{\norm@on}/Off {\norm@off}>>\fi
      \ifx\efappr@roll\@empty\else/R <<%
        \eq@On\space{\roll@on}/Off {\roll@off}>>\fi
      \ifx\efappr@down\@empty\else/D <<%
        \eq@On\space{\down@on}/Off {\down@off}>>\fi
    \space>>
  }%
}
%    \end{macrocode}%
% \subsection{Miscellaneous keys}
% The \texttt{autocenter}\IndexKey{autocenter} is a feature of eforms. Use \texttt{autocenter=yes} (the default) to center the bounding
% box, and use \texttt{autocenter=no} otherwise.
%    \begin{macrocode}
\define@choicekey{eforms}{autocenter}[\val\nr]{yes,no}
{%
  \ifcase\nr\relax\@eqautoCenter{y}\or
  \@eqautoCenter{n}\fi
}{}
\define@choicekey{eforms}{inline}[\val\nr]{yes,no}
{%
  \ifcase\nr\relax\@eqinline{y}\or
  \@eqinline{n}\fi
}{}
\define@choicekey{eforms}{mlfix}[\val\nr]{yes,no}
{%
  \ifcase\nr\relax\@eqmlfix{y}\or
  \@eqmlfix{n}\fi
}{}
\define@choicekey{eforms}{importicons}[\val\nr]{yes,no}
{%
  \ifcase\nr\relax\@eqimportIcons{y}\or
  \@eqimportIcons{n}\fi
}{}
\define@key{eforms}{mlstrut}[\strut]{\@eqmlstrut{#1}}
\define@key{eforms}{mlcrackat}[]{\@eqmlcrackat{#1}}
\define@key{eforms}{mlcrackinat}[]{\@eqmlcrackinsat{#1}}
\define@choicekey{eforms}{mlhyph}[\val\nr]{yes,no}
{%
  \ifcase\nr\relax\@eqmlhyph{y}\or
  \@eqmlhyph{n}\fi
}{}
%    \end{macrocode}
%    \changes{v2.10}{2019/03/16}{Added \string\cs{cmd} to optional args of forms}
%    This key passes its argument directly into the stream for processing
%    \begin{macrocode}
\define@key{eforms}{cmd}[]{\@eqcmd{#1}}
%    \end{macrocode}
% Set presets\IndexKey{presets} from inside a \cs{ui} argument. For example,
%\begin{verbatim}
%\def\myUIOptsi{%
%    border=visible,         % Link Type
%    linktxtcolor=blue,      % blue link color
%    linewidth=medium,       % Line thickness
%    highlight=outline,      % Highlight style
%    linestyle=dashed,       % Line style
%    bordercolor={1 0 0},    % border color
%    js={app.alert("Good, good, good!")},
%}
%\end{verbatim}
% Then we can say,
%\begin{verbatim}
%\setLinkText[\ui{presets={\myUIOptsii}}]{Press Me Again!!}
%\end{verbatim}
%    \begin{macrocode}
\define@key{eforms}{presets}{\ef@jstoks=\expandafter{#1}%
  \edef\ef@temp@expand{\noexpand\setkeys{eforms}{\the\ef@jstoks}}%
  \ef@temp@expand
}
%    \end{macrocode}
%    \changes{v2.10}{2019/03/16}{epresets, the ui counterpart to \string\cs{epresets}}
%     epresets, the ui counterpart to \cs{epresets}
%    \begin{macrocode}
\define@key{eforms}{epresets}{\ef@jstoks=\expandafter{#1}%
  \edef\@rgs{#1}\ef@jstoks=\expandafter{\@rgs}%
  \edef\ef@temp@expand{\noexpand\setkeys{eforms}{\the\ef@jstoks}}%
  \ef@temp@expand
}
%    \end{macrocode}
% \texttt{symbolchoice}\IndexKey{symbolchoice} is used with a checkbox or radio button field. This sets the symbol
% that appears in the field with the box is checked, choices are
% \texttt{check}, \texttt{circle}, \texttt{cross}, \texttt{diamond},
% \texttt{square}, and \texttt{star}.
%    \begin{macrocode}
\define@choicekey{eforms}{symbolchoice}[\val\nr]%
  {check,circle,cross,diamond,square,star}
  {\expandafter\@eqsymbolchoice\expandafter{\val}}{}
%    \end{macrocode}
%\paragraph*{Set rectangle keys}
%The dimensions of the bounding rectangle is generally passed to the widget
%using required arguments; however, for \pkg{exerquiz}, those arguments are
%not directly accessible. The \IndexKey{rectW}\key{rectW} and \IndexKey{rectH}\key{rectH}
%may be used to pass the dimensions of the rectangle through the optional
%optional argument of the field.
%    \begin{macrocode}
\define@key{eforms}{rectW}{\@eqrectW{#1}}
\define@key{eforms}{rectH}{\@eqrectH{#1}}
%    \end{macrocode}
%\paragraph*{Resizing keys}
%These keys resize the field while preserving the aspect ratio; these are
%\IndexKey{width}\key{width}, \IndexKey{height}\key{height}, and \IndexKey{scalefactor}\key{scalefactor}
%    \begin{macrocode}
\define@key{eforms}{width}{\@eqwidth{#1}}
\define@key{eforms}{height}{\@eqheight{#1}}
\define@key{eforms}{scalefactor}{\@eqscalefactor{#1}}
%    \end{macrocode}
%
% \subsection{The Signed Tab}
%
% A signature field has a signed tab. On that tab is an option to mark a set of fields
% as readonly (locked). The locked key controls that option.\medskip
%
% {\noindent}The \texttt{lock}\IndexKey{lock} key is used with signature fields.
% Typical entries, using \emph{raw PDF markup}, are,
%\begin{Verbatim}[xleftmargin=\parindent,fontsize=\small,
%commandchars={!~@},codes={\catcode`\%=9}]
%lock={/Action/All}          !% !textsf~lock all fields in the doc@
%lock={/Action/Include       !% !textsf~lock all fields listed in Fields@
%      /Fields [(field1)(field2)...]}
%lock={/Action/Exclude       !% !textsf~lock  all fields not listed in Fields@
%      /Fields [(field1)(field2)...]}
%\end{Verbatim}
%Simplified syntax for UF key-values:
%\changes{v2.3.6}{2020/10/10}{Addition support for the \string\texttt{lock} key}
%\begin{Verbatim}[xleftmargin=\parindent,fontsize=\small,
%commandchars={!~@},codes={\catcode`\%=9}]
%lock={action=all}             !% !textsf~lock all fields in the doc@
%lock={action=include,
%   fields={field1,field2,...}}!% !textsf~lock all fields listed in Fields@
%lock={action=exclude,
%   fields{field1,field2,...}} !% !textsf~lock all fields not listed in Fields@
%\end{Verbatim}
%
%    \begin{macrocode}
\define@key{eforms}{lock}{\setkeys{lckflds}{#1}%
  \ifx\lckFlds\@empty
    \edef\ActnArgs{\noexpand\Action\noexpand\All}\else
    \edef\ActnArgs{\noexpand\Action\expandafter
      \noexpand\lckActnArg{\lckFlds}}\fi
  \expandafter\@eqLock\expandafter{\ActnArgs}%
}
\define@choicekey{lckflds}{action}[\val\nr]{all,include,exclude}
{%
  \ifcase\nr\relax
  \def\lckActnArg{\All}\or
  \def\lckActnArg{\IncludeFields}\or
  \def\lckActnArg{\ExcludeFields}\fi
}{}
\define@key{lckflds}{fields}{\def\lckFlds{#1}}
\let\lckFlds\@empty
%    \end{macrocode}
%
% Another option that is included in the Signed tab is titled ``This script executes
% when field is signed.''
%
% This is an option that, through the user interface, is mutually exclusive from
% locking fields. This option appears to be implemented through the format event.
% Thus, to populate this option with JavaScript use \texttt{format}. For example,
%\begin{verbatim}
%   format={app.alert("Thank you for signing this field.");}
%\end{verbatim}
%
% \subsection{Specifying Actions}
% Here, we offer up various common actions for a document author to choose from.
%
% The \texttt{goto}\IndexKey{goto} key is a combines \textsf{GoTo} and \textsf{GoToR} action. There are options for jumping to a page
% with a particular view, a latex label, or a named destination.
%
% The goto key first calls |\setkeys{efgoto}{#1}|, which parses the parameters. Following the
% reading of the parameters, we determined whether this is a GoTo or a GoToR request, whether
% we are jumping to a page, a named destination or a latex label.
%
% Documentation for the parameters of \texttt{goto} follow this definition.
%
% Process the \texttt{goto} key using \pkg{conv-xkv}.
% \changes{v2.9g}{2017/01/03}{Process the \string\texttt{goto} key using \string\pkg{conv-xkv}}
%    \begin{macrocode}
\define@key{eforms}{goto}[]{%
%    \end{macrocode}
%    (2017/01/03) Use \cs{cxkvsetkeys} for \texttt{goto}
%    \changes{v2.9g}{2017/01/03}{Use \string\cs{cxkvsetkeys} for \string\texttt{goto}}
%    \begin{macrocode}
    \cxkvsetkeys{efgoto}{#1}%
    \ifx\ef@goto@url\@empty
        \ifx\ef@goto@file\@empty
            % Jump within the file
            \def\ef@subtype{/S/GoTo }%
            \ifcase\eq@drivernum
              \def\ef@formatpage{{Page\ef@page}}\or
              \def\ef@formatpage{\pdfpageref\ef@page\space\space 0 R}\or
              \def\ef@formatpage{@page\ef@page}\fi
            \let\ef@open\@empty
            \let\ef@formatfile\@empty
        \else
            % Jump to another PDF
            \def\ef@subtype{/S/GoToR }%
            \count0=\ef@page\advance\count0by-1
            \edef\ef@formatpage{\the\count0 }%
            \def\ef@formatfile{/F (\ef@goto@file)}%
        \fi
        \ifx\ef@goto@targetdest\@empty
            \ifx\ef@labeldest\@empty
            % we will jump to a page, it might be the default page
              \expandafter\@eqA\expandafter{\ef@subtype
                /D[\ef@formatpage\ef@view]\ef@formatfile\ef@open}\else
            % jump to a label
              \expandafter\@eqA\expandafter{\ef@subtype%
                /D (\labelRef{\ef@labeldest})%
                \ef@formatfile\ef@open}\fi
        \else
        % jump to a target
          \expandafter\@eqA\expandafter{\ef@subtype%
            /D (\ef@goto@targetdest)\ef@formatfile\ef@open}%
        \fi
    \else % go to url
        \ifx\ef@goto@openparams\@empty
          \@eqA{/S/URI/URI(\ef@goto@url)}\else
          \@eqA{/S/URI/URI(\ef@goto@url\#\ef@goto@openparams)}%
        \fi
    \fi
}
%    \end{macrocode}
% The keys for \texttt{goto} are \texttt{file},
% \texttt{targetdest}, \texttt{labeldest}, \texttt{page},
% \texttt{view}, and \texttt{open}.
% Specify a relative path\IndexKey{file} to the PDF file. This will work on the Web if
% the position is the same relative to the calling file.
%    \begin{macrocode}
\define@key{efgoto}{file}[]{\def\ef@goto@file{#1}}
\let\ef@goto@file\@empty
%    \end{macrocode}
% Specify a url\IndexKey{url} to create a weblink
%    \begin{macrocode}
\define@key{efgoto}{url}[]{%
    \if\ef@linktxtcolor@set0\@eqlinktxtcolor{\@urlcolor}\fi
    \def\ef@goto@url{#1}%
}
\let\ef@goto@url\@empty
%    \end{macrocode}
% Specify a relative path to the PDF file with \texttt{openparams}\IndexKey{openparams}.
% This will work on the Web if the position is the same relative to the calling file.
%    \begin{macrocode}
\define@key{efgoto}{openparams}[]{\def\ef@goto@openparams{#1}}
\let\ef@goto@openparams\@empty
%    \end{macrocode}
%    Jump to a target\IndexKey{targetdest}, perhaps created by |\hypertarget|. For example,
%    if we say |\hypertarget{acrotex}{Welcome!}|, we can then jump to
%    \texttt{acrotex} by specifying \texttt{targetdest=acrotex}.
%    \begin{macrocode}
\define@key{efgoto}{targetdest}[]{\def\ef@goto@targetdest{#1}}
\let\ef@goto@targetdest\@empty
%    \end{macrocode}
%    \texttt{labeldest}\IndexKey{labeldest} is the
%    same as targetdest, but now we jump to a destination specified by
%    a latex label. For example, |\section{AcroTeX}\label{acrotex}|,
%    we can jump to this section by specifying \texttt{labeldest=acrotex}.
%    \begin{macrocode}
\define@key{efgoto}{labeldest}[]{\def\ef@labeldest{#1}}
\let\ef@labeldest\@empty
%    \end{macrocode}
% The page number\IndexKey{page} that you want to jump to. If we set \texttt{page=1},
% we will jump to the first page of the document.
%    \begin{macrocode}
\define@key{efgoto}{page}[1]{\def\ef@page{#1}}
\def\ef@page{1}
\def\ef@view{/Fit}%
%    \end{macrocode}
% The view\IndexKey{view} can be set when the page key is used. Possible values are
% \texttt{fitpage}, \texttt{actualsize}, \texttt{fitwidth},
% \texttt{fitvisible}, and \texttt{inheritzoom}. These terms correspond
% to Acrobat's UI. When jumping to a destination, the view is set by the
% destination code.
%\begin{verbatim}
%\def\dl@fitpage{/Fit}
%\def\dl@actualsize{/XYZ -32768 -32768 1.0}
%\def\dl@fitwidth{/FitH -32768}
%\def\dl@fitvisible{/FitBH -32768}
%\def\dl@inheritzoom{/XYZ 0 0 0}
%\end{verbatim}
%    \begin{macrocode}
\define@choicekey{efgoto}{view}[\val\nr]{fitpage,actualsize,fitwidth,%
    fitheight,fitvisible,inheritzoom}
{%
    \edef\ef@view{\csname dl@\val\endcsname}%
}{}
%    \end{macrocode}
% The \texttt{open}\IndexKey{open} key is use when you specify the \texttt{file} key. The
% open key determines if a new window is opened or not when we
% jump to the file. Possible values are \texttt{userpref} (use user preferences),
% \texttt{new} (open new window), \texttt{existing} (use the existing window).
%    \begin{macrocode}
\define@choicekey{efgoto}{open}[\val\nr]{userpref,new,existing}
{%
  \ifcase\nr\relax
    \let\ef@open\@empty\or
    \def\ef@open{/NewWindow true }\or
    \def\ef@open{/NewWindow false }\fi
}{}
\let\ef@open\@empty
%    \end{macrocode}
% Through the \texttt{launch}\IndexKey{launch} key, we support the subtype \textbf{Launch}. The launch action dictionary
% contains six keys, four of which we support: \texttt{S}, \texttt{F}, \texttt{Win},
% and \texttt{New Window} (boolean). The Win key is a dictionary with keys \texttt{F}, \texttt{D}, \texttt{O},
% and \texttt{P}. If the \texttt{Win} key is used, the F key in the launch dictionary is optional. The parameters
% of the Win dictionary are passed to the \textsf{ShellExecute}.
%
% Process the \texttt{launch} key using \pkg{conv-xkv}.
% \changes{v2.9g}{2017/01/03}{Process the \string\texttt{launch} key using \string\pkg{conv-xkv}}
%    \begin{macrocode}
\define@key{eforms}{launch}[]{%
%    \end{macrocode}
%    (2017/01/03) Use \cs{cxkvsetkeys} for \texttt{launch}
%    \changes{v2.9g}{2017/01/03}{Use \string\cs{cxkvsetkeys} for \string\texttt{launch}}
%    \begin{macrocode}
  \cxkvsetkeys{eflaunch}{#1}%
  \@eqA{/S/Launch\ifx\ef@launch@file\@empty\else
    /F(\ef@launch@file)\fi\ef@launch@open
    \ifx\ef@launch@win\@empty\else
      /Win<<\ifx\ef@launchwin@file\@empty
      /F(\ef@launch@file)\else/F(\ef@launchwin@file)\fi
      \ifx\ef@launchwin@params\@empty\else
        /P(\ef@launchwin@params)\fi
      \ifx\ef@launchwin@open\@empty\else
        /O(\ef@launchwin@open)\fi
      \ifx\ef@launchwin@dir\@empty\else
        /D(\ef@launchwin@dir)\fi>>
    \fi
  }%
}
%    \end{macrocode}
%    The value of the \texttt{file} key can be an application or a file. If a
%    document, the operating system will launch an application associated with
%    the file extension. The path can be given relative to the current source
%    folder.
%    \begin{macrocode}
\define@key{eflaunch}{file}[]{\def\ef@launch@file{#1}}
\let\ef@launch@file\@empty
%    \end{macrocode}
%    The value of the \texttt{open} key is ignored, if the value of the
%    \texttt{file} key is not a PDF file.  If the file is a PDF, a new window
%    can be optionally opened, or not. The default is to use the user
%    preferences.
%    \begin{macrocode}
\define@choicekey{eflaunch}{open}[\val\nr]{userpref,new,existing}
{%
  \ifcase\nr\relax
    \let\ef@launch@open\@empty\or
    \def\ef@launch@open{/NewWindow true }\or
    \def\ef@launch@open{/NewWindow false }\fi
}{}
\let\ef@launch@open\@empty
%    \end{macrocode}
%    The PDF Specification allows for additional parameters to be passed on
%    the windows operating system. (No such key is available for the Mac or
%    for Unix.) This is a windows-only feature. The \texttt{winParams} key
%    itself takes key values pairs; these keys are \texttt{file} (\texttt{F}),
%    \texttt{directory} (\texttt{D}), \texttt{open} (\texttt{O}), and
%    \texttt{params} (\texttt{P}), these keys are defined below.
%    \begin{macrocode}
\define@key{eflaunch}{winParams}[]{\def\ef@launch@win{#1}%
    \setkeys{eflaunchwin}{#1}%
}\let\ef@launch@win\@empty
%    \end{macrocode}
%    As far as I can see, the \texttt{file} (application or file) must have a
%    full path (absolute path). The path should be enclosed in double quotes if
%    the path contains any spaces.
%    \begin{macrocode}
\define@key{eflaunchwin}{file}[]{\def\ef@launchwin@file{#1}}
\let\ef@launchwin@file\@empty
%    \end{macrocode}
% As far as I can see, this key does nothing. The value of the key
% is a path to the (startup) folder.
%    \begin{macrocode}
\define@key{eflaunchwin}{directory}[]{\def\ef@launchwin@dir{#1}}
\let\ef@launchwin@dir\@empty
%    \end{macrocode}
%    The \texttt{open} key takes one of two (documented) values: \texttt{open}
%    (the default) or \texttt{print}. But because these parameters are passed
%    to Window's \texttt{ShellExecute} a value of explore is recognized as well
%    (when the value of \texttt{file} is a path to a folder).
%    \begin{macrocode}
\define@key{eflaunchwin}{open}[]{\def\ef@launchwin@open{#1}}
\let\ef@launchwin@open\@empty
%    \end{macrocode}
% The launch parameters. Any paths must be absolute and enclosed in double quotes, if the
% path contains a space.
%    \begin{macrocode}
\define@key{eflaunchwin}{params}[]{\def\ef@launchwin@params{#1}}
\let\ef@launchwin@params\@empty
%    \end{macrocode}
% The \texttt{js}\IndexKey{js} key is used to execute JavaScript actions on a mouse up trigger.
% The argument is a JavaScript text string: |js={app.alert("Hello World!"}|.
% The value of \texttt{js} may be a macro containing JavaScript, which would include
% a macro created by the \env{defineJS} environment of \textsf{insdljs}.
%    \begin{macrocode}
\define@key{eforms}{js}[]{\@eqA{\JS{#1}}}
%    \end{macrocode}
% Next up are additional actions, and there are a lot of them. All these
% take JavaScript code as their values.%
% \begin{itemize}
%   \item \texttt{mouseup}\IndexKey{mouseup}: Executes its code with a mouse up event. If there is a JavaScript
%         action defined by the \texttt{js} key (or the |\A| key), the \texttt{js} (|\A|) action is executed.
%   \item \texttt{mousedown}\IndexKey{mousedown}: Executes when the mouse is hovering over the field and the user clicks
%         on the mouse.
%   \item \texttt{onenter}\IndexKey{onenter}: Executes its code when the user moves the mouse into the form field (the bounding rectangle).
%   \item \texttt{onexit}\IndexKey{onexit}: Executes its code when the user moves the mouse out of the form field (the bounding rectangle).
%   \item \texttt{onfocus}\IndexKey{onfocus}: Executes its code when the user brings the field into focus.
%   \item \texttt{onblur}\IndexKey{onblur}: Executes its code when the user brings the field loses focus (the user tabs away from
%         the field, or click outside the field).
%   \item \texttt{format}\IndexKey{format}: JavaScript to format the text that appears to
%           the user in a text field or editable combo box.
%   \item \texttt{keystroke}\IndexKey{keystroke}: JavaScript to process each keystroke in a text field or editable combo box.
%   \item \texttt{validate}\IndexKey{validate}: JavaScript to validate the committed data input into a text field or editable combo box.
%   \item \texttt{calculate}\IndexKey{calculate}: JavaScript to make calculates based on the values of other fields.
%   \item \texttt{pageopen}\IndexKey{pageopen}: JavaScript that executes when the page containing the field is opened.
%   \item \texttt{pageclose}\IndexKey{pageclose}: JavaScript that executes when the page containing the field is closed.
%   \item \texttt{pagevisible}\IndexKey{pagevisible}: JavaScript that executes when the page containing the field first becomes visible to the user.
%   \item \texttt{pageinvisible}\IndexKey{pageinvisible}: JavaScript that executes when the page containing the field is no longer visible to the user.
% \end{itemize}
%    \begin{macrocode}
\@tfor\ef@AActions:={{mouseup}{AAmouseup}}{{mousedown}{AAmousedown}}%
  {{onenter}{AAmouseenter}}{{onexit}{AAmouseexit}}%
  {{onfocus}{AAonfocus}}{{onblur}{AAonblur}}%
  {{format}{AAformat}}{{keystroke}{AAkeystroke}}%
  {{validate}{AAvalidate}}{{calculate}{AAcalculate}}%
  {{pageopen}{AApageopen}}{{pageclose}{AApageclose}}%
  {{pagevisible}{AApagevisible}}%
  {{pageinvisible}{AApageinvisible}}\do{%
  \expandafter\getargs\ef@AActions\ef@jstoks={#1}%
  \edef\temp@expand@def{\noexpand\define@key{eforms}{\aeb@argi}[]%
  {\noexpand\csname @eq\aeb@argii\noexpand\endcsname%
      {\the\ef@jstoks}}}%
  \temp@expand@def
}
%    \end{macrocode}
% The \texttt{objdef}\IndexKey{objdef} key is used to create indirect references to the form field.
% The value of this key must be unique throughout the whole document. Both \texttt{objdef}
% and \texttt{taborder}\IndexKey{taborder} are used for structured tabbing. (Distiller only)
%    \begin{macrocode}
\define@key{eforms}{objdef}{\@eqobjdef{#1}}
\define@key{eforms}{taborder}{\@eqtaborder{#1}}
%    \end{macrocode}
%    \begin{macrocode}
%</userinterface>
%    \end{macrocode}
%
%    \section{Input Driver Specific Code}
% Now bring in driver dependent macros that support form fields.
% \changes{v2.9j}{2017/01/22}{\string\cs{eq@rectH} and \string\cs{eq@rectW} pass
% through \string\cs{setlength} to enable arithmetic for the arguments of those commands}
%    \begin{macrocode}
%<*package>
%    \end{macrocode}
%    \begin{macrocode}
\input{\eq@drivercode}
%    \end{macrocode}
%    \begin{macrocode}
%</package>
%    \end{macrocode}
%    \subsection{For the \texttt{dvips} and \texttt{dvipsone} options}
%    \begin{macrocode}
%<*epdfmark>
%    \end{macrocode}
% This is the code for the \texttt{dvipsone} and \texttt{dvips}
% options.  These two are done together.  \textsf{hyperref}
% redefines the macro \cmd{\literalps@out} appropriate to each of
% these drivers.  Both use pdfmarks, so we can put them together.
% \par\medskip\noindent
% This sets the rectangle size allowing for a literal
% hyperlink---meaning we can insert arbitrary links actions.
%    \begin{macrocode}
\def\Rect#1{\pdf@rect{\textcolor{\@linkcolor}{#1}}}
%    \end{macrocode}
% Code to hide the solutions page to a quiz that has solutions.
%    \begin{macrocode}
\def\noPeek#1#2{\literalps@out{%
  \AEB@psMrk{ThisPage} << \noPeekAction{#1}{#2} >> /PUT pdfmark}}
%    \end{macrocode}
% We create an object definition for each field, there is an option for
% the author to specify a objdef name, and for a calculation field, this
% is done automatically.
%    \begin{macrocode}
\def\ef@getobjdef{%
  \HyField@AdvanceAnnotCount
  \ifisCalculate\edef\eq@objdefName{\Fld@name}\else
  \ifx\eq@objdef\@empty
    \edef\eq@objdefName{\annot@type\HyField@TheAnnotCount}\fi\fi
  \edef\eq@objdef{/_objdef \string{\eq@objdefName\string}}%
}
%    \end{macrocode}
% Driver dependent code (distiller) for choice fields, list and combo.
%    \begin{macrocode}
\def\eq@choice@driver
{%
  \Hy@pdfstringtrue
  \ifx\@vertRotate\ef@One
    \let\W@temp\eq@rectW\edef\eq@rectW{\eq@rectH}%
    \edef\eq@rectH{\W@temp}\fi\centerWidget\eq@rectH
  \ifx\autoCenter\ef@n\eqcenterWidget=0pt\fi
  \ef@getobjdef
  \pdf@rect{\lower\eqcenterWidget\ef@Bbox{\eq@rectW}{\eq@rectH}}%
  \literalps@out{\AEB@psMrk
    \eq@objdef/Rect [pdf@llx pdf@lly pdf@urx pdf@ury]
    \common@choiceCode
    /ANN pdfmark
    \AEB@psMrk{afields} {\eq@objdefName} /APPEND pdfmark
    \ifisCalculate\AEB@psMrk{corder} {\Fld@name} /APPEND pdfmark\fi
  }\to@insertStrucTabOrder{Form}\endgroup
  \dl@restorespcs
}
%    \end{macrocode}
% Driver dependent code (distiller) for push button fields.
%    \begin{macrocode}
\def\eq@Button@driver
{%
  \Hy@pdfstringtrue
  \ifx\eq@rectW\@empty\def\eq@rectW{\wd\eq@tmpbox}\ef@djXPD\fi
  \ifx\@vertRotate\ef@One\let\W@temp\eq@rectW\edef\eq@rectW{\eq@rectH}%
      \edef\eq@rectH{\W@temp}\fi\centerWidget\eq@rectH
  \ifx\autoCenter\ef@n\eqcenterWidget=0pt\fi
  \ef@getobjdef
  \pdf@rect{\lower\eqcenterWidget\ef@Bbox{\eq@rectW}{\eq@rectH}}%
  \literalps@out{\AEB@psMrk
    \eq@objdef/Rect [pdf@llx pdf@lly pdf@urx pdf@ury]
    \common@pushButtonCode
    /ANN pdfmark
    \AEB@psMrk{afields} {\eq@objdefName} /APPEND pdfmark
  }\to@insertStrucTabOrder{Form}\endgroup
  \dl@restorespcs
}
%    \end{macrocode}
% Driver dependent code (distiller) for radio and button fields.
%    \begin{macrocode}
\def\parentRef#1#2{\csarg\gdef{parent@#1}{#2}}
\def\ef@radioData#1#2{\immediate
  \write\@mainaux{\string\parentRef{#1}{#2}}}
\def\eq@Radio@driver
{%
  \Hy@pdfstringtrue
  \ifx\@vertRotate\ef@One
    \let\W@temp\eq@rectW\edef\eq@rectW{\eq@rectH}%
    \edef\eq@rectH{\W@temp}\fi\centerWidget\eq@rectH
  \ifx\autoCenter\ef@n\eqcenterWidget=0pt\fi
  \ef@getobjdef
%    \end{macrocode}
%    Additional code version dated 2019/06/14 or later.
%    \begin{macrocode}
  \ifuseNewRadios
    \ifx\isRadioParent\ef@YES\expandafter
      \ifx\csname radio@\Fld@name\endcsname\relax\else
        \literalps@out{\AEB@psMrk
          /_objdef{parent@rad\HyField@TheAnnotCount}
            /type/dict/OBJ pdfmark
          \AEB@psMrk{parent@rad\HyField@TheAnnotCount}
            << \radio@parent\space >>/PUT pdfmark
        }\ef@radioData{\Fld@name}{{parent@rad\HyField@TheAnnotCount}}%
        \ifx\eq@V\@empty\else
          \csarg\xdef{value@\Fld@name}%
            {\@nameuse{OnVal@\Fld@name}}%
        \fi
      \fi
      \ifx\ef@multigroupradios\ef@YES
        \csarg\gdef{multigroup@\Fld@name}%
          {\let\ef@multigroupradios\ef@YES}%
      \else
        \csarg\gdef{multigroup@\Fld@name}%
          {\let\ef@multigroupradios\ef@NO}%
      \fi
      \ifx\isRadiosInUnison\ef@YES
        \csarg\gdef{uniradios@\Fld@name}%
          {\let\isRadiosInUnison\ef@YES}%
      \else
        \csarg\gdef{uniradios@\Fld@name}%
          {\let\isRadiosInUnison\ef@NO}%
      \fi
    \fi
  \fi
  \ifuseNewRadios\expandafter\ef@NewRadiosLateOpts\fi
  \ifx\eq@rectW\@empty\def\eq@rectW{\wd\eq@tmpbox}\fi
  \pdf@rect{\lower\eqcenterWidget\ef@Bbox{\eq@rectW}{\eq@rectH}}%
  \literalps@out{\AEB@psMrk
    \eq@objdef/Rect [pdf@llx pdf@lly pdf@urx pdf@ury]
    \common@RadioCode
    /ANN pdfmark
  \ifuseNewRadios\else
    \AEB@psMrk{afields} {\eq@objdefName} /APPEND pdfmark
  \fi
  }\to@insertStrucTabOrder{Form}%
%    \end{macrocode}
%    Save the object reference to this kid
%    \begin{macrocode}
  \ifuseNewRadios
    \edef\x{\noexpand\g@addto@macro\noexpand
      \ef@KidsArray{{\eq@objdefName}\space}}\x
    \csarg\xdef{kid@\Fld@name}{\ef@KidsArray}%
  \fi
  \endgroup
  \dl@restorespcs
}
\def\eq@Check@driver
{%
  \Hy@pdfstringtrue
  \ifx\@vertRotate\ef@One
    \let\W@temp\eq@rectW\edef\eq@rectW{\eq@rectH}%
    \edef\eq@rectH{\W@temp}\fi\centerWidget\eq@rectH
  \ifx\autoCenter\ef@n\eqcenterWidget=0pt\fi
  \ef@getobjdef
  \ifx\eq@rectW\@empty\def\eq@rectW{\wd\eq@tmpbox}\fi
  \pdf@rect{\lower\eqcenterWidget\ef@Bbox{\eq@rectW}{\eq@rectH}}%
  \literalps@out{\AEB@psMrk
    \eq@objdef/Rect [pdf@llx pdf@lly pdf@urx pdf@ury]
    \common@CheckCode
    /ANN pdfmark
    \AEB@psMrk{afields} {\eq@objdefName} /APPEND pdfmark
  }\to@insertStrucTabOrder{Form}\endgroup
  \dl@restorespcs
}
\def\eq@l@check@driver
{%
  \ef@getobjdef
  \pdf@rect{\makebox[\eq@tmpdima]{\phantom{\link@@Content}}}%
  \literalps@out{\AEB@psMrk
    \eq@objdef/Rect [pdf@llx pdf@lly pdf@urx pdf@ury]
    \common@CheckCode
    /ANN pdfmark
    \AEB@psMrk{afields} {\eq@objdefName} /APPEND pdfmark
  }\endgroup
  \dl@restorespcs
}
%    \end{macrocode}
% Driver dependent code for text fields.
%    \begin{macrocode}
\def\eq@TextField{\Hy@pdfstringtrue
  \ifx\@vertRotate\ef@One
    \let\W@temp\eq@rectW\edef\eq@rectW{\eq@rectH}%
    \edef\eq@rectH{\W@temp}\fi\centerWidget\eq@rectH
  \ifx\autoCenter\ef@n\eqcenterWidget=0pt\fi
  \ef@getobjdef
  \pdf@rect{\lower\eqcenterWidget\ef@Bbox{\eq@rectW}{\eq@rectH}}%
  \literalps@out{\AEB@psMrk
    \eq@objdef/Rect [pdf@llx pdf@lly pdf@urx pdf@ury]
    \common@TextFieldCode
    /ANN pdfmark
    \AEB@psMrk{afields} {\eq@objdefName} /APPEND pdfmark
    \ifisCalculate\AEB@psMrk{corder} {\Fld@name} /APPEND pdfmark\fi
  }\to@insertStrucTabOrder{Form}\endgroup
  \dl@restorespcs
}
%    \end{macrocode}
% \changes{v2.5p}{2012/09/25}{Corrected a bug in \string\cs{eq@SigField} for
% the dvipdfm-type drivers}
%    \begin{macrocode}
\def\eq@SigField{\Hy@pdfstringtrue
  \ifx\@vertRotate\ef@One
    \let\W@temp\eq@rectW\edef\eq@rectW{\eq@rectH}%
    \edef\eq@rectH{\W@temp}\fi\centerWidget\eq@rectH
  \ifx\autoCenter\ef@n\eqcenterWidget=0pt\fi
  \ef@getobjdef
  \pdf@rect{\lower\eqcenterWidget\ef@Bbox{\eq@rectW}{\eq@rectH}}%
  \literalps@out{\AEB@psMrk
    \eq@objdef/Rect [pdf@llx pdf@lly pdf@urx pdf@ury]
    \common@SigFieldCode
    /ANN pdfmark
    \AEB@psMrk{afields} {\eq@objdefName} /APPEND pdfmark
  }\to@insertStrucTabOrder{Form}\endgroup
  \dl@restorespcs
}
%    \end{macrocode}
% For processing the \texttt{pdfmark} with distiller, the key \texttt{/Action} is
% required (not \texttt{/A}). This macro converts \texttt{/A} to \texttt{/Action},
% and is used for the drivers using distiller.
%    \begin{macrocode}
\def\convertAToAction/A#1\@nil{\def\eq@A{/Action#1}}
%    \end{macrocode}
% Driver dependent code for links.
%    \begin{macrocode}
\def\setLink@driver
{%
  \ifx\eq@A\@empty\else\expandafter\convertAToAction\eq@A\@nil\fi
  \@eqBS{}%
  \pdf@rect{\link@@Box}%
  \literalps@out{\AEB@psMrk
    \eq@objdef/Rect [pdf@llx pdf@lly pdf@urx pdf@ury]
  \eq@Border
  \common@LinkCode
  /Subtype /Link
  /ANN pdfmark}%
  \to@insertStrucTabOrder{Link}\endgroup
  \dl@restorespcs
}
%    \end{macrocode}
%   (2018/03/22) Defined \cs{pboxRect} to support
%    \cs{setLinkPbox}.
%    \changes{v2.9.19}{2018/03/22}{Defined \string\cs{pboxRect} to support
%    \string\cs{setLinkPbox}}
%    \begin{macrocode}
\def\pboxRect{/Rect [\par@@Rect]}
%    \end{macrocode}
%    Added \cs{mllnkcontainer} to support textsf{aeb\_mlink} package.
%    \changes{v2.9.17}{2018/03/14}{Added \string\cs{mllnkcontainer}.}
%    \begin{macrocode}
\def\mllnkcontainer#1{#1}
\def\setLinkPbox@driver
{%
  \ifx\eq@A\@empty\else\expandafter\convertAToAction\eq@A\@nil\fi
  \@eqBS{}%
  \literalps@out{\mllnkcontainer{%
  \AEB@psMrk\eq@objdef\pboxRect
  \eq@Border
  \eq@QuadPoints  % QuadPoints
  \common@LinkCode
  /Subtype /Link
  /ANN pdfmark}}%
  \to@insertStrucTabOrder{Link}\endgroup
  \dl@restorespcs
}
%    \end{macrocode}
%    \begin{macrocode}
%</epdfmark>
%    \end{macrocode}
%    \subsection{For the \texttt{pdftex} option}
%    \begin{macrocode}
%<*epdftex>
%    \end{macrocode}
% Code used in the case of the \texttt{pdftex} option.
%\par\medskip\noindent
% Code to hide the solutions page to a quiz that has solutions.
%    \begin{macrocode}
\def\noPeek#1#2{\global\pdfpageattr=\expandafter{\noPeekAction{#1}{#2}}}
%    \end{macrocode}
% Support for automatic calculation fields for the \textsf{pdflatex} driver. The command
% \cs{HyField@AddToFields}, not modified here, is inserted at the end of the code
% for \cs{eq@choice@driver} and \cs{eq@TextField} below.
%\changes{v2.8}{2014/11/23}{Modified a macro from hyperref (2012/11/06) to support
%automatic calculation of fields using eforms (and hyperref).}
%    \begin{macrocode}
\def\HyField@@AddToFields#1{%
  \HyField@AfterAuxOpen{%
    \if@filesw
      \write\@mainaux{%
        \string\HyField@AuxAddToFields{#1}}%
% added for eforms
    \ifisCalculate\write\@mainaux{%
        \string\HyField@AuxAddToCoFields{}{#1}}\fi
% end eforms
    \fi
  }%
}%
%    \end{macrocode}
% driver dependent code for choice fields
%    \begin{macrocode}
\def\eq@choice@driver
{%
  \Hy@pdfstringtrue
  \ifx\@vertRotate\ef@One
    \let\W@temp\eq@rectW\edef\eq@rectW{\eq@rectH}%
    \edef\eq@rectH{\W@temp}\fi\centerWidget\eq@rectH
  \ifx\autoCenter\ef@n\eqcenterWidget=0pt\fi
  \hbox{\pdfstartlink user{\common@choiceCode}%
  \lower\eqcenterWidget\ef@Bbox{\eq@rectW}{\eq@rectH}\pdfendlink}%
  \HyField@AddToFields
  \endgroup
  \dl@restorespcs
}
\def\eq@Button@driver
{%
  \Hy@pdfstringtrue
  \ifx\eq@rectW\@empty\def\eq@rectW{\wd\eq@tmpbox}\ef@djXPD\fi
  \ifx\@vertRotate\ef@One
    \let\W@temp\eq@rectW\edef\eq@rectW{\eq@rectH}%
    \edef\eq@rectH{\W@temp}\fi\centerWidget\eq@rectH
  \ifx\autoCenter\ef@n\eqcenterWidget=0pt\fi
  \ifx\eq@rectW\@empty\def\eq@rectW{\wd\eq@tmpbox}\fi
  \hbox{\pdfstartlink user{\common@pushButtonCode}%
  \lower\eqcenterWidget\ef@Bbox{\eq@rectW}{\eq@rectH}\pdfendlink}%
  \HyField@AddToFields
  \endgroup
  \dl@restorespcs
}
%    \end{macrocode}
%    \begin{macrocode}
\def\parentRef#1#2{\csarg\gdef{parent@#1}{#2 0 R}}
\def\ef@radioData#1#2{\expandafter
  \HyField@@AddToFields\expandafter{#2}%
  \immediate\write\@mainaux{\string\parentRef{#1}{#2}}}
\def\eq@Radio@driver{\Hy@pdfstringtrue
  \ifx\@vertRotate\ef@One
    \let\W@temp\eq@rectW\edef\eq@rectW{\eq@rectH}%
    \edef\eq@rectH{\W@temp}\fi\centerWidget\eq@rectH
  \ifx\autoCenter\ef@n\eqcenterWidget=0pt\fi
%    \end{macrocode}
%    Additional code version dated 2019/06/14 or later.
%    \begin{macrocode}
  \ifuseNewRadios
    \ifx\isRadioParent\ef@YES\expandafter
      \ifx\csname radio@\Fld@name\endcsname\relax\else
        \immediate\pdfobj{<< \radio@parent\space >>}%
        \ef@radioData{\Fld@name}{\the\pdflastobj}%
        \ifx\eq@V\@empty\else
            \csarg\xdef{value@\Fld@name}%
              {\@nameuse{OnVal@\Fld@name}}%
        \fi
      \fi
      \ifx\ef@multigroupradios\ef@YES
        \csarg\gdef{multigroup@\Fld@name}%
          {\let\ef@multigroupradios\ef@YES}%
      \else
        \csarg\gdef{multigroup@\Fld@name}%
          {\let\ef@multigroupradios\ef@NO}%
      \fi
      \ifx\isRadiosInUnison\ef@YES
        \csarg\gdef{uniradios@\Fld@name}%
          {\let\isRadiosInUnison\ef@YES}%
      \else
        \csarg\gdef{uniradios@\Fld@name}%
          {\let\isRadiosInUnison\ef@NO}%
      \fi
    \fi
  \fi
  \ifuseNewRadios\expandafter\ef@NewRadiosLateOpts\fi
  \ifx\eq@rectW\@empty\def\eq@rectW{\wd\eq@tmpbox}\fi
  \hbox{\pdfstartlink user{\common@RadioCode}%
  \lower\eqcenterWidget\ef@Bbox{\eq@rectW}{\eq@rectH}\pdfendlink}%
%    \end{macrocode}
%    Save the object reference to this kid
%    \begin{macrocode}
  \ifuseNewRadios
    \edef\x{\noexpand\g@addto@macro\noexpand
      \ef@KidsArray{\the\pdflastlink\space 0 R\space}}\x
    \csarg\xdef{kid@\Fld@name}{\ef@KidsArray}%
  \else
    \HyField@AddToFields
  \fi
  \endgroup
  \dl@restorespcs
}
\def\eq@Check@driver
{%
  \Hy@pdfstringtrue
  \ifx\@vertRotate\ef@One
    \let\W@temp\eq@rectW\edef\eq@rectW{\eq@rectH}%
    \edef\eq@rectH{\W@temp}\fi\centerWidget\eq@rectH
  \ifx\autoCenter\ef@n\eqcenterWidget=0pt\fi
  \ifx\eq@rectW\@empty\def\eq@rectW{\wd\eq@tmpbox}\fi
  \hbox{\pdfstartlink user{\common@CheckCode}%
  \lower\eqcenterWidget\ef@Bbox{\eq@rectW}{\eq@rectH}\pdfendlink}%
  \HyField@AddToFields
  \endgroup
  \dl@restorespcs
}
\def\eq@l@check@driver
{%
  \pdfstartlink user{\common@CheckCode}%
  \makebox[\eq@tmpdima]{\phantom{\link@@Content}}%
  \pdfendlink\HyField@AddToFields\endgroup
  \dl@restorespcs
}
\def\eq@TextField{\Hy@pdfstringtrue
  \ifx\@vertRotate\ef@One
    \let\W@temp\eq@rectW\edef\eq@rectW{\eq@rectH}%
    \edef\eq@rectH{\W@temp}\fi\centerWidget\eq@rectH
  \ifx\autoCenter\ef@n\eqcenterWidget=0pt\fi
  \leavevmode
  \hbox{\pdfstartlink user{\common@TextFieldCode}%
  \lower\eqcenterWidget\ef@Bbox{\eq@rectW}{\eq@rectH}\pdfendlink}%
  \HyField@AddToFields
  \endgroup
  \dl@restorespcs
}
\def\eq@SigField{\Hy@pdfstringtrue
  \ifx\@vertRotate\ef@One
    \let\W@temp\eq@rectW\edef\eq@rectW{\eq@rectH}%
    \edef\eq@rectH{\W@temp}\fi\centerWidget\eq@rectH
  \ifx\autoCenter\ef@n\eqcenterWidget=0pt\fi
  \leavevmode\hbox{\pdfstartlink user{\common@SigFieldCode}%
  \lower\eqcenterWidget\ef@Bbox{\eq@rectW}{\eq@rectH}\pdfendlink}%
  \HyField@AddToFields
  \endgroup
  \dl@restorespcs
}
\def\setLink@driver
{%
  \@eqBS{}%
  \leavevmode\pdfstartlink
  attr {\eq@Border}%
  user{/Subtype/Link \common@LinkCode}%
  \Hy@colorlink{\@linkcolor}\link@@Box
  \close@pdflink
  \endgroup
  \dl@restorespcs
}
\def\ef@setTabOrder{\ifx\ef@taborder\@empty\else
  \edef\ef@tmp@toks{\the\pdfpageattr\space/Tabs/\ef@taborder}%
  \global\pdfpageattr=\expandafter{\ef@tmp@toks}%
  \fi\endgroup
}
%    \end{macrocode}
%    \begin{macrocode}
%</epdftex>
%    \end{macrocode}
%    \subsection{For the \texttt{dvipdfm}, \texttt{dvipdfmx}, \texttt{xetex} options}
%    \begin{macrocode}
%<*edvipdfm>
%    \end{macrocode}
% Code to hide the solutions page to a quiz that has solutions.
%    \begin{macrocode}
\def\noPeek#1#2{\@pdfm@mark{put @thispage << \noPeekAction{#1}{#2} >> }}
%    \end{macrocode}
%    (2016/12/22) Removed \cs{ef@adjHWxetex} in favor of \cs{ef@djXPD}.
%    \changes{v2.9d}{2016/12/22}{Removed \string\cs{ef@adjHWxetex} in favor of \string\cs{ef@djXPD}}
%    \begin{macrocode}
\let\ef@adjHWxetex\relax
%    \end{macrocode}
%    \begin{macrocode}
\def\eq@choice@driver{\ef@adjHWxetex
    \Hy@pdfstringtrue
  \ifx\@vertRotate\ef@One
    \let\W@temp\eq@rectW\edef\eq@rectW{\eq@rectH}%
    \edef\eq@rectH{\W@temp}\fi\centerWidget\eq@rectH
  \ifx\autoCenter\ef@n\eqcenterWidget=0pt\fi
  \leavevmode\setbox\pdfm@box=%
      \hbox{\lower\eqcenterWidget\ef@Bbox{\eq@rectW}{\eq@rectH}}%
  \@pdfm@mark{ann @\Fld@name\space\dvipdfm@setdim
  <<\common@choiceCode>>}\unhbox\pdfm@box\relax%
  \@pdfm@mark{put @afields @\Fld@name}% record in @afields array
  \ifisCalculate\@pdfm@mark{put @corder @\Fld@name}\fi
  \dl@restorespcs
  \endgroup
}
%    \end{macrocode}
% (2013/06/09) xelatex apparently includes the boundary in its width and height
% calculations. So we must too.
%    \begin{macrocode}
\def\eq@Button@driver{\Hy@pdfstringtrue
  \ifx\eq@rectW\@empty\def\eq@rectW{\wd\eq@tmpbox}\ef@djXPD\fi
  \ifx\@vertRotate\ef@One
    \let\W@temp\eq@rectW\edef\eq@rectW{\eq@rectH}%
    \edef\eq@rectH{\W@temp}\fi\centerWidget\eq@rectH
  \ifx\autoCenter\ef@n\eqcenterWidget=0pt\fi
  \setbox\pdfm@box=%
      \hbox{\lower\eqcenterWidget\ef@Bbox{\eq@rectW}{\eq@rectH}}%
  \@pdfm@mark{ann @\Fld@name\space\dvipdfm@setdim
  << \common@pushButtonCode >>}\unhbox\pdfm@box\relax%
  \@pdfm@mark{put @afields @\Fld@name}% record in @afields array
  \endgroup
  \dl@restorespcs
}
%    \end{macrocode}
%    \begin{macrocode}
\def\parentRef#1#2{\csarg\gdef{parent@#1}{#2}}
\def\ef@radioData#1#2{%\expandafter
%  \HyField@@AddToFields\expandafter{#2}%
  \immediate\write\@mainaux{\string\parentRef{#1}{#2}}}
%    \end{macrocode}
%    \begin{macrocode}
\def\eq@Radio@driver{\ef@adjHWxetex\Hy@pdfstringtrue
  \ifx\@vertRotate\ef@One
    \let\W@temp\eq@rectW\edef\eq@rectW{\eq@rectH}%
    \edef\eq@rectH{\W@temp}\fi\centerWidget\eq@rectH
  \ifx\autoCenter\ef@n\eqcenterWidget=0pt\fi
%    \end{macrocode}
%    Additional code version dated 2019/06/14 or later.
%    \begin{macrocode}
  \ifuseNewRadios
    \ifx\isRadioParent\ef@YES\expandafter
      \ifx\csname radio@\Fld@name\endcsname\relax\else
        \immediate\@pdfm@mark{obj @parentobj\HyField@TheAnnotCount
          << \radio@parent\space >>}%
        \ef@radioData{\Fld@name}{@parentobj\HyField@TheAnnotCount}%
        \ifx\eq@V\@empty\else
            \csarg\xdef{value@\Fld@name}%
              {\@nameuse{OnVal@\Fld@name}}%
        \fi
      \fi
      \ifx\ef@multigroupradios\ef@YES
        \csarg\gdef{multigroup@\Fld@name}%
          {\let\ef@multigroupradios\ef@YES}%
      \else
        \csarg\gdef{multigroup@\Fld@name}%
          {\let\ef@multigroupradios\ef@NO}%
      \fi
      \ifx\isRadiosInUnison\ef@YES
        \csarg\gdef{uniradios@\Fld@name}%
          {\let\isRadiosInUnison\ef@YES}%
      \else
        \csarg\gdef{uniradios@\Fld@name}%
          {\let\isRadiosInUnison\ef@NO}%
      \fi
    \fi
  \fi
  \ifuseNewRadios\expandafter\ef@NewRadiosLateOpts\fi
  \ifx\eq@rectW\@empty\def\eq@rectW{\wd\eq@tmpbox}\fi
  \HyField@AdvanceAnnotCount
  \setbox\pdfm@box=%
    \hbox{\lower\eqcenterWidget\ef@Bbox{\eq@rectW}{\eq@rectH}}%
  \ifx\annot@type\annot@type@checkbox\def\btn@type{check}\else
    \def\btn@type{radio}\fi
  \@pdfm@mark{ann @\btn@type\HyField@TheAnnotCount\space\dvipdfm@setdim
  <<\common@RadioCode>>}\unhbox\pdfm@box\relax%
%    \end{macrocode}
%    Save the object reference to this kid
%    \begin{macrocode}
  \ifuseNewRadios
    \edef\x{\noexpand\g@addto@macro\noexpand
      \ef@KidsArray{@parentobj\HyField@TheAnnotCount\space}}\x
    \csarg\xdef{kid@\Fld@name}{\ef@KidsArray}%
  \else
    \@pdfm@mark{put @afields @\btn@type\HyField@TheAnnotCount}%
  \fi
  \endgroup
  \dl@restorespcs
}
%    \end{macrocode}
%    \begin{macrocode}
\def\eq@Check@driver{\ef@adjHWxetex\Hy@pdfstringtrue
  \ifx\@vertRotate\ef@One
    \let\W@temp\eq@rectW\edef\eq@rectW{\eq@rectH}%
    \edef\eq@rectH{\W@temp}\fi\centerWidget\eq@rectH
  \ifx\autoCenter\ef@n\eqcenterWidget=0pt\fi
  \ifx\eq@rectW\@empty\def\eq@rectW{\wd\eq@tmpbox}\fi
  \HyField@AdvanceAnnotCount
  \setbox\pdfm@box=%
    \hbox{\lower\eqcenterWidget\ef@Bbox{\eq@rectW}{\eq@rectH}}%
  \ifx\annot@type\annot@type@checkbox\def\btn@type{check}\else
    \def\btn@type{radio}\fi
  \@pdfm@mark{ann @\btn@type\HyField@TheAnnotCount\space\dvipdfm@setdim
  <<\common@CheckCode>>}\unhbox\pdfm@box\relax%
  \@pdfm@mark{put @afields @\btn@type\HyField@TheAnnotCount}%
  \endgroup
  \dl@restorespcs
}
\def\eq@l@check@driver{%
  \HyField@AdvanceAnnotCount
  \setbox\pdfm@box=%
    \hbox{\makebox[\eq@tmpdima]{\phantom{\link@@Content}}}%
  \@pdfm@mark{ann @check\HyField@TheAnnotCount\space
    \dvipdfm@setdim<<\common@CheckCode>>}%
  \unhbox\pdfm@box\relax
  \@pdfm@mark{put @afields @check\HyField@TheAnnotCount}%
  \endgroup
}
\def\eq@TextField{\ef@adjHWxetex\Hy@pdfstringtrue
  \ifx\@vertRotate\ef@One
    \let\W@temp\eq@rectW\edef\eq@rectW{\eq@rectH}%
    \edef\eq@rectH{\W@temp}\fi\centerWidget\eq@rectH
  \ifx\autoCenter\ef@n\eqcenterWidget=0pt\fi
  \leavevmode\setbox\pdfm@box=%
    \hbox{\lower\eqcenterWidget\ef@Bbox{\eq@rectW}{\eq@rectH}}%
  \@pdfm@mark{ann @\Fld@name\space\dvipdfm@setdim
    << \common@TextFieldCode >>}\unhbox\pdfm@box\relax%
  \@pdfm@mark{put @afields @\Fld@name}% record in @afields array
  \ifisCalculate\@pdfm@mark{put @corder @\Fld@name}\fi
  \endgroup
  \dl@restorespcs
}
\def\eq@SigField{\ef@adjHWxetex\Hy@pdfstringtrue
  \ifx\@vertRotate\ef@One
    \let\W@temp\eq@rectW\edef\eq@rectW{\eq@rectH}%
    \edef\eq@rectH{\W@temp}\fi\centerWidget\eq@rectH
  \ifx\autoCenter\ef@n\eqcenterWidget=0pt\fi
  \leavevmode\setbox\pdfm@box=%
    \hbox{\lower\eqcenterWidget\ef@Bbox{\eq@rectW}{\eq@rectH}}%
  \@pdfm@mark{ann @\Fld@name\space\dvipdfm@setdim
    << \common@SigFieldCode >>}\unhbox\pdfm@box\relax%
  \@pdfm@mark{put @afields @\Fld@name}% record in @afields array
  \endgroup
  \dl@restorespcs
}
\def\setLink@driver{%
  \@eqBS{}\leavevmode
  \@pdfm@mark{bann
    <</Subtype/Link\eq@Border\common@LinkCode>>}%
    \Hy@colorlink{\@linkcolor}\link@@Box\Hy@endcolorlink
  \@pdfm@mark{eann}%
  \endgroup
  \dl@restorespcs
}
\def\ef@setTabOrder{\ifx\ef@taborder\@empty\else
  \@pdfm@mark{ put @thispage << /Tabs/\ef@taborder >> }%
  \fi\endgroup
}
%</edvipdfm>
%    \end{macrocode}
%    \begin{macrocode}
%<*setcorder>
%    \end{macrocode}
%    \section{Document JavaScripts}
%    \subsection{Support for setting calculation order}
%    \begin{macrocode}
\begin{insDLJS}{cojs}{eforms: JavaScript to set calculation order}
var debugCalc=false;
ef_setCalcOrder.lastIndex=0;
function ef_setCalcOrder (a) {
  var o1, o2, f;
  while ( a.length > 0) {
    if (a.length > 1) {
      f=a.shift();
      o1=this.getField(f);
      if ( o1 == null ) {
        ef_CalcOrderErr(f);
        continue;
      }
      f = a[0];
      o2=this.getField(f);
      if ( o2 == null ) {
        ef_CalcOrderErr(f);
        a.shift();
        continue;
      }
      if (  o2.calcOrderIndex < o1.calcOrderIndex ) {
        o2.calcOrderIndex=o1.calcOrderIndex+1;
        ef_setCalcOrder.lastIndex=o2.calcOrderIndex;
      }
    } else {
      f=a.shift();
      o1=this.getField(f);
      if ( o1 == null ) {
        ef_CalcOrderErr(f);
        continue;
      }
      o1.calcOrderIndex=ef_setCalcOrder.lastIndex;			
    }
  }
}
function ef_CalcOrderErr(f) {
  console.show(); app.beep(0);
  console.println("calcOrder: the field \""+ f
    + "\" does not exist in this document, skipping it.\n\n"
    + "calcOrder: Check the case sensitive spelling of the field.");
}
var _EfCalcOrder=\efCalcOrder;
ef_setCalcOrder(_EfCalcOrder);
\end{insDLJS}
%</setcorder>
%    \end{macrocode}
%    \begin{macrocode}
%<*package>
\inputCalcOrderJS
\catcode`\$=\ef@CatChngs
%</package>
%    \end{macrocode}
%  \Finale
\endinput
