% \iffalse
% makeindex -s gglo.ist -o insdljs.gls insdljs.glo
% makeindex -s gind.ist -o insdljs.ind insdljs.idx
%<*copyright>
%%%%%%%%%%%%%%%%%%%%%%%%%%%%%%%%%%%%%%%%%%%%%%%%%%%%%%%%%%
%% insdljs.sty package,                                 %%
%% Copyright (C) 2001-2021  D. P. Story                 %%
%%   dpstory@uakron.edu                                 %%
%%                                                      %%
%% This program can redistributed and/or modified under %%
%% the terms of the LaTeX Project Public License        %%
%% Distributed from CTAN archives in directory          %%
%% macros/latex/base/lppl.txt; either version 1 of the  %%
%% License, or (at your option) any later version.      %%
%%%%%%%%%%%%%%%%%%%%%%%%%%%%%%%%%%%%%%%%%%%%%%%%%%%%%%%%%%
%</copyright>
%<package>\NeedsTeXFormat{LaTeX2e}
%<package>\ProvidesPackage{insdljs}
%<package> [2021/06/19 v2.5.5 Insert Document Level JavaScripts (dps)]
%<*driver>
\documentclass{ltxdoc}
\usepackage[colorlinks,hyperindex=false]{hyperref}
\usepackage{fancyvrb}
\makeatletter
\let\@latex@warning@no@line\@gobble
\renewcommand{\paragraph}{\@startsection{paragraph}{4}{0pt}{6pt}{-3pt}{\bfseries}}
\makeatother
\usepackage{makeidx}
\makeindex
\def\EXPT{!}
\pdfstringdefDisableCommands{\let\\\textbackslash}%
% \OnlyDescription  % comment out for implementation details
\EnableCrossrefs
\CodelineIndex
\RecordChanges
\gdef\darg#1{\texttt{\char123\relax#1\char125\relax}}
\let\env\texttt
\let\opt\texttt
\let\app\textsf
\let\pkg\textsf
\let\uif\textsf
\def\visispace{\symbol{32}}
\def\ameta#1{\ensuremath{\langle\textit{\texttt{#1}}\rangle}}
\def\meta#1{\textsl{\texttt{#1}}}
\def\SUB#1{\ensuremath{{}_{\mbox{\scriptsize\ttfamily#1}}}}
\InputIfFileExists{aebdocfmt.def}{\PackageInfo{insdljs}{Inputting aebdocfmt.def}}
    {\def\IndexOpt{\DescribeMacro}\def\IndexKey{\DescribeMacro}\let\setupFullwidth\relax
     \PackageInfo{insdljs}{aebdocfmt.def cannot be found}}
\begin{document}
  \GetFileInfo{insdljs.sty}
  \title{The \texttt{insDLJS} Package}
  \author{D. P. Story\\
    Email: \texttt{dpstory@uakron.edu}}
  \date{processed \today}
  \maketitle
  \tableofcontents
  \let\Email\texttt
  \let\pkg\textsf
  \DocInput{insdljs.dtx}
\IfFileExists{\jobname.ind}{\newpage\setupFullwidth\PrintIndex}{\paragraph*{Index} The index goes here.\\Execute
    \texttt{makeindex -s gind.ist -o insdljs.ind insdljs.idx} on the command line and recompile
    \texttt{insdljs.dtx}.}
\IfFileExists{\jobname.gls}{\PrintChanges}{\paragraph*{Change History} The list of changes goes here.\\Execute
    \texttt{makeindex -s gglo.ist -o insdljs.gls insdljs.glo} on the command line and recompile
    \texttt{insdljs.dtx}.}
\end{document}
%</driver>
% \fi
%
% \InputIfFileExists{aebdonotindex.def}{\PackageInfo{insdljs}{Inputting aebdonotindex.def}}
%    {\PackageInfo{insdljs}{aebdonotindex.def cannot be found}}
%
%    \section{Introduction}
% This package defines a new environment, \texttt{insDLJS}, used
% for inserting Acrobat JavaScript into a PDF file created from a
% \LaTeX{} source. This package works correctly for users of
% \textsf{pdftex} or \textsf{dvipdfm}. For those that use the
% \textsf{Acrobat Distiller} (specifically, those that use either
% \textsf{dvips} or \textsf{dvipsone} to produce a postscript file,
% which is then distilled), you are required to have Acrobat~5.0 (or
% later).
% \changes{v2.5.1}{2021/02/04}{Updated \string\texttt{aeb\_man.tex} and
% \string\texttt{eformman.tex} to reflect additional security restriction
% on \string\app{Acrobat}.}
%
% \section{Features of the \textsf{insdljs} package}
% We present a few of the several features of the package to get some feel for its
% purpose.
%
% \subsection{The \texttt{insDLJS} Environment}
%
% The following is a quick illustration of the use of the new environment.
%\begin{Verbatim}[xleftmargin=\parindent,fontsize=\small,
%commandchars={!~@},codes={\catcode`\%=9}]
%\documentclass{article}
%\usepackage{hyperref}
%\usepackage{insdljs}
%\newcommand\tugHello{Welcome to TUG 2001!}
%\begin{insDLJS}[HelloWorld]{mydljs}{My Private DLJS}
%function HelloWorld(){app.alert("\tugHello", 3);}
%\end{insDLJS}
%\begin{document}
%\begin{Form}     !%!textsf~ needed for !cs~PushButton@@
%
%\section{Test of the \textsf{insDLJS} Package}
%
%!%!textsf~ use the built-in form button of hyperref@
%Push \PushButton[name=myButton,onclick={HelloWorld();}]{Button}
%
%\end{Form}
%\end{document}
%\end{Verbatim}
% This environment takes three parameters, the first of which is required
% for users of \texttt{dvips} and \texttt{dvipsone} and optional otherwise.
% Within the environment, `|\|', is the escape character, and `|%|' is the comment
% character, as they are in \TeX{} and \LaTeX.
% See the documentation preceding the definition of the
% \texttt{\hyperlink{insDLJS}{insDLJS}} environment.
%
% \subsection{Debugging}
%
% Another feature of this package is the ability to place debugging markers
% within the JavaScript. When the \texttt{debug} option is used, the markers
% show up in the document level JavaScript; otherwise, they are not written
% to the PDF document. Here is a simple example,
%\begin{Verbatim}[xleftmargin=\parindent,fontsize=\small,
%commandchars={*~@},codes={\def\1{\quad}\catcode`\%=9}]
%\begin{insDLJS}{mydljs}{My Private DLJS}
%function HelloWorld() {
%*1app.alert("\tugHello", 3);
%*1var x = "\\";
%\db console.println("Entered HelloWorld function and x = " + x);\db!%
%}
%\end{insDLJS}
%\end{Verbatim}
% The last line of the function will appear in the DLJS if compiled under
% the \texttt{debug} option; otherwise, it is removed. Additional discussion
% of this debugging device is given below.
%
% \subsection{Open Action}
%
% This package also defines an \cs{OpenAction} command to introduce actions that are
% executed when the PDF document is  opened on page~1. The open action command only
% applies to page~1.
%\paragraph*{Usage}
%\begin{Verbatim}[xleftmargin=\parindent,fontsize=\small,
%codes={\catcode`\%=9}]
%\OpenAction{/S/JavaScript/JS(app.alert("Hello World!");)}
%\end{Verbatim}
% The package defines a \cs{JS} command that takes one argument. The code |\JS{#1}| expands to
% |/S/JavaScript/JS (#1)|, and so the above example may be written as,
%\begin{Verbatim}[xleftmargin=\parindent,fontsize=\small,
%codes={\catcode`\%=9}]
%\OpenAction{\JS{app.alert("Hello World!");}}
%\end{Verbatim}%
% Multiple JavaScript commands can be entered. For nice formatting use \cs{r}
% and \cs{t} for carriage return and tab, respectively.
%\goodbreak
%\begin{Verbatim}[xleftmargin=\parindent,fontsize=\small,
%commandchars={*~@},codes={\def\1{\quad}\catcode`\%=9}]
%\OpenAction{\JS{*%
%*1app.alert("Hello World!");\r
%*1app.alert("Good Day to You!");
%}}
%\end{Verbatim}
% The open actions are placed in a token list. Additional actions are added to the token list, so
% you say
%\begin{Verbatim}[xleftmargin=\parindent,fontsize=\small,
%commandchars={*~@},codes={\def\1{\quad}\catcode`\%=9}]
%\OpenAction{\JS{app.alert("Hello World!");}}
%\OpenAction{\JS{app.alert("Good Day to You!");}}
%\end{Verbatim}
% the two messages will appear each time you open page~1 of the document.
%
% In the case of users of the distiller, \textsf{insdljs} uses this mechanism to
% define an open action. To avoid overwriting this definition, the \cs{OpenAction}
% command is necessary in insert additional actions.
%
% See \Nameref{openaction} for more details.
%
% \subsection{How \textsf{insdljs} Works}
%
% Let me describe in rough terms what goes on behind the scenes. In the discussion
% below, the following example will be used:
%\begin{Verbatim}[xleftmargin=\parindent,fontsize=\small,
%commandchars={*~@},codes={\def\1{\quad}\catcode`\%=9}]
%\newcommand\tugHello{"Hello World!"}
%\begin{insDLJS}{mydljs}{My Private DLJS}
%function HelloWorld(){
%*1app.alert("\tugHello", 3);
%}
%\end{insDLJS}
%\end{Verbatim}
%
% \subsubsection{For the \texttt{pdftex} and \texttt{dvipdfm} Options}
%
% Both \texttt{pdftex} and \texttt{dvipdfm} have primitives/macros for inserting
% JavaScript at the document level into the PDF document. The work of the package,
% in this case, is to take the JavaScript and place it into the correct form
% for the application (\textsf{pdftex} or \textsf{dvipdfm}).
%
% Material within the \texttt{insDLJS} environment is written verbatim to a file.
% The name of this file is \texttt{mydljs.djs}. The file base name comes from the
% first required parameter of the \texttt{insDLJS} environment, see example above.
% The file extension \texttt{.djs} stands for ``define javascript''.
%
% At begin document, the file \texttt{mydljs.djs} is input back into the document after
% special definitions have already been read. These special definitions are the content
% of the file \texttt{dljscc.def}.  The script is placed in the appropriate construct
% for insertion by the application at the document level.
%
% For more information about \texttt{dljscc.def}, see documentation given in the Section~\ref*{dljscc},
% \Nameref{dljscc}.
%
% \subsubsection{For the \texttt{dvipsone} and \texttt{dvips} Options}
%
% When you use the \texttt{dvipsone} or \texttt{dvips}, this means you are going to convert
% your document to postscript and distill it to create a PDF document. There is no \texttt{pdfmark}
% construct for inserting document level JavaScripts; however, the recent release of Acrobat~5.0
% gives postscript users an avenue for inserting DLJS.
%
% The material within the \texttt{insDLJS} environment is written verbatim to the file
% \texttt{mydljs.djs}, as above. It is input back into the document where macros are allowed
% to expand, then written back out to another file. The name of this new file is \texttt{mydljs.fdf}.
% The file base name comes from the first required parameter of the \texttt{insDLJS} environment.
% The file extension \texttt{.fdf} stands for ``forms data format''. This is an extension defined
% and recognized by the Acrobat line of products.
%
% The file \texttt{mydljs.fdf} contains the document level JavaScript in a form that
% the Acrobat application (version~5.0 or later) can import. Part of the work of the package
% is to define an open page action. When the PDF document is opened for the first time in
% Acrobat, the JavaScript is imported.  Usually, you open the document following distillation.
% After the \texttt{.fdf} file(s) have been imported, you need to save the document, usually using
% the ``SaveAs'' file option, this saves the JavaScript code with the file. (There is no need for the
% \texttt{.fdf} file(s) at this point.)
%
% \subsection{Comments on JavaScript}
% A few words on JavaScript.
%
% \subsubsection{What is Document Level JavaScript?}
% The document level is a location in the PDF document where script can be stored.
% When the PDF document is opened, the document level functions are scanned, and any
% ``exposed script'' is executed.
%
% Normally, the type of scripts you would place at the document level are
% general purpose JavaScript functions, functions that are called repeatedly or large special
% purpose functions.  Functions at the document level
% are known throughout the document, so they can be called by links, form buttons, page open
% actions, etc.
%
% Variables declared within a JavaScript function have local scope, they are not known outside
% that function. However, if you can declare variables and initialize them at the document level outside
% of a function, these variables will have document wide scope. Throughout the document, the values of these
% global variables are known. For example
%\begin{Verbatim}[xleftmargin=\parindent,fontsize=\small,
%commandchars={*~@},codes={\def\1{\quad}\catcode`\%=9}]
%\begin{insDLJS}{mydljs}{My Private DLJS}
%var myVar = 17;                //*textsf~ defined outside a function, global scope@
%function HelloWorld(){
%*1var x = 3;                   //*textsf~ defined inside a function, local scope@
%*1app.alert("\tugHello", 3);
%}
%\end{insDLJS}
%\end{Verbatim}
% Both the function \texttt{HelloWorld()} and the variable \texttt{myVar} are known throughout
% the document. The function \texttt{HelloWorld()} can be called by a mouse up button action;
% some form field, executing some JavaScript, may access the value of \texttt{myVar} and/or
% change its value.  The variable \texttt{x} is not known outside of the \texttt{HelloWorld()} function.
%
% \subsubsection{Access and Debugging}
% For those who do not have \textsf{Acrobat}, the application,
% unless you are writing very simple code, writing and debugging
% JavaScript will be very difficult.  From the Acrobat Reader,
% there is not access to the document level JavaScript. You will be
% pretty much writing blind. You can use the debug feature of
% \textsf{insdljs} and insert some debugging code to try to give you
% some insight into what is going wrong. Even so, debugging will be a problem.
%
% Normally, I develop the JavaScript from within Acrobat. The GUI editor does check for
% syntax errors, giving you a chance to correct some simple errors as you go. After I am satisfied
% with my code, I copy it from the editor and paste it into a \texttt{insDLJS} environemnt. This is
% how the JavaScript code of \textsf{exerquiz} was developed.
%
% In my opinion, if you want to develop rather complicated code, having the full Acrobat product
% is a must. (This implies that the Windows or Mac platform is needed!)
%
% \subsubsection{JavaScript References}
% The JavaScript used by Acrobat consists of the core JavaScript plus Acrobat's JavaScript extensions.
% Acrobat~5.0 uses core JavaScript~1.5, see
%\begin{quote}
%   \footnotesize\url{https://developer.mozilla.org/en/JavaScript}
%\end{quote}
% and documentation of the Acrobat extensions can be found in the ``JavaScript for Acrobat API Reference,''
% see
%\begin{quote}
%\href{http://livedocs.adobe.com/acrobat_sdk/9.1/Acrobat9_1_HTMLHelp/wwhelp/wwhimpl/common/html/wwhelp.htm?context=Acrobat9_HTMLHelp&file=JS_API_AcroJSPreface.87.1.html}
%{JavaScript for Acrobat API Reference}
%\end{quote}
%
% \subsection{The \textsf{execJS} Environment}
% This is an environment useful to PDF developers who want to tap into the power of JavaScript.
% To use this environment, the developer needs Acrobat~5.0 or higher, or Acrobat Approval~5.0
% or higher. \textsf{pdftex} or \textsf{dvipdfm} can be used to produce the PDF document, but one of the
% applications (not the Reader) listed above is needed for this environment to do anything.
%
% The \texttt{execJS} is used primarily for post-distillation processing (post-creation processing, in the case of
% \textsf{pdftex} and \textsf{dvipdfm}). The \texttt{execJS} environment can be used, for example, to automatically import
% named icons into the document, which can, in turn, be used for an animation.
%
% The \textsf{execJS} is an environment in which you can write verbatim JavaScript code. This environment
% is a variation on \textsf{insdljs}, it writes a couple of auxiliary files to disk; in particular, the
% environment creates an \texttt{.fdf} file. When the newly produced PDF is loaded for the first time
% into the viewer (Acrobat or Approval, not Reader), the \texttt{.fdf} file generated by the \textsf{execJS}
% environment is imported, and the JavaScript executed. This JavaScript is \emph{not} saved with
% the document.
%
% The environment takes one required argument, the base name of the auxiliary files to be generated.
%
%
% For more details on this environment, see
% \Nameref{execJS}. Also, see the demo file \texttt{execjstst.tex} and \texttt{execjstst.pdf} for
% a short tutorial on the \texttt{execJS} environment, and an animation example, built entirely from
% {\LaTeX} commands.
%
% \paragraph*{Security Note:} Executing arbitrary
%JavaScript through a menu is potentially a security hole. If a
%malicious hacker knows that you have a menu item with the name of
%\texttt{MenuProc}, and that you use the variable
%\texttt{\_MenuProc} in the way described above, the hacker could
%distribute a PDF that executes (security restricted) JavaScript
%through the menu in this way. Therefore, you should select
%\texttt{private names} for \texttt{MenuProc} and \texttt{\_MenuProc}.
%
% \subsection{The \textsf{defineJS} Environment}
%
%When you create a form element (button, text field, etc.), you sometimes want to attach
%JavaScript.  The \textsf{defineJS} environment aids you in writing your Field level JavaScript.
%It too is a verbatim environment, however, this environment does not write to this, but
%saves the contents in a token register. The contents of the register are used in defining
%a macro that expands to the verbatim listing.
%
% The \textsf{defineJS} environment takes two parameters, the first optional.
% the required parameter is the cmd name of the command you want to define.
% You can use the optional first parameter to modify the verbatim environment,
% as illustrated.  The \textsf{defineJS} is a complete verbatim environment: no escape,
% and no comment characters are defined. You can use the optional parameter to create
% an escape character.  You can pretty much use any character you wish, \emph{except}
% the usual one `\verb+\+', backslash.
%
% The following environment, defines a command \cs{myCmd} what expands to the verbatim
% contents of the environment.
%\begin{Verbatim}[xleftmargin=\parindent,fontsize=\small,
%commandchars={!()},codes={\catcode`\%=9}]
%\begin{defineJS}{\myCmd}
%...
%!ameta(JavaScript code)
%...
%\end{defineJS}
%\end{Verbatim}
%
% There is also a \textsf{localJS} environment that makes any definitions within the
% environment local.
%
%The following examples uses some macros from the \textsf{eforms} package:
%\begin{Verbatim}[xleftmargin=\parindent,fontsize=\small,
%commandchars={*~$},codes={\def\1{\quad}\catcode`\%=9}]
%*%*uif~ Define a text macro that will be expanded from within$
%*%*uif~ the *env~defineJS$ environment$
%\def\HelloWorld{Hello World}
%*%*uif~ Now define some JavaScript to execute with a button to be created.$
%*%*uif~ We enclose the *env~defineJS$ and the *cs~pushButton$ in the *env~localJS$ environment$
%*%*uif~ That way, the macro definitions, *cs~JSA$, *cs~JSAAE$, *cs~JSAAX$ will be local to$
%*%*uif~ this group$
%\begin{localJS}
%*%*uif~ Make @ the escape so we can demonstrate the optional parameter.$
%\begin{defineJS}[\makeesc\@]{\JSA}
%var sum = 0;
%for (var i = 0; i < 10; i++)
%{
%*1sum += i;
%*1console.println("@HelloWorld i = " + i );
%}
%console.println("sum = "+sum);
%\end{defineJS}
%\begin{defineJS}{\JSAAE}
%console.println("Enter the button area");
%\end{defineJS}
%\begin{defineJS}{\JSAAX}
%console.println("Exiting the button area");
%\end{defineJS}
%\pushButton[*%
%*1\A{\JS{\JSA}}
%*1\AA{\AAMouseEnter{\JS{\JSAAE}}
%*1*1  \AAMouseExit{\JS{\JSAAX}}
%*1}
%]{myButton}{30bp}{15bp}
%\end{localJS}
%\end{Verbatim}
%
% See the section \Nameref{defineJS} for details of these two environments.
% \section{Package Options and Requirements}
%
% \subsection{Package Options}
%
% The options are \texttt{dvipsone}, \texttt{dvips}, \texttt{pdftex} and
% \texttt{dvipdfm}. The default is \texttt{dvipsone}/\texttt{dvips}.
% \changes{v2.3g}{2017/09/03}{Added ifluatex package}
%    \begin{macrocode}
%<*package>
\usepackage{xkeyval}
\RequirePackage{ifpdf}[2006/02/20]
\RequirePackage{ifxetex}[2006/08/21]
\RequirePackage{ifluatex}
%    \end{macrocode}
%
% Standard driver options.\par\medskip\noindent
% \textbf{Those using Distiller}
%   \IndexOpt{dvips}
%   \IndexOpt{dvipsone}
%   \IndexOpt{textures}
%    \begin{macrocode}
\DeclareOptionX{dvipsone}{\def\dljs@drivernum{0}%
    \AtBeginDocument{\dvips@marker}}
\DeclareOptionX{dvips}{\def\dljs@drivernum{0}%
    \AtBeginDocument{\dvips@marker}}
\DeclareOptionX{textures}{\def\dljs@drivernum{0}%
    \AtBeginDocument{\dvips@marker}}
%    \end{macrocode}
%\textbf{Those not using Distiller}
%    \IndexOpt{pdftex}
%    \IndexOpt{luatex}
%    \IndexOpt{dvipdfm}
%    \IndexOpt{dvipdfmx}
%    \IndexOpt{xetex}
% \changes{v2.3g}{2017/09/03}{Added luatex option}
%    \begin{macrocode}
\DeclareOptionX{pdftex}{\def\dljs@drivernum{1}}
\DeclareOptionX{luatex}{\def\dljs@drivernum{1}}
\DeclareOptionX{dvipdfm}{\def\dljs@drivernum{2}}
\DeclareOptionX{dvipdfmx}{\def\dljs@drivernum{2}}
\DeclareOptionX{xetex}{\def\dljs@drivernum{2}}
%    \end{macrocode}
% Set the default value of the driver: assume the distiller is used.
%    \begin{macrocode}
%\def\dljs@drivernum{0}
%    \end{macrocode}
% Some switch definitions
%    \begin{macrocode}
\let\dl@YES=y \let\dl@NO=n
\let\dl@One=1 \let\dl@Zero=0
\let\dljs@YES=y \let\dljs@NO=n
\edef\dl@restoreCats{%
  \catcode`\noexpand\"=\the\catcode`\"\relax
  \catcode`\noexpand\'=\the\catcode`\'\relax
  \catcode`\noexpand\,=\the\catcode`\,\relax
  \catcode`\noexpand\(=\the\catcode`\(\relax
  \catcode`\noexpand\!=\the\catcode`\!\relax
  \catcode`\noexpand\_=\the\catcode`\_\relax
}
\@makeother\"\@makeother\'\@makeother\,%
\@makeother\(\@makeother\!\@makeother\_
%    \end{macrocode}
%    \leavevmode\DescribeMacro{\dlpkgInfo} A command to identify package date
%    from within document level Javascript
%\changes{v2.4.7}{2019/10/23}{Added \string\cs{dlkgInfo} to identify version}
%    \begin{macrocode}
\def\dlpkgInfo{\@nameuse{ver@\@currname.\@currext}}
\def\dlSetPkgInfo{%
  \edef\dlpkgInfoExpd{\dlpkgInfo}%
  \expandafter\dlparsePkgInfo\dlpkgInfoExpd\@nil}
\def\dlparsePkgInfo#1 #2 #3:#4\@nil{%
  \def\dlPkgInfoDate{#1}%
  \def\dlPkgInfoVer{#2}%
  \def\dlPkgInfoPkg{#3}%
  \def\dlPkgInfoDesc{#4}}
%    \end{macrocode}
% When using pdfmarks, the open action is initiated by the document JavaScript code,
% when there was not document JavaScript, the open action did not get executed.
% This was a bad idea, here is a work around. When the \texttt{insDLJS} environment is used,
% we |\let\dljspresent=y|, and if at the beginning of the document, \cs{dljspresent} is still
% \texttt{n}, we'll issue a special execution of \cs{@OAction}.
%    \begin{macrocode}
\def\dvips@marker{\ifx\dljspresent\dl@NO\@OAction\fi}
%    \end{macrocode}
% This command sequence will mark the passing of the first open event that is defined. The rest of the open events,
% if any, will be in a \texttt{Next} dictionary.
%    \begin{macrocode}
\let\isOpenAction\dl@NO
%    \end{macrocode}
%    \leavevmode\IndexOpt{nodljs}^^A
% Option to cancel the insertion of DLJS, useful for a document meant to be
% printed only, or one that does not use the \texttt{shortquiz} or
% \texttt{quiz} environments, for example.
%    \begin{macrocode}
\DeclareOptionX{nodljs}{\let\importdljs\dl@NO}
\let\importdljs\dl@YES
%    \end{macrocode}
% Need to place certain code only once. This switch will make sure of that.
%    \begin{macrocode}
\let\firstdljs\dl@YES
\let\dljspresent\dl@NO
%    \end{macrocode}
%    \leavevmode\IndexOpt{debug}^^A
% Use this option to help debug DLJS.
%    \begin{macrocode}
\DeclareOptionX{debug}{\let\dljs@debug\dl@YES}
\let\dljs@debug\dl@NO
%    \end{macrocode}
% If the option \IndexOpt{execJS}\opt{execJS} is taken, then the JavaScript code for any \env{execJS} environment
% will be executed following PDF creation (\app{Acrobat} required). The default is no.
% More recently (2019/08/06) \DescribeMacro\execJSOn\cs{execJSOn} and \DescribeMacro\execJSOff\cs{execJSOff}
% were added to turn on (or off) execution of JS in this way. Useful when \pkg{insdljs} is loaded by other
% packages and you cannot access the optional arguments of \cs{usepackage}. These commands are only effective
% in the preamble.
% \changes{v2.4.6}{2019/08/06}{Added \string\cs{execJSOn} and \string\cs{execJSOff}}
%    \begin{macrocode}
\DeclareOptionX{execJS}{\execJSOn}
\def\execJSOn{\let\execjs\dl@YES}
\def\execJSOff{\let\execjs\dl@NO}
\execJSOff
\DeclareOptionX{nopro}{\let\dl@gopro\dl@NO}
\let\dl@gopro\dl@YES
%    \end{macrocode}
%    We set the driver, according to the following algorithm.
%    \begin{macrocode}
\@ifpackageloaded{web}{%
  \ExecuteOptionsX{\eq@driver@name}%
}{%
  \@ifpackageloaded{eforms}{%
    \ExecuteOptionsX{\ef@driver}%
  }{%
    \ifpdf\ExecuteOptionsX{pdftex}\else
    \ifxetex\ExecuteOptionsX{xetex}\else
    \@ifundefined{l@tex@@@@driver}{\ExecuteOptionsX{dvips}}
      {\ExecuteOptionsX{dvipsone}}\fi\fi
  }%
}
%    \end{macrocode}
%    \leavevmode\IndexOpt{reqjs}^^A
%    Beginning with PDF 1.7, a document can specify requirements, here we declare
%    the document uses JavaScripts. The option \texttt{reqjs} is taken by default,
%    that is, JavaScripts are required for this document
%     \medskip\par\noindent
%    \DescribeMacro{!reqjs}^^A
%    The  option \texttt{!reqis} turns off this feature; that is, there is no notification
%    via the \textsf{YMB} (\textsf{Yellow Message Bar}) that Javascripts are used in the document, and the
%    document may not function as designed.
%    \changes{v2.2d}{2016/04/06}{Added \string\texttt{reqjs} and \string\texttt{{\string\EXPT}reqjs} options.}
%    \begin{macrocode}
\DeclareOptionX{reqjs}{\let\dl@reqJS\dl@YES}
\let\dljs@reqJS\dl@YES
\DeclareOptionX{!reqjs}{\let\dl@reqJS\dl@NO}
%    \end{macrocode}
%    The \texttt{usealtadobe} option\IndexOpt{usealtadobe} defined alternate names
%    for some of Adobe's formatting functions. This is useful only if the document
%    author has the Adobe application and know how to use its form edit capabilities.
%    The name of each of the selected function begins with `\texttt{AF}', the name of the new
%    function is the same, but begins with `\texttt{EF}'. For example, the keystroke function
%    \texttt{AFNumber\_Keystroke} is also known under the name of \texttt{EFNumber\_Keystroke}.
%    See \hyperref[s:AltAdbFncs]{Section~\ref*{s:AltAdbFncs}} for the list of functions.
%    \changes{v2.4.6}{2019/08/06}{Added switch \string\cs{usedAdbFuncs} to help detect
%    the usealtadobe option}
%    \begin{macrocode}
\DeclareOptionX{usealtadobe}{\let\usedAdbFuncs\dl@YES
\def\inputAltAdbFncs{\InputIfFileExists{altadbfncs.def}
  {\PackageInfo{insdljs}{Inputting code for usealtadobe option}}
  {\PackageWarning{insdljs}{Cannot find altadbfncs.def.\MessageBreak
    Reinstall or refresh your file name database.}}}}
\let\usedAdbFuncs\dl@NO
\let\inputAltAdbFncs\relax
%    \end{macrocode}
%    \textbf{Process the package options}
%    \begin{macrocode}
\ProcessOptionsX
%    \end{macrocode}
%    This package uses the double quote as a double quote, other packages the make double quote
%    an active character.
%    \begin{macrocode}
\edef\dl@dquoteCat{\the\catcode`\"}
\catcode`\"=12\relax
%    \end{macrocode}
%    Issue a package error if there is no driver specified or implied.
%    \begin{macrocode}
\@ifundefined{dljs@drivernum}{\PackageError{insdljs}%
  {You have not specified dvips, dvipsone, pdftex,\MessageBreak
  dvipdfm, dvipdfmx, or xetex in the option list
  \MessageBreak of the insdljs package}
  {Place one of the drivers dvips, dvipsone, pdftex,  dvipdfm,
  dvipdfmx, or xetex
  \MessageBreak in the option list of the insdljs package.}
}{}
%    \end{macrocode}
% \subsection{Required Packages}
% We need \texttt{hyperref} to provide some fundamental code for the
% various drivers and \texttt{verbatim} to help write verbatim code
% to a file.
% \changes{v2.2h}{2016/06/09}{Added \string\cs{ifpdfmarkup} to easily detect
% use of pdfmarks}
%    \begin{macrocode}
\@ifundefined{ifpdfmarkup}{\newif\ifpdfmarkup}{}\pdfmarkupfalse
\ifpdf\else\ifxetex\else\pdfmarkuptrue\fi\fi
\RequirePackage{hyperref}
%    \end{macrocode}
% LuaTeX v0.85 has made some changes to primitives, the definition below follows
% the pattern of \texttt{hluatex.def} of \textsf{hyperref}.
%    \changes{v2.3}{2016/06/09}{Fixed \string\cs{pdfnames} corresponding to
%    changes in primitives of LuaTeX v0.85}
%    \changes{v2.3a}{2016/07/31}{Added luatex85 definition of \string\cs{pdfpageref}}
%    \begin{macrocode}
\ifpdf\ifx\pdfextension\@undefined\else
  \protected\def\pdfcatalog{\pdfextension catalog } % l3
  \protected\def\pdfnames{\pdfextension names }
  \protected\def\pdfobj{\pdfextension obj }  % l3
  \protected\def\pdflastobj{\numexpr\pdffeedback lastobj\relax}  % l3
  \def\pdfpageref{\pdffeedback pageref}\fi\fi
%    \end{macrocode}
%     The critical JavaScript files are no longer distributed with this package,
%     but are distributed by the \pkg{acrotex-js} package. This package must be
%     installed, and the instructions in the \texttt{docs} folder followed to
%     propery install the critical JavaScript file \texttt{aeb.js}.
%     \changes{v2.5.5}{2021/06/19}{Require acrotex-js when dvips is used}
%    \begin{macrocode}
\ifpdfmarkup
\IfFileExists{acrotex-js.sty}{\let\reqpkg\relax}
%    \end{macrocode}
%    If the style file \texttt{acrotex-js.sty} is detected, we assume the document
%    author has correctly installed \texttt{aeb.js}, if not, we require
%    this package, which may be automatically installed on some {\TeX} systems.
%    \begin{macrocode}
  {\PackageWarningNoLine{insdljs}{%
      The acrotex-js package is required.\MessageBreak
      Before continuing, install this package,\MessageBreak
      read the documentation, and place\MessageBreak
      aeb.js in the expected folder
    }\def\reqpkg{\usepackage{acrotex-js}[2021/06/19]}
  }
\expandafter\reqpkg\fi
\RequirePackage{verbatim}
%    \end{macrocode}
%\changes{v2.4.1}{2019/02/11}{Save \string\cs{comment}\space as \string\cs{dlcomment}}
%    \begin{macrocode}
\let\dlcomment\comment
\let\enddlcomment\endcomment
%    \end{macrocode}
%    (2017/01/03) The \pkg{insdljs} package does not use \pkg{conv-xkv} but
%    is fundamental to other packages (\pkg{eforms}, \pkg{aeb\_pro}) that require
%    \pkg{insdljs} and \pkg{conv-xkv}.
%    \changes{v2.3c}{2017/01/03}{Require \string\pkg{conv-xkv}}
%    \begin{macrocode}
\usepackage{conv-xkv}[2017/01/03]
%    \end{macrocode}
%    Last minute adjustments in the driver.
%    \begin{macrocode}
\@ifundefined{dljs@drivernum}{%
  \@ifundefined{eq@drivernum}{%
    \PackageWarning{insdljs}
      {No driver specified, will check for pdftex.}
    \@ifundefined{pdfoutput}{%
      \@ifundefined{@pdfm@mark}{\def\eq@drivernum{2}%
      \def\dljs@drivernum{1}\let\isOpenAction\dl@NO
      \PackageWarning{insdljs}{No pdftex, but detected dvipdfm}}%
      {\PackageWarning{insdljs}{Neither pdftex nor dvipdfm,
          assuming dvips/dvipsone.}%
      \def\eq@drivernum{0}\def\dljs@drivernum{0}%
      \let\isOpenAction\dl@YES}
    }{%
      \ifcase\pdfoutput
        \def\eq@drivernum{1}\def\dljs@drivernum{1}%
        \let\isOpenAction\dl@NO
        \PackageInfo{insdljs}{Pdftex detected.}%
      \else
        \def\eq@drivernum{0}\def\dljs@drivernum{0}%
        \let\isOpenAction\dl@YES
        \PackageWarning{insdljs}{Pdftex engine detected,
        but is outputting dvi. Assuming dvips as the driver.}%
      \fi
    }%
  }{%
    \begingroup
      \count0=\eq@drivernum
      \xdef\dljs@drivernum{\ifcase\count0 0\or1\or2\or0\else0\fi}
      \ifnum\dljs@drivernum=0\global\let\isOpenAction\dl@YES
        \else\global\let\isOpenAction\dl@NO\fi
    \endgroup
  }%
}{}
%    \end{macrocode}
% I've always had problems with the pdftex option and inserting open page actions.
% The problem is that the open page action, which is defined using the token list
% \cs{pdfpageattr} appears not only on the first page, but every page thereafter.
% To solve this problem, I've used \textsf{everyshi}, a nice and useful package by
% Martin Schr\"oder, to set the open page action with \cs{AtNextShipout}.
% This approach seems to work.
%    \begin{macrocode}
\ifnum\dljs@drivernum=1\relax
  \RequirePackage{everyshi}
\fi
%    \end{macrocode}
% If \texttt{reqjs} is in effect, we execute the following code, otherwise, we
% do not. The \textbf{\textsf{Requirements}} dictionary goes in the document \textbf{\textsf{Catalog}}.
%    \changes{v2.2d}{2016/04/06}{Code to support \string\texttt{reqjs} and \string\texttt{{\string\EXPT}reqjs} options.}
%    \begin{macrocode}
\def\AEB@psMrk{[%]
  \space}
\ifx\dl@reqJS\dl@YES
  \def\dl@@reqJS{/Requirements[<</Type/Requirement/S/EnableJavaScripts>>]}
  \ifpdf\pdfcatalog{\dl@@reqJS}\else
  \ifxetex\immediate\special{pdf:put @catalog <<\dl@@reqJS>>}\else
  \literalps@out{\AEB@psMrk{Catalog} <<\dl@@reqJS>> /PUT\space pdfmark}\fi\fi
\fi
%    \end{macrocode}
%     Added switch to mark whether djs is inserted
%     \changes{v2.4}{2018/07/21}{Added switch}{Added \string\cs{ifthereisdjs}}
%    \begin{macrocode}
\newif\ifthereisdjs \thereisdjsfalse
\def\applydljs#1{\ifthereisdjs\else#1
  \global\thereisdjstrue\fi}
%    \end{macrocode}
%
%    \section{Main Code}
%
% Before we begin, we need to document the problems that need to be addressed by this package.
%
% \subsection{Complications}
%
% There is a complication with writing JavaScript and \TeX{} together; they both use
% the backslash character, `|\|', as the escape character. The following two tables
% list escaped characters that have special meanings in different situations. These two tables
% were taken from the book \textsl{JavaScript, The Definitive Guide}, by David Flanagan, 4th Edition,
% O'Reilly publishers.
%
%\begin{flushleft}
%\begin{minipage}{\linewidth}
% \textbf{Escape Sequences in String Literals}
%
% \begin{tabular}{@{}ll@{}}
% Sequence & Character represented \\
% \cs{0}   & the NULL character (\cs{u0000}) \\
% \cs{b}   & backspace (\cs{u0008}) \\
% \cs{t}   & horizontal tab (\cs{u0009}) \\
% \cs{n}   & newline (\cs{u000A}) \\
% \cs{v}   & vertical tab (\cs{u000B}) \\
% \cs{f}   & form feed (\cs{u000C}) \\
% \cs{r}   & carriage return (\cs{u000D}) \\
% \cs{"}   & double quote (\cs{u0022}) \\
% \cs{'}   & apostrophe or single quote (\cs{u0027}) \\
% |\\|     & backslash (\cs{u005C})\\
% \cs{xXX} & the Latin-1 character specified by the two hexadecimal digits \texttt{XX} \\
% \cs{uXXXX}& the unicode character specified by the four hexadecimal digits \texttt{XXXX} \\
% \cs{XXX} & the Latin-1 character specified by the octal digits \texttt{XXX}, between \\
%          & $1$ and $377$. \\
% \end{tabular}
%\end{minipage}
%\end{flushleft}
%
%\begin{flushleft}
%\begin{minipage}{\linewidth}
% \textbf{Regular Expression Literal characters}
%
% \begin{tabular}{ll}
% Character & Matches \\
% Alphanumeric & Itself \\
% \cs{0}   & the NULL character (\cs{u0000}) \\
% \cs{t}   & horizontal tab (\cs{u0009}) \\
% \cs{n}   & newline (\cs{u000A}) \\
% \cs{v}   & vertical tab (\cs{u000B}) \\
% \cs{f}   & form feed (\cs{u000C}) \\
% \cs{r}   & carriage return (\cs{u000D}) \\
% \cs{xXX} & the Latin-1 character specified by the two hexadecimal\\
%          & digits \texttt{XX} \\
% \cs{uXXXX}& the unicode character specified by the four hexadecimal\\
%           & digits \texttt{XXXX} \\
% \cs{cXX}  & the control character |^X|
% \end{tabular}
%\end{minipage}
%\end{flushleft}
%
% Again, both JavaScript and \TeX, certain punctuation marks have special meaning; in the case of
% JavaScript, punctuation has a special meaning within regular expressions:
%\paragraph*{Special Punctuation in Regular Expressions}
%\begin{verbatim}
%     ^ $ . * + ? = ! : | \ / ( ) [ ] { }
%\end{verbatim}
% Complications continued. The distiller method (the \texttt{dvipsone} or the
% \texttt{dvips} option), and the \textsf{pdftex}/\textsf{dvipdfm} applications
% handle the backslash differently. The following example illustrates the problem.
% {\small
%\begin{verbatim}
%dvips
%LaTeX Source      As appears in PDF
%var x = "\";  --> var x = "\";  (SyntaxError: Unterminated string literal)
%var x = "\\"; --> var x = "\\"; (Correct)
%\end{verbatim}}
% When we type \texttt{"\string\"}, we are beginning a string (with the first
% double quotes, but then we have a literal \texttt{\string\"}, so we have not
% closed the opening string, hence the error message. We have to type
% `\verb+\\+' to get the single backslash.
% {\small
%\begin{verbatim}
%pdftex/xetex
%LaTeX Source        As appears in PDF
%var x = "\";    --> var x = "";  (\" is the literal ")
%var x = "\\";   --> var x = "\"; (SyntaxError: Unterminated string literal)
%var x = "\\\\"; --> var x = "\\"; (Correct)
%\end{verbatim}}
% These applications write raw PDF code to the \texttt{.pdf} file.
% All special characters need to be escaped. You can see, that \textsf{pdftex}
% and \textsf{dvipdfm} require extra backslashs.
%
% The solution to this problem---the problem of how distiller and the two applications
% handle backslashes---is to define control sequences for all the sequences JavaScript uses
% and to adjust their definitions depending on the driver option. As a result, of these
% background definitions, the JavaScript writer does not worry too much about these details,
% for example
%\begin{Verbatim}[xleftmargin=\parindent,fontsize=\small,
%commandchars={*~@},codes={\def\1{\quad}\catcode`\%=9}]
%\begin{insDLJS}{mydljs}{My Private DLJS}
%function HelloWorld()
%{
%*1app.alert("\tugHello", 3);
%*1var x = "\\";
%*1app.alert("x = " + x);
%}
%\end{insDLJS}
%\end{Verbatim}
% So much for the complication.
%
%\subsection{Begin Code}
%
% Some control sequences that are useful for various
% actions. All appear within the \cs{A} key, eg, |\A{\JS{var x=17;}}|. Note the arguments
% of some are delimited by parentheses, others by braces. Sorry for the inconsistency,
% but these evolved over a period of years. Some keys require additional arguments which
% are typically inserted in raw form; eg, |\A{\GoToR(|\ameta{dest}|)/F(example.pdf)}|.
%    \begin{macro}{\JS}\hskip-\marginparsep\texttt{\darg{\ameta{code}}}
%    A convenience command for writing a JavaScript action
%    \begin{macrocode}
\providecommand\JS[1]{/S/JavaScript/JS(#1)}
%    \end{macrocode}
%    \end{macro}
%    \begin{macro}{\Named}\hskip-\marginparsep\texttt{\darg{\ameta{name}}}
%    A named action, eg, |\A{\Named{NextPage}}|
%    \begin{macrocode}
\providecommand{\Named}[1]{/S/Named/N/#1}
%    \end{macrocode}
%    \end{macro}
%    \begin{macro}{\GoTo}\hskip-\marginparsep\texttt{(\ameta{name})}
%    A go-to-action; eg, |\A{\GoTo(|\ameta{dest}|)}|. The argument can actually be a name
%    (enclosed in parentheses), a byte string, or an array.
%    \begin{macrocode}
\providecommand{\GoTo}{/S/GoTo/D}
%    \end{macrocode}
%    \end{macro}
%    \begin{macro}{\GoToD}\hskip-\marginparsep\texttt{(\ameta{name})}
%    A go-to-action destination; eg, |\A{\GoToD(|\ameta{dest}|)}| (same as \cs{GoTo})
%    \begin{macrocode}
\providecommand{\GoToD}{/S/GoTo/D}
%    \end{macrocode}
%    \end{macro}
%    \begin{macro}{\GoToR}\hskip-\marginparsep\texttt{(\ameta{name})/F(\ameta{file-spec})} A go-to-remote action,
%    where \ameta{name} is the destination in the remote file \ameta{file-spec}; eg, |\A{\GoToR(|\ameta{dest}|)/F(example.pdf)}|
%
%    \begin{macrocode}
\providecommand{\GoToR}{/S/GoToR}
%    \end{macrocode}
%    \end{macro}
%    \begin{macro}{\Thread}\hskip-\marginparsep\texttt{\darg{\ameta{thread-title}}}
%   \changes{v2.4.9}{2020/07/09}{Added \string\cs{Thread}}
%    A thread action; this command is used in the \pkg{artthreads} package.
%    \begin{macrocode}
\providecommand{\Thread}[1]{/S/Thread/D(#1)}
%    \end{macrocode}
%    \end{macro}
%    \begin{macro}{\URI}\hskip-\marginparsep\texttt{\darg{\ameta{string}}}
%    A URI action, eg, |\A{\URI{http://www.example.com}}|
%    \begin{macrocode}
\providecommand{\URI}[1]{/S/URI/URI(#1)}
%    \end{macrocode}
%    \end{macro}
%    \begin{macro}{\Launch}\hskip-\marginparsep\texttt{\darg{\ameta{file-spec}}}
% \changes{v2.5}{2020/12/20}{Added \string\cs{Launch}}
%    A launch action
%    \begin{macrocode}
\providecommand{\Launch}[1]{/S/Launch/F(#1)}
%    \end{macrocode}
%    \end{macro}
%    \begin{macrocode}
\def\previewMiKTeX{\def\jsR{\string\r}\def\jsT{\string\t}}
{\obeylines %
\gdef\js@@R{\noexpand
}}
%    \end{macrocode}
% We define space\DescribeMacro\pdfSp, left\DescribeMacro\pdfLBr, and right \DescribeMacro\pdfRBr
% braces in pdf (PDFDocEncoding); these are also defined
% in \pkg{exerquiz} but under different names (\cs{eqSP}, \cs{eqLBr}, and \cs{eqRBr}).
% \changes{v2.4.2}{2019/03/16}{Added \string\cs{pdfSP}, and others, and \string\cs{dl@sp@ce}}
% \changes{v2.4.10}{2020/11/28}{\string\cs{pdfSpacesOn} on by default,
%   only for \string\app{dvips}; otherwise, \string\cs{pdfSpacesOff}}
%    \begin{macrocode}
\def\pdfSPDef{\string\040}
\def\dl@sp@ce{ }
%    \end{macrocode}
%    Beginning with \pkg{insdljs} dated 2019/03/16, we solve an age-old problem with \app{dvips}: This problem
%    is that is wraps code around to the next line, according to some unknown algorithm. Normally this is not
%    a problem, but occasionally it will wrap some JavaScript code that breaks the code. Solve this problem,
%    we make the space character active, and replace it with \cs{pdfSP} (\cs{040}), which is octal for the space
%    character. Now there are no free spaces for \app{dvips} to break the line at in the PS file. The feature
%    is turned on by default, and may be turned off with \DescribeMacro\pdfSpacesOff\cs{pdfSpacesOff}
%    and back on by \DescribeMacro\pdfSpacesOn\cs{pdfSpacesOn}.
%    \begin{macrocode}
\newif\ifpdfspaces\pdfspacesfalse
\def\pdfSpacesOn{\global\pdfspacestrue\let\pdfSP\pdfSPDef}
\def\pdfSpacesOff{\global\pdfspacesfalse\let\pdfSP\dl@sp@ce}
\pdfSpacesOff
\ifpdf\else\ifxetex\else\pdfSpacesOn\fi\fi
\def\pdfLBr{\string\173}
\def\pdfRBr{\string\175}
%    \end{macrocode}
%   \DescribeMacro\dl@restorespcs restores the obey spaces space to its default,
%   saved earlier. This command is placed at the end of each of the form field
%   commands to reverse the earlier change to \cs{pdfSP}.
%    \begin{macrocode}
\bgroup\obeyspaces
\gdef\dl@restorespcs{\global\let =\dl@sp@ce}\egroup
%    \end{macrocode}
% \cs{dljstfor} is used in the \texttt{cc4js} segment.
%    \begin{macrocode}
\let\dljstfor\@tfor
%    \end{macrocode}
% \leavevmode\DescribeMacro{\dfnJSCR}\hskip-\marginparsep\texttt{\darg{\ameta{eol}}}
% A help macro to define \cs{dfnJSR} in a convenient way; \cs{dfnJSR} is
% used in the \env{defineJS} environment.
% \changes{v2.5}{2020/12/20}{Helper commands \string\cs{dfnJSCR} to define
% \string\cs{defineJSjsR} and \string\cs{dfnJSCRDef} to set default.}
%    \begin{macrocode}
\def\dfnJSCR#1{\gdef\dfnJSR{#1}\gdef\defineJSjsR{#1}}
\def\dfnJSCRDef{\dfnJSCR{\string\r\eqbs^^J}}
\dfnJSCRDef
\def\jsR{\string\r\string\n}
%    \end{macrocode}
% \cs{r} is recognized by hyperref when \cs{pdfstringdef} is used. Hyperref
% then modifies the expansion of \cs{r}. So, we'll use \cs{n} in those situations.
% \changes{v2.0l}{2012/11/10}{Added \string\cs{jsN}}
%    \begin{macrocode}
\def\jsN{\string\n}
{\catcode`\^^I\active %
\gdef^^I{\noexpand^^I}%
\gdef\js@@T{\noexpand^^I}%
}%  end of \catcode`\^^I
\let\jsT=\js@@T
%    \end{macrocode}
% Need to write to files. This stream will be used for that purpose.
%    \begin{macrocode}
\newwrite\js@verbatim@out
\def\iwvo#1{\immediate\write\js@verbatim@out{#1}}
%    \end{macrocode}
% \paragraph*{Convenience command\DescribeMacro{\dl@EForAF4} that support \opt{usealtadobe} option.} This helper
% command make it convenient to switch from the `AF' to the `EF' versions.
% \changes{v2.5.3}{2021/05/10}{Added \string\cs{dl@EForAF4}}
% \changes{v2.5.4}{2021/05/13}{Added \string\cs{dl@EForAFMergeChange}}
%    \begin{macrocode}
\def\dl@EForAF4#1_{\ifx\usedAdbFuncs\dl@YES EF\else AF\fi#1_}
\def\dl@EForAFMergeChange{%
  \ifx\usedAdbFuncs\dl@YES EF\else AF\fi MergeChange}
%    \end{macrocode}
% \subsection{Some JavaScript Support}
%
%    \begin{macro}{\flJSStr}
%    \begin{macro}{\dlJSStr}
%
% A simple command for creating JavaScript strings that recognize unicode escape
% sequences \cs{uXXXX}. This command defines a macro with name contained in \texttt{\#1},
% the first argument, the second argument \texttt{\#2} is the JavaScript string. Enclosing
% the string with double quotes is not necessary, this is done automatically. Any expandable
% commands within string are expanded at define time, unless they are protected with \cs{protect}.
% The string can be formatted using \cs{r} and \cs{t}. Also, \verb!\\! expands to a backslash
% should the need arise.
%\changes{v2.0b}{2008/06/19}
%{
%  A simple command for creating JS strings with unicode \string\cs{uXXXX}.
%}
% \changes{v2.0m}{2013/12/23}{Added \string\cs{jscsDflt} and \string\cs{jscsDLJS}, incorporated into \string\cs{dl@defJSStr}.}
% Added \string\cs{jscsDflt} and \string\cs{jscsDLJS}, incorporated them into \string\cs{dl@defJSStr}.
%    \begin{macrocode}
\def\dl@lparen{(}\def\dl@rparen{)}
\def\genericNL{^^J}
\def\defjsLB{\gobiv}
\def\genericLB{\eqbs\eqbs^^J}
\def\pdfmarkLB{\eqbs^^J}
\def\jscsDflt{\let\protect\noexpand\let\jslit\string
  \let\~\dl@tilda\let\^\dl@sup
  \def\\{\eqbs\eqbs}\def\cs##1{\\\\##1}%
  \def\r{\\r}\def\t{\\t}\def\n{\\n}%
  \let\{\dl@lBrace\let\}\dl@rBrace
  \ifx\jsstr@useUni\dl@One\def\Uni{\\u}\else\def\u{\\u}\fi
  \ifx\jsstr@parens\dl@One\let\l@paren\dl@lparen
    \let\r@paren\dl@rparen\else
  \let\l@paren\@empty\let\r@paren\@empty\fi
  \dl@paramlocal\let\LB\genericLB\let\NL\genericNL
}
\def\jscsDLJS{\let\protect\noexpand\let\jslit\string
  \let\~\dl@tilda\let\^\dl@sup
  \ifxetex\def\\{\eqbs\eqbs}\else\def\\{\eqbs}\fi\def\cs##1{\\\\##1}%
  \def\r{\\r}\def\t{\\t}\def\n{\\n}%
  \let\{\dl@lBrace\let\}\dl@rBrace
  \ifx\jsstr@useUni\dl@One\def\Uni{\\u}\else\def\u{\\u}\fi
  \ifx\jsstr@parens\dl@One\let\l@paren\dl@lparen
    \let\r@paren\dl@rparen\else
  \let\l@paren\@empty\let\r@paren\@empty\fi
  \ifpdfmarkup\let\LB\pdfmarkLB\else
    \let\LB\genericLB\fi
  \let\NL\genericNL
}
\ifpdf\let\jscsDLJS\jscsDflt\fi
\let\dl@JSCS\jscsDflt
%    \end{macrocode}
% \DescribeMacro{\flJSStr} This form is designed for writing literal strings
% at the field level.
% When the \texttt{*}-form is used, \texttt{\#2} is passed through \cs{pdfstringdef},
% needed if there are some latex to pdf strings to be converted.
% \changes{v2.4.2}{2019/03/16}{Implement tex to pdf spacing}
%    \begin{macrocode}
\bgroup\obeyspaces
\gdef\flJSStr{\bgroup\global\let =\pdfSP
\let\dl@JSCS\jscsDflt
\obeyspaces\@ifstar{\def\isStar{*}\dl@defJSStr}%
{\let\isStar\relax\dl@defJSStr}}\egroup
\let\fieldJSStr\flJSStr
%    \end{macrocode}
% For backward compatibility, we \cs{let} \cs{defineJSStr} to \cs{flJSStr}.
%    \begin{macrocode}
\let\defineJSStr\flJSStr
%    \end{macrocode}
% \DescribeMacro{\dlJSStr} This form is designed for writing strings to the document
% level JavaScript, as embedded using the \texttt{insDLJS} environment.
% When the \texttt{*}-form is used, \texttt{\#2} is passed through \cs{pdfstringdef},
% needed if there are some latex to pdf strings to be converted.
%    \begin{macrocode}
\newcommand{\dlJSStr}{\bgroup
  \@ifstar{\def\isStar{*}\let\dl@JSCS\jscsDLJS\dl@defJSStr}
  {\let\isStar\relax\let\dl@JSCS\jscsDLJS\dl@defJSStr}}
%    \end{macrocode}
% Some simple key-values for both commands above. Both commands, by default,
% insert double quotations marks at the beginning and end of the string as a convenience.
% \IndexKey{quotes}\texttt{quotes} option just affirms that default; it need not be specified.
% The \IndexKey{noquotes}\texttt{noquotes} option signals \textsf{eforms} not to
% insert said double quotation marks. The other option, \texttt{useUni} is implemented, but not
% documented publicly, to change the command for unicode from \cs{u} to \cs{Uni} in case there
% is a clash of definitions with some accents, \u{a} (\verb!\u{a}!), for example. Hyperref has a
% \cs{u} command defined in the PU encoding, but \cmd{\flJSStr} and \cmd{\dlJSStr} through
% \cmd{\dl@defJSStr} sets \cmd{\Hy@unicodefalse} to support PD1 encoding only. The user can
% access the full range of unicode glyphs through \cs{u}.
%    \begin{macrocode}
\define@key{jsstr}{quotes}[]{\let\jsstr@quotes\dl@One}
\define@key{jsstr}{noquotes}[]{\let\jsstr@quotes\dl@Zero}
\define@key{jsstr}{parens}[]{\let\jsstr@parens\dl@One}
\define@key{jsstr}{noparens}[]{\let\jsstr@parens\dl@Zero}
\let\jsstr@quotes\dl@One\let\jsstr@parens\dl@Zero % noparens the default
\define@key{jsstr}{useUni}[]{\let\jsstr@useUni\dl@One}
\let\jsstr@useUni\dl@Zero
%    \end{macrocode}
% The common part to both \cs{flJSStr} and \cs{dlJSStr}.
% \changes{v2.4.2}{2019/03/16}{Implement tex to pdf spacing. Remove the use of
% parentheses in \string\cs{dl@defJSStr}.}
%    \begin{macrocode}
\newcommand{\dl@defJSStr}[3][quotes]{\def\dl@argi{#1}\ifx\dl@argi\@empty
  \else\ifx\dl@JSCS\jscsDLJS\let\jsstr@parens\dl@Zero\fi
  \setkeys{jsstr}{#1}\fi\dl@JSCS
  \ifx\jsstr@quotes\dl@One
    \xdef#2{\l@paren"#3"\r@paren}\else
    \xdef#2{\l@paren#3\r@paren}\fi
  \if\isStar*\gdef\dl@afterAction{{\Hy@unicodefalse\expandafter
    \pdfstringdef\expandafter#2\expandafter{#2}}}\else
    \global\let\dl@afterAction\relax\fi
  \aftergroup\dl@afterAction\egroup\dl@restorespcs}
%    \end{macrocode}
%    \end{macro}
%    \end{macro}
%    \begin{macro}{\dlTC}\hspace{-\marginparsep}\thinspace
%    \darg{\ameta{code}} A convenience command to insert a \texttt{try/catch} pair around
%    \ameta{code} with no error report.
%    \changes{v2.3e}{2017/02/27}{Added convenience command \string\cs{dlTC}}
%    \begin{macrocode}
\def\dlTC#1{try{#1}catch(e){}}
%    \end{macrocode}
%    \end{macro}
%
% \subsection{A Macro for Debugging JS}
%    \begin{macro}{\db}\hskip-\marginparsep{\,\texttt{...\cs{db}}}
% The \cs{db} macro can be used within the \texttt{insDLJS} to insert
% addition JS commands to help debug the code. Usage:
%\begin{Verbatim}[xleftmargin=\parindent,fontsize=\small,
%commandchars={*~@},codes={\def\1{\quad}\catcode`\%=9}]
%\db console.println("myVal = " + myVal);\db*%
%\end{Verbatim}
% Any material (on one line) that is between the two \verb+\db+ will either be
% written to the DLJS (if \cmd{\dljs@debug} expands to \texttt y), or be removed from
% the final output (if \cmd{\dljs@debug} does not expand to \texttt y).
%
% Note that the comment character (|%|) following the terminating \verb+\db+.
% This comment keeps a carriage return from being invoked; if \cmd{dljs@debug} is false
% then this will not create an empty line in your JavaScript.
%    \begin{macrocode}
\def\js@R{\ifcase\dljs@drivernum^^J\else\jsR\fi}
%    \end{macrocode}
% The three spaces (\cs{space}) is meant to align the debug statement, since |\db| takes
% up three spaces.
%    \begin{macrocode}
\def\db#1\db{\ifx\dljs@debug\dl@YES\space\space\space#1\js@R\fi}
%    \end{macrocode}
%    \end{macro}
% Acrobat (and Reader) expect the JavaScript code will be listed in the JavaScript
% array in sorted form.  \texttt{pdftex} and \texttt{dvipdfm} do not sort this array
% and it would be difficult for \TeX{} to do the sorting; therefore a workaround.
% Each set of DLJS will be numbered in the order in which they are defined. This puts
% them naturally in proper sorted order. The following counter, numbers the JS.
%    \begin{macrocode}
\newcounter{dljs@cnt}
\newcounter{dljssegs}\setcounter{dljssegs}{2}
%    \end{macrocode}
% \subsection{Some Verbatim Write Environments}
% Here is a verbatim write environment, based on an example in the
% \texttt{verbatim} package. One modification: The output stream is
% not opened by this environment, this must be done prior. This is necessary
% to write some other stuff prior to the verbatim write.
%
% This verbatim is used internally to write the \texttt{.djs} files, containing
% the JavaScript.
%    \begin{macrocode}
\newenvironment{js@verbatimwrite}{% writes to current \js@verbatim@out
  \@bsphack
  \let\do\@makeother\dospecials\catcode`\^^M\active
  \def\verbatim@processline{\immediate
    \write\js@verbatim@out{\the\verbatim@line}}\verbatim@start
}{\@esphack}
%    \end{macrocode}
% Unlike the example of verbatimwrite in the \texttt{verbatim} package, we do not
% automatically close the stream when we finish out \texttt{js@verbatimwrite} environment.
% We need to write more to this file, so we have an explicit close.
%    \begin{macrocode}
\def\closejs@verbatim@out{\immediate\closeout\js@verbatim@out}
%    \end{macrocode}
% Same as above, except we expand the line with an \cmd{\edef}.
%
% For the case of \texttt{dvipsone/dvips}, which use the pdfmark operator, we have
% a couple of definitions for escape (esc). It turns out, I find it convenient to have
% two versions, a \cs{eqesc}, and a \cs{eqesci}, the latter one is only used once, the former
% one is used numerous times, see the \texttt{\hyperref[dljscc]{dljscc.def}} file.
%    \begin{macrocode}
\begingroup
\catcode`\@=0 @catcode`@\=12
@gdef@eqbs{\}
@gdef@ccpdfmark{@gdef@eqesc{\}@gdef@eqesci{}}
@endgroup
%    \end{macrocode}
% Here's a definition of left and right braces that will be used in writing JavaScript code.
%    \begin{macrocode}
\begingroup
    \catcode`<=1 \catcode`\>=2 \@makeother\{ \@makeother\}
\gdef\definebraces<\def\{<\eqesc{>\def\}<\eqesc}>>
\endgroup
%    \end{macrocode}
% My own special dos and don'ts. Preserve |\| and \texttt{\%}
%    \begin{macrocode}
\def\eqdospecials{\do\ \do\{\do\}\do\$\do\&%
  \do\#\do\^\do\_\do\~}
%    \end{macrocode}
% The \texttt{jsexpverbatimwrite} environment is used to write a quasi-verbatim
% file. We keep original default definitions of |\| and |%| (they are not included in the
% \cs{eqdospecials} above). The code for \texttt{jsexpverbatimwrite} is based on that found
% in the \texttt{verbatim} package. The trouble is, this package assumes that |\| has been changed
% to catcode 12, other.  Consequently, I had a devil of a time trying to stop the verbatim write
% with |\| as an escape rather than other. I finally decided it was necessary to rewrite some of
% the main macros in \texttt{verbatim} so they would properly stop.  Below are the modifications
% designed to stop when \texttt{*end} encountered rather than \cs{end}. These changes seem to
% work o.k.
%    \begin{macrocode}
\begingroup
\catcode`\~=\active \lccode`\~=`\^^M
\lowercase{\endgroup
  \def\eqverbatim@#1~{\verbatim@@#1*end\@nil}%
  \def\eqverbatim@@#1*end{%
  \verbatim@addtoline{#1}%
  \futurelet\dl@next\eqverbatim@@@}%
  \def\eqverbatim@@@#1\@nil{%
  \ifx\dl@next\@nil
    \verbatim@processline
    \verbatim@startline
    \let\dl@next\eqverbatim@
  \else
    \def\@tempa##1*end\@nil{##1}%
    \@temptokena{*end}%
    \def\dl@next{\expandafter\verbatim@test\@tempa#1\@nil~}%
  \fi\dl@next}%
}%
\def\jsexpverbatimwrite
{% writes to current \js@verbatim@out
  \@bsphack
  \ccpdfmark
  \catcode`\<=12 \catcode`\>=12
  %%
%% This is file `dljscc.def',
%% generated with the docstrip utility.
%%
%% The original source files were:
%%
%% insdljs.dtx  (with options: `copyright,cc4js')
%% 
%%%%%%%%%%%%%%%%%%%%%%%%%%%%%%%%%%%%%%%%%%%%%%%%%%%%%%%%%%
%% insdljs.sty package,                                 %%
%% Copyright (C) 2001-2021  D. P. Story                 %%
%%   dpstory@uakron.edu                                 %%
%%                                                      %%
%% This program can redistributed and/or modified under %%
%% the terms of the LaTeX Project Public License        %%
%% Distributed from CTAN archives in directory          %%
%% macros/latex/base/lppl.txt; either version 1 of the  %%
%% License, or (at your option) any later version.      %%
%%%%%%%%%%%%%%%%%%%%%%%%%%%%%%%%%%%%%%%%%%%%%%%%%%%%%%%%%%
\def\r{\eqesc r\ckivspace}
\def\t{\eqesc t\ckivspace}
\def\n{\eqesc n\ckivspace}
\def\f{\eqesc f\ckivspace}
\def\v{\eqesc v\ckivspace}
\def\b{\eqesc b\ckivspace}
\def\B{\eqesc B\ckivspace}
\def\d{\eqesc d\ckivspace}
\def\D{\eqesc D\ckivspace}
\def\s{\eqesc s\ckivspace}
\def\S{\eqesc S\ckivspace}
\def\w{\eqesc w\ckivspace}
\def\W{\eqesc W\ckivspace}
\begingroup
\catcode`\:=12\relax
\dljstfor\dlhex:={AA}{AB}{AC}{AD}{AE}{AF}%
    {BA}{BB}{BC}{BD}{BE}{BF}{CA}{CB}{CC}{CD}{CE}{CF}%
    {DA}{DB}{DC}{DD}{DE}{DF}{EA}{EB}{EC}{ED}{EE}{EF}%
    {FA}{FB}{FC}{FD}{FE}{FF}\do{%
    \expandafter\xdef\csname x\dlhex\endcsname{\noexpand\eqesc x\dlhex}%
}
\endgroup
\def\u{\eqesc u\ckivspace}
\def\0{\eqesc0}
\def\1{\eqesc1}
\def\2{\eqesc2}
\def\3{\eqesc3}
\definebraces
\def\({\eqesc\eqesci(}
\def\){\eqesc\eqesci)}
\def\.{\eqesc.}
\def\/{\eqesc/}
\def\[{\eqesc[}
\def\]{\eqesc]}
\def\|{\eqesc|}
\def\+{\eqesc+}
\def\*{\eqesc*}
\def\-{\eqesc-}
\def\?{\eqesc?}
\def\${\eqesc$}
\def\^{\eqesc^}
\def\\{\eqesc\eqesc}
\def\'{\eqesc'}
\catcode`\"=12
\def\"{\eqesc"}
\catcode`\<=12
\def\<{\eqesc<}
\catcode`\>=12
\def\>{\eqesc>}
\catcode`\%=12
\def\%{%}
\catcode`\%=14
\catcode`\& = 12
\def\&{\eqesc&}
\endinput
%%
%% End of file `dljscc.def'.
%
% Attempt to redefine some macros from the verbatim package
  \let\verbatim@=\eqverbatim@
  \let\verbatim@@=\eqverbatim@@
  \let\verbatim@@@=\eqverbatim@@@
% end redefine of verbatim package
  \let\do\@makeother\eqdospecials%
  \catcode`\^^M=\active\catcode`\^^I=12%
  \def\verbatim@processline{%
    \edef\expVerb{\the\verbatim@line}%
    \immediate\write\js@verbatim@out{\expVerb}}%
    \verbatim@start%
}
\def\endjsexpverbatimwrite{\immediate\closeout\js@verbatim@out\@esphack}
%    \end{macrocode}
%    \begin{macro}{\insPath}
% Set this macro in the \textbf{preamble} of your document to
% direct (almost) all auxiliary files of insDLJS to the specified
% path. The macro takes one parameter, a path on your local hard
% disk.  Usage:
%\begin{Verbatim}[xleftmargin=\parindent,fontsize=\small,
%commandchars={*~@},codes={\def\1{\quad}\catcode`\%=9}]
%\insPath{c:/temp/}
%\end{Verbatim}
% Be sure to use only forward slashes, and don't forget to finish
% the string with a final forward slash, as illustrated above.
%
% All \texttt{*.fdf} files are written to this folder. The
% \texttt{.djs} files of any \texttt{insDLJS} environments created
% after the \cmd{\insPath} will also be written to the path. Any
% \texttt{.djs} created by packages loaded earlier (such as
% \textsf{exerquiz}) will be written to the current directory.
%
% This macro may be useful for users of Distiller~5.0, and is of marginal
% value to users of \textsf{pdftex/dvipdfm}.
%    \begin{macrocode}
\let\js@Path\@empty
\def\insPath#1{\def\js@Path{#1}}
%    \end{macrocode}
%    \end{macro}
% \DescribeMacro{\makeJSspecials}
% This command is typically executed at the beginning of each field
% or link (see \cs{eq@setButtonProps} and \cs{eq@setWidgetProps}).
% They are used for formatting JavaScript code. The command
% \cs{makespecialJS} is defined in \textsf{insdljs}.
% \changes{v2.1}{2015/06/15}{Moved \string\cs{makeJSspecials} from \string\textsf{eforms}
% to here.}
%    \begin{macrocode}
\begingroup
\catcode`\{=12 \catcode`\}=12 \catcode`\<=1 \catcode`\>=2
\gdef\dl@lBrace<\string\{> \gdef\dl@rBrace<\string\}>
\gdef\dl@tilda<\string~> \gdef\dl@sup<\string^>
\endgroup
\def\makeJSspecials{%
  \def\\{\eqbs\eqbs}%         % define \\ to be \\
  \let\jslit\string
%    \end{macrocode}
% \changes{v2.0n}{2014/05/20}{Remake the definition of \string\cs{\char`\{} and \string\cs{\char`\}}}
% In an attempt to avoid the use of \verb~\jslit\{~ and \verb~\jslit\}~, we remake
% the definition of \verb~\{} and \verb~\}~.
%    \begin{macrocode}
  \let\{\dl@lBrace\let\}\dl@rBrace
%    \end{macrocode}
% \changes{v2.1}{2015/07/12}{\string\cs{let} \string\cs{u} to \string\cs{@empty}.}
% \cs{let} \cs{u} to \cs{@empty}.
%    \begin{macrocode}
  \let\u\@empty
  \def\({\eqbs(}\def\){\eqbs)}%
  \makespecialJS              % make \r=\jsR and \t=\jsT
}
%    \end{macrocode}
% \subsection{Open Document Actions}\label{opendocactions}
% Early in the process of opening a document, the `\texttt{/OpenAction}' event occurs, where
% various action-types may be executed. These actions occur before the document JavaScripts
% are read and before any pages are opened. We give support for this event, originally part
% of the \textsf{aeb\_pro} package.
%\changes{v2.1}{2015/06/15}{Moved definition of \string\cs{addToDocOpen} from
% \textsf{aeb\_pro} to \textsf{insdljs}, added support for \string\texttt{pdftex} and \string\texttt{xetex}
% options.}
% \paragraph*{Page views}
% The view can be set when the page key is used. Possible values are
% \texttt{fitpage}, \texttt{actualsize}, \texttt{fitwidth},
% \texttt{fitvisible}, and \texttt{inheritzoom}. These terms correspond
% to Acrobat's UI. When jumping to a destination, the view is set by the
% destination code.
%    \begin{macrocode}
\def\dl@fitpage{/Fit}
\def\dl@actualsize{/XYZ null null 1.0}  %{/XYZ -32768 -32768 1.0}
\def\dl@fitwidth{/FitH null}            %{/FitH -32768}
\def\dl@fitheight{/FitV null}           %{/FitV -32768}
\def\dl@fitvisible{/FitBH null}         %{/FitBH -32768}
\def\dl@inheritzoom{/XYZ 0 0 0}
\def\dl@mag#1{/XYZ null null #1}
\edef\dl@Page#1{\ifcase\dljs@drivernum
  {Page#1}\or\noexpand\pdfpageref#1\space\space 0 R\or
  @page#1\fi}
\def\dl@preProcDefns{\let\Page\dl@Page\let\mag\dl@mag
  \let\fitpage\dl@fitpage\let\actualsize\dl@actualsize
  \let\fitwidth\dl@fitwidth\let\fitheight\dl@fitheight
  \let\fitvisible\dl@fitvisible\let\inheritzoom\dl@inheritzoom
}
\let\aeb@additionalOpenAction\@empty
\let\aeb@addActionCode\@empty
\let\addActionObj\@empty
\let\aebpopentoks\@empty
\let\aebpFA\dl@NO
\def\aebp@rightDelimiters{}
%    \end{macrocode}
% The argument of \DescribeMacro\addToDocOpen\cmd{\addToDocOpen} is an action-type, usually
% \textbf{JavaScript} or a \textbf{GoTo}. The command is accumulative, it can be invoked several types
% with different action-types. Must be executed in the preamble or it has no effect.
% It populates the value of the key \textbf{/OpenAction} in the \textbf{Catalog} of the
% PDF file.
%\begin{Verbatim}[xleftmargin=\parindent,fontsize=\small,
%commandchars={*~@},codes={\def\1{\quad}\catcode`\%=9}]
%\addToDocOpen{\JS{console.println("Open the doc!");}}
%\addToDocOpen{/S/GoTo/D[\Page{2}\fitpage]}
%\end{Verbatim}
%    \begin{macrocode}
\newcommand{\addToDocOpen}[1]{%
  \ifx\aebpFA\dl@NO
    \g@addto@macro\aebpopentoks{#1}%
    \def\aeb@additionalOpenAction{/Next {pubAddToDocOpen}}%
\ifpdf
    \def\aeb@addActionCode{\immediate\pdfobj{<<
    \aebpopentoks\aebp@rightDelimiters >>}%
    \edef\addActionObj{\the\pdflastobj\space 0 R}}%
\else\ifxetex
    \def\aeb@addActionCode{\immediate
      \special{pdf:obj @objAddActionObj
        << \aebpopentoks\aebp@rightDelimiters >> }}%
      \def\addActionObj{@objAddActionObj}%
\else % postscript
    \def\aeb@addActionCode{\literalps@out{%
    \AEB@psMrk/_objdef {pubAddToDocOpen}
      /type/dict /OBJ pdfmark^^J%
    \AEB@psMrk{pubAddToDocOpen} <<
    \aebpopentoks\aebp@rightDelimiters >> /PUT pdfmark}}%
    \def\addActionObj{{pubAddToDocOpen}}%
\fi\fi
    \global\let\aebpFA\dl@YES
    \let\@pdfstartview\@empty
  \else
      \edef\aebp@dljstmp{\aebp@rightDelimiters}%
      \xdef\aebp@rightDelimiters{\aebp@dljstmp >> }%
      \g@addto@macro\aebpopentoks{/Next << #1 }%
  \fi
}
\def\dljs@OpenActions{/OpenAction \addActionObj}
\def\dljs@setOpenActions{\bgroup\makeJSspecials\dl@preProcDefns
  \aeb@addActionCode
  \ifpdf\pdfcatalog{\dljs@OpenActions}\else
  \ifxetex\immediate
      \special{pdf:put @catalog <<\dljs@OpenActions>>}\else
  \literalps@out{\AEB@psMrk
   {Catalog} <<\dljs@OpenActions>> /PUT pdfmark}\fi\fi\egroup}
%    \end{macrocode}
% If \textsf{aeb\_pro} is not loaded, we set the open actions. If \textsf{aeb\_pro}
% is loaded, we do nothing for \textsf{aeb\_pro} has its own contributions to the
% PDF \textbf{Catalog}.
%    \changes{v2.3d}{2017/01/08}{Use \string\cs{aebpFA} to detect open actions}
%    \begin{macrocode}
\@ifpackageloaded{aeb_pro}{}
  {\AtBeginDocument{\ifx\aebpFA\dl@YES\expandafter
    \dljs@setOpenActions\fi}}
%    \end{macrocode}
%
% \subsection{Open Page Actions}\label{openaction}
%
% In order to get automatic insertion of DLJS using Distiller~5.0 or greater, it is necessary
% to use a ``Page Open Action'' for page 1 of the document. This section contains some commands
% for defining an open page action. Originally, these macros were defined only for Distiller
% users (\textsf{dvips} and \textsf{dvipsone} users), later, I extended their use to
% \textsf{pdftex} and \textsf{dvipdfm}.
%
%\subsubsection{\texorpdfstring{\cs{OpenAction}}{\textbackslash{OpenAction}}}
%
% Here are the macros that actually place the open page action into the PDF document. One for
% the \textsf{Distiller}, one for \textsf{pdftex}, and one for \textsf{dvipdfm}.
%    \begin{macrocode}
\let\@CloseAction\@empty
\def\@OAction@pdfmark{\literalps@out{%
   \AEB@psMrk{ThisPage} << /AA << /O << \theFirstAction\space
  \opentoks\@rightDelimiters >> \@CloseAction >> >> /PUT pdfmark}}
\def\@POAction@pdfmark{\literalps@out{%
  \AEB@psMrk{ThisPage} << /AA << \@thisOpenAction
  \@thisCloseAction >> >> /PUT pdfmark}%
}
\def\@OAction@pdftex{%
  \ifx\isOpenAction\dl@YES
    \xdef\pdftexOAction{/AA << /O << \theFirstAction\space
    \opentoks\@rightDelimiters >> \@CloseAction  >>}%
    \AtNextShipout{\pdfpageattr=\expandafter{\pdftexOAction}}\fi
}
\def\@POAction@pdftex{%
  \xdef\thisPageActionpdftex{/AA << \@thisOpenAction
    \@thisCloseAction >>}%
  \AtNextShipout{\pdfpageattr=\expandafter{\thisPageActionpdftex}}%
}
%    \end{macrocode}
% For \textsf{dvipdfm}
%    \begin{macrocode}
\def\@OAction@dvipdfm{\ifx\isOpenAction\dl@YES
  \special{pdf:put @thispage << /AA << /O << \theFirstAction\space
  \opentoks\@rightDelimiters >> \@CloseAction >> >>}\fi
}
\def\@POAction@dvipdfm{%
  \special{pdf:put @thispage << /AA << \@thisOpenAction
  \@thisCloseAction >> >>}%
}
%    \end{macrocode}
% Now we choose the one appropriate to the user's driver option.
%    \begin{macrocode}
\ifcase\dljs@drivernum
  \let\@OAction\@OAction@pdfmark
  \let\@POAction\@POAction@pdfmark
  \gdef\theFirstAction{/S/JavaScript/JS (\the\importfdftoks)}
  \let\isOpenAction\dl@YES
\or
  \let\@OAction\@OAction@pdftex
  \let\@POAction\@POAction@pdftex
  \AtBeginDocument{\@OAction@pdftex}
\or
  \let\@OAction\@OAction@dvipdfm
  \let\@POAction\@POAction@dvipdfm
  \AtBeginDocument{\@OAction@dvipdfm}
\fi
%    \end{macrocode}
%    \begin{macro}{\OpenAction}
% In order for the DLJS to be inserted, \texttt{insdljs} places an open page action
% on the first page.  If the document author wants his/her own open page action, inserting
% it with the pdfmark operator may lead to unpredictable results.  For the uses of the
% Distiller (dvipsone and dvips users of the distiller), if additional open page action
% is needed on the first page, you can insert this action using the \cmd{\nextOAction} command.
%
%\paragraph*{Usage}
%\begin{Verbatim}[xleftmargin=\parindent,fontsize=\small,
%commandchars={*~@},codes={\def\1{\quad}\catcode`\%=9}]
%\OpenAction{\JS{app.beep(-1); app.alert("Welcome to my Page!",3);}}
%\end{Verbatim}
% The macro takes two parameters, the \cmd{\Next} followed by the desired action, in this
% case it is a JavaScript action.
%
% From the code below, you can see that \cmd{\nextOAction} ends calling itself. If the next token
% is a \cmd{\Next}, additional code is entered into the open page action. Thus, for example, the
% following code is effectively the same as the above example, but the individual lines are
% listed separately in the GUI action dialog of Acrobat.
%\begin{Verbatim}[xleftmargin=\parindent,fontsize=\small,
%commandchars={*~@},codes={\def\1{\quad}\catcode`\%=9}]
%\OpenAction{\JS{app.beep(-1);}}
%\Next{\JS{app.alert("Welcome to my Page!",3);}}
%\end{Verbatim}
% In the above example, several open page actions are  ``chained'' together.
%
% You are not restricted to JavaScript actions, for example, you can perform a \texttt{Named}
% action:
%\begin{Verbatim}[xleftmargin=\parindent,fontsize=\small,
%commandchars={*~@},codes={\def\1{\quad}\catcode`\%=9}]
%\OpenAction{/S/Named/N/Open}
%\end{Verbatim}
% There is a |\Named| command that expands to |/S /Named /N#1|, so the above
% example can be written
%\begin{verbatim}
%   \OpenAction{\Named{Open}}
%\end{verbatim}
%\cs{opentoks} accumulates the ``action''.
%    \begin{macrocode}
\def\opentoks{}
%    \end{macrocode}
% The open action is really not meant for anything too fancy, such as searching using regular
% expressions. Of the special characters, we include only \cs{r} and \cs{t} for formatting
% purposes. If you want to use a regular expression in the open page action (not likely),
% put it at the document level as a JavaScript function, and call that function from the
% open page action.
%    \begin{macrocode}
\def\makespecialJS{%
  \let\r\jsR\let\t\jsT\let\n\jsN}
\def\@rightDelimiters{}
%    \end{macrocode}
% If the next token is \cs{Next} then we gobble it up, and begin \cs{@OpenAction}
% otherwise we begin \cs{@OpenAction}. This allows the ``chaining'' as illustrated above.
%    \begin{macrocode}
\def\OpenAction{\@ifnextchar\Next
  {\expandafter\@OpenAction\@gobble}{\@OpenAction}}
%    \end{macrocode}
% \cs{@OpenAction}: This is the macro that actually does the work. We differentiate between whether the
% current argument \texttt{\#1} is the first of all open actions, or not. If the former,
% we define it as \cs{theFirstAction}, and reset \cs{isOpenAction}, which keeps track of
% whether there has been a \textit{prior} open action defined. If the latter, we place it in
% the \cs{opentoks} token register.
%    \begin{macrocode}
\def\@OpenAction#1{%
  \ifx\isOpenAction\dl@NO
    {\makespecialJS\xdef\theFirstAction{#1}}%
    \global\let\isOpenAction\dl@YES
  \else
    \edef\dljstmp{\@rightDelimiters}%
    \xdef\@rightDelimiters{\dljstmp >> }%
    {\makespecialJS\xdef\dljstmp{\opentoks /Next << #1 }}%
    \xdef\opentoks{\dljstmp}%
  \fi
  \@nextOpenAction
}
%    \end{macrocode}
% As its last act, \cs{@OpenAction} calls \cs{@nextOpenAction}, which is nearly identical to
% \cs{OpenAction}, except if there is no \cs{Next} token, the macro terminates.
%    \begin{macrocode}
\def\@nextOpenAction{\@ifnextchar\Next{\expandafter
  \@OpenAction\@gobble}{}}
%    \end{macrocode}
%    \end{macro}
% Here is a last example that uses the \texttt{insDLJS} environment and the \cs{OpenAction} command. This code
% will cause the following action: Ten seconds after the document is opened, an annoying
% message is placed on the screen. The annoying message appears only \texttt{once}, otherwise,
% it would really be annoying. Reader (Acrobat)~5.0 is required for this code.
%\begin{verbatim}
%\begin{insDLJS}[DoIt]{doit}{DoIt}
%var timeout;
%var didIt = false;
%var myWelcome="Welcome to your PDF Page!\r\r You can remove this annoying"
%     +" message by sending \u00A3100 to my Swiss bank account!"
%function startToDoIt()
%{
%    if(!didIt) timeout = app.setInterval("DoIt();", 10000);
%}
%function DoIt()
%{
%    app.beep(-1);
%    app.alert(myWelcome,1);
%    app.clearInterval(timeout);
%    didIt = true;
%}
%\end{insDLJS}
%\OpenAction{/S /JavaScript /JS(startToDoIt();)}
%\end{verbatim}
% See the ``Acrobat JavaScript Object Specification'' for the definitions of some
% of these JavaScript methods.
%
%\subsubsection{\texorpdfstring{\cs{thisPageAction}}{\textbackslash{thisPageAction}}}
% \DescribeMacro{\thisPageAction} places page actions on the current page.
% The first argument is on open page action, the second one is a closed page action.
%    \begin{macrocode}
\def\protectJSCtrls{\def\r{\string\r\string\n}%
  \def\t{\string\t}\def\n{\string\n}}
\def\dljs@tstlp{\@ifnextchar\bgroup
  {\let\dljs@multi\dl@YES\dljs@tstlpi}%
  {\let\dljs@multi\dl@NO\dljs@tstlpi}}
\def\dljs@tstlpi#1\@nil{}%
% #1=arg, #2=O|C, #3=cmd
\def\dljs@bldScript#1#2#3{\bgroup
  \dljs@tstlp#1\@nil
  \ifx\dljs@multi\dl@NO{\protectJSCtrls\xdef#3{/#2<<#1>>}}\else
    \toks0={}\let\dljs@rAngles\@empty\count0=0
    \@tfor\dljs@arg:=#1\do{%
      \ifnum\count0=0
        \toks0=\expandafter{\dljs@arg}\else
        \toks2=\expandafter{\dljs@arg}%
        \edef\holdtokstmp{\the\toks0
          /Next<<\the\toks2 }%
        \edef\dljs@rAngles{\dljs@rAngles\space>>}%
        \toks0=\expandafter{\holdtokstmp}%
      \fi
      \advance\count0by\@ne
    }{\protectJSCtrls\xdef#3{/#2<<\the\toks0 \dljs@rAngles>>}%
     \xdef#3{#3}}%
  \fi
\egroup}
\newcommand{\thisPageAction}[2]{%
  \let\@CloseAction\@empty
  \let\@thisOpenAction\@empty
  \let\@thisCloseAction\@empty
  \def\dljs@argi{#1}\def\dljs@argii{#2}%
%    \end{macrocode}
% On the first page, \cs{OpenAction} is used to process the open action
%    \begin{macrocode}
  \ifnum\value{page}=1\relax
    \ifx\dljs@argii\@empty\else
      \dljs@bldScript{#2}{C}{\@CloseAction}\fi
    \ifx\dljs@argi\@empty\else\dljs@tstlp#1\@nil
      \ifx\dljs@multi\dl@NO\OpenAction{#1}\else
        \@tfor\dljs@arg:=#1\do{\expandafter
          \OpenAction\expandafter{\dljs@arg}%
        }\fi\fi
  \else
%    \end{macrocode}
% If this is not the first page, we just populate the \texttt{/O} dictionary
%    \begin{macrocode}
    \ifx\dljs@argi\@empty\else
      \dljs@bldScript{#1}{O}{\@thisOpenAction}\fi
    \ifx\dljs@argii\@empty\else
      \dljs@bldScript{#2}{C}{\@thisCloseAction}\fi
    \@POAction
  \fi
}
%    \end{macrocode}
% \subsection{The \texttt{insDLJS} Environments}
%
%    \begin{macro}{insDLJS}\hypertarget{insDLJS}{}
%    \begin{macro}{insDLJS*}
% This is the main environment defined by this package. These environments
% first set some global variables, then set the program flow to driver-dependent
% environments. There are two forms of this environment, the \texttt{insDLJS} and the
% \texttt{insDLJS*}.
%
% The \texttt{insDLJS} is the simplest of the two environments. Any material
% within the environment, eventually ends up in the DLJS section of the
% PDF document
%
% The environment takes the \texttt{\ameta{base\_name}} and writes the
% file \texttt{\ameta{base\_name}.djs}. This file contains a verbatim
% listing of the JS within the environment, plus some changing of
% catcodes. This file is then input back into the document at
% \cmd{\AtBeginDocument} with the necessary code for
% \textsf{pdftex} and \textsf{dvipdfm} properly place the JS.
%
% The case of \textsf{dvipsone} and \textsf{dvips} is a little different. A
% \texttt{\ameta{base\_name}.djs} is written and input back, and a second file
% \texttt{\ameta{base\_name}.fdf} is written. This second file is later input
% into the PDF document after distillation.
%
% The syntax of usage for this environment, which takes three
% arguments, is given next.\goodbreak
%\begin{Verbatim}[xleftmargin=\parindent,fontsize=\small,
%commandchars={!()},codes={\def\1{\quad}\catcode`\%=9}]
%\begin{insDLJS}[!ameta(function)]{!ameta(base_name)}{!ameta(script_name)}
%!1!ameta(JavaScript functions or exposed code)
%\end{insDLJS}
%\end{Verbatim}
% where,
% \begin{description}
% \item[\texttt{\#1:}] This optional legacy parameter, its use is discouraged.
% Its value must be the name of
% one of the functions defined in the environment. This is used to
% detect whether the DLJS has already be loaded by Acrobat. This parameter
% is now automatically created.
% \item[\texttt{\#2}:] This parameter, \ameta{base\_name}, is  an
% alphabetic word with no spaces and limited to eight characters. It
% is used to build the names of auxiliary files and to build the
% names of macros used by the environment.
% \item[\texttt{\#3}:] The \ameta{script\_name} of your JavaScript.
% This title will appear in the document-level JavaScript dialog of
% Acrobat.
% \end{description}
% Within the insDLJS environment, there are two types of comment characters:
% (1) a \TeX{} comment (|%|) and (2) a JavaScript comment.   The JavaScript
% comments are `\texttt{//}', a line comment, and `\texttt{/*...*/}' for more
% extensive commenting. These comments will survive and be placed into the
% PDF file. In JavaScript the `|%|' is used as well, use |\%| when you want to use
% the percent character in a JavaScript statement, for example
% |app.alert("\%.2f", 3.14159);|, this statement will appear within your JavaScript
% code as  |app.alert("%.2f", 3.14159);|.
% \changes{v2.2c}{2015/11/13}{Turned \string\cs{Hy@unicodefalse} so unicode is not written
% in the name of the script.}
%    \begin{macrocode}
\let\insdljsloadVar\@empty
\newenvironment{insDLJS}[3][]
{%
  \gdef\detectdljs{#1}\gdef\dljsBase{#2}%
  \ifxetex{\def\1{\string\1}\def\2{\string2}%
  \def\3{\string\3}\xdef\dljsName{#3}}\else
  \Hy@unicodefalse\pdfstringdef{\dljsName}{#3}\fi
%    \end{macrocode}
% If the optional argument is not passed,
% we generate one for the user.
%\changes{v2.0k}{2012/06/21}{If the optional argument is not passed,
% we generate one for the user.}
%    \begin{macrocode}
\ifx\detectdljs\@empty
  \stepcounter{dljs@cnt}%
  \xdef\detectdljs{_bInsdljsload\arabic{dljs@cnt}}%
  \edef\insdljsloadVar{var \detectdljs=true;}%
  \addtocounter{dljs@cnt}{-1}%
\fi
  \global\let\multisegments\dl@NO\setcounter{dljssegs}{2}%
  \global\dljsobjtoks={}\expandafter
  \ifx\csname dljs\dljsBase\endcsname\relax
    \else\@insjserrDuplicate\fi
  \ifcase\dljs@drivernum
    \let\insert@DLJS\insert@DLJS@pdfmark
    \let\endinsDLJS\endinsert@DLJS@pdfmark
    \let\newsegment\newsegment@pdfmark
    \let\endnewsegment\endnewsegment@pdfmark
  \or
    \let\insert@DLJS\insert@DLJS@pdftex
    \let\endinsDLJS\endinsert@DLJS@pdftex
    \let\newsegment\newsegment@pdftex
    \let\endnewsegment\endnewsegment@pdftex
  \or
    \let\insert@DLJS\insert@DLJS@dvipdfm
    \let\endinsDLJS\endinsert@DLJS@dvipdfm
    \let\newsegment\newsegment@dvipdfm
    \let\endnewsegment\endnewsegment@dvipdfm
  \fi
  \insert@DLJS
}{}
%    \end{macrocode}
%
% The \texttt{insDLJS*} environment can be used to better organize,
% edit and debug your JavaScript.  If you have the full Acrobat
% product, you can open the DLJS edit dialog. There you will see a
% listing of all DLJS contained in the document. When you double
% click on one of the \textsl{script names}, you enter the edit
% window, where you can edit all JavaScript contained under that
% name. Each \texttt{insDLJS} environment creates a new listing
% within this   DLJS dialog; and, each environment creates a
% `\texttt{.djs}' file and possibly  a  \texttt{.fdf}' file. Each
% \texttt{insDLJS*} environment also creates a `\texttt{.djs}' file
% and possibly a `\texttt{.fdf}' file too, but within the
% \texttt{insDLJS*} environment you can create JavaScript under
% different \textsl{script names}.
%
% The syntax is
%\begin{Verbatim}[xleftmargin=\parindent,fontsize=\small,
%commandchars={!()},codes={\def\1{\quad}\catcode`\%=9}]
%\begin{insDLJS*}[!ameta(function)]{!ameta(base_name)}
%\begin{newsegment}{!ameta(script_name)}
%!1!ameta(JavaScript functions or exposed code)
%\end{newsegment}
%\begin{newsegment}{!ameta(script_name)}
%!1!ameta(JavaScript functions or exposed code)
%\end{newsegment}
%...
%...
%\begin{newsegment}{<script_name>}
%!1!ameta(JavaScript functions or exposed code)
%\end{newsegment}
%\end{insDLJS*}
%\end{Verbatim}
% where,
% \begin{description}
% \item[\texttt{\#1:}] This optional parameter, \texttt{<function>},
% is \emph{required} for the \texttt{dvipsone} and \texttt{dvips}
% options; otherwise it is ignored. Its value must be the name of
% one of the functions defined in the environment. This is used to
% detect whether the DLJS has already be loaded by Acrobat.
% \item[\texttt{\#2}:] This parameter, \texttt{<base\_name>}, is  an
% alphabetic word with no spaces and limited to eight characters. It
% is used to build the names of auxiliary files and to build the
% names of macros used by the environment.
% \item[\texttt{\#3}:] The \texttt{<script\_name>} of your JavaScript.
% This title will appear in the Document level JavaScript dialog of
% Acrobat.
% \end{description}
%    \begin{macrocode}
\newenvironment{insDLJS*}[2][]
{%
  \gdef\detectdljs{#1}\gdef\dljsBase{#2}%
%    \end{macrocode}
% If the optional argument is not passed,
% we generate one for the user.
%\changes{v2.0k}{2012/06/21}{If the optional argument is not passed,
% we generate one for the user.}
%    \begin{macrocode}
\ifx\detectdljs\@empty
  \stepcounter{dljs@cnt}%
  \xdef\detectdljs{_bInsdljsload\arabic{dljs@cnt}}%
  \edef\insdljsloadVar{var \detectdljs=true;}%
  \addtocounter{dljs@cnt}{-1}%
\fi
  \global\let\multisegments\dl@YES\setcounter{dljssegs}{2}%
  \global\dljsobjtoks={}%
  \expandafter\ifx\csname dljs\dljsBase\endcsname\relax
    \else\@insjserrDuplicate\fi
  \ifcase\dljs@drivernum
    \let\insert@DLJS\insert@DLJS@pdfmark\expandafter
    \let\csname endinsDLJS*\endcsname\endinsert@DLJS@pdfmark
    \let\newsegment\newsegment@pdfmark
    \let\endnewsegment\endnewsegment@pdfmark
  \or
    \let\insert@DLJS\insert@DLJS@pdftex\expandafter
    \let\csname endinsDLJS*\endcsname\endinsert@DLJS@pdftex
    \let\newsegment\newsegment@pdftex
    \let\endnewsegment\endnewsegment@pdftex
  \or
    \let\insert@DLJS\insert@DLJS@dvipdfm\expandafter
    \let\csname endinsDLJS*\endcsname\endinsert@DLJS@dvipdfm
    \let\newsegment\newsegment@dvipdfm
    \let\endnewsegment\endnewsegment@dvipdfm
  \fi
  \insert@DLJS
}{}
\def\@insjserrDuplicate{%
  \typeout{^^J! insdljs Package error.}
  \typeout{! insDLJS environment: On line number \the\inputlineno,}
  \typeout{! the base name `\dljsBase' has already been chosen.}
  \typeout{! A DLJS earlier defined has been overwritten!}
  \typeout{! Choose another name for the first required argument}
  \typeout{! of the insDLJS environment.^^J}
}
%    \end{macrocode}
%    \end{macro}
%    \end{macro}
% \subsection{The \texttt{execJS} Environment}\label{execJS}
% This environment works only for those document authors using Acrobat 5.0 or
% Acrobat Approval. The \texttt{execJS} environment writes a verbatim \texttt{.djs} and
% \texttt{.fdf} files using the same scheme as \texttt{insDLJS} environment.
% The \texttt{.fdf} file is imported into the newly created PDF document and the JavaScript contained
% within the environments are executed.
%
% In order for the JavaScript to be imported and executed, the \texttt{execJS} must be used. The
% JavaScript is executed once, when the document is first opened in Acrobat.
%
% This feature is potentially useful for developmental purposes.
%    \begin{macrocode}
\newif\ifisdljs \isdljsfalse
\begingroup
\catcode`\~=12 \catcode`\#=12
\gdef\efdlspecials{\catcode`\~=12 \catcode`\#=12}
\endgroup
\def\ckivspace#1{\if\noexpand#1\space\else\expandafter#1\fi}
%    \end{macrocode}
% \changes{v2.2b}{2015/08/03}{Added a hook \string\cs{DLspecialDefs} into the DJS file}
% (2015/08/03) Added the \string\cs{DLspecialDefs} hook into the DJS file, used in
% \textsf{aeb\_pro}.
%    \begin{macrocode}
\let\DLspecialDefs\@empty
\def\fdfAfterheader#1{%
  \iwvo{\string\begingroup\string\efdlspecials}
  \iwvo{\string\DLspecialDefs\string#1}
  \iwvo{\string\isdljstrue}
  \iwvo{\string\makeatletter}
  \iwvo{\string\immediate\string\openout\string
\js@verbatim@out=\string\js@Path\space\dljsBase.fdf}%
  \iwvo{\string\begin{jsexpverbatimwrite}}
  \iwvo{\string\firstFDFline}
  \iwvo{1 0 obj}
  \iwvo{<< /FDF << /JavaScript << /Doc 2 0 R /After 3 0 R >> >> >> }
  \iwvo{endobj}
  \iwvo{2 0 obj}
%    \end{macrocode}
%    When a non-Distiller workflow, ensure \env{execJS}s are listed first.
%    \changes{v2.2h}{2016/06/09}{When a non-Distiller workflow, ensure \string\env{execJS}s are listed first}
%    \begin{macrocode}
  \let\x\@empty
  \applydljs{\def\x{^^Jvar _thereisdjs=true;}}%
  \ifpdfmarkup
    \iwvo{[ (ExecJS \dljsBase) (var _\dljsBase\space = true;\x) ] }\else
    \iwvo{[ (00 ExecJS \dljsBase) (var _\dljsBase\space = true;\x)] }\fi
  \iwvo{endobj}
  \iwvo{3 0 obj}
  \iwvo{<<>>}
  \iwvo{stream}
}
%    \end{macrocode}
%    \begin{environment}{execJS}
%    \hskip-\marginparsep\texttt{[\ameta{\cs{cmd}}]\darg{\ameta{base-name}}}
% The parameter \texttt{\#2} (\ameta{base-name}) is the base name for this environment.
% The base name will be used to create a filename to save the \texttt{.fdf} file under.
%    \begin{macrocode}
%    \end{macrocode}
%    Added \DescribeMacro\restoreDLspecialsDefs\cs{restoreDLspecialsDefs} to restore
%    \DescribeMacro\DLspecialDefs\cs{DLspecialDefs} to its default of \cs{@empty}.
%    \changes{v2.4.4}{2019/07/14}{Added \string\cs{restoreDLspecialsDefs}}
%    The \env{execJS} environment now has an optional argument, the optional argument must be
%    a command (\ameta{\cs{cmd}}) that is then passed on to \cs{fdfAfterheader}, which then writes it to
%    \cs{dljsBase.djs}. The purpose is to make special definitions at the time the script
%    is expanded.
%    \changes{v2.4.5}{2019/07/15}{\string\env{execJS} now has an optional argument}
%    \begin{macrocode}
\def\restoreDLspecialDefs{\let\DLspecialDefs\@empty}
\newenvironment{execJS}[2][\relax]
{%
  \def\x{#1}\ifx\x\@empty\def\x{\relax}\fi
  \gdef\detectdljs{\string_#2}\gdef\dljsBase{#2}%
  \global\dljsobjtoks={}%
  \expandafter\gdef\csname dljs\dljsBase\endcsname{}%
  \ifx\importdljs\dl@YES\ifx\execjs\dl@YES
    \ifx\dl@gopro\dl@YES
    \ifnum\dljs@drivernum=0\relax
      \addImportAnFDF\importAnFDFTemplate
    \else
      \OpenAction{/S/JavaScript/JS (\importAnFDFTemplate)}%
    \fi\fi
  \fi\fi
  \immediate\openout \js@verbatim@out \js@Path\dljsBase.djs
  \expandafter\fdfAfterheader\expandafter{\x}%
  \js@verbatimwrite
}{%
  \fdfendstreamobj
  \endjs@verbatimwrite
  \fdftrailer
%    \end{macrocode}
%    Added \cs{restoreDLspecialsDefs} to restore \cs{DLspecialDefs} to its default of \cs{@empty}.
%    \changes{v2.4.4}{2019/07/14}{Added \string\cs{restoreDLspecialsDefs}}
%    \begin{macrocode}
  \iwvo{\string\restoreDLspecialDefs}%
  \closejs@verbatim@out
  \expandafter\xdef\csname\dljsBase OBJ\endcsname{\the\dljsobjtoks}%
  \edef\@dljstmp{\noexpand\AtBeginDocument{%
    \noexpand\input{\js@Path\dljsBase.djs}}}\@dljstmp
%    \end{macrocode}
%\changes{v2.2g}{2016/05/11}{Removed \string\cs{AtBeginDocument} code
%from the \string\texttt{execJS} environment. This causes problems with
%the order of \string\texttt{execJS} and \string\texttt{insDLJS}.}
%(2016/05/11) Remove the following code, seens to cause problems with
%the order of \texttt{execJS} and \texttt{insDLJS}.
%\begin{verbatim}
%    \ifx\firstdljs\dl@YES
%        \AtBeginDocument{\importAnFDF}%
%        \global\let\firstdljs\dl@NO
%    \fi
%\end{verbatim}
%    \begin{macrocode}
}
%    \end{macrocode}
%    \end{environment}
% \subsection{The \texttt{defineJS} Environment}\label{defineJS}
% When we create a form field that has a JavaScript action attached, it would be nice
% to be able to write verbatim JavaScript code into the action. Due to the limitations
% of \TeX, this is not possible. The other possibility is to create an environment for
% writing JavaScript. This section introduced a couple of environments, the
% \texttt{defineJS} and \texttt{localJS} environments.
%    \begin{environment}{defineJS}
% The \texttt{defineJS} environment takes two parameters, the first one optional.
%\begin{description}
%\item[\ttfamily\#1 = ] This optional parameter can be used to change catcodes
%   before the verbatim read begins.
%\item[\ttfamily\#2 = ] The command to be defined.
%\end{description}
% The material within the environment is read verbatim and a new command by the name of \#2
% is defined.
% \changes{v2.4.2}{2019/03/16}{Changed space token to \string\cs{pdfSP} in \string\env{defineJS}}
%    \begin{macrocode}
\newtoks\JStoks
%    \end{macrocode}
%   \leavevmode\DescribeMacro{\bParams}\hskip-\marginparsep\texttt{\darg{\ameta{token\SUB1}}\darg{\ameta{token\SUB2}}...\darg{\ameta{token\SUB{n}}}\cs{eParams}}
%   The \env{defineJS} environments do not
%   have parameters/arguments as a normal environment (or command) do. To enable the ability
%   to  modify the JavaScript code within the environment of \env{defineJS}, we introduce
%   the \cs{bParams}/\allowbreak\cs{eParams} construct. When you use one of the \env{defineJS} environments
%   to define field level JavaScript, you can include parameters \cs{p(1)}, \cs{p(2)}, etc., within the body
%   of the environment. (Note, within the body, there is no escape character unless you change catcode of another character
%   within the optional argument of the \env{defineJS} environments.)
%\begin{verbatim}
%   \begin{defineJS}[\makeesc\!]{\myCode}
%   var p1=!p(1), p2=!p(2);
%   app.alert("p1 + p2 = " + Number(p1+p2) );
%   \end{defineJS}
%\end{verbatim}
%   The substitution of the parameter values does not take place until the code is expanded (\cs{myCode})
%\begin{verbatim}
%   \bParams{1}{16}\eParams
%   \pushButton[\A{\JS{\myCode}}]{pbfld1}{.5in}{11bp}
%\end{verbatim}
%When the button is pressed, an alert box appears and states that `\texttt{p1 + p2 = 17}'. The same code
%can be used later in another button, for example,
%\begin{verbatim}
%   \bParams{77}{11}\eParams
%   \pushButton[\A{\JS{\myCode}}]{pbfld2}{.5in}{11bp}
%\end{verbatim}
%Using the (new) \cs{cmd} key for \pkg{eforms}, all these declarations can be made locally:
%\begin{verbatim}
%   \pushButton[\cmd{\bParams{77}{11}\eParams}
%     \A{\JS{\myCode}}]{pbfld3}{.5in}{11bp}
%\end{verbatim}
% The definitions of the parameters are made using an \cs{edef} (using a control
% name, \cs{p@r@m\ameta{cnt}}, that is probably not otherwise used). This last
% command is generally accessed through \DescribeMacro\dl@param\cs{dl@param}.
% Within all
% the form fields, as defined in \pkg{eforms},
% \DescribeMacro\dl@paramlocal\cs{dl@paramlocal} is expanded within a group; it is
% also expanded within \cs{flJSStr}. Expansion of \cs{dl@paramlocal} \cs{let}s
% \cs{p} to \cs{dl@param}; hence, \cs{p(1)}, \cs{p(2)}, etc., are defined.  Great
% care must be used in managing the use of this feature; if you are not careful
% an earlier definition of a parameter is used when it was not meant to be so;
% or a \cs{p(n)} is not defined.
%    \begin{macrocode}
\def\dl@param(#1){\csname p@r@m#1\endcsname}
\def\dl@paramlocal{\let\p\dl@param}
%    \end{macrocode}
%    This command uses the counter \texttt{dljs@cnt} and the token register
%    \cs{JStoks}, both of which are used elsewhere in this package; however,
%    there should be no conflict in using them here.
%    \begin{macrocode}
\let\eParams\relax
%    \end{macrocode}
%    When there is only one token, the braces, if any, get lost. To avoid
%    this, we determine how many tokens there are: one or more. If only one, we
%    handle separately, otherwise, we pass the flow on to \cs{bParams@i}.
%    \begin{macrocode}
\def\bParams#1{\def\p@r@m{p@r@m}\c@dljs@cnt\z@
  \@ifnextchar\eParams{\expandafter
    \def\csname\p@r@m1\endcsname{#1}}
    {\bParams@i{#1}}}
%    \end{macrocode}
%    The main definition of \cs{bParams}, we \cs{def\cs{p@r@m}\ameta{cnt}\darg{\ameta{token}}}
%    \begin{macrocode}
\def\bParams@i#1\eParams{\@tfor\x:=#1\do{\advance\c@dljs@cnt\@ne
  \JStoks=\expandafter{\x}\expandafter
  \edef\csname\p@r@m\the\c@dljs@cnt\endcsname{\the\JStoks}}}
%    \end{macrocode}
%    \DescribeMacro{\dlcombine}\hskip-\marginparsep\texttt{(\#1)(\#2)}
%    (In this discussion, it is assumed `\texttt{@}' is the escape character
%    within \env{defineJS}.)
%    Within the \env{defineJS} environment, spaces are obeyed; at times, we want to say
%    \verb*|@currQuiz MyField|, for example, which within \env{defineJS} gets expanded to
%    \verb*|q1 MyField|, not \texttt{q1MyField} as it should be. To solve this and other
%    problems like it, use \verb|@dlcombine(@currQuiz)(MyField)| which does expand
%    to the intended expression.
%    \changes{v2.4.3}{2019/06/27}{Added \string\cs{dlcombine}}
%    \begin{macrocode}
\def\dlcombine(#1)(#2){#1#2}
%    \end{macrocode}
%    \DescribeMacro{\dlcontig}\hskip-\marginparsep\texttt{(\#1)} Perhaps a better method than
%    just described (\cs{dlcombine}) is to have a single argument delimited by parentheses.
%    Thus |@currQuiz@dlcontig(MyField)| also produces \texttt{q1MyField}.
%    \begin{macrocode}
\def\dlcontig(#1){#1}
%    \end{macrocode}
%    When you declare a character to be an escape within the optional argument of \env{defineJS},
%    that character may be needed in the code; for example if we declare |\makeesc\!|, and
%    within the JavaScript code we have \texttt{if (x != 0 )...}, the JavaScript
%    code is broken. As a fix for this problem, you can declare \DescribeMacro\escIs|\escIs{!}|,
%    which in turn defines \DescribeMacro\Eschr\cs{Eschr} to expand to `\texttt!'. The above code
%    may then be written as \texttt{if (x !Eschr= 0 )...}; messy, but it works. The lines
%    below are the recommended method of declaring \cs{escIs}:
%\begin{verbatim}
%   \pushButton[\cmd{\escIs{!}\bParams{77}{11}\eParams}
%     \A{\JS{\myCode}}]{pbfld2}{.5in}{11bp}
%\end{verbatim}
%   These declarations make everything local to the group within which the button is built.
%    \begin{macrocode}
\def\escIs#1{\def\Eschr{#1}}
%    \end{macrocode}
%    \DescribeMacro\makeesc\hskip-\marginparsep\texttt{\cs{\ameta{char}}}
%    \cs{dl@makeesc} is a helper command that sets the escape character,
%     Within the \env{defineJS} environments, it is \cs{let} to \cs{makeesc};
%     eg, |\makeesc\!|.
%    \changes{v2.4.3}{2019/06/27}{Added \string\cs{dl@makeesc}}
%    \begin{macrocode}
\def\dl@makeesc#1{\catcode`#1=0\relax}
%    \end{macrocode}
%     \DescribeMacro\makecmt\hskip-\marginparsep\texttt{\cs{\ameta{char}}} Likewise,
%     you can conviently declare the comment character, if any; eg, |\makecmt\%|.
%    \changes{v2.4.10}{2020/11/28}{Added \string\cs{dl@makecmt}}
%    \begin{macrocode}
\def\dl@makecmt#1{\catcode`#1=14\relax}
%    \end{macrocode}
%    \paragraph*{Incorporating {\LaTeX} conditionals in \env{defineJS}.} {\LaTeX} conditionals
%    need to be handled carefull so as not to produce any specious spaces. Normally, there is
%    a comment character declared (|\makecmt\%|). Use the comment character to suppress the beginning
%    of a new line until it is known whether there is content on that next line. Eg,
%\begin{Verbatim}[xleftmargin=\parindent,fontsize=\small,
%commandchars={!()},codes={\def\1{\quad}\catcode`\%=9}]
%\begin{defineJS}[\makeesc\*\makecmt\!%]{\X}
%var w=1;!%
%*ifx*Y*@empty*else
%var x=1;
%var y=1;*fi
%var z=1;
%\end{defineJS}
%\end{Verbatim}
%When \cs{Y} is empty, the content expands to,
%\begin{Verbatim}[xleftmargin=\parindent,fontsize=\small,
%commandchars={!()},codes={\def\1{\quad}\catcode`\%=9}]
%var w=1;
%var z=1;
%\end{Verbatim}
%and when \cs{Y} is nonempty, it expands to,
%\begin{Verbatim}[xleftmargin=\parindent,fontsize=\small,
%commandchars={!()},codes={\def\1{\quad}\catcode`\%=9}]
%var w=1;
%var x=1;
%var y=1;
%var z=1;
%\end{Verbatim}
%    The following two commands are no longer used by this package.
%    \changes{v2.4.8}{2019/10/30}{Added \string\cs{dlgobToFi} to help format
%    \string\env{defineJS} code}
%    \begin{macrocode}
\def\dlgobToFi#1\fi{\fi\defjsLB}
\let\gobiv\@gobblefour
%    \end{macrocode}
%    \textbf{Typesetting content with the \cs{typeset} keyword macro.}
%    To aid typesetting the contents of a \env{defineJS} environment,
%    the keyword macro \DescribeMacro{\typeset}\cs{typeset}, which is defined only within the environment,
%    can be used. Pass \cs{typeset} to \env{defineJS} through the optional argument.
%    \changes{v2.5}{2020/12/20}{Added the command \string\cs{typeset} to typeset the
%    contents of the \string\env{defineJS} environment}
%    \begin{macrocode}
\newif\ifdlfortypeset \dlfortypesetfalse
%    \end{macrocode}
%    The macro \cs{dl@typesetthis} defined here, is the external version of the local command
%    \cs{typeset}. It globally sets the switch \cs{ifdlfortypeset} to true and defines the value
%    of \cs{dfnJSR} and the legacy command \cs{defineJSjsR}. For \cs{dfnJSCR}, we use only {\TeX}
%    primitives to maintain compatibility with the now defunct YandY System, which I still use.
%    \begin{macrocode}
\def\dl@typesetthis{\global\dlfortypesettrue
  \dfnJSCR{\hfil\break\null}}
%    \end{macrocode}
%    Next, we define some components common to all define-js environments. First, we initialize
%    the environment with several \cs{let}s.
%    \begin{macrocode}
\def\@initdfnJS{\let\jsFrstLne\dl@One\let\makeesc\dl@makeesc
  \let\makecmt\dl@makecmt\let\typeset\dl@typesetthis}
%    \end{macrocode}
%    (2020/12/20) Insert \cs{defineJSJsR} (an alias of \cs{dfnJSR}) at the \emph{beginning} of the second line
%    onward. This gives better formatting of the code within the editor, without
%    an empty final line.
%    \changes{v2.5}{2020/12/20}{Modified \string\cs{verbatim@line} in \string\env{defineJS}}
%    \begin{macrocode}
\def\@vrbsetupdfnJS{\def\verbatim@processline{\ifx\jsFrstLne\dl@One
  \let\jsFrstLne\dl@Zero
  \xdef\JS@temp{\the\JStoks\the\verbatim@line}\else
  \xdef\JS@temp{\the\JStoks\defineJSjsR\the\verbatim@line}\fi
  \global\JStoks=\expandafter{\JS@temp}}%
  \let\do\@makeother\dospecials\catcode`\^^M\active}
%    \end{macrocode}
%   We \cs{let} \cs{space} to \cs{pdfSP}, and declare \cs{obeyspaces}, in this way, the space
%   character is active and is equal to \cs{040} (space). Start the verbatim environment.
%   We expand \cs{obeyspaces} only if \cs{pdfSpacesOn} is in effect.
%   \changes{v2.5.2}{2021/02/07}{Modify \string\cs{b@vrbdfnJS}, remove \string\cs{global}
%   on \string\cs{let}}
%    \begin{macrocode}
\def\b@vrbdfnJS#1{\ifpdfspaces\let\space\pdfSP
  \makeatletter\obeyspaces\fi%
%    \end{macrocode}
%     Pass the first argument here. If this environment is for typesetting,
%     initialize the tokens \cs{JStoks} to be in a group, and such. The insertion
%     of \cs{dl@paramlocal} is to support the \cs{bParams}/\allowbreak\cs{eParams} construct.
%    \begin{macrocode}
  #1\ifdlfortypeset%
  \global\JStoks={\bgroup\leavevmode\dl@paramlocal}\fi%
  \verbatim@start}
%    \end{macrocode}
%    The end of the environment code
%    \begin{macrocode}
\def\e@vrbdfnJS{\gdef\eq@JStemp{\ifdlfortypeset\global
  \JStoks=\expandafter{\the\JStoks\egroup}\fi\expandafter
  \edef\defineJSArg{\the\JStoks}\global
  \dlfortypesetfalse}\global\let\space\dl@sp@ce\aftergroup\eq@JStemp}
%    \end{macrocode}
%    Finally, we get to the definition of \env{defineJS}, and its compatriates.
%    \begin{macrocode}
\newenvironment{defineJS}[2][]{\@initdfnJS
  \expandafter\@ifundefined\expandafter{%
  \expandafter\@gobble\string#2}{}{\PackageWarning{insdljs}
    {The command \string#2 already defined}}%
  \gdef\defineJSArg{#2}\global\JStoks={}%
  \@vrbsetupdfnJS
  \b@vrbdfnJS{#1}}{\e@vrbdfnJS}
%    \end{macrocode}
%    \end{environment}
%    \begin{environment}{@defineJS}
% Silent version of \texttt{defineJS}, can repeatedly be used to redefine the same
% macro. Note: |\makeesc\@| cannot be declared with the \env{@defineJS} environment.
%    \begin{macrocode}
\newenvironment{@defineJS}[2][]{\@initdfnJS
  \gdef\defineJSArg{#2}\global\JStoks={}%
  \@vrbsetupdfnJS
  \b@vrbdfnJS{#1}}{\e@vrbdfnJS}
%    \end{macrocode}
%    \end{environment}
%    \begin{environment}{defineJS*}
%    A silent public version of \env{defineJS}.
%   The \env{defineJS*} environment cannot use |\makeesc\*| in its optional argument.
% \changes{v2.4.2}{2019/03/16}{define \string\env{defineJS*} a public version of \string\env{@defineJS}}
%    \begin{macrocode}
\newenvironment{defineJS*}[2][]{\@initdfnJS
  \gdef\defineJSArg{#2}\global\JStoks={}%
  \@vrbsetupdfnJS
  \b@vrbdfnJS{#1}}{\e@vrbdfnJS}
%    \end{macrocode}
%    \end{environment}
% An author might want these definitions to be local in definition, so the
% \texttt{localJS} environment is supplied.
%    \begin{environment}{localJS}
% This environment does nothing more than to enclose the material
% in \cs{begingroup} and \cs{endgroup}.
%    \begin{macrocode}
\newenvironment{localJS}[1][]{}{\ifvmode\else\unskip\fi}
%    \end{macrocode}
%    \end{environment}
%
% \subsection{Set Document Actions}
% We provide simple environments for writing document actions.
%    \begin{environment}{willClose}
% Execute a script just before closing the document.
%    \begin{environment}{willSave}
% Execute script just before a save.
%    \begin{environment}{didSave}
% Execute a script just after a save
%    \begin{environment}{willPrint}
% Execute a script just before a print job
%    \begin{environment}{didPrint}
% Execute a script just after a print job
%    \begin{macrocode}
\newenvironment{willClose}[1][]
{\@defineJS[#1]{\will@Close}}
{\end@defineJS}
\let\will@Close\@empty
\newenvironment{willSave}[1][]
{\@defineJS[#1]{\will@Save}}
{\end@defineJS}
\let\will@Save\@empty
\newenvironment{didSave}[1][]
{\@defineJS[#1]{\did@Save}}
{\end@defineJS}
\let\did@Save\@empty
\newenvironment{willPrint}[1][]
{\@defineJS[#1]{\will@Print}}
{\end@defineJS}
\let\will@Print\@empty
\newenvironment{didPrint}[1][]
{\@defineJS[#1]{\did@Print}}
{\end@defineJS}
\let\did@Print\@empty
\def\dljs@willDidActions{/AA <<
  \ifx\test@will@Close\@empty\else/WC <<
    \JS{\will@Close} >>\fi
  \ifx\test@will@Save\@empty\else/WS <<
    \JS{\will@Save} >>\fi
  \ifx\test@did@Save\@empty\else/DS <<
    \JS{\did@Save} >>\fi
  \ifx\test@will@Print\@empty\else/WP <<
    \JS{\will@Print} >>\fi
  \ifx\test@did@Print\@empty\else/DP <<
    \JS{\did@Print} >>\fi
  >>
}
\newcommand\dljs@setDocActions{%
  \edef\test@will@Close{\will@Close}%
  \edef\test@will@Save{\will@Save}%
  \edef\test@did@Save{\did@Save}%
  \edef\test@will@Print{\will@Print}%
  \edef\test@did@Print{\did@Print}%
%    \end{macrocode}
% Driver dependent code to put any document actions into the additional actions
% dictionary of the document catalog.
%    \begin{macrocode}
  \ifpdf\pdfcatalog{\dljs@willDidActions}\else
  \ifxetex\immediate
    \special{pdf:put @catalog <<\dljs@willDidActions>>}\else
  \literalps@out{\AEB@psMrk
   {Catalog} <<\dljs@willDidActions>> /PUT pdfmark}\fi\fi
}
\@ifpackageloaded{aeb_pro}{}
    {\AtBeginDocument{\dljs@setDocActions}}
%    \end{macrocode}
%    \end{environment}
%    \end{environment}
%    \end{environment}
%    \end{environment}
%    \end{environment}
%
%    \subsection{For \texttt{dvipsone}/\texttt{dvips}}
% The method of automatic insertion of document level JavaScript is to
% write an FDF file (Forms Data Format), then import this FDF into the
% document using the JavaScript method \texttt{Doc.importAnFDF()}.
%
% We break the FDF file into three parts: the \cmd{\fdfheader} (the stuff prior
% to the JavaScript code); the JavaScript code itself; and the
% \cmd{\fdftrailer}, the stuff that follows the code.
%    \begin{macrocode}
\begingroup
\catcode`\%=12
\gdef\firstFDFline{%FDF-1.2}
\gdef\lastFDFline{%%EOF}
\endgroup
\def\fdfheader
{%
  \iwvo{\string\begingroup\string\efdlspecials}
  \iwvo{\string\isdljstrue}
  \iwvo{\string\makeatletter}
  \iwvo{\string\immediate\string\openout\string
    \js@verbatim@out=\string\js@Path\space\dljsBase.fdf}%
  \iwvo{\string\begin{jsexpverbatimwrite}}
  \iwvo{\string\firstFDFline}
  \iwvo{1 0 obj}
  \iwvo{<< /FDF << /JavaScript << /Doc 2 0 R >> >> >>}
  \iwvo{endobj}
  \iwvo{2 0 obj}
  \iwvo{[ \string\csname\string\@gobble\space\dljsBase%
    OBJ\string\endcsname]}
  \iwvo{endobj}
}
\def\fdfbeginstreamobj
{%
  \iwvo{\thedljssegs\space 0 obj}
  \iwvo{<<>>}
  \iwvo{stream}
  \applydljs{\iwvo{var _thereisdjs=true;}}
%    \end{macrocode}
% If the user has not passed the optional
% name, we generated one for him, now we insert it into the \texttt{stream}.
%\changes{v2.0k}{2012/06/21}{If the user has not passed the optional
% name, we generated one for him, now we insert it into the \string\texttt{stream}.}
%    \begin{macrocode}
  \ifx\insdljsloadVar\@empty\else
    \iwvo{\insdljsloadVar}
    \global\let\insdljsloadVar\@empty\fi
}
\def\fdfendstreamobj{%
  \iwvo{endstream}
  \iwvo{endobj}
}
\def\fdftrailer{%
  \iwvo{trailer}
  \iwvo{<< /Root 1 0 R >>}
  \iwvo{\string\lastFDFline}
%    \end{macrocode}
% Here we write \texttt{*end{jsexpverbatimwrite}} as a signal for our modified verbatim write
% code to stop.
%    \begin{macrocode}
  \iwvo{*end{jsexpverbatimwrite}}
  \iwvo{\string\endgroup}
}
%    \end{macrocode}
% There may be more than one use of the \texttt{insDLJS}
% environment and we need to be able to import each of the resultant
% FDF files. The \cmd{\importfdftoks} accumulates material to be
% used after all chance for the user to insert JS code has been
% exhausted.
%    \begin{macrocode}
\newtoks\importfdftoks \importfdftoks={}
\newtoks\dljsobjtoks \dljsobjtoks={}
%    \end{macrocode}
% This is a template for detecting whether the DLJS has been imported into
% the document.
%    \begin{macrocode}
\def\importAnFDFTemplate{%
if(typeof \detectdljs\space == "undefined")\jsR\jsT
%    \end{macrocode}
% Beginning with Acrobat 8.1, the \texttt{Doc.importAnFDF} has additional security. We must call
% it through a trusted function which is in the file aeb.js, which resides in the user JavaScript folder.
%\changes{v2.0h}{2006/10/03}
%{
%   Created aeb.js, which must be installed for users of distiller. The importAnFDF method is now called
%   through a trusted function.
%}
%    \begin{macrocode}
  ( app.viewerVersion > 8 ) ? %
    aebTrustedFunctions( this, aebImportAnFDF, %
    "\js@Path\dljsBase.fdf") : %
    this.importAnFDF("\js@Path\dljsBase.fdf");\jsR
}
%    \end{macrocode}
% Add in another template into \cmd{\importfdftoks}.
%    \begin{macrocode}
\def\addImportAnFDF#1{%
  \ifx\importdljs\dl@YES
    \edef\importAnFDFtmp{\the\importfdftoks#1}%
    \global\importfdftoks=\expandafter{\importAnFDFtmp}\fi
}
%    \end{macrocode}
% This is used by \texttt{insert@DLJS@pdfmark} and is placed
% \cmd{\AtBeginDocument}.
%    \begin{macrocode}
\def\importAnFDF{\ifx\importdljs\dl@YES\@OAction\fi}
%    \end{macrocode}
% The \texttt{insert@DLJS@pdfmark} environment writes the \cmd{\dljsBase.djs} file
% which is in turn input back in, and rewritten as \cmd{\dljsBase.fdf}. This is
% necessary to give the user a chance to modify the JavaScript code in an
% authorized way.
%    \begin{macrocode}
\newenvironment{newsegment@pdfmark}[1]{%
  \addtocounter{dljssegs}{1}\addtocounter{dljs@cnt}{1}%
  \Hy@unicodefalse\pdfstringdef{\dljsName}{#1}%
  \edef\@dljstmp{\the\dljsobjtoks(\dljsName)
    \thedljssegs\space 0 R\space}%
  \global\dljsobjtoks=\expandafter{\@dljstmp}%
  \fdfbeginstreamobj
  \js@verbatimwrite
}{%
  \fdfendstreamobj
  \endjs@verbatimwrite
}
\def\insert@DLJS@pdfmark{%
  \global\let\dljspresent\dl@YES
  \expandafter\gdef\csname dljs\dljsBase\endcsname{}%
  \ifx\importdljs\dl@YES\ifx\dl@gopro\dl@YES
    \addImportAnFDF\importAnFDFTemplate\fi\fi
  \immediate\openout \js@verbatim@out \js@Path\dljsBase.djs
  \fdfheader
  \ifx\multisegments\dl@NO\expandafter
    \newsegment\expandafter{\expandafter\dljsName\expandafter}\fi
}
\def\endinsert@DLJS@pdfmark{%
  \ifx\importdljs\dl@YES
    \ifx\multisegments\dl@NO\expandafter\endnewsegment\fi
    \fdftrailer
    \closejs@verbatim@out
    \expandafter\xdef\csname\dljsBase%
      OBJ\endcsname{\the\dljsobjtoks}%
    \edef\@dljstmp{\noexpand\AtBeginDocument{%
      \noexpand\input{\js@Path\dljsBase.djs}}}%
    \@dljstmp
    \ifx\firstdljs\dl@YES
      \AtBeginDocument{\edef\@dljstmp{\importAnFDF}%
        \@dljstmp}\global\let\firstdljs\dl@NO
    \fi
  \fi
}
%    \end{macrocode}
%
%    \subsection{For \texttt{pdftex}/\texttt{dvipdfm}}
%
% Again, we break the problem of creating DLJS into four parts: \cmd{\begindljs},
% \cmd{\enddljs}, the code itself, driver dependent material, \cmd{\write@pdftex@obj}
% and \cmd{\write@dvipdfm@obj}.
%
% The following is used by both \texttt{pdftex} and \texttt{dvipdfm}.
%    \begin{macrocode}
\begingroup
\catcode`\@=0 @catcode`@\=12
@gdef@ccpdftex{@gdef@eqesc{\\}@gdef@eqesci{\}}
@endgroup
\def\begindljs
{%
  \iwvo{\string\begingroup\string\efdlspecials}
  \iwvo{\string\isdljstrue}
%    \end{macrocode}
%\changes{v2.2f}{2016/05/07}{Include \string\cs{makeletter} in verbatim write}
%    \begin{macrocode}
  \iwvo{\string\makeatletter}
  {\uccode`c=`\%\uppercase{%
    \iwvo{\string\obeyspaces\string\obeylines
      \string\let\string^\string^M=\string\jsR c}}}
  {\escapechar=-1 \lccode`C=`\%\lowercase{%
    \iwvo{\string\\catcode`\string\\"=12C}}}
}
\def\beginseg{%
  {\lccode`P=`\{\lccode`C=`\%\lowercase{%
    \iwvo{\string\gdef\string\dljs\dljsBase\roman{dljssegs}PC}}}%
  \applydljs{\iwvo{var _thereisdjs=true;}}%
}
%    \end{macrocode}
% With \cs{enddsljs}, we now finish the macro definition with a closing right brace, followed by a
% comment, `\texttt\%, and an end of group.
%    \begin{macrocode}
\def\endseg{{\uccode`c=`\%\uccode`p=`\}\uppercase{\iwvo{pc}}}}
\def\enddljs{\iwvo{\string\endgroup}}
%    \end{macrocode}
%\changes{v2.0j}{2011/06/27}{made some changes as suggested by Christian Feuersaenger, who reported some
% problem clashes between \string\textsf{babel} (spanish) and \string\textsf{insdljs}.}
%    \begin{macrocode}
\def\@roman#1{\romannumeral#1}
\def\write@objs{%
  \iwvo{\begingroup}
  \iwvo{\string\catcode`\string\<=12 }
  \iwvo{\string\catcode`\string\>=12 }
  {\lccode`C=`\%\lowercase{\iwvo{\string\ccpdftex C}}}
  {\lccode`C=`\%\lowercase{\iwvo{%
    \string%%
%% This is file `dljscc.def',
%% generated with the docstrip utility.
%%
%% The original source files were:
%%
%% insdljs.dtx  (with options: `copyright,cc4js')
%% 
%%%%%%%%%%%%%%%%%%%%%%%%%%%%%%%%%%%%%%%%%%%%%%%%%%%%%%%%%%
%% insdljs.sty package,                                 %%
%% Copyright (C) 2001-2021  D. P. Story                 %%
%%   dpstory@uakron.edu                                 %%
%%                                                      %%
%% This program can redistributed and/or modified under %%
%% the terms of the LaTeX Project Public License        %%
%% Distributed from CTAN archives in directory          %%
%% macros/latex/base/lppl.txt; either version 1 of the  %%
%% License, or (at your option) any later version.      %%
%%%%%%%%%%%%%%%%%%%%%%%%%%%%%%%%%%%%%%%%%%%%%%%%%%%%%%%%%%
\def\r{\eqesc r\ckivspace}
\def\t{\eqesc t\ckivspace}
\def\n{\eqesc n\ckivspace}
\def\f{\eqesc f\ckivspace}
\def\v{\eqesc v\ckivspace}
\def\b{\eqesc b\ckivspace}
\def\B{\eqesc B\ckivspace}
\def\d{\eqesc d\ckivspace}
\def\D{\eqesc D\ckivspace}
\def\s{\eqesc s\ckivspace}
\def\S{\eqesc S\ckivspace}
\def\w{\eqesc w\ckivspace}
\def\W{\eqesc W\ckivspace}
\begingroup
\catcode`\:=12\relax
\dljstfor\dlhex:={AA}{AB}{AC}{AD}{AE}{AF}%
    {BA}{BB}{BC}{BD}{BE}{BF}{CA}{CB}{CC}{CD}{CE}{CF}%
    {DA}{DB}{DC}{DD}{DE}{DF}{EA}{EB}{EC}{ED}{EE}{EF}%
    {FA}{FB}{FC}{FD}{FE}{FF}\do{%
    \expandafter\xdef\csname x\dlhex\endcsname{\noexpand\eqesc x\dlhex}%
}
\endgroup
\def\u{\eqesc u\ckivspace}
\def\0{\eqesc0}
\def\1{\eqesc1}
\def\2{\eqesc2}
\def\3{\eqesc3}
\definebraces
\def\({\eqesc\eqesci(}
\def\){\eqesc\eqesci)}
\def\.{\eqesc.}
\def\/{\eqesc/}
\def\[{\eqesc[}
\def\]{\eqesc]}
\def\|{\eqesc|}
\def\+{\eqesc+}
\def\*{\eqesc*}
\def\-{\eqesc-}
\def\?{\eqesc?}
\def\${\eqesc$}
\def\^{\eqesc^}
\def\\{\eqesc\eqesc}
\def\'{\eqesc'}
\catcode`\"=12
\def\"{\eqesc"}
\catcode`\<=12
\def\<{\eqesc<}
\catcode`\>=12
\def\>{\eqesc>}
\catcode`\%=12
\def\%{%}
\catcode`\%=14
\catcode`\& = 12
\def\&{\eqesc&}
\endinput
%%
%% End of file `dljscc.def'.
C\the\dljsobjtoks}}}
  \iwvo{\endgroup}
}
%    \end{macrocode}
%    \subsubsection{\texttt{pdftex} Specific Code}
%    \begin{macrocode}
\newenvironment{newsegment@pdftex}[1]{%
  \addtocounter{dljssegs}{1}%
  \addtocounter{dljs@cnt}{1}%
  \edef\tmp{^^J\string\immediate\string\pdfobj{ << /S/JavaScript/JS%
  (\string\dljs\dljsBase\roman{dljssegs}) >> }}%
  \edef\@dljstmp{\the\dljsobjtoks\tmp}%
  \global\dljsobjtoks=\expandafter{\@dljstmp}%
  \edef\tmp{^^J\string\xdef\string\obj\dljsBase\roman{dljssegs}%
      {\string\the\string\pdflastobj\string\space 0 R}}%
  \edef\@dljstmp{\the\dljsobjtoks\tmp}%
  \global\dljsobjtoks=\expandafter{\@dljstmp}%
  \edef\curr@Cnt{\ifnum\arabic{dljs@cnt}<10 0\fi\arabic{dljs@cnt}}%
  \Hy@unicodefalse\pdfstringdef{\dljsName}{#1}%
  \edef\dljspdftextmp
  {\the\importfdftoks (\curr@Cnt\space\dljsName)
    \noexpand\csname obj\dljsBase\roman{dljssegs}%
    \noexpand\endcsname\space}%
  \global\importfdftoks=\expandafter{\dljspdftextmp}%
  \beginseg
  \js@verbatimwrite
}{%
  \endjs@verbatimwrite
  \endseg
}
%    \end{macrocode}
% The main branch of the \texttt{insDLJS} for \texttt{pdftex}. This
% environment writes to the file \cmd{\dljsBase.djs} all the necessary code, then
% is input back into the file using \cmd{\AtBeginDocument}.
%    \begin{macrocode}
\newenvironment{insert@DLJS@pdftex}{%
  \expandafter\gdef\csname dljs\dljsBase\endcsname{}%
  \immediate\openout \js@verbatim@out \js@Path\dljsBase.djs
  \begindljs
  \ifx\multisegments\dl@NO\expandafter
    \newsegment\expandafter{\expandafter\dljsName\expandafter}\fi
}{%
  \ifx\multisegments\dl@NO\expandafter\endnewsegment\fi
  \enddljs
  \write@objs
  \endjs@verbatimwrite
  \closejs@verbatim@out
  \edef\@dljstmp{\noexpand\AtBeginDocument{%
    \noexpand\input{\js@Path\dljsBase.djs}}}%
  \@dljstmp
  \ifx\importdljs\dl@YES
    \ifx\firstdljs\dl@YES
      \AtEndDocument{\edef\@dljstmp{\setDLJSRef@pdftex}\@dljstmp}
      \global\let\firstdljs\dl@NO
    \fi
  \fi
}
%    \end{macrocode}
% This code places the \texttt{/JavaScript} key-value in the \texttt{/Names} dictionary
% of the PDF document. This code is inserted into the document in \texttt{insert@DLJS@pdftex}
% using \cmd{\AtEndDocument}.
%    \begin{macrocode}
\def\setDLJSRef@pdftex{\noexpand
  \immediate\noexpand\pdfobj {%
    << /Names [\the\importfdftoks] >> }%
  \edef\noexpand\objNames{\noexpand\the\noexpand\pdflastobj\space 0 R}%
  \pdfnames {/JavaScript \noexpand\objNames}%
}
%    \end{macrocode}
%    \subsubsection{\texttt{dvipdfm} Specific Code}
% Begin by writing driver-specific code to \cmd{\dljsBase.djs}.
%    \begin{macrocode}
\newenvironment{newsegment@dvipdfm}[1]
{%
  \addtocounter{dljssegs}{1}\addtocounter{dljs@cnt}{1}%
  \edef\tmp{^^J\string\immediate\string
    \csname\space @pdfm@mark\string\endcsname
    {obj @obj\dljsBase\roman{dljssegs}\space << /S/JavaScript%
    /JS(\string\dljs\dljsBase\roman{dljssegs}) >> }}%
  \edef\@dljstmp{\the\dljsobjtoks\space\tmp}%
  \global\dljsobjtoks=\expandafter{\@dljstmp}%
  \edef\curr@Cnt{\ifnum\arabic{dljs@cnt}<10 0\fi\arabic{dljs@cnt}}%
  {\def\1{\string\1}\def\2{\string2}%
  \def\3{\string\3}\xdef\dljsName{#1}}%
  \edef\dljspdftextmp
  {\the\importfdftoks (\curr@Cnt\space \dljsName) %
    @obj\dljsBase\roman{dljssegs}\space}%
  \global\importfdftoks=\expandafter{\dljspdftextmp}%
  \beginseg
  \js@verbatimwrite
}{%
  \endjs@verbatimwrite
  \endseg
}
%    \end{macrocode}
% This code places the \texttt{/JavaScript} key-value in the \texttt{/Names} dictionary
% of the PDF document. This code is inserted into the document in \texttt{insert@DLJS@dvipdfm}
% using \cmd{\AtEndDocument}.
%    \begin{macrocode}
\def\setDLJSRef@dvipdfm
{%
  \immediate\special{pdf:obj @objnames %
    << /Names [\the\importfdftoks] >> }%
  \special{pdf:put @names %
    << /JavaScript @objnames >> }%
}
%    \end{macrocode}
% The main branch of the \texttt{insDLJS} for \texttt{dvipdfm}. This
% environment writes to the file \cmd{dljsBase.djs} all the necessary code, then
% is input back into the file using \cmd{\AtBeginDocument}.
%    \begin{macrocode}
\newenvironment{insert@DLJS@dvipdfm}
{%
  \expandafter\gdef\csname dljs\dljsBase\endcsname{}%
  \immediate\openout \js@verbatim@out \js@Path\dljsBase.djs
  \begindljs
  \ifx\multisegments\dl@NO\expandafter\newsegment\expandafter{%
    \expandafter\dljsName\expandafter}\fi
}{%
  \ifx\multisegments\dl@NO\expandafter\endnewsegment\fi
  \enddljs
  \write@objs
  \endjs@verbatimwrite
  \closejs@verbatim@out
  \edef\@dljstmp{\noexpand\AtBeginDocument{%
    \noexpand\input{\js@Path\dljsBase.djs}}}%
  \@dljstmp
  \ifx\importdljs\dl@YES
    \ifx\firstdljs\dl@YES
%    \end{macrocode}
% 2012/06/09 Changed positioning of \cmd{\setDLJSRef@dvipdfm}.
%    \begin{macrocode}
      \AtBeginDocument{\setDLJSRef@dvipdfm}%
      \global\let\firstdljs\dl@NO
    \fi
  \fi
}
\catcode`\"=\dl@dquoteCat
%</package>
%    \end{macrocode}
%    \begin{macrocode}
%<*cc4js>
%    \end{macrocode}
%\section{Command Changes for JavaScript}\label{dljscc}
% Regular expressions are a very important part of JavaScript, but they do present
% some problems for \LaTeX.  Ideally, we would like to type JavaScript code into the
% \texttt{insDLJS} environment using standard JavaScript syntax.
% (Whether I am successful remains to be seen.) As a result, some new definitions
% and changes in old definitions are necessary.  All definitions take place within a
% group, so they are unknown outside the \texttt{insDLJS} environment.
%
%\subsection{Regular Expressions}
% Regular expressions can be constructed in two ways, by using (1) a literal text format; or (2)
% using the \texttt{RegExp} constructor function. The literal text format:
%\begin{flushleft}
%\texttt/\textsl{pattern}\texttt/\textsl{flags}
%\end{flushleft}
% The \texttt{RegExp} constructor method:
%\begin{flushleft}\ttfamily
%new RegExp("\textsl{pattern}" [, "\textsl{flags}"])
%\end{flushleft}
% For additional details on regular expressions, see
%\begin{flushleft}\small
%\url{http://developer.netscape.com/docs/manuals/js/core/jsref15/regexp.html}
%\end{flushleft}
% There is a further complication when dealing with regular expressions. To quote from the above
% document:\par\medskip\noindent
% \textbf{Description }
% When using the constructor function, the normal string escape rules (preceding special characters with `|\|' when included in a string) are necessary.
% For example, the following are equivalent:
% \begin{flushleft}
% \ttfamily\obeylines
% re = new RegExp("\string\\w+")
% re = /\string\w+/
%\end{flushleft}
%\subsection{Begin Definitions}
% \subsubsection{Handling of Formatting Sequences}
% The macro \cmd{\eqesc} expands to `|\|' in the case of the distiller and to `|\\|' in the
% case of \textsf{pdftex} or \textsf{dvipdfm}. This group of special characters formats the
% output and they require a different definition than most of the others.
%
% The \cs{ckivspace} (``check for space'') macro sees whether the next token is a space,
% if yes, it absorbs it, if not, it replaces it. I implemented this in order to get
% final output identical to the JavaScript. For example, the following is legal JavaScript:
%\begin{Verbatim}[xleftmargin=\parindent,fontsize=\small,
%codes={\def\1{\quad}\catcode`\%=9}]
%var str = "First line.\r\rSkip a line.     // (1)
%\end{Verbatim}
% This is a problem for \TeX. In order to delimit the macro, \TeX{} users need to write the
% above line as,
%\begin{Verbatim}[xleftmargin=\parindent,fontsize=\small,
%codes={\def\1{\quad}\catcode`\%=9}]
% var str = "First line.\r\r Skip a line.    // (2)
%\end{Verbatim}
% which then introduces a spurious space into the JavaScript code. The \cs{ckivspace} solves this
% problem, I hope. With the definition of \cs{r} given below, after expansion of the line (2)
% occurs, we have line (1), the space following the second \cs{r} gets absorbed. Note: (2014/05/20)
% I moved this to the main code thread of \textsf{insdljs}.
% \changes{v2.0n}{2014/05/20}{Moved code for \string\cs{ckivspace} to main code thread of \string\textsf{insdljs}.}
%    \begin{macrocode}
%\def\ckivspace#1{\if\noexpand#1\space\else\expandafter#1\fi}
%    \end{macrocode}
%    \begin{macro}{\r}
% Matches a carriage return.
%    \begin{macrocode}
\def\r{\eqesc r\ckivspace}
%    \end{macrocode}
%    \end{macro}
%    \begin{macro}{\t}
% Matches a tab.
%    \begin{macrocode}
\def\t{\eqesc t\ckivspace}
%    \end{macrocode}
%    \end{macro}
%    \begin{macro}{\n}
% Matches a linefeed.
%    \begin{macrocode}
\def\n{\eqesc n\ckivspace}
%    \end{macrocode}
%    \end{macro}
%    \begin{macro}{\f}
% Matches a form-feed.
%    \begin{macrocode}
\def\f{\eqesc f\ckivspace}
%    \end{macrocode}
%    \end{macro}
%    \begin{macro}{\v}
% Matches a form-feed.
%    \begin{macrocode}
\def\v{\eqesc v\ckivspace}
%    \end{macrocode}
%    \end{macro}
% \subsubsection{Matching a Word Boundary or Not}
%    \begin{macro}{\b}
% Matches a word boundary, such as a space. (Not to be confused with |[\b]|, which
% matches a backspace.)
%    \begin{macrocode}
\def\b{\eqesc b\ckivspace}
%    \end{macrocode}
%    \end{macro}
% The next group are all defined in the same way, pretty much. The macro \cmd{\eqesc} expands to `|\|'
% in the case of the distiller options, and to `|\\\|' otherwise.
%    \begin{macro}{\B}
% Matches a non-word boundary.
%    \begin{macrocode}
\def\B{\eqesc B\ckivspace}
%    \end{macrocode}
%    \end{macro}
%    \begin{macro}{\cX}
% Where X is a letter from A - Z. Matches a control character in a string.  Doesn't seem to have
% much use in Acrobat JavaScript, will comment this one out.   Usage: \cs{cM}
%    \begin{macrocode}
% \def\doAlpha{\do A\do B\do C\do D\do E\do F%
%   \do G\do H\do I\do J\do K\do K\do M\do N%
%   \do O\do P\do Q\do R\do S\do T\do U\do V\do W\do X\do Y\do Z}
% \def\mkDefns#1{\expandafter\def\csname c#1\endcsname{\eqesc c#1}}
% \let\do=\mkDefns\doAlpha
%    \end{macrocode}
%    \end{macro}
% \subsubsection{Matching Digits or Not}
%    \begin{macro}{\d}
% Matches a digit character. Equivalent to \texttt{[0-9]}.
%    \begin{macrocode}
\def\d{\eqesc d\ckivspace}
%    \end{macrocode}
%    \end{macro}
%    \begin{macro}{\D}
% Matches any non-digit character. Equivalent to |[^0-9]|.
%    \begin{macrocode}
\def\D{\eqesc D\ckivspace}
%    \end{macrocode}
%    \end{macro}
%\subsubsection{Matching Whitespaces or Not}
%    \begin{macro}{\s}
% Matches a single white space character, including space, tab, form feed, line feed.
% Equivalent to |[\f\n\r\t\u00A0\u2028\u2029]|.
%    \begin{macrocode}
\def\s{\eqesc s\ckivspace}
%    \end{macrocode}
%    \end{macro}
%    \begin{macro}{\S}
% Matches a single character other than white space.\newline
% Equivalent to |[^ \f\n\r\t\u00A0\u2028\u2029]|.
%    \begin{macrocode}
\def\S{\eqesc S\ckivspace}
%    \end{macrocode}
%    \end{macro}
%\subsubsection{Matching Alphanumeric Characters or Not}
%    \begin{macro}{\w}
% Matches any alphanumeric character including the underscore. Equivalent to |[A-Za-z0-9\_]|.
%    \begin{macrocode}
\def\w{\eqesc w\ckivspace}
%    \end{macrocode}
%    \end{macro}
%    \begin{macro}{\W}
% Matches any non-word character. Equivalent to |[^A-Za-z0-9_]|.
%    \begin{macrocode}
\def\W{\eqesc W\ckivspace}
%    \end{macrocode}
%    \end{macro}
% \subsubsection{Handling of Character Codes}
%    \begin{macro}{\xXX}
% Matches the character with the code \texttt{XX} (two hexadecimal digits).
% The character with the Latin-1 encoding specified by the two hexadecimal digits \texttt{XX}
% between 00 and FF. For example, \cs{xA9} is the hexadecimal sequence for the copyright
% symbol.
%\begin{flushleft}
%\textbf{Usage}
%\begin{verbatim}
%\x9A
%\end{verbatim}
%if the most significant digit is a number \texttt{[0-9]},
%\begin{verbatim}
%\x A9
%\end{verbatim}
%if the most significant digit is a letter \texttt{[a-fA-F]}
%\end{flushleft}
%\changes{v2.3b}{2016/10/13}{JPC reported bug with French-Babel, traced it to this code.}
%To avoid a French-Babel clash (reported by JPC), we change \cs{catcode} of colon (:) to other, enclose in a group
%and do \cs{xdef} instead of \cs{edef}.
%    \begin{macrocode}
\begingroup
\catcode`\:=12\relax
\dljstfor\dlhex:={AA}{AB}{AC}{AD}{AE}{AF}%
    {BA}{BB}{BC}{BD}{BE}{BF}{CA}{CB}{CC}{CD}{CE}{CF}%
    {DA}{DB}{DC}{DD}{DE}{DF}{EA}{EB}{EC}{ED}{EE}{EF}%
    {FA}{FB}{FC}{FD}{FE}{FF}\do{%
    \expandafter\xdef\csname x\dlhex\endcsname{\noexpand\eqesc x\dlhex}%
}
\endgroup
%    \end{macrocode}
%    \end{macro}
%    \begin{macro}{\uXXXX}
% In a regular expression search, \cs{uXXXX} matches the character
% with unicode character code \texttt{XXXX} (four hexadecimal
% digits). For example, \cs{u00A9} is the Unicode sequence for the
% copyright symbol. See Unicode Escape Sequences. http://www.unicode.org
%\begin{flushleft}
%\textbf{Usage}
%\begin{verbatim}
%\u00A9
%\end{verbatim}
%if the most significant digit is a number \texttt{[0-9]},
%\begin{verbatim}
%\u F0A9
%\end{verbatim}
%if the most significant digit is a letter \texttt{[a-fA-F]}
%\end{flushleft}
%    \begin{macrocode}
\def\u{\eqesc u\ckivspace}
%    \end{macrocode}
%    \end{macro}
%    \begin{macro}{\XXX}
% The character with the Latin-1 encoding specified by up to three octal digits \texttt{XXX} between
% 0 and 377. For example, \cs{251} is the octal sequence for the copyright symbol.
%
% Since this goes only up to octal 377, it suffices to make only the following four definitions.
%    \begin{macrocode}
\def\0{\eqesc0}
\def\1{\eqesc1}
\def\2{\eqesc2}
\def\3{\eqesc3}
%    \end{macrocode}
%    \end{macro}
%\subsubsection{Other Special Characters}
%    \begin{macro}{braces}
% Braces are a problem when they are unbalanced.  In a regular expression, |\{| works properly, but |\\{| works
% for the distiller options but not otherwise. For example, if I wanted to replace every instance
% of \texttt{\{} with \texttt{(}, we can do this by
%\begin{Verbatim}[xleftmargin=\parindent,fontsize=\small,
%codes={\def\1{\quad}\catcode`\%=9}]
%myString = myString.replace( /\{/, "\(");
%\end{Verbatim}
% This works for all options, even though the braces are not balanced. However, when the constructor
% function is used, we must double-escape:
%\begin{Verbatim}[xleftmargin=\parindent,fontsize=\small,
%codes={\def\1{\quad}\catcode`\%=9}]
%var re = new RegExp("\\{", "g");
%myString = myString.replace( re, "\(");
%\end{Verbatim}
% This works for the distiller options, but not for \texttt{pdftex/dvipdfm}. The reason |\{| works is that
% it is a control sequence, defined by the \cmd{\definebraces} command below. Where as |\\{| consists
% of the control sequence |\\|, which is defined below, followed by a left brace.
% \TeX{} still requires the braces to be balanced.
%
% If you are searching for balanced braces, there should't be a problem, for example,
% the code
%\begin{verbatim}
%     var re = new RegExp("\\{|\\}", "g");
%     myString = myString.replace( re, "\(");
%\end{verbatim}
% works correctly.
%
% To continue the discussion of the problem with searching for
% unbalanced braces, one workaround is to balance out the left
% brace with a right brace in the form of a JavaScript comment.
% This ``fools'' \TeX!  For example,
%\begin{Verbatim}[xleftmargin=\parindent,fontsize=\small,
%codes={\def\1{\quad}\catcode`\%=9}]
%var re = new RegExp("\\{", "g");           // }
%myString = myString.replace( re, "\(");
%\end{Verbatim}
% Finally, we have\par\medskip
% \noindent\textbf{The Recommended Workaround:} Use |\\\{| and |\\\}|, e.g.,
%\begin{Verbatim}[xleftmargin=\parindent,fontsize=\small,
%codes={\def\1{\quad}\catcode`\%=9}]
%var re = new RegExp("\\\{", "g");
%myString = myString.replace( re, "\(");
%\end{Verbatim}
% This is more verbose than needed, but it works for all options! These comments are only for regular expression
% strings within the constructor function. Note the replacement text, |\(|, is only escaped.
%
% Here we make the definitions for left and right braces. See the definition of \cs{definebraces}
% given earlier.
%    \begin{macrocode}
\definebraces
%    \end{macrocode}
%    \end{macro}
%    \begin{macro}{parentheses}
% Just as braces can give us problems with \TeX, so too, parentheses can give us problems with
% Acrobat JavaScript interpreter.  There seems to be a requirement that parentheses be balanced
% unless they are properly escaped.  Generally, we have no problems in the case of distiller options,
% once again \textsf{pdftex} and \textsf{dvipdfm} are causing (me) problems.
%
% No problems when the regular expression uses the literal string format
%\begin{Verbatim}[xleftmargin=\parindent,fontsize=\small,
%codes={\def\1{\quad}\catcode`\%=9}]
%myString = myString.replace(/\(/g, "\[");
%\end{Verbatim}
% The above code works for all options. However, for the constructor function we have some problems
% in non-distiller options. If we type instead
%\begin{Verbatim}[xleftmargin=\parindent,fontsize=\small,
%codes={\def\1{\quad}\catcode`\%=9}]
%var re = new RegExp("\\(", "g");
%myString = myString.replace(re, "\[");
%\end{Verbatim}
% The above code works for the distiller folks, but not for the
% others. In the case of \textsf{pdftex}, I cannot access the UI to
% the document level JavaScripts, and the script is undefined. I suspect
% the lack of balanced parentheses is the curprit.  For example,
% consider the code
%\begin{Verbatim}[xleftmargin=\parindent,fontsize=\small,
%codes={\def\1{\quad}\catcode`\%=9}]
%var re = new RegExp("\\(|\\)", "g");
%myString = myString.replace(re, "\[");
%\end{Verbatim}
% works. I can access the UI to the DLJS from Acrobat, and the script is defined.  Another
% example, we can fool the JavaScript interpreter as we did the \TeX{} compiler:
%\begin{Verbatim}[xleftmargin=\parindent,fontsize=\small,
%codes={\def\1{\quad}\catcode`\%=9}]
%var re = new RegExp("\\(", "g");  // )
%myString = myString.replace(re, "\[");
%\end{Verbatim}
% Here I have placed a balancing right parenthesis as a JavaScript comment. Both sets of code
% work correctly.
%\par\medskip
%\noindent\texttt{Recommended Workaround:} As in the case of parentheses, a more verbose version
% of the script works of all options.
%\begin{Verbatim}[xleftmargin=\parindent,fontsize=\small,
%codes={\def\1{\quad}\catcode`\%=9}]
%var re = new RegExp("\\\(", "g");
%myString = myString.replace(re, "\[");
%\end{Verbatim}
% This is only for the case of the constructor function.
%    \begin{macrocode}
\def\({\eqesc\eqesci(}
\def\){\eqesc\eqesci)}
%    \end{macrocode}
%    \end{macro}
% The rest of these definitions are for characters that have a
% special meaning when they appear in a regular expression. See
%\begin{flushleft}\small
%\url{http://developer.netscape.com/docs/manuals/js/core/jsref15/regexp.html}
%\end{flushleft}
% for a description of the meaning for the special characters given below (and above).
% To ``escape'' the special meaning of any character, ``escape'' them.
%    \begin{macrocode}
\def\.{\eqesc.}
\def\/{\eqesc/}
\def\[{\eqesc[}
\def\]{\eqesc]}
%    \end{macrocode}
%    \begin{macrocode}
\def\|{\eqesc|}
\def\+{\eqesc+}
\def\*{\eqesc*}
\def\-{\eqesc-}
\def\?{\eqesc?}
\def\${\eqesc$}
\def\^{\eqesc^}
\def\\{\eqesc\eqesc}
%    \end{macrocode}
% The next group of characters have no special meaning in a regular expression, but are using elsewhere
% in JavaScript. These definitions may come in handy.
%    \begin{macrocode}
\def\'{\eqesc'}
%    \end{macrocode}
% The \texttt{german} package makes doublequotes, |"|, active, so within the \texttt{insDLJS}
% environment, we change its catcode to 12.
%    \begin{macrocode}
\catcode`\"=12
\def\"{\eqesc"}
%    \end{macrocode}
% We de-activate angle brackets.
%    \begin{macrocode}
\catcode`\<=12
\def\<{\eqesc<}
\catcode`\>=12
\def\>{\eqesc>}
%    \end{macrocode}
% Here, we define `\texttt{\%}' differently. The percent has no special meaning in regular expressions,
% but is used by the \texttt{util.printf()} method.
%    \begin{macrocode}
\catcode`\%=12
\def\%{%}
\catcode`\%=14
%    \end{macrocode}
% The `and' symbol has not special meaning for regular expressions, but is used as a logical `and'.
% Probably don't need this escape.
%    \begin{macrocode}
\catcode`\& = 12
\def\&{\eqesc&}
%    \end{macrocode}
%    \begin{macrocode}
%</cc4js>
%    \end{macrocode}
%    \begin{macrocode}
%<*altadbfncs>
%    \end{macrocode}
%    \section{Alternate names for some Adobe functions}\label{s:AltAdbFncs}
%    (2017/03/02) Added alternate names for some common Adobe functions. Used for
%    creating custom keystroke and formatting events.
%    \changes{v2.3f}{2017/03/02}{Alternate names for Adobe functions added}
%    \begin{macrocode}
\begin{insDLJS}{altadbfncs}{indljs: Alternate Adobe Function Names}
var EFNumber_Keystroke=AFNumber_Keystroke;
var EFNumber_Format=AFNumber_Format;
var EFPercent_Keystroke=AFPercent_Keystroke;
var EFPercent_Format=AFPercent_Format;
var EFDate_Format=AFDate_Format;
var EFDate_Keystroke=AFDate_Keystroke;
var EFDate_FormatEx=AFDate_FormatEx;
var EFTime_Keystroke=AFTime_Keystroke;
var EFTime_Format=AFTime_Format;
var EFTime_FormatEx=AFTime_FormatEx;
var EFDate_KeystrokeEx=AFDate_KeystrokeEx;
var EFSpecial_Keystroke=AFSpecial_Keystroke;
var EFSpecial_Format=AFSpecial_Format;
var EFSpecial_KeystrokeEx=AFSpecial_KeystrokeEx;
var EFRange_Validate=AFRange_Validate;
var EFSimple_Calculate=AFSimple_Calculate;
var EFMergeChange=AFMergeChange;
\end{insDLJS}
%</altadbfncs>
%    \end{macrocode}
%    \begin{macrocode}
%<*package>
\inputAltAdbFncs
\dl@restoreCats
%</package>
%    \end{macrocode}
%  \Finale
%\endinput
